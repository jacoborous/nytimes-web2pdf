Sections

SEARCH

\protect\hyperlink{site-content}{Skip to
content}\protect\hyperlink{site-index}{Skip to site index}

\href{https://myaccount.nytimes.com/auth/login?response_type=cookie\&client_id=vi}{}

\href{https://www.nytimes.com/section/todayspaper}{Today's Paper}

\href{/section/opinion}{Opinion}\textbar{}Trump Supporters Make Their
Case for 2020

\href{https://nyti.ms/2Xwus9w}{https://nyti.ms/2Xwus9w}

\begin{itemize}
\item
\item
\item
\item
\item
\end{itemize}

Advertisement

\protect\hyperlink{after-top}{Continue reading the main story}

transcript

Back to The Argument

bars

0:00/0:00

-0:00

transcript

\hypertarget{trump-supporters-make-their-case-for-2020}{%
\subsection{Trump Supporters Make Their Case for
2020}\label{trump-supporters-make-their-case-for-2020}}

\hypertarget{with-ross-douthat}{%
\subsubsection{With Ross Douthat}\label{with-ross-douthat}}

\hypertarget{conservatives-helen-andrews-and-daniel-mccarthy-join-ross-for-a-reelection-roundtable}{%
\paragraph{Conservatives Helen Andrews and Daniel McCarthy join Ross for
a reelection
roundtable.}\label{conservatives-helen-andrews-and-daniel-mccarthy-join-ross-for-a-reelection-roundtable}}

Thursday, August 6th, 2020

\begin{itemize}
\item
  ross douthat\\
  I'm Ross Douthat, and this is ``The Argument.'' {[}MUSIC PLAYING{]}

  If you've ever taken the time to read this podcast's description,
  you'll know that our mission is to host arguments from across the
  political spectrum. If you've listened to the show every week, you'll
  also notice that I'm usually holding up the rightmost end of our
  debates, but not this week. While my more liberal co-hosts, Michelle
  and Frank, are on vacation, I'm conducting a temporary right wing
  coup. I've asked two writers who are somewhat to my right, meaning,
  among other things, mostly pro-Trump rather than ``Never Trump'' to
  join the show. We'll talk about the president the pandemic, the Black
  Lives Matter protests, 2020 elections, and the future of conservatism.
  {[}MUSIC PLAYING{]}

  First is Dan McCarthy, the editor of Modern Age, a conservative
  quarterly of ideas, a director of journalism fellowships at the fund
  for American Studies, and a frequent times op-ed contributor. He's
  also been a guest on this show. Dan, welcome back.
\item
  daniel mccarthy\\
  Thanks, Ross.
\item
  ross douthat\\
  And joining Dan is Helen Andrews, who is a senior editor at the
  American Conservative, and the author of ``Boomers, The Men and Women
  Who Promised Freedom and Delivered Disaster,'' which is out next
  January. Helen, thanks so much for coming on ``The Argument.''
\item
  helen andrews\\
  Thanks so much for having me.
\item
  ross douthat\\
  You're very welcome. So let's talk about some issues where I can
  hopefully play the totally unaccustomed role of moderate squish. So I
  thought we could start with how President Trump has handled the
  coronavirus crisis, because given that he's currently seven or eight
  or nine points down in the polls to Joe Biden, I suspect that we can
  all agree that he could have handled it better, but since both of you
  have been somewhat skeptical of lockdowns and other tough public
  health measures, I also suspect that we disagree about what a better
  strategy would have been. So starting with you, Dan, what do you think
  the president could have done differently over the last five months or
  so?
\item
  daniel mccarthy\\
  Well, you know, I don't think there's any kind of silver bullet here.
  If we look at the responses that other countries have had to COVID-19,
  while there are places you can point to that have had considerable
  success, there are places that are very different from the United
  States, that have leadership that is very different, places like
  Sweden, for example, places like the United Kingdom, where you have a
  conservative leader, but one who can recite ``The Iliad'' from memory.
  Quite a contrast from Donald Trump, perhaps. There are places like
  Italy, which has a government that's always rather rickety. You can
  see a wide swath of different governments responding in different ways
  to COVID-19, and a great many of them are getting it wrong. And the
  ones that have gotten it right, even they are starting to have a
  resurgence of the virus and other difficulties. So I just don't know
  that there is anything Donald Trump could have done that would have
  been the one right way of approaching this. And of course, this is
  something where the role of the governors in our 50 states is also of
  the utmost importance, and in fact, is of rather more importance than
  the president himself. Perhaps the easiest thing to say about what the
  president could have done is to perhaps act as psychiatrist in chief,
  not as physician in chief, but someone who could have made some
  reassuring statements that might have made people simply feel better
  about this crisis. But I don't think that would have had much
  practical effect. It seems to me that some of these crises that we're
  facing are --- they're acts of God. They are things that it's very
  hard for any kind of leadership simply to overcome.
\item
  ross douthat\\
  That's some pretty deep fatalism, Dan. Helen, I'm hoping that you will
  make a more explicit case for why we should have simply gone full
  Sweden.
\item
  helen andrews\\
  Unfortunately, I'm going to have to disappoint you there, and also to
  disagree with the premise of your question. I don't think the
  president has handled the coronavirus crisis badly. I think he's
  actually handled it pretty well. One of the great frustrations for
  people who generally like and support this president is that he seems
  to be someone who has a hard time understanding that often, the
  correct response to a situation is to say nothing. He seems to have a
  real problem grasping that sometimes silence is the best policy. But
  here, with the coronavirus, he seems to have finally grasped that
  lesson. He hasn't made himself the point man on every policy question,
  and I think that that's a good thing, because a lot of this is stuff
  that should be decided by governors anyway. So there's no reason for a
  president to have a position on a national mask mandate or something
  like that. And there are a lot of things that we know now that we
  wouldn't know about comparisons between different strategies because
  we allowed a diversity of responses on the part of different states.
  So I think the fact that the president has left space for that to
  happen is actually a pretty good decision on his part.
\item
  ross douthat\\
  How would you make the case for the president's re-election in the
  context of, let's say, by the time the fall rolls around, up to
  200,000 or so deaths from the coronavirus? Because I think it's fair
  to say that if you had flashback to say February or early March when
  we were first having conversations/panics about COVID-19, that if you
  had said, well, over the course of the next six months, 200,000
  Americans are going to die from this disease, a lot of people would
  have assumed that the president's re-election efforts were doomed, and
  I don't think either of you think his re-election effort is doomed.
\item
  daniel mccarthy\\
  Well, his re-election chances aren't doomed, but there is a limit to
  what any president can do in the face of certain public assumptions,
  which are of a almost divine nature. So I'm a very strong critic of
  President George W Bush. I was very much against most of his domestic
  policies, and even more so, his foreign policy, especially the Iraq
  war. One thing, however, that I've generally not blamed President Bush
  for was his response to Hurricane Katrina. And there were certainly
  mistakes he made there that were quite big but fundamentally, when you
  have a natural disaster, the president is not a god. He can't change
  those outcomes. I mean, sometimes, these discussions, it's almost like
  being a medieval priest and you're being confronted by a sort of angry
  group of people who say, why didn't the king cure this man's scrofula,
  and the fact is, well, actually, kings can't cure scrofula. That's
  something beyond their power. So the president is necessarily going to
  take a hit to his approval ratings and his re-election chances as a
  result of the COVID-19 crisis. That doesn't mean, however, that there
  was really much that he could have done one way or the other that
  would have greatly altered what he could --- the response that he
  would get from whatever actions he might have taken. Now, in terms of
  what case can be made for his re-election, it's simply that
  fundamentally, what he had warned us about back in 2016 regarding the
  possible consequences of globalism have all proven true as a result of
  the COVID-19 crisis. We've had a economic collapse that has perhaps
  been worse than it had to be as a result of our interconnectedness
  with China and with other parts of the world, and of course, the fact
  that the coronavirus came in from outside of the country is another
  sort of point against globalism and point in favor of Donald Trump's
  fundamental populism and nationalism. So I would say that Donald Trump
  should reiterate his original 2016 campaign points, his themes, and
  that what we've seen happen this year, as disastrous as it has been,
  is, in fact, a vindication of his early warnings back in 2016.
\item
  ross douthat\\
  OK, sure, it's a vindication of his early warnings, but in response to
  those early warnings, who did we elect president? Seems like we
  elected Donald Trump president of the United States, right? So I mean,
  it seems to me that in this crisis, he was, indeed, handed a challenge
  whose parameters did sort of vindicate portions of his 2016 world
  view, but then once confronted with this challenge, from a reasonable
  perspective, it seems like he's failed. That seems to be the problem.
  You're simultaneously saying, this is an act of God that no president
  could hope to deal with, and Trump sort of saw it coming, but if he
  sort of saw it coming, then it's not just an act of God. It's a
  predictable challenge that maybe we should have been better prepared
  for and had a response to.
\item
  daniel mccarthy\\
  No, but again Ross we have a government of 50 states, not just one
  central federal government. And the central federal government that we
  do have is divided between the president and Congress. Now, I would
  criticize the president for having sort of lost the initiative early
  on when he did have control of Congress, but he's had the opposition
  party in charge of the US house of representatives since two years
  ago. So while you can say that President Trump could have had some
  sort of dictatorial powers and done everything that he might have
  wished to do, that was never really the case. It certainly has not
  been the case with respect to the states and their policies, but it
  hasn't even been the case with respect to the federal government. He
  had a very short window where he had plenipotentiary power, and that
  went by the wayside two years ago.
\item
  helen andrews\\
  And just to be clear, the mere fact that the coronavirus crisis
  vindicated many of Trump's campaign policies doesn't necessarily mean
  that he could have done anything about it once it hit the United
  States. I mean, it's not as if he can, in the space of a month, make
  American supply chains less dependent on China. Even if, in
  retrospect, we had done that, we might have been in better shape, the
  American people will forgive our president for not controlling things
  that he's unable to control. They don't expect presidents to be gods.
\item
  ross douthat\\
  They don't? I would dispute that premise too, honestly. I think ---
\item
  daniel mccarthy\\
  Yeah, I might agree with you, Ross. Perhaps they do expect presidents
  to be gods.
\item
  ross douthat\\
  I mean, I think it's --- at the very least, though --- at the very
  least, though, I mean, you compared this to Hurricane Katrina, but
  COVID-19 is something where it's a rolling crisis, where every day in
  the United States, a certain number of people become infected, and
  every day, a certain number of people die, and the choices that are
  made, the public health choices that are made over the course of those
  months presumably have some impact on it. And I think that both of
  you, but I think especially you, Helen, have argued that whether this
  was Trump's choice or not, that a lot of the choices we made were the
  wrong choices, that sort of locking down as extremely as we did in the
  months of March and April was actually the wrong approach. Is that
  what you think?
\item
  helen andrews\\
  That is, but even though I think the lockdowns were probably more
  extreme than they had to be, I am perfectly willing to make allowances
  for decision-makers who have to operate with imperfect information. I
  think that you can't expect people to make perfect decisions when they
  don't yet have all the facts that they need. But we're now a few
  months into the crisis, and we know better what needs to happen next.
  And so now is the time when I think the American people are going to
  start grading their politicians a little more toughly on the decisions
  that they make. And the one thing that they won't forgive is people
  who make politicized decisions in response to a health crisis. Which
  is why I think the biggest opportunity for the president going forward
  in terms of how he should handle coronavirus to improve his
  re-election chances, is to focus on schools, because that's definitely
  a case where you have teachers unions and a lot of the pro-lockdown
  side making decisions that look to a lot of parents like they're
  politically motivated, and also, like they'd be perfectly happy to
  keep schools shut, sometimes, it seems, indefinitely, and prevent
  private and charter schools from opening up either. So taking on the
  teachers unions who are making politicized decisions might be a good
  strategy for the president going forward.
\item
  ross douthat\\
  Do you think those decisions are politicized in the sense that they're
  being made in effect to spite the president, or do you think they're
  politicized --- I mean, it seems to me much more that they're made in
  a spirit of panic and anxiety about the disease that may be
  overstated, or maybe sort of missing the importance of having schools
  open and the importance of balancing that against the public health
  risk, which is not exactly the same as being politicized. It's more
  disease anxiety driven.
\item
  helen andrews\\
  Well, when I look at the list of demands from Randi Weingarten and the
  A.F.T., it includes a lot of things that don't have anything to do
  with education, like eviction moratoriums, a wish list that you would
  expect from activist Democrats. So that's pretty clearly politicized.
  And I'm in favor of a diversity of approaches. So if a local school
  board wants to keep schools closed or remote until November, that's
  their decision to make. But when I see those same school boards and
  county health organizations trying to shut private and charter schools
  as well and not allowing those schools to make their own decisions and
  maybe find their own path forward, that to me does look like teachers
  unions trying to kneecap their opposition, because they're afraid that
  competition from private schools and religious schools will draw away
  pupils. So, yeah, I think got a lot of parents do detect some
  politicization in decisions like that, and I think they're right.
\item
  ross douthat\\
  Right. There was the case just now in Maryland where they announced
  that--- this has since been overturned by the governor, I guess, in
  defiance of your call for local decision-making in some sense, Helen,
  but they first announced that private and parochial schools would have
  to stay closed till October 1. And one of the theories for why they
  chose that date was that that's the date at which enrollments are set
  in public schools, and they were hoping to effectively maintain or
  increase public school enrollments for budgetary reasons. So I think
  there's obviously some sort of effective political competition within
  the education system going on in this totally weird environment. But
  let me pull you back to the macro level. So, fine, you think Trump
  should be making a kind of campaign on school openings. At the same
  time, you are saying that fundamentally, these are local and state
  decisions, which, of course, they are. So in a sense, you're making an
  argument for Trump using the bully pulpit, using sort of rhetorical
  powers to make a particular case. But why doesn't that apply to the
  entire public health situation? Certainly, Trump can't institute a
  national mask requirement, but it would certainly have been possible
  for him to, start wearing a mask in public a month and a half earlier
  than he did and not discourage mask wearing tacitly in all the various
  ways that he's done. And he has, throughout this disease, had the
  power of sort of public speechifying, the power of making statements.
  He attempted to sort of command the situation with his nightly press
  conferences for a while. In that sense, if he can use the bully pulpit
  for his re-election, then I don't see how you can sort of absolve him
  for the way he has or hasn't used the bully pulpit in the last few
  months. Do you think there's been any kind of consistent public health
  message emanating, not from the experts around Trump, but from the
  president himself?
\item
  helen andrews\\
  The very simple reason why Trump should not use the bully pulpit more
  is that when he wades in to political debates, he very often makes
  them worse. Not so much because anything he does or says is
  particularly reckless, but because of the way other people react to
  him. An unfortunate fact is that there are a lot of people involved in
  our political discourse who will immediately take the opposite
  position to whatever the president adopts. And we saw that just in the
  past few weeks. A month ago, it seemed like everyone was kind of
  pretty much on board with the idea that children were low-risk
  spreaders, and therefore opening up schools should be one of the first
  things on our list so we can get people back to work. And yet, when
  the president came down on the side of reopening the schools, a lot of
  people rushed to the other side. So yeah, I think when the president
  weighs in on things, a lot of people immediately leap to say the
  opposite. And that may say bad things about them, but that's just a
  fact of our political discourse we have to deal with. So I think very
  often, the president does better by the country by staying out of
  certain debates.
\item
  daniel mccarthy\\
  Well, and I'll sort of jump in here and underscore something that
  Helen is getting at. This idea that we have to house our national
  media pointing to Donald Trump and saying, you know what? COVID is
  really about Donald Trump. It's about what Donald Trump does, what he
  says, the signals he sends, the policies he makes. It's simply a
  misreading of our actual structure of government. It's an actual
  misreading of the way in which the private sector works here, and we
  really do need people to be making prudential decisions at the very
  --- not just the local level in terms of government, but the molecular
  level in terms of our society. People need to be making smart
  decisions about how to discriminate between high-risk and low-risk
  groups, how to keep a relatively sound degree of separation between
  these groups if there's a chance that someone in a low-risk group may
  be infected and could perhaps harm someone in a high-risk group. All
  of these things have to be done at the societal level. They can't
  simply be mandated from Washington D.C. Whether it's Donald Trump, or
  if Joe Biden were president, it would not make any difference. It is
  something that fundamentally requires personal responsibility. I think
  Americans have been trained and have been told that they should not be
  exercising this kind of personal responsibility. They should not be
  thinking for themselves. They should be looking to some sort of expert
  or god figure and deferring to that person.
\item
  ross douthat\\
  But again, isn't this --- I mean, I watched Donald Trump's 2016
  campaign where he literally used that kind of rhetoric, I think very
  effectively in many cases, in ways that a, shall we say, more
  libertarian or states' rights oriented style of Republican would not,
  right? Like, Trump basically said, look there is a national role, a
  national federal role in dealing with things like the opioid epidemic.
  You need a national strategy on the offshoring and factory closures.
  He would go around the country and give speeches in towns that had
  factories closed, and people would say a version of what you're saying
  now, Dan. Like, oh, well, this is an organic process, and it's a
  molecular process, and the decisions that are made by individual
  actors can't be micromanaged in D.C., and Trump basically said, to
  hell with that. The national government is powerful, and we actually
  need a policy to deal with people taking jobs overseas, right?
\item
  daniel mccarthy\\
  Well, Ross, no ---
\item
  ross douthat\\
  It just seems weird to retreat to that rhetoric in the face of a more
  immediate crisis.
\item
  daniel mccarthy\\
  No but you're fixated on a single point here, which is always, if
  Trump has a power in one field, he must have a power in every field.
  He must be a kind of omnipotent figure, and that's not correct. The
  federal government does ---
\item
  ross douthat\\
  I have certainly never claimed that Donald Trump is an omnipotent
  figure. I just want to put that on the record.
\item
  daniel mccarthy\\
  It is the --- no, it is the presumption of this conversation right
  now. But in fact, the federal government has a role in trade policy
  historically that exceeds what its role in health care policy has
  been. And that's just a fact of the way the U.S. Constitution works
  and the way our governments work. The United States is facing foreign
  competitors, facing entities like the People's Republic of China,
  which cannot be confronted by a locality or by a state, but which can,
  in fact, be confronted by the federal government. COVID-19 is a threat
  of a different kind. It's coming not from the outside at this point.
  It's now coming from within the country. And it's coming not from a
  sort of conscious entity from the outside. It's rather a phenomenon of
  individuals communicating the disease to one another right here. So
  the fact that there is a division between what Donald Trump can do
  with national power and what he can't do does not mean that there's an
  inconsistency here. It just means that there really is a pattern to
  what he is able to apply his abilities to. And COVID-19, some of these
  natural disasters, I think, are not in that category.
\item
  ross douthat\\
  If you go back to the period of, let's say, one to two months between
  when the coronavirus first appeared in China and the fact that it was
  dangerous became clear, and the period when it started to really hit
  New York City, you had this zone where the president did do something
  that used his actual federal powers. He instituted some kind of travel
  ban on people coming from China, and we can debate how effective that
  was, but that was real. He did that. And then there was a six-week
  period where if you looked at his rhetoric, he consistently downplayed
  the likely severity of the problem, made what turned out to be totally
  false predictions about having the virus under control, and that was
  the period when --- and this this part was not Trump's fault per se,
  although it is his executive branch --- but where the C.D.C. and the
  F.D.A., but especially the C.D.C., sort of fell down on the job in
  terms of testing and containing the virus. And at the end of that
  period, you had the huge outbreak in New York City, which also
  reflected disastrous policy choices on the part of local officials and
  state officials there. But the state and local officials right now, I
  mean, especially Andrew Cuomo, have pretty high approval ratings, in
  part just because they rhetorically seemed to take the virus
  seriously. So it seems to me that there both was, first, a clear
  window for some kind of federal action where the president missed the
  boat, and second, that there is a political upside to even if you're
  doing a bad job, or even if it's an act of God and beyond your
  control, just seeming sort of rhetorically in control in a way this
  president struggles to do. Do you think that's wrong?
\item
  daniel mccarthy\\
  I do think it's wrong. First of all, regarding his window of
  opportunity to act, he took action, and the actions he was able to
  take were marginal. So the president is very proud of what he did in
  terms of trying to restrict travel into the United States as the
  coronavirus was breaking out. We can say, OK, well, even if that's
  succeeded, even if that was exactly what Donald Trump thought needed
  to be done and he did it, obviously, it was not sufficient to stop the
  disease from reaching the levels that it has reached. Now, if he had
  not done that, if he had just let more people come in with the virus,
  what effect would that have had? Well, would it make things even
  worse. So the fact that Donald Trump has limited powers that can have
  a marginal effect is, I think, just a reality that has to be accepted
  here rather than saying that the overall situation is something over
  which he has control. Regarding the relative popularity ratings of
  President Trump and Governor Cuomo, I mean, sometimes, a democracy
  gives you an injustice, and in the case here, I think Andrew Cuomo has
  a lot more lives on his hands from having sent people with the
  coronavirus into nursing homes and other places that couldn't turn
  patients away, or people away who had the disease and then spread it
  to the most vulnerable elderly populations. That is catastrophic, and
  it's something he should face serious electoral repercussions for.
  It's a question of whether the American people are going to look
  seriously enough at where the responsibility actually lies in our
  federal system relative the president versus the governor.
\item
  ross douthat\\
  But surely, this goes back to the question of rhetoric and
  presentation as a part of statesmanship?
\item
  daniel mccarthy\\
  No, it doesn't. Because not everything is a matter --- no, absolutely
  not. This idea that you can magically cure things with words has to be
  resisted.
\item
  ross douthat\\
  No, no, no, but that's not even --- but that's not --- that's not per
  se the point. I started this conversation by asking both of you what
  the president could have done differently, and Dan, your answer was
  nothing, and Helen, your answer was he's done well to the extent that
  he's minimized his public presentation. But what I'm saying is that
  what you're describing as the injustice, Dan, of Cuomo being popular
  in his disastrous response reflects Cuomo's use of the art of public
  rhetoric and statesmanship, which is part of the president's job
  description, indeed, a central part.
\item
  daniel mccarthy\\
  Well, look, I mean, you're asking --- yes, you're asking a basic
  Machiavellian question here. Is it important to seem virtuous even if
  you're not?
\item
  ross douthat\\
  Yes, sure.
\item
  daniel mccarthy\\
  And in Cuomo's case, that's what he's done. He's been able to create
  the appearance, perhaps, of a certain kind of competence. But I'll
  mention two things here. The first is that in terms of the way that
  the media has reported the COVID-19 crises, and even the way this
  conversation is going right now, you can certainly see that there is a
  tendency to, in all of the elite media, to give Donald Trump the worst
  possible reading and to be rather more generous towards Democrats.
\item
  helen andrews\\
  I have to agree with Dan, here. Ross, you seem to be proposing that if
  only Trump were a better political communicator, he could be more like
  that great hero Andrew Cuomo, to which I can only respond that being
  more like Andrew Cuomo is not a thing that I would like my politicians
  to be interested in. And I also think the media plays a bigger role
  here than you are giving it credit for. Trump could have literally
  lifted the script from every Andrew Cuomo press conference and it
  would have been covered very, very differently.
\item
  ross douthat\\
  First of all, I'm not saying that wouldn't it have been great if the
  national response had been more like New York's response in those
  first few crucial weeks. I am saying that if your political goal is to
  govern the country successfully, which requires, among other things,
  being re-elected, then yes, presumably, the Republican Party and its
  position would be better if Donald Trump had some of the communication
  skills of Andrew Cuomo, which, yeah, it is a Machiavellian point, I
  suppose, but it's kind of an essential one. And it's just very ---
  it's very hard for me to imagine a world in which, maybe not the two
  of you, but where most Republicans or conservatives were willing to
  look at 200,000 dead Americans in the final year of a Democratic
  president's first term in office and say, well, it's time for the
  American public to grow up and realize that President Obama doesn't
  have magical powers and there was nothing that he could have done. I
  mean, if you had 200,000 dead Americans under Democratic precedent,
  Republicans would be screaming bloody murder about it, and I think
  completely understandably so.
\item
  daniel mccarthy\\
  Well, Ross, look, you could say that, yeah, if the president had the
  ability to sort of bamboozle the public and to give them a sense that
  they are sort of healthier than they actually are through very
  persuasive language that that would make us all feel better---
\item
  ross douthat\\
  Or give them the sense that --- give them the sense that he knows
  what's going on and he has a plan to deal with it, right? I mean,
  that's ---
\item
  daniel mccarthy\\
  No, that's not what he should do, because he doesn't know what's going
  on. {[}DOUTHAT LAUGHS{]} Nobody knows what's going on. No, I'm quite
  serious about this. I think we're in a very dangerous position, and in
  fact, not just us, but the whole world, because we're assuming that we
  know things about this disease that we really don't, and we should be
  very careful here and we should be very cautious with our local
  circumstances as opposed to assuming that we've got some sort of
  master plan that's going to deal with this. I don't think any of us
  actually knew six months ago that this was going to be quite as severe
  a problem right now as it still is. And we need to also keep in mind
  the costs of policies like shutdowns, for example, on psychology, on
  suicide rates, on other deaths of despair. There are any number of
  contingencies and complexities here which I think we need to be honest
  about. We can't simply say, you know what, we're just going to clap
  for Tinkerbell and that's going to resurrect ---
\item
  ross douthat\\
  OK, good. No, but let's stick with that point, right? So forget Donald
  Trump for a minute. Just articulated in your own --- from your own
  point of view, what have we learned? We've learned something about the
  disease over the last four to six months. We have a lot of --- we have
  a wide range of outcomes in different countries. We have observed and
  absorbed many of the costs of lockdowns right now. So accepting that
  this varies from state to state, what should be our policy going into
  the fall? Let's say we have a residual lockdown right now with some
  limits on gatherings, and possibly, we'll end up with school closures
  and so on. Is that a mistake? I mean, were the swedes right to assume
  that we could reach herd immunity at a pace that would save our
  economy. What have we learned? What have you guys learned?
\item
  daniel mccarthy\\
  Well, look, Sweden still has per million more COVID-19 fatalities than
  we have, and maybe that's going to change over the course of the next
  several months, but I'm not quite prepared to say that the Swedish
  model is the correct model. On the other hand, I am less confident
  than you are, Ross, that we already have as clear lessons as we would
  like to have about this disease and how best to approach it, and
  that's one reason why I'm emphasizing that this sort of most localist
  approach possible is what I think is the only thing you can do.
\item
  helen andrews\\
  I also think that's the biggest thing we've learned, is how big that
  difference in risk is for different demographic groups. There are
  still today a lot of states where most of their coronavirus deaths
  took place in nursing homes, a majority in facilities that house less
  than 1 percent of the U.S. population. That's a huge thing that we've
  learned in the past few months, and maybe that should tell us that
  focusing on those high risk groups would have been better to do from
  the beginning, and certainly, better to do from here on out. Because
  unlike you, Ross, I'm not entirely confident that we have weathered
  the worst of the economic storm. I think that a lot of the jobs that
  are currently being thought about as layoffs or temporary furloughs
  are going to end up being permanent. A lot of small businesses are
  going to close no matter what happens with unemployment insurance or
  any other mitigating measures from here on out. So I think the
  economic cost of lockdown could be a lot higher than we're thinking
  right now. So between that and the concentration on nursing homes, I
  think we should start looking at opening up.
\item
  ross douthat\\
  But we have started opening up, right? And the challenge is precisely
  that. I'll flip the script and argue a version of you guys's case,
  which is that the power of the state as manifested in lockdowns is
  maybe less significant than the choices that people make themselves to
  socially distance, not go to restaurants, not go shopping, not go to
  the mall and so on. So you have, if you look at trends in foot traffic
  and OpenTable reservations and all of these things, you see sort of
  steep declines before states implement lockdowns, and what you see now
  in states that have opened up, or partially opened up, is that when
  there's a surge of cases of the virus, people react to some extent the
  way that they did before, and you get a kind of cycle. You get a
  reopening, people start going to bars and restaurants again, then the
  virus surges, then people stop. But that means that the economic pain
  exists to some extent independent of public policy. It exists so long
  as the virus is an active force in American society. So I mean, to me,
  the stronger anti-lockdown case is psychological rather than economic.
  I don't see how you avoid the massive economic hit from this virus
  with or without formal lockdowns. But the lockdown policy, the sort of
  extremity of it does seem to generate an intense psychological toll
  that is manifest in the most extreme form in rising suicide rates and
  so on, but I think is also manifest in the protests of the summer. I
  think there was a reasonable argument that you don't get massive
  protests across the U.S., even though they are officially about racial
  justice and police brutality, that you just don't get those protests
  without the two-month lockdown policy beforehand. Helen, do you think
  the lockdowns caused the protests?
\item
  helen andrews\\
  Absolutely. A fact about the Russian revolution that a lot of people
  don't know is that the winter of 1916 and 17 was one of the harshest
  on record, and that everybody in St. Petersburg had basically been in
  lockdown because of the weather and the harsh winter for months and
  months and months. Until the end of February, when suddenly,
  temperatures shot up, and everybody went outside, and a week later,
  the Romanov Dynasty was no more. {[}LAUGHING{]} So yeah, when people
  are locked up for a long time, and then suddenly, you let them out,
  they run into the streets and they go crazy.
\item
  daniel mccarthy\\
  Well, and you only let them out for the protests. I mean, that's one
  of the other key things here. People can't really go to work as
  normal. They can't really go to school or whatever educational
  opportunities they would have as normal. They can't socialize as
  normal. Practically the only thing they're allowed to do is to
  congregate outside in large numbers and burn down a police department.
  If that's your only option, and that's the only kind of social
  activity you can take part in, a lot of people are going to
  opportunistically glom onto it.
\item
  ross douthat\\
  We're going to take a quick break, and when we come back, we're going
  to talk about the fall election and whether conservatives should wish
  for a Donald Trump second term. We'll be right back.

  {[}MUSIC PLAYING{]}

  And we're back. So I want to start with the larger cultural upheaval
  that's affecting maybe elite institutions in particular, but also,
  corporate America, sort of the general tug to the left that's going on
  in American institutions right now as a reaction in part against the
  presidency of Donald Trump. That just as you see tugs to the right
  under Democratic presidents, you get tugs to the left under Republican
  presidents. And I'm going to submit and then have my guests disagree
  that the strength of this tug and the weakness and incapacity of Trump
  himself in response to it, his inability to sort of harness or marshal
  public opinion against the leftward swing in American life is a reason
  for conservatives to wish that he loses the election in November,
  because a Trump second term would probably be an extension of the end
  of his first term where the right clings to political power in
  Washington D.C. and continues to lose cultural ground just about
  everywhere you look. Helen, I suspect that you disagree and that you
  will be supporting Donald Trump in November. Tell me why I'm wrong.
\item
  helen andrews\\
  Yeah, I disagree vehemently. I disagree very strongly. I think any
  conservative out there who's wishing for a Donald Trump loss is just
  flat out crazy. Winning is always better than losing. And if you're
  somebody who supports the things that Donald Trump campaigned on, you
  especially should be helping that he wins so that the G.O.P. doesn't
  just revert to business as usual. I kind of knew when Trump was
  elected that there would be a lot of growing pains, more than for most
  presidents because he was such an outsider, but I do think there is an
  observable learning curve. I don't know how much he's gotten better,
  but the people on his staff have gotten better and four years wiser,
  and I think a second term would be a lot smoother. I also think that I
  have been just more disappointed that I thought I would be by the
  behavior of the left in response to Donald Trump. The inability of so
  many people in the Democratic party, and not a few in the Republican
  establishment, to accept the outcome of the 2016 election and just
  pitching tantrums and throwing bogus impeachments. And I really think
  it's important that that kind of behavior not be rewarded. So those
  are two good reasons to be hoping that Donald Trump wins.
\item
  ross douthat\\
  Dan, can you give me more?
\item
  daniel mccarthy\\
  Yeah. Well, certainly, if you look at policy, if you voted for Donald
  Trump in 2016, and you wanted to see a degree of change in our foreign
  policy, if you wanted to see conservative justices put on the Supreme
  Court, or at least, as close as the Republican Party seems able to get
  to putting conservative justices on the Supreme Court, then it seems
  to me that you have to support Donald Trump and hope that he will see
  through the project that he began in 2016. In foreign policy, it looks
  to me as if Trump now has a stronger handle on things, that we're not
  going to get appointments of neoconservatives like John Bolton in a
  second Trump administration. I see some promising personnel moves in
  terms of the next ambassador, for example, that Trump wants to appoint
  to Germany. And in various other places as well, I think you're
  actually starting to see the kind of braintrust forming to see through
  a more restrained foreign policy in a second Trump administration. And
  of course, foreign policy is the thing that the president really has
  the most direct control over. As far as the courts are concerned, this
  is an existential matter for conservatives. In terms of religious
  liberty, it's certainly an existential question in terms of the lives
  of the unborn. Conservatives have been bitterly disappointed by
  Justice John Roberts, and to a somewhat lesser extent, also by Neil
  Gorsuch. However, the idea of having Joe Biden in there, probably with
  a Democratic Senate, and having more justices like Kagan or like Ruth
  Bader Ginsburg, that would obviously be even worse for conservatives.
  So the conservative approach to the courts has been a mixed success,
  and in the eyes of many, it's been a failure, but it hasn't been as
  great a failure as you would get with a court who had appointees
  coming from the left wing of the Democratic party.
\item
  ross douthat\\
  Talk a little bit more about the idea of a braintrust, Dan, because I
  think we disagree on this.
\item
  daniel mccarthy\\
  Well Donald Trump came from out --- yeah.
\item
  ross douthat\\
  I mean, imagine a Trump second term. Right now, the Trump cabinet has
  an awful lot of acting non-confirmed cabinet rank officers. The Trump
  White House has partially emptied. Now, I'm sure you're glad of some
  of the emptying, because it reflects, as you say, people who Trump
  hired who are either very conventional establishment Republicans, or
  in Bolton's case, much more hawkish than the president himself. I
  think that's reasonable, but it doesn't seem to me that there is a
  populist braintrust around the president. And by populism, I mean
  people who would agree with your perspective on foreign policy, which
  is to say, containing China, and otherwise very restrained in military
  adventurism, and in domestic policy, being willing to break with sort
  of Reaganite views on things like infrastructure spending, let's say.
  I think such a populous braintrust exists outside the White House. If
  you put me in charge of assembling such a braintrust for Trump in his
  second term, I think I could do a decent job, and maybe would hire the
  two of you, but I don't see any evidence that Trump himself sort of
  sees that as his mission or that there's a group coalescing around him
  that has that kind of clear agenda. Trump's most sort of successful in
  terms of duration cabinet official is Mike Pompeo, who as far as I can
  tell, is still obsessed with conflict with Iran. So who is the
  braintrust?
\item
  daniel mccarthy\\
  Well, so I think that you're beginning to see the right kind of
  appointments being contemplated, and in some cases, being made. And
  obviously, on the eve of an election, you're not having as much action
  in terms of appointments as you would get after an election, after you
  have a second term coming on. But no, I mean someone like Douglas
  MacGregor, for example, being mooted as appointee for ambassador to
  Germany. That's very promising. And there are a lot of other people I
  hear in the pipeline of similar caliber and similar views. Donald
  Trump took a couple of years here, and I regret that it took as long
  as it did to find out that people like John Bolton really are not on
  his side. And Mike Pompeo, he has certainly been persistent. He's
  stuck around for a long time, and he may stick around for a longer
  time yet. And he is someone who comes from a more conventional
  Republican background, but I don't think he's the whole story, and I
  think there's actually a lot of things --- interesting things
  happening on the personnel side of the administration, sometimes in
  the less high-profile roles, that are indicative of a new approach to
  staffing.
\item
  ross douthat\\
  Who do you --- who do you trust? Who do you trust in the White House?
  Give me somebody who you trust, who you ---
\item
  daniel mccarthy\\
  First of all, in terms of things like trade policy, I think you've had
  a braintrust from the beginning. Robert Lighthizer, for example, I
  think, is an indication that the Trump administration has had its act
  together with respect to its trade policy almost from the beginning I
  get the impression that with appointments like Douglas MacGregor, they
  now have the same sort of focus with respect to the foreign policy as
  well. Mike Pompeo is a more conventional Republican, yes, but I think
  that a lot of people around him are going to be more in the Trump
  estate. There have been a lot of battles lately where people are
  saying that Trump is appointing people who are too conservative. The
  former head of the Claremont Institute, for example. But this is
  actually what Trump was elected to do, to make sure that the message
  that we're sending out to the world is a message that reflects the
  changes here in this country with respect to our foreign policy, that
  we no longer are in the business of militarily promoting democracy and
  trying to engage in regime change operations here, there, and
  everywhere, but rather, we're talking about America's fundamental
  values, and we hope that that example is what will change countries,
  as opposed to skullduggery to kind of forcibly alter other regimes.
\item
  ross douthat\\
  Helen, let me go back to your initial point, that winning is always
  better than losing, which is a powerful point. So in 2016, Donald
  Trump ran, as among many other things, the candidate of standing for
  the national anthem, right? I mean, I think that that's --- that was
  sort of a condensed symbol of Trump's campaign in 2016, where he used
  Colin Kaepernick as a foil to attack athletes who didn't stand for the
  national anthem. Now, here we are four years later, and we're in a
  cultural landscape where kneeling for the national anthem is the norm
  and standing for the national anthem is now seen as sort of the act of
  protest. Now, this is a small, condensed symbol that obviously doesn't
  have immediate policy implications, but it seems like a pretty
  striking cultural defeat that conservatives have absorbed, and I think
  would not have absorbed had Donald Trump not been president, in part,
  because of the fact that Trump's unpopularity, the fact that most
  Americans they really don't like him has created all of this space for
  left wing argument and left wing movements to effectively gain large
  amounts of cultural ground. Now, I know you don't think that's totally
  wrong. Tell me why that doesn't just happen even more so across the
  next four years, where you're effectively trading maybe one more
  Supreme Court seat for even more sweeping cultural defeat.
\item
  helen andrews\\
  I'm actually not sure I agree with any part of the premise of your
  question. First of all, your logic sounds like an argument for
  electing McGovern in 1972. And the `60s would have ended sooner if
  we'd gone further to the left, which just doesn't sound right as a
  matter of history. Also, as a matter of history, I disagree that none
  of this would be happening if Hillary Clinton had won in 2016. I think
  it's entirely possible that the fringe left would have felt empowered
  by that victory and we would be seeing many of the same things that
  we're seeing now. But that's actually not the important question. The
  relevant question is, will these forces be stronger or weaker if Biden
  wins in 2020? I think that the Trump campaign's message of you won't
  be safe in Joe Biden's America is exactly the right one. I don't think
  anybody expects that Joe Biden himself is secretly a closet Sandinista
  radical, but I do think a lot of people are worried that he's so old
  and out of it that he would be a figurehead. So whatever aggravating
  effect Donald Trump might have on the psyches of leftists who can't
  stand that he won in 2016, I think that effect just pales next to the
  empowerment these forces would feel under a Biden presidency.
\item
  ross douthat\\
  But it's not just about the empowerment of those particular forces,
  right? It's also about public opinion writ large and the pressures
  that are put on, for instance, mayors and governors in terms of
  dealing with riots and urban unrest. where right now, there's this
  sense that every event is about Trump, and every choice that, lets
  say, a blue state governor or a blue state mayor makes is seen in
  light of whether it helps or hurts Trump, right? So you have this ---
  you have pressure right now on blue state mayors to refuse federal aid
  or something in terms of dealing with soaring murder rates, because
  that's seen as boosting Trump in some sense, right? In a Biden
  presidency, that kind of thing goes away. In a Biden presidency, you
  get a thermostatic swing in public opinion, the way you do in almost
  every presidency, where moderate voters who are right now very worried
  about Donald Trump's capacities to handle the coronavirus and haven't
  yet internalized the lessons that you guys were preaching in the first
  segment of our show would swing to being more worried about the
  overreach potentially of liberalism and would be more likely to vote
  for Republican candidates in 2022. And I just want to push on your
  historical analogy, Helen, which is that, yes, if Donald Trump were
  Richard Nixon, which is to say, for all his faults, a very effective
  politician who was capable of building, in the end, a 60 percent
  majority coalition against McGovern, then of course, it would be
  ludicrous for Republicans or conservatives to want the liberal
  democrat to win. But if Trump is a figure more like George Wallace,
  sort of a representative of a sort of permanent minority faction who
  can only win through electoral college luck, then having him in power
  doesn't--- I mean, it's a different --- it's a different case, right?
  It's like if you were a conservative in 1968, would you rather have
  Hubert Humphrey or George Wallace? Maybe you'd rather have Hubert
  Humphrey. This is the question with Trump. At some point, if you're
  going to actually govern the country, you actually need to win a
  majority of the country. You can't just rely on electoral college
  minorities for generations yet to come and having your standard bearer
  be this deeply unpopular figure who constantly hands cultural
  victories to the left. Seems like it pushes that moment ever further
  out of reach.
\item
  helen andrews\\
  I have high hopes that the standard bearer for populist conservatism
  that comes after Trump will be a lot more normal and more competent
  than he is. I have every hope in the next generation. But I really am
  struggling to grasp where you're coming from on this line of
  questioning, Ross, just because are you really saying that if Biden
  wins, the local government of Portland is finally, for the first time
  in its existence, going to start cracking down on Antifa terrorism and
  not let these 20 somethings in black roam the streets with impunity?
  That just doesn't seem realistic. That just doesn't sound like them.
\item
  ross douthat\\
  I mean, I think it's less the government of Portland per se. Portland,
  I think, is a distinctive case in the sense that they have had a kind
  of anarchist protest culture that the government has basically
  tolerated for a long time before the George Floyd protests. But no, I
  mean, I'm thinking more of cities like Atlanta, Chicago, Boston, New
  York and so on. Cities that have not been run the way Portland has
  been run for the last few years, to put it mildly, cities that right
  now have murder rates spiking that they need to bring under control.
  Those are the cities where Trump creates particular pressures not to
  be seen as cooperating with what liberalism has decided is a white
  nationalist president. Those are the kind of specific pressures that
  I'm thinking of --- Portland is a special case. I think that applies
  to other institutions too. I mean, obviously, I have a personal stake
  in this since I work for a institution that is widely reputed to be
  liberal and has had its own Trump era internal controversies. But just
  about every liberal institution right now, in higher education, in
  media, and so on, there is this pressure that Trump himself creates
  where everyone has to be onside against the great threat of Trump. And
  so all of those institutions in the Biden presidency, I think, would
  feel potentially some of that pressure relaxed. Not necessarily.
\item
  daniel mccarthy\\
  No, not at all. I mean, come on, you're saying that everything was
  going smoothly for conservatives in higher education, for example,
  until Donald Trump came along, and then the inflamed left suddenly
  decided to start acting out. That's not the way it's happened. In
  fact, conservatives have been losing ground in these institutions, the
  media and higher education in particular, for generations, and the
  fact that now it's being advertised in a way that it perhaps wasn't
  quite as much during the Obama years is not a change in the substance.
  The thing that really matters here is the sort of control of the
  institutions themselves and what kind of ethos they embody, whether or
  not that's being expressed as sort of flamboyantly as it is now, or
  whether it's deeper in the institutional power structure, as it was
  during Obama or during the Clinton years or whatever. But
  conservatives actually are in a better position knowing who their
  enemies are and knowing just how biased the media is against them,
  knowing just how ruthless higher education is in preventing
  conservatives from getting tenured positions and so forth. I think
  there's a great deal to be said for the sort of confrontation with
  reality that Donald Trump is bringing about. And it does seem to me
  that as you see crime rates spike by 24 percent in the 50 largest
  cities or across the country right now, you're going to see a
  significant change in public opinion, and there's a limit to how much
  demagoguing against Donald Trump is actually going to be effective
  before people say, wait a minute, the street crime that I'm seeing in
  New York or in Chicago is not being caused by Donald Trump. It's being
  caused by mayors and by city councils that are far too lenient on
  criminals.
\item
  ross douthat\\
  So let's talk about that scenario in a world where Trump loses before
  we wrap up. Helen, you you mentioned the idea that there exists after
  Trump potentially a more competent and serious form of populist
  conservatism waiting in the wings. Talk a little bit about that. Who
  leads the post-Trump Republican Party in 2024 or beyond?
\item
  helen andrews\\
  Yeah, I think that there are a lot of candidates, and there will be
  more candidates for that spot if Trump wins, which is another reason
  why I hope that that happens, because it will show that that's the
  path forward for victory for the GOP. It used to be my consolation
  when I thought about whether there was a chance that the G.O.P. would
  just revert back to the Romney-Ryan default that I thought, you know,
  they can't just do that, because if the G.O.P. goes back to the
  Romney-Ryan default, it will lose. If you go back to those policies,
  you're not going to win in Pennsylvania or Wisconsin. But as I've
  watched the G.O.P. over the last few years of the Trump presidency, I
  have realized that there are a lot of people who would sort of be fine
  with that, who would be OK if the G.O.P. went back to losing under
  that set of failed economic policies. You look at people in the never
  Trump movement, it seems like some of them would be very happy for the
  GOP to go back to losing. They really are the Washington generals of
  policy. So that's why I'm really hoping that Donald Trump wins in
  2020, to show that these ideas are the way forward for the party.
  Because if he shows that, then I think it matters less who actually
  picks up the standard, if it's somebody as bright and articulate as
  Senator Josh Hawley, or whether it's somebody else.
\item
  ross douthat\\
  Dan, your prophecies?
\item
  daniel mccarthy\\
  Yeah, I'll underscore what Helen has just said, in that there is a
  certain large component, actually, of the conservative movement which
  stands to gain if conservative leaders lose. It's notorious that
  political magazines, for example, get more subscriptions when the
  opposing party is in office. If you're writing fundraising copy for a
  think tank, it's much easier to write an attack on someone than it is
  to write a defense of some policy. So in a way the conservative
  movement as an institution is invested in failure, and we see the
  results of that. I think that's one reason why conservatism was in
  such a decrepit shape that someone with no political experience
  whatsoever like Donald Trump could come in and actually knock over all
  of these highly credentialed and highly articulate conservative
  leaders that he ran against in 2016. Now, this institutional
  corruption is something that hasn't gone away. As Helen mentions, you
  see it never Trump. You see it also in some of the opportunistic moves
  that people have made towards Trump in some cases. And whatever
  happens in 2020, this institutional problem is going to remain on the
  right. So in addition to seeing sort of a Trump plus, sort of someone
  with Trump's themes, but with a very competent, smooth execution in
  the future we also, I think, need to see conservative institutions
  that are geared towards achieving things in policy and in culture as
  opposed to simply fattening their pockets with panicked direct mail
  pitches about how the Democrats are bringing socialism back and all
  sorts of other things that rile up the 70 plus year old donors. But I
  would be very cautious, however, of naming who the sort of great
  populist or national conservative hope of the future is going to be in
  that nobody saw Donald Trump coming. I think Ross and I were both
  very, very surprised to see that someone with Donald Trump's
  background became not only the Republican nominee, but became
  president. And I think in the future, you may actually see more
  surprises like that, and that the old sense of political
  professionalism is decaying as a result of many changes taking place
  within the country, and also within the media and with the rise of
  social media. So maybe Kanye West is going to be the future. Maybe
  Tucker Carlson. But I'm a little bit skeptical of some of the
  politicians who become kind of flashy in the last few years, but still
  seem a bit untried and wet behind the ears.
\item
  ross douthat\\
  Well, so then let this be the last point that I press you both on. So
  one, I think --- I think, Dan, you're absolutely right about the
  extent to which there still exists this strong conservative
  infrastructure that just wants to fundraise against the threat of
  liberalism, doesn't want to govern, and would be perfectly happy in
  that sense with a Biden presidency, even if Biden himself is not the
  ideal foil. On the political side, though, I think if you brought
  Helen's great hope Josh Hawley onto our show and somehow hooked him up
  to a lie detector, which politicians sadly won't let you do, he would,
  I think, much rather run in 2024 if Donald Trump had lost in 2020. For
  a certain kind of politician, at least, I don't think victory is
  always better than defeat, because American politics moves in cycles,
  and you have openings and opportunities when the opposition party is
  in power that you don't have when you yourself--- when your own party
  is in power, especially if the politician leading your party is
  catastrophically unpopular. And I mean, this, to me, remains the
  conservative case for Trump losing, is that if you want Josh Hawley to
  win an election in 2024, or a figure like him, then he's more likely
  to do it running against an even more decrepit Biden or Biden's
  running mate in 2024 than he is as Trump's heir, and then especially
  --- and this is the last point I'll make --- especially if what Trump
  --- what Trump represents is not the sort of policies that he's
  embraced or half embraced, but a kind of ``own the libs'' celebrity
  culture, in which case, his likely heir is not Hawley. It's not even
  Tucker Carlson. It's someone like his own son, Don Jr., which just
  sort of propagates the side the cycle of maybe this guy can govern,
  no, actually, he can't, ever deeper into the future. So after that
  rant, I'll give you both the last word to explain once again why I'm
  wrong.
\item
  daniel mccarthy\\
  Well, first of all, I think that is extremely a very cynical
  statement, Ross. If you think that you know Josh Hawley has such a
  hollow core that he would rather advance his own personal prospects
  with your scenario for 2024 at the expense of seeing Joe Biden put a
  few people in Supreme Court, and at the expense of seeing what Joe
  Biden will do to foreign policy. That strikes me as a sacrifice ---
\item
  ross douthat\\
  I'll plead guilty. I'll plead guilty to that cynicism.
\item
  daniel mccarthy\\
  But look, if that's your view of Josh Hawley, that he's that cynical
  of a politician, then I don't want Josh Hawley to be the Republican
  nominee if that's who he is. I don't want a guy who is basically no
  different from a Mitt Romney, for example, who had a history of
  changing his positions and changing his ideological complexion based
  on what he thought would win. And in fact, I really dislike and will
  push back against one of the premises of this conversation, which is
  that popularity is sort of the most important thing, and the fact that
  polls show that Donald Trump is unpopular in a large part of the
  country is, therefore, the last word on Donald Trump. I don't think
  that's the case at all. It seems to me that Donald Trump, he's an
  imperfect tool, but he is actually trying to change things in a very
  dramatic way with respect to our economic policy, our foreign policy,
  and to see through the conservative promises on the Supreme Court. And
  those are going to be --- there's going to be some turbulence in that
  project. You're necessarily going to take some hits. Ronald Reagan was
  very unpopular in 1982. These are the sorts of headwinds you just have
  to confront and push through if you're ever actually going to achieve
  anything and be a success at the end of your years like Ronald Reagan
  was. And the idea that because you are unpopular at a given time means
  that you should just kind of give up and do something that's more
  popular is not only cynical, it's a recipe for losing. It's suicidal.
\item
  ross douthat\\
  I have some thoughts in response, but I'll just give Helen the last
  word for the show.
\item
  helen andrews\\
  {[}LAUGHS{]} Running to succeed a two-term president from the same
  party is always a challenge, but I don't think that's a reason for
  even the most cynical version of Josh Hawley to root for a Biden
  victory. Because after four years of Joe Biden's Supreme Court
  nominations, that's going to drastically limit what a president wholly
  can do. What the state of religious freedom precedents would be after
  those four years would probably leave his Supreme Court a lot less
  wiggle room to defend faithful Christians, or four years of Joe Biden
  foreign policy. That would drastically limit what a President Hawley
  could do in terms of furthering the populist agenda. So I stick by my
  cardinal principle of politics, which I think applies to everyone in
  both parties at all times, which is winning is always better than
  losing, and it is this time too.
\item
  ross douthat\\
  I think that's an excellent note on which to say, that's our show for
  this week. Thank you so much for listening. Dan, Helen, thank you so
  much for coming on.
\item
  daniel mccarthy\\
  Thanks, Ross.
\item
  helen andrews\\
  Thanks, Ross.
\item
  ross douthat\\
  You're very welcome. If you have a question you want to hear us debate
  in the future, share it with us in a voicemail by calling
  347-915-4324. You can also email us at
  \href{mailto:argument@nytimes.com}{\nolinkurl{argument@nytimes.com}}.
  ``The Argument'' is a production of The New York Times Opinion
  Section. Our team includes Phoebe Lett, Paula Szuchman, and Pedro
  Rafael Rosado. Special thanks to Brad Fisher and Kristin Lin. And
  don't worry, listeners, this is just a one-episode right-wing coup,
  and regular programming will resume next week. {[}MUSIC PLAYING{]}
\end{itemize}

\href{https://www.nytimes.com/column/the-argument}{\includegraphics{https://static01.nyt.com/images/2018/10/03/opinion/the-argument-album-art/the-argument-album-art-square320-v3.png}The
Argument}Subscribe:

\begin{itemize}
\tightlist
\item
  \href{https://itunes.apple.com/us/podcast/id1438024613}{Apple
  Podcasts}
\item
  \href{https://www.google.com/podcasts?feed=aHR0cHM6Ly9yc3MuYXJ0MTkuY29tL3RoZS1hcmd1bWVudA\%3D\%3D}{Google
  Podcasts}
\end{itemize}

\hypertarget{trump-supporters-make-their-case-for-2020-1}{%
\section{Trump Supporters Make Their Case for
2020}\label{trump-supporters-make-their-case-for-2020-1}}

\hypertarget{conservatives-helen-andrews-and-daniel-mccarthy-join-ross-for-a-reelection-roundtable-1}{%
\subsection{Conservatives Helen Andrews and Daniel McCarthy join Ross
for a reelection
roundtable.}\label{conservatives-helen-andrews-and-daniel-mccarthy-join-ross-for-a-reelection-roundtable-1}}

With Ross Douthat

Transcript

transcript

Back to The Argument

bars

0:00/0:00

-0:00

transcript

\hypertarget{trump-supporters-make-their-case-for-2020-2}{%
\subsection{Trump Supporters Make Their Case for
2020}\label{trump-supporters-make-their-case-for-2020-2}}

\hypertarget{with-ross-douthat-1}{%
\subsubsection{With Ross Douthat}\label{with-ross-douthat-1}}

\hypertarget{conservatives-helen-andrews-and-daniel-mccarthy-join-ross-for-a-reelection-roundtable-2}{%
\paragraph{Conservatives Helen Andrews and Daniel McCarthy join Ross for
a reelection
roundtable.}\label{conservatives-helen-andrews-and-daniel-mccarthy-join-ross-for-a-reelection-roundtable-2}}

Thursday, August 6th, 2020

\begin{itemize}
\item
  ross douthat\\
  I'm Ross Douthat, and this is ``The Argument.'' {[}MUSIC PLAYING{]}

  If you've ever taken the time to read this podcast's description,
  you'll know that our mission is to host arguments from across the
  political spectrum. If you've listened to the show every week, you'll
  also notice that I'm usually holding up the rightmost end of our
  debates, but not this week. While my more liberal co-hosts, Michelle
  and Frank, are on vacation, I'm conducting a temporary right wing
  coup. I've asked two writers who are somewhat to my right, meaning,
  among other things, mostly pro-Trump rather than ``Never Trump'' to
  join the show. We'll talk about the president the pandemic, the Black
  Lives Matter protests, 2020 elections, and the future of conservatism.
  {[}MUSIC PLAYING{]}

  First is Dan McCarthy, the editor of Modern Age, a conservative
  quarterly of ideas, a director of journalism fellowships at the fund
  for American Studies, and a frequent times op-ed contributor. He's
  also been a guest on this show. Dan, welcome back.
\item
  daniel mccarthy\\
  Thanks, Ross.
\item
  ross douthat\\
  And joining Dan is Helen Andrews, who is a senior editor at the
  American Conservative, and the author of ``Boomers, The Men and Women
  Who Promised Freedom and Delivered Disaster,'' which is out next
  January. Helen, thanks so much for coming on ``The Argument.''
\item
  helen andrews\\
  Thanks so much for having me.
\item
  ross douthat\\
  You're very welcome. So let's talk about some issues where I can
  hopefully play the totally unaccustomed role of moderate squish. So I
  thought we could start with how President Trump has handled the
  coronavirus crisis, because given that he's currently seven or eight
  or nine points down in the polls to Joe Biden, I suspect that we can
  all agree that he could have handled it better, but since both of you
  have been somewhat skeptical of lockdowns and other tough public
  health measures, I also suspect that we disagree about what a better
  strategy would have been. So starting with you, Dan, what do you think
  the president could have done differently over the last five months or
  so?
\item
  daniel mccarthy\\
  Well, you know, I don't think there's any kind of silver bullet here.
  If we look at the responses that other countries have had to COVID-19,
  while there are places you can point to that have had considerable
  success, there are places that are very different from the United
  States, that have leadership that is very different, places like
  Sweden, for example, places like the United Kingdom, where you have a
  conservative leader, but one who can recite ``The Iliad'' from memory.
  Quite a contrast from Donald Trump, perhaps. There are places like
  Italy, which has a government that's always rather rickety. You can
  see a wide swath of different governments responding in different ways
  to COVID-19, and a great many of them are getting it wrong. And the
  ones that have gotten it right, even they are starting to have a
  resurgence of the virus and other difficulties. So I just don't know
  that there is anything Donald Trump could have done that would have
  been the one right way of approaching this. And of course, this is
  something where the role of the governors in our 50 states is also of
  the utmost importance, and in fact, is of rather more importance than
  the president himself. Perhaps the easiest thing to say about what the
  president could have done is to perhaps act as psychiatrist in chief,
  not as physician in chief, but someone who could have made some
  reassuring statements that might have made people simply feel better
  about this crisis. But I don't think that would have had much
  practical effect. It seems to me that some of these crises that we're
  facing are --- they're acts of God. They are things that it's very
  hard for any kind of leadership simply to overcome.
\item
  ross douthat\\
  That's some pretty deep fatalism, Dan. Helen, I'm hoping that you will
  make a more explicit case for why we should have simply gone full
  Sweden.
\item
  helen andrews\\
  Unfortunately, I'm going to have to disappoint you there, and also to
  disagree with the premise of your question. I don't think the
  president has handled the coronavirus crisis badly. I think he's
  actually handled it pretty well. One of the great frustrations for
  people who generally like and support this president is that he seems
  to be someone who has a hard time understanding that often, the
  correct response to a situation is to say nothing. He seems to have a
  real problem grasping that sometimes silence is the best policy. But
  here, with the coronavirus, he seems to have finally grasped that
  lesson. He hasn't made himself the point man on every policy question,
  and I think that that's a good thing, because a lot of this is stuff
  that should be decided by governors anyway. So there's no reason for a
  president to have a position on a national mask mandate or something
  like that. And there are a lot of things that we know now that we
  wouldn't know about comparisons between different strategies because
  we allowed a diversity of responses on the part of different states.
  So I think the fact that the president has left space for that to
  happen is actually a pretty good decision on his part.
\item
  ross douthat\\
  How would you make the case for the president's re-election in the
  context of, let's say, by the time the fall rolls around, up to
  200,000 or so deaths from the coronavirus? Because I think it's fair
  to say that if you had flashback to say February or early March when
  we were first having conversations/panics about COVID-19, that if you
  had said, well, over the course of the next six months, 200,000
  Americans are going to die from this disease, a lot of people would
  have assumed that the president's re-election efforts were doomed, and
  I don't think either of you think his re-election effort is doomed.
\item
  daniel mccarthy\\
  Well, his re-election chances aren't doomed, but there is a limit to
  what any president can do in the face of certain public assumptions,
  which are of a almost divine nature. So I'm a very strong critic of
  President George W Bush. I was very much against most of his domestic
  policies, and even more so, his foreign policy, especially the Iraq
  war. One thing, however, that I've generally not blamed President Bush
  for was his response to Hurricane Katrina. And there were certainly
  mistakes he made there that were quite big but fundamentally, when you
  have a natural disaster, the president is not a god. He can't change
  those outcomes. I mean, sometimes, these discussions, it's almost like
  being a medieval priest and you're being confronted by a sort of angry
  group of people who say, why didn't the king cure this man's scrofula,
  and the fact is, well, actually, kings can't cure scrofula. That's
  something beyond their power. So the president is necessarily going to
  take a hit to his approval ratings and his re-election chances as a
  result of the COVID-19 crisis. That doesn't mean, however, that there
  was really much that he could have done one way or the other that
  would have greatly altered what he could --- the response that he
  would get from whatever actions he might have taken. Now, in terms of
  what case can be made for his re-election, it's simply that
  fundamentally, what he had warned us about back in 2016 regarding the
  possible consequences of globalism have all proven true as a result of
  the COVID-19 crisis. We've had a economic collapse that has perhaps
  been worse than it had to be as a result of our interconnectedness
  with China and with other parts of the world, and of course, the fact
  that the coronavirus came in from outside of the country is another
  sort of point against globalism and point in favor of Donald Trump's
  fundamental populism and nationalism. So I would say that Donald Trump
  should reiterate his original 2016 campaign points, his themes, and
  that what we've seen happen this year, as disastrous as it has been,
  is, in fact, a vindication of his early warnings back in 2016.
\item
  ross douthat\\
  OK, sure, it's a vindication of his early warnings, but in response to
  those early warnings, who did we elect president? Seems like we
  elected Donald Trump president of the United States, right? So I mean,
  it seems to me that in this crisis, he was, indeed, handed a challenge
  whose parameters did sort of vindicate portions of his 2016 world
  view, but then once confronted with this challenge, from a reasonable
  perspective, it seems like he's failed. That seems to be the problem.
  You're simultaneously saying, this is an act of God that no president
  could hope to deal with, and Trump sort of saw it coming, but if he
  sort of saw it coming, then it's not just an act of God. It's a
  predictable challenge that maybe we should have been better prepared
  for and had a response to.
\item
  daniel mccarthy\\
  No, but again Ross we have a government of 50 states, not just one
  central federal government. And the central federal government that we
  do have is divided between the president and Congress. Now, I would
  criticize the president for having sort of lost the initiative early
  on when he did have control of Congress, but he's had the opposition
  party in charge of the US house of representatives since two years
  ago. So while you can say that President Trump could have had some
  sort of dictatorial powers and done everything that he might have
  wished to do, that was never really the case. It certainly has not
  been the case with respect to the states and their policies, but it
  hasn't even been the case with respect to the federal government. He
  had a very short window where he had plenipotentiary power, and that
  went by the wayside two years ago.
\item
  helen andrews\\
  And just to be clear, the mere fact that the coronavirus crisis
  vindicated many of Trump's campaign policies doesn't necessarily mean
  that he could have done anything about it once it hit the United
  States. I mean, it's not as if he can, in the space of a month, make
  American supply chains less dependent on China. Even if, in
  retrospect, we had done that, we might have been in better shape, the
  American people will forgive our president for not controlling things
  that he's unable to control. They don't expect presidents to be gods.
\item
  ross douthat\\
  They don't? I would dispute that premise too, honestly. I think ---
\item
  daniel mccarthy\\
  Yeah, I might agree with you, Ross. Perhaps they do expect presidents
  to be gods.
\item
  ross douthat\\
  I mean, I think it's --- at the very least, though --- at the very
  least, though, I mean, you compared this to Hurricane Katrina, but
  COVID-19 is something where it's a rolling crisis, where every day in
  the United States, a certain number of people become infected, and
  every day, a certain number of people die, and the choices that are
  made, the public health choices that are made over the course of those
  months presumably have some impact on it. And I think that both of
  you, but I think especially you, Helen, have argued that whether this
  was Trump's choice or not, that a lot of the choices we made were the
  wrong choices, that sort of locking down as extremely as we did in the
  months of March and April was actually the wrong approach. Is that
  what you think?
\item
  helen andrews\\
  That is, but even though I think the lockdowns were probably more
  extreme than they had to be, I am perfectly willing to make allowances
  for decision-makers who have to operate with imperfect information. I
  think that you can't expect people to make perfect decisions when they
  don't yet have all the facts that they need. But we're now a few
  months into the crisis, and we know better what needs to happen next.
  And so now is the time when I think the American people are going to
  start grading their politicians a little more toughly on the decisions
  that they make. And the one thing that they won't forgive is people
  who make politicized decisions in response to a health crisis. Which
  is why I think the biggest opportunity for the president going forward
  in terms of how he should handle coronavirus to improve his
  re-election chances, is to focus on schools, because that's definitely
  a case where you have teachers unions and a lot of the pro-lockdown
  side making decisions that look to a lot of parents like they're
  politically motivated, and also, like they'd be perfectly happy to
  keep schools shut, sometimes, it seems, indefinitely, and prevent
  private and charter schools from opening up either. So taking on the
  teachers unions who are making politicized decisions might be a good
  strategy for the president going forward.
\item
  ross douthat\\
  Do you think those decisions are politicized in the sense that they're
  being made in effect to spite the president, or do you think they're
  politicized --- I mean, it seems to me much more that they're made in
  a spirit of panic and anxiety about the disease that may be
  overstated, or maybe sort of missing the importance of having schools
  open and the importance of balancing that against the public health
  risk, which is not exactly the same as being politicized. It's more
  disease anxiety driven.
\item
  helen andrews\\
  Well, when I look at the list of demands from Randi Weingarten and the
  A.F.T., it includes a lot of things that don't have anything to do
  with education, like eviction moratoriums, a wish list that you would
  expect from activist Democrats. So that's pretty clearly politicized.
  And I'm in favor of a diversity of approaches. So if a local school
  board wants to keep schools closed or remote until November, that's
  their decision to make. But when I see those same school boards and
  county health organizations trying to shut private and charter schools
  as well and not allowing those schools to make their own decisions and
  maybe find their own path forward, that to me does look like teachers
  unions trying to kneecap their opposition, because they're afraid that
  competition from private schools and religious schools will draw away
  pupils. So, yeah, I think got a lot of parents do detect some
  politicization in decisions like that, and I think they're right.
\item
  ross douthat\\
  Right. There was the case just now in Maryland where they announced
  that--- this has since been overturned by the governor, I guess, in
  defiance of your call for local decision-making in some sense, Helen,
  but they first announced that private and parochial schools would have
  to stay closed till October 1. And one of the theories for why they
  chose that date was that that's the date at which enrollments are set
  in public schools, and they were hoping to effectively maintain or
  increase public school enrollments for budgetary reasons. So I think
  there's obviously some sort of effective political competition within
  the education system going on in this totally weird environment. But
  let me pull you back to the macro level. So, fine, you think Trump
  should be making a kind of campaign on school openings. At the same
  time, you are saying that fundamentally, these are local and state
  decisions, which, of course, they are. So in a sense, you're making an
  argument for Trump using the bully pulpit, using sort of rhetorical
  powers to make a particular case. But why doesn't that apply to the
  entire public health situation? Certainly, Trump can't institute a
  national mask requirement, but it would certainly have been possible
  for him to, start wearing a mask in public a month and a half earlier
  than he did and not discourage mask wearing tacitly in all the various
  ways that he's done. And he has, throughout this disease, had the
  power of sort of public speechifying, the power of making statements.
  He attempted to sort of command the situation with his nightly press
  conferences for a while. In that sense, if he can use the bully pulpit
  for his re-election, then I don't see how you can sort of absolve him
  for the way he has or hasn't used the bully pulpit in the last few
  months. Do you think there's been any kind of consistent public health
  message emanating, not from the experts around Trump, but from the
  president himself?
\item
  helen andrews\\
  The very simple reason why Trump should not use the bully pulpit more
  is that when he wades in to political debates, he very often makes
  them worse. Not so much because anything he does or says is
  particularly reckless, but because of the way other people react to
  him. An unfortunate fact is that there are a lot of people involved in
  our political discourse who will immediately take the opposite
  position to whatever the president adopts. And we saw that just in the
  past few weeks. A month ago, it seemed like everyone was kind of
  pretty much on board with the idea that children were low-risk
  spreaders, and therefore opening up schools should be one of the first
  things on our list so we can get people back to work. And yet, when
  the president came down on the side of reopening the schools, a lot of
  people rushed to the other side. So yeah, I think when the president
  weighs in on things, a lot of people immediately leap to say the
  opposite. And that may say bad things about them, but that's just a
  fact of our political discourse we have to deal with. So I think very
  often, the president does better by the country by staying out of
  certain debates.
\item
  daniel mccarthy\\
  Well, and I'll sort of jump in here and underscore something that
  Helen is getting at. This idea that we have to house our national
  media pointing to Donald Trump and saying, you know what? COVID is
  really about Donald Trump. It's about what Donald Trump does, what he
  says, the signals he sends, the policies he makes. It's simply a
  misreading of our actual structure of government. It's an actual
  misreading of the way in which the private sector works here, and we
  really do need people to be making prudential decisions at the very
  --- not just the local level in terms of government, but the molecular
  level in terms of our society. People need to be making smart
  decisions about how to discriminate between high-risk and low-risk
  groups, how to keep a relatively sound degree of separation between
  these groups if there's a chance that someone in a low-risk group may
  be infected and could perhaps harm someone in a high-risk group. All
  of these things have to be done at the societal level. They can't
  simply be mandated from Washington D.C. Whether it's Donald Trump, or
  if Joe Biden were president, it would not make any difference. It is
  something that fundamentally requires personal responsibility. I think
  Americans have been trained and have been told that they should not be
  exercising this kind of personal responsibility. They should not be
  thinking for themselves. They should be looking to some sort of expert
  or god figure and deferring to that person.
\item
  ross douthat\\
  But again, isn't this --- I mean, I watched Donald Trump's 2016
  campaign where he literally used that kind of rhetoric, I think very
  effectively in many cases, in ways that a, shall we say, more
  libertarian or states' rights oriented style of Republican would not,
  right? Like, Trump basically said, look there is a national role, a
  national federal role in dealing with things like the opioid epidemic.
  You need a national strategy on the offshoring and factory closures.
  He would go around the country and give speeches in towns that had
  factories closed, and people would say a version of what you're saying
  now, Dan. Like, oh, well, this is an organic process, and it's a
  molecular process, and the decisions that are made by individual
  actors can't be micromanaged in D.C., and Trump basically said, to
  hell with that. The national government is powerful, and we actually
  need a policy to deal with people taking jobs overseas, right?
\item
  daniel mccarthy\\
  Well, Ross, no ---
\item
  ross douthat\\
  It just seems weird to retreat to that rhetoric in the face of a more
  immediate crisis.
\item
  daniel mccarthy\\
  No but you're fixated on a single point here, which is always, if
  Trump has a power in one field, he must have a power in every field.
  He must be a kind of omnipotent figure, and that's not correct. The
  federal government does ---
\item
  ross douthat\\
  I have certainly never claimed that Donald Trump is an omnipotent
  figure. I just want to put that on the record.
\item
  daniel mccarthy\\
  It is the --- no, it is the presumption of this conversation right
  now. But in fact, the federal government has a role in trade policy
  historically that exceeds what its role in health care policy has
  been. And that's just a fact of the way the U.S. Constitution works
  and the way our governments work. The United States is facing foreign
  competitors, facing entities like the People's Republic of China,
  which cannot be confronted by a locality or by a state, but which can,
  in fact, be confronted by the federal government. COVID-19 is a threat
  of a different kind. It's coming not from the outside at this point.
  It's now coming from within the country. And it's coming not from a
  sort of conscious entity from the outside. It's rather a phenomenon of
  individuals communicating the disease to one another right here. So
  the fact that there is a division between what Donald Trump can do
  with national power and what he can't do does not mean that there's an
  inconsistency here. It just means that there really is a pattern to
  what he is able to apply his abilities to. And COVID-19, some of these
  natural disasters, I think, are not in that category.
\item
  ross douthat\\
  If you go back to the period of, let's say, one to two months between
  when the coronavirus first appeared in China and the fact that it was
  dangerous became clear, and the period when it started to really hit
  New York City, you had this zone where the president did do something
  that used his actual federal powers. He instituted some kind of travel
  ban on people coming from China, and we can debate how effective that
  was, but that was real. He did that. And then there was a six-week
  period where if you looked at his rhetoric, he consistently downplayed
  the likely severity of the problem, made what turned out to be totally
  false predictions about having the virus under control, and that was
  the period when --- and this this part was not Trump's fault per se,
  although it is his executive branch --- but where the C.D.C. and the
  F.D.A., but especially the C.D.C., sort of fell down on the job in
  terms of testing and containing the virus. And at the end of that
  period, you had the huge outbreak in New York City, which also
  reflected disastrous policy choices on the part of local officials and
  state officials there. But the state and local officials right now, I
  mean, especially Andrew Cuomo, have pretty high approval ratings, in
  part just because they rhetorically seemed to take the virus
  seriously. So it seems to me that there both was, first, a clear
  window for some kind of federal action where the president missed the
  boat, and second, that there is a political upside to even if you're
  doing a bad job, or even if it's an act of God and beyond your
  control, just seeming sort of rhetorically in control in a way this
  president struggles to do. Do you think that's wrong?
\item
  daniel mccarthy\\
  I do think it's wrong. First of all, regarding his window of
  opportunity to act, he took action, and the actions he was able to
  take were marginal. So the president is very proud of what he did in
  terms of trying to restrict travel into the United States as the
  coronavirus was breaking out. We can say, OK, well, even if that's
  succeeded, even if that was exactly what Donald Trump thought needed
  to be done and he did it, obviously, it was not sufficient to stop the
  disease from reaching the levels that it has reached. Now, if he had
  not done that, if he had just let more people come in with the virus,
  what effect would that have had? Well, would it make things even
  worse. So the fact that Donald Trump has limited powers that can have
  a marginal effect is, I think, just a reality that has to be accepted
  here rather than saying that the overall situation is something over
  which he has control. Regarding the relative popularity ratings of
  President Trump and Governor Cuomo, I mean, sometimes, a democracy
  gives you an injustice, and in the case here, I think Andrew Cuomo has
  a lot more lives on his hands from having sent people with the
  coronavirus into nursing homes and other places that couldn't turn
  patients away, or people away who had the disease and then spread it
  to the most vulnerable elderly populations. That is catastrophic, and
  it's something he should face serious electoral repercussions for.
  It's a question of whether the American people are going to look
  seriously enough at where the responsibility actually lies in our
  federal system relative the president versus the governor.
\item
  ross douthat\\
  But surely, this goes back to the question of rhetoric and
  presentation as a part of statesmanship?
\item
  daniel mccarthy\\
  No, it doesn't. Because not everything is a matter --- no, absolutely
  not. This idea that you can magically cure things with words has to be
  resisted.
\item
  ross douthat\\
  No, no, no, but that's not even --- but that's not --- that's not per
  se the point. I started this conversation by asking both of you what
  the president could have done differently, and Dan, your answer was
  nothing, and Helen, your answer was he's done well to the extent that
  he's minimized his public presentation. But what I'm saying is that
  what you're describing as the injustice, Dan, of Cuomo being popular
  in his disastrous response reflects Cuomo's use of the art of public
  rhetoric and statesmanship, which is part of the president's job
  description, indeed, a central part.
\item
  daniel mccarthy\\
  Well, look, I mean, you're asking --- yes, you're asking a basic
  Machiavellian question here. Is it important to seem virtuous even if
  you're not?
\item
  ross douthat\\
  Yes, sure.
\item
  daniel mccarthy\\
  And in Cuomo's case, that's what he's done. He's been able to create
  the appearance, perhaps, of a certain kind of competence. But I'll
  mention two things here. The first is that in terms of the way that
  the media has reported the COVID-19 crises, and even the way this
  conversation is going right now, you can certainly see that there is a
  tendency to, in all of the elite media, to give Donald Trump the worst
  possible reading and to be rather more generous towards Democrats.
\item
  helen andrews\\
  I have to agree with Dan, here. Ross, you seem to be proposing that if
  only Trump were a better political communicator, he could be more like
  that great hero Andrew Cuomo, to which I can only respond that being
  more like Andrew Cuomo is not a thing that I would like my politicians
  to be interested in. And I also think the media plays a bigger role
  here than you are giving it credit for. Trump could have literally
  lifted the script from every Andrew Cuomo press conference and it
  would have been covered very, very differently.
\item
  ross douthat\\
  First of all, I'm not saying that wouldn't it have been great if the
  national response had been more like New York's response in those
  first few crucial weeks. I am saying that if your political goal is to
  govern the country successfully, which requires, among other things,
  being re-elected, then yes, presumably, the Republican Party and its
  position would be better if Donald Trump had some of the communication
  skills of Andrew Cuomo, which, yeah, it is a Machiavellian point, I
  suppose, but it's kind of an essential one. And it's just very ---
  it's very hard for me to imagine a world in which, maybe not the two
  of you, but where most Republicans or conservatives were willing to
  look at 200,000 dead Americans in the final year of a Democratic
  president's first term in office and say, well, it's time for the
  American public to grow up and realize that President Obama doesn't
  have magical powers and there was nothing that he could have done. I
  mean, if you had 200,000 dead Americans under Democratic precedent,
  Republicans would be screaming bloody murder about it, and I think
  completely understandably so.
\item
  daniel mccarthy\\
  Well, Ross, look, you could say that, yeah, if the president had the
  ability to sort of bamboozle the public and to give them a sense that
  they are sort of healthier than they actually are through very
  persuasive language that that would make us all feel better---
\item
  ross douthat\\
  Or give them the sense that --- give them the sense that he knows
  what's going on and he has a plan to deal with it, right? I mean,
  that's ---
\item
  daniel mccarthy\\
  No, that's not what he should do, because he doesn't know what's going
  on. {[}DOUTHAT LAUGHS{]} Nobody knows what's going on. No, I'm quite
  serious about this. I think we're in a very dangerous position, and in
  fact, not just us, but the whole world, because we're assuming that we
  know things about this disease that we really don't, and we should be
  very careful here and we should be very cautious with our local
  circumstances as opposed to assuming that we've got some sort of
  master plan that's going to deal with this. I don't think any of us
  actually knew six months ago that this was going to be quite as severe
  a problem right now as it still is. And we need to also keep in mind
  the costs of policies like shutdowns, for example, on psychology, on
  suicide rates, on other deaths of despair. There are any number of
  contingencies and complexities here which I think we need to be honest
  about. We can't simply say, you know what, we're just going to clap
  for Tinkerbell and that's going to resurrect ---
\item
  ross douthat\\
  OK, good. No, but let's stick with that point, right? So forget Donald
  Trump for a minute. Just articulated in your own --- from your own
  point of view, what have we learned? We've learned something about the
  disease over the last four to six months. We have a lot of --- we have
  a wide range of outcomes in different countries. We have observed and
  absorbed many of the costs of lockdowns right now. So accepting that
  this varies from state to state, what should be our policy going into
  the fall? Let's say we have a residual lockdown right now with some
  limits on gatherings, and possibly, we'll end up with school closures
  and so on. Is that a mistake? I mean, were the swedes right to assume
  that we could reach herd immunity at a pace that would save our
  economy. What have we learned? What have you guys learned?
\item
  daniel mccarthy\\
  Well, look, Sweden still has per million more COVID-19 fatalities than
  we have, and maybe that's going to change over the course of the next
  several months, but I'm not quite prepared to say that the Swedish
  model is the correct model. On the other hand, I am less confident
  than you are, Ross, that we already have as clear lessons as we would
  like to have about this disease and how best to approach it, and
  that's one reason why I'm emphasizing that this sort of most localist
  approach possible is what I think is the only thing you can do.
\item
  helen andrews\\
  I also think that's the biggest thing we've learned, is how big that
  difference in risk is for different demographic groups. There are
  still today a lot of states where most of their coronavirus deaths
  took place in nursing homes, a majority in facilities that house less
  than 1 percent of the U.S. population. That's a huge thing that we've
  learned in the past few months, and maybe that should tell us that
  focusing on those high risk groups would have been better to do from
  the beginning, and certainly, better to do from here on out. Because
  unlike you, Ross, I'm not entirely confident that we have weathered
  the worst of the economic storm. I think that a lot of the jobs that
  are currently being thought about as layoffs or temporary furloughs
  are going to end up being permanent. A lot of small businesses are
  going to close no matter what happens with unemployment insurance or
  any other mitigating measures from here on out. So I think the
  economic cost of lockdown could be a lot higher than we're thinking
  right now. So between that and the concentration on nursing homes, I
  think we should start looking at opening up.
\item
  ross douthat\\
  But we have started opening up, right? And the challenge is precisely
  that. I'll flip the script and argue a version of you guys's case,
  which is that the power of the state as manifested in lockdowns is
  maybe less significant than the choices that people make themselves to
  socially distance, not go to restaurants, not go shopping, not go to
  the mall and so on. So you have, if you look at trends in foot traffic
  and OpenTable reservations and all of these things, you see sort of
  steep declines before states implement lockdowns, and what you see now
  in states that have opened up, or partially opened up, is that when
  there's a surge of cases of the virus, people react to some extent the
  way that they did before, and you get a kind of cycle. You get a
  reopening, people start going to bars and restaurants again, then the
  virus surges, then people stop. But that means that the economic pain
  exists to some extent independent of public policy. It exists so long
  as the virus is an active force in American society. So I mean, to me,
  the stronger anti-lockdown case is psychological rather than economic.
  I don't see how you avoid the massive economic hit from this virus
  with or without formal lockdowns. But the lockdown policy, the sort of
  extremity of it does seem to generate an intense psychological toll
  that is manifest in the most extreme form in rising suicide rates and
  so on, but I think is also manifest in the protests of the summer. I
  think there was a reasonable argument that you don't get massive
  protests across the U.S., even though they are officially about racial
  justice and police brutality, that you just don't get those protests
  without the two-month lockdown policy beforehand. Helen, do you think
  the lockdowns caused the protests?
\item
  helen andrews\\
  Absolutely. A fact about the Russian revolution that a lot of people
  don't know is that the winter of 1916 and 17 was one of the harshest
  on record, and that everybody in St. Petersburg had basically been in
  lockdown because of the weather and the harsh winter for months and
  months and months. Until the end of February, when suddenly,
  temperatures shot up, and everybody went outside, and a week later,
  the Romanov Dynasty was no more. {[}LAUGHING{]} So yeah, when people
  are locked up for a long time, and then suddenly, you let them out,
  they run into the streets and they go crazy.
\item
  daniel mccarthy\\
  Well, and you only let them out for the protests. I mean, that's one
  of the other key things here. People can't really go to work as
  normal. They can't really go to school or whatever educational
  opportunities they would have as normal. They can't socialize as
  normal. Practically the only thing they're allowed to do is to
  congregate outside in large numbers and burn down a police department.
  If that's your only option, and that's the only kind of social
  activity you can take part in, a lot of people are going to
  opportunistically glom onto it.
\item
  ross douthat\\
  We're going to take a quick break, and when we come back, we're going
  to talk about the fall election and whether conservatives should wish
  for a Donald Trump second term. We'll be right back.

  {[}MUSIC PLAYING{]}

  And we're back. So I want to start with the larger cultural upheaval
  that's affecting maybe elite institutions in particular, but also,
  corporate America, sort of the general tug to the left that's going on
  in American institutions right now as a reaction in part against the
  presidency of Donald Trump. That just as you see tugs to the right
  under Democratic presidents, you get tugs to the left under Republican
  presidents. And I'm going to submit and then have my guests disagree
  that the strength of this tug and the weakness and incapacity of Trump
  himself in response to it, his inability to sort of harness or marshal
  public opinion against the leftward swing in American life is a reason
  for conservatives to wish that he loses the election in November,
  because a Trump second term would probably be an extension of the end
  of his first term where the right clings to political power in
  Washington D.C. and continues to lose cultural ground just about
  everywhere you look. Helen, I suspect that you disagree and that you
  will be supporting Donald Trump in November. Tell me why I'm wrong.
\item
  helen andrews\\
  Yeah, I disagree vehemently. I disagree very strongly. I think any
  conservative out there who's wishing for a Donald Trump loss is just
  flat out crazy. Winning is always better than losing. And if you're
  somebody who supports the things that Donald Trump campaigned on, you
  especially should be helping that he wins so that the G.O.P. doesn't
  just revert to business as usual. I kind of knew when Trump was
  elected that there would be a lot of growing pains, more than for most
  presidents because he was such an outsider, but I do think there is an
  observable learning curve. I don't know how much he's gotten better,
  but the people on his staff have gotten better and four years wiser,
  and I think a second term would be a lot smoother. I also think that I
  have been just more disappointed that I thought I would be by the
  behavior of the left in response to Donald Trump. The inability of so
  many people in the Democratic party, and not a few in the Republican
  establishment, to accept the outcome of the 2016 election and just
  pitching tantrums and throwing bogus impeachments. And I really think
  it's important that that kind of behavior not be rewarded. So those
  are two good reasons to be hoping that Donald Trump wins.
\item
  ross douthat\\
  Dan, can you give me more?
\item
  daniel mccarthy\\
  Yeah. Well, certainly, if you look at policy, if you voted for Donald
  Trump in 2016, and you wanted to see a degree of change in our foreign
  policy, if you wanted to see conservative justices put on the Supreme
  Court, or at least, as close as the Republican Party seems able to get
  to putting conservative justices on the Supreme Court, then it seems
  to me that you have to support Donald Trump and hope that he will see
  through the project that he began in 2016. In foreign policy, it looks
  to me as if Trump now has a stronger handle on things, that we're not
  going to get appointments of neoconservatives like John Bolton in a
  second Trump administration. I see some promising personnel moves in
  terms of the next ambassador, for example, that Trump wants to appoint
  to Germany. And in various other places as well, I think you're
  actually starting to see the kind of braintrust forming to see through
  a more restrained foreign policy in a second Trump administration. And
  of course, foreign policy is the thing that the president really has
  the most direct control over. As far as the courts are concerned, this
  is an existential matter for conservatives. In terms of religious
  liberty, it's certainly an existential question in terms of the lives
  of the unborn. Conservatives have been bitterly disappointed by
  Justice John Roberts, and to a somewhat lesser extent, also by Neil
  Gorsuch. However, the idea of having Joe Biden in there, probably with
  a Democratic Senate, and having more justices like Kagan or like Ruth
  Bader Ginsburg, that would obviously be even worse for conservatives.
  So the conservative approach to the courts has been a mixed success,
  and in the eyes of many, it's been a failure, but it hasn't been as
  great a failure as you would get with a court who had appointees
  coming from the left wing of the Democratic party.
\item
  ross douthat\\
  Talk a little bit more about the idea of a braintrust, Dan, because I
  think we disagree on this.
\item
  daniel mccarthy\\
  Well Donald Trump came from out --- yeah.
\item
  ross douthat\\
  I mean, imagine a Trump second term. Right now, the Trump cabinet has
  an awful lot of acting non-confirmed cabinet rank officers. The Trump
  White House has partially emptied. Now, I'm sure you're glad of some
  of the emptying, because it reflects, as you say, people who Trump
  hired who are either very conventional establishment Republicans, or
  in Bolton's case, much more hawkish than the president himself. I
  think that's reasonable, but it doesn't seem to me that there is a
  populist braintrust around the president. And by populism, I mean
  people who would agree with your perspective on foreign policy, which
  is to say, containing China, and otherwise very restrained in military
  adventurism, and in domestic policy, being willing to break with sort
  of Reaganite views on things like infrastructure spending, let's say.
  I think such a populous braintrust exists outside the White House. If
  you put me in charge of assembling such a braintrust for Trump in his
  second term, I think I could do a decent job, and maybe would hire the
  two of you, but I don't see any evidence that Trump himself sort of
  sees that as his mission or that there's a group coalescing around him
  that has that kind of clear agenda. Trump's most sort of successful in
  terms of duration cabinet official is Mike Pompeo, who as far as I can
  tell, is still obsessed with conflict with Iran. So who is the
  braintrust?
\item
  daniel mccarthy\\
  Well, so I think that you're beginning to see the right kind of
  appointments being contemplated, and in some cases, being made. And
  obviously, on the eve of an election, you're not having as much action
  in terms of appointments as you would get after an election, after you
  have a second term coming on. But no, I mean someone like Douglas
  MacGregor, for example, being mooted as appointee for ambassador to
  Germany. That's very promising. And there are a lot of other people I
  hear in the pipeline of similar caliber and similar views. Donald
  Trump took a couple of years here, and I regret that it took as long
  as it did to find out that people like John Bolton really are not on
  his side. And Mike Pompeo, he has certainly been persistent. He's
  stuck around for a long time, and he may stick around for a longer
  time yet. And he is someone who comes from a more conventional
  Republican background, but I don't think he's the whole story, and I
  think there's actually a lot of things --- interesting things
  happening on the personnel side of the administration, sometimes in
  the less high-profile roles, that are indicative of a new approach to
  staffing.
\item
  ross douthat\\
  Who do you --- who do you trust? Who do you trust in the White House?
  Give me somebody who you trust, who you ---
\item
  daniel mccarthy\\
  First of all, in terms of things like trade policy, I think you've had
  a braintrust from the beginning. Robert Lighthizer, for example, I
  think, is an indication that the Trump administration has had its act
  together with respect to its trade policy almost from the beginning I
  get the impression that with appointments like Douglas MacGregor, they
  now have the same sort of focus with respect to the foreign policy as
  well. Mike Pompeo is a more conventional Republican, yes, but I think
  that a lot of people around him are going to be more in the Trump
  estate. There have been a lot of battles lately where people are
  saying that Trump is appointing people who are too conservative. The
  former head of the Claremont Institute, for example. But this is
  actually what Trump was elected to do, to make sure that the message
  that we're sending out to the world is a message that reflects the
  changes here in this country with respect to our foreign policy, that
  we no longer are in the business of militarily promoting democracy and
  trying to engage in regime change operations here, there, and
  everywhere, but rather, we're talking about America's fundamental
  values, and we hope that that example is what will change countries,
  as opposed to skullduggery to kind of forcibly alter other regimes.
\item
  ross douthat\\
  Helen, let me go back to your initial point, that winning is always
  better than losing, which is a powerful point. So in 2016, Donald
  Trump ran, as among many other things, the candidate of standing for
  the national anthem, right? I mean, I think that that's --- that was
  sort of a condensed symbol of Trump's campaign in 2016, where he used
  Colin Kaepernick as a foil to attack athletes who didn't stand for the
  national anthem. Now, here we are four years later, and we're in a
  cultural landscape where kneeling for the national anthem is the norm
  and standing for the national anthem is now seen as sort of the act of
  protest. Now, this is a small, condensed symbol that obviously doesn't
  have immediate policy implications, but it seems like a pretty
  striking cultural defeat that conservatives have absorbed, and I think
  would not have absorbed had Donald Trump not been president, in part,
  because of the fact that Trump's unpopularity, the fact that most
  Americans they really don't like him has created all of this space for
  left wing argument and left wing movements to effectively gain large
  amounts of cultural ground. Now, I know you don't think that's totally
  wrong. Tell me why that doesn't just happen even more so across the
  next four years, where you're effectively trading maybe one more
  Supreme Court seat for even more sweeping cultural defeat.
\item
  helen andrews\\
  I'm actually not sure I agree with any part of the premise of your
  question. First of all, your logic sounds like an argument for
  electing McGovern in 1972. And the `60s would have ended sooner if
  we'd gone further to the left, which just doesn't sound right as a
  matter of history. Also, as a matter of history, I disagree that none
  of this would be happening if Hillary Clinton had won in 2016. I think
  it's entirely possible that the fringe left would have felt empowered
  by that victory and we would be seeing many of the same things that
  we're seeing now. But that's actually not the important question. The
  relevant question is, will these forces be stronger or weaker if Biden
  wins in 2020? I think that the Trump campaign's message of you won't
  be safe in Joe Biden's America is exactly the right one. I don't think
  anybody expects that Joe Biden himself is secretly a closet Sandinista
  radical, but I do think a lot of people are worried that he's so old
  and out of it that he would be a figurehead. So whatever aggravating
  effect Donald Trump might have on the psyches of leftists who can't
  stand that he won in 2016, I think that effect just pales next to the
  empowerment these forces would feel under a Biden presidency.
\item
  ross douthat\\
  But it's not just about the empowerment of those particular forces,
  right? It's also about public opinion writ large and the pressures
  that are put on, for instance, mayors and governors in terms of
  dealing with riots and urban unrest. where right now, there's this
  sense that every event is about Trump, and every choice that, lets
  say, a blue state governor or a blue state mayor makes is seen in
  light of whether it helps or hurts Trump, right? So you have this ---
  you have pressure right now on blue state mayors to refuse federal aid
  or something in terms of dealing with soaring murder rates, because
  that's seen as boosting Trump in some sense, right? In a Biden
  presidency, that kind of thing goes away. In a Biden presidency, you
  get a thermostatic swing in public opinion, the way you do in almost
  every presidency, where moderate voters who are right now very worried
  about Donald Trump's capacities to handle the coronavirus and haven't
  yet internalized the lessons that you guys were preaching in the first
  segment of our show would swing to being more worried about the
  overreach potentially of liberalism and would be more likely to vote
  for Republican candidates in 2022. And I just want to push on your
  historical analogy, Helen, which is that, yes, if Donald Trump were
  Richard Nixon, which is to say, for all his faults, a very effective
  politician who was capable of building, in the end, a 60 percent
  majority coalition against McGovern, then of course, it would be
  ludicrous for Republicans or conservatives to want the liberal
  democrat to win. But if Trump is a figure more like George Wallace,
  sort of a representative of a sort of permanent minority faction who
  can only win through electoral college luck, then having him in power
  doesn't--- I mean, it's a different --- it's a different case, right?
  It's like if you were a conservative in 1968, would you rather have
  Hubert Humphrey or George Wallace? Maybe you'd rather have Hubert
  Humphrey. This is the question with Trump. At some point, if you're
  going to actually govern the country, you actually need to win a
  majority of the country. You can't just rely on electoral college
  minorities for generations yet to come and having your standard bearer
  be this deeply unpopular figure who constantly hands cultural
  victories to the left. Seems like it pushes that moment ever further
  out of reach.
\item
  helen andrews\\
  I have high hopes that the standard bearer for populist conservatism
  that comes after Trump will be a lot more normal and more competent
  than he is. I have every hope in the next generation. But I really am
  struggling to grasp where you're coming from on this line of
  questioning, Ross, just because are you really saying that if Biden
  wins, the local government of Portland is finally, for the first time
  in its existence, going to start cracking down on Antifa terrorism and
  not let these 20 somethings in black roam the streets with impunity?
  That just doesn't seem realistic. That just doesn't sound like them.
\item
  ross douthat\\
  I mean, I think it's less the government of Portland per se. Portland,
  I think, is a distinctive case in the sense that they have had a kind
  of anarchist protest culture that the government has basically
  tolerated for a long time before the George Floyd protests. But no, I
  mean, I'm thinking more of cities like Atlanta, Chicago, Boston, New
  York and so on. Cities that have not been run the way Portland has
  been run for the last few years, to put it mildly, cities that right
  now have murder rates spiking that they need to bring under control.
  Those are the cities where Trump creates particular pressures not to
  be seen as cooperating with what liberalism has decided is a white
  nationalist president. Those are the kind of specific pressures that
  I'm thinking of --- Portland is a special case. I think that applies
  to other institutions too. I mean, obviously, I have a personal stake
  in this since I work for a institution that is widely reputed to be
  liberal and has had its own Trump era internal controversies. But just
  about every liberal institution right now, in higher education, in
  media, and so on, there is this pressure that Trump himself creates
  where everyone has to be onside against the great threat of Trump. And
  so all of those institutions in the Biden presidency, I think, would
  feel potentially some of that pressure relaxed. Not necessarily.
\item
  daniel mccarthy\\
  No, not at all. I mean, come on, you're saying that everything was
  going smoothly for conservatives in higher education, for example,
  until Donald Trump came along, and then the inflamed left suddenly
  decided to start acting out. That's not the way it's happened. In
  fact, conservatives have been losing ground in these institutions, the
  media and higher education in particular, for generations, and the
  fact that now it's being advertised in a way that it perhaps wasn't
  quite as much during the Obama years is not a change in the substance.
  The thing that really matters here is the sort of control of the
  institutions themselves and what kind of ethos they embody, whether or
  not that's being expressed as sort of flamboyantly as it is now, or
  whether it's deeper in the institutional power structure, as it was
  during Obama or during the Clinton years or whatever. But
  conservatives actually are in a better position knowing who their
  enemies are and knowing just how biased the media is against them,
  knowing just how ruthless higher education is in preventing
  conservatives from getting tenured positions and so forth. I think
  there's a great deal to be said for the sort of confrontation with
  reality that Donald Trump is bringing about. And it does seem to me
  that as you see crime rates spike by 24 percent in the 50 largest
  cities or across the country right now, you're going to see a
  significant change in public opinion, and there's a limit to how much
  demagoguing against Donald Trump is actually going to be effective
  before people say, wait a minute, the street crime that I'm seeing in
  New York or in Chicago is not being caused by Donald Trump. It's being
  caused by mayors and by city councils that are far too lenient on
  criminals.
\item
  ross douthat\\
  So let's talk about that scenario in a world where Trump loses before
  we wrap up. Helen, you you mentioned the idea that there exists after
  Trump potentially a more competent and serious form of populist
  conservatism waiting in the wings. Talk a little bit about that. Who
  leads the post-Trump Republican Party in 2024 or beyond?
\item
  helen andrews\\
  Yeah, I think that there are a lot of candidates, and there will be
  more candidates for that spot if Trump wins, which is another reason
  why I hope that that happens, because it will show that that's the
  path forward for victory for the GOP. It used to be my consolation
  when I thought about whether there was a chance that the G.O.P. would
  just revert back to the Romney-Ryan default that I thought, you know,
  they can't just do that, because if the G.O.P. goes back to the
  Romney-Ryan default, it will lose. If you go back to those policies,
  you're not going to win in Pennsylvania or Wisconsin. But as I've
  watched the G.O.P. over the last few years of the Trump presidency, I
  have realized that there are a lot of people who would sort of be fine
  with that, who would be OK if the G.O.P. went back to losing under
  that set of failed economic policies. You look at people in the never
  Trump movement, it seems like some of them would be very happy for the
  GOP to go back to losing. They really are the Washington generals of
  policy. So that's why I'm really hoping that Donald Trump wins in
  2020, to show that these ideas are the way forward for the party.
  Because if he shows that, then I think it matters less who actually
  picks up the standard, if it's somebody as bright and articulate as
  Senator Josh Hawley, or whether it's somebody else.
\item
  ross douthat\\
  Dan, your prophecies?
\item
  daniel mccarthy\\
  Yeah, I'll underscore what Helen has just said, in that there is a
  certain large component, actually, of the conservative movement which
  stands to gain if conservative leaders lose. It's notorious that
  political magazines, for example, get more subscriptions when the
  opposing party is in office. If you're writing fundraising copy for a
  think tank, it's much easier to write an attack on someone than it is
  to write a defense of some policy. So in a way the conservative
  movement as an institution is invested in failure, and we see the
  results of that. I think that's one reason why conservatism was in
  such a decrepit shape that someone with no political experience
  whatsoever like Donald Trump could come in and actually knock over all
  of these highly credentialed and highly articulate conservative
  leaders that he ran against in 2016. Now, this institutional
  corruption is something that hasn't gone away. As Helen mentions, you
  see it never Trump. You see it also in some of the opportunistic moves
  that people have made towards Trump in some cases. And whatever
  happens in 2020, this institutional problem is going to remain on the
  right. So in addition to seeing sort of a Trump plus, sort of someone
  with Trump's themes, but with a very competent, smooth execution in
  the future we also, I think, need to see conservative institutions
  that are geared towards achieving things in policy and in culture as
  opposed to simply fattening their pockets with panicked direct mail
  pitches about how the Democrats are bringing socialism back and all
  sorts of other things that rile up the 70 plus year old donors. But I
  would be very cautious, however, of naming who the sort of great
  populist or national conservative hope of the future is going to be in
  that nobody saw Donald Trump coming. I think Ross and I were both
  very, very surprised to see that someone with Donald Trump's
  background became not only the Republican nominee, but became
  president. And I think in the future, you may actually see more
  surprises like that, and that the old sense of political
  professionalism is decaying as a result of many changes taking place
  within the country, and also within the media and with the rise of
  social media. So maybe Kanye West is going to be the future. Maybe
  Tucker Carlson. But I'm a little bit skeptical of some of the
  politicians who become kind of flashy in the last few years, but still
  seem a bit untried and wet behind the ears.
\item
  ross douthat\\
  Well, so then let this be the last point that I press you both on. So
  one, I think --- I think, Dan, you're absolutely right about the
  extent to which there still exists this strong conservative
  infrastructure that just wants to fundraise against the threat of
  liberalism, doesn't want to govern, and would be perfectly happy in
  that sense with a Biden presidency, even if Biden himself is not the
  ideal foil. On the political side, though, I think if you brought
  Helen's great hope Josh Hawley onto our show and somehow hooked him up
  to a lie detector, which politicians sadly won't let you do, he would,
  I think, much rather run in 2024 if Donald Trump had lost in 2020. For
  a certain kind of politician, at least, I don't think victory is
  always better than defeat, because American politics moves in cycles,
  and you have openings and opportunities when the opposition party is
  in power that you don't have when you yourself--- when your own party
  is in power, especially if the politician leading your party is
  catastrophically unpopular. And I mean, this, to me, remains the
  conservative case for Trump losing, is that if you want Josh Hawley to
  win an election in 2024, or a figure like him, then he's more likely
  to do it running against an even more decrepit Biden or Biden's
  running mate in 2024 than he is as Trump's heir, and then especially
  --- and this is the last point I'll make --- especially if what Trump
  --- what Trump represents is not the sort of policies that he's
  embraced or half embraced, but a kind of ``own the libs'' celebrity
  culture, in which case, his likely heir is not Hawley. It's not even
  Tucker Carlson. It's someone like his own son, Don Jr., which just
  sort of propagates the side the cycle of maybe this guy can govern,
  no, actually, he can't, ever deeper into the future. So after that
  rant, I'll give you both the last word to explain once again why I'm
  wrong.
\item
  daniel mccarthy\\
  Well, first of all, I think that is extremely a very cynical
  statement, Ross. If you think that you know Josh Hawley has such a
  hollow core that he would rather advance his own personal prospects
  with your scenario for 2024 at the expense of seeing Joe Biden put a
  few people in Supreme Court, and at the expense of seeing what Joe
  Biden will do to foreign policy. That strikes me as a sacrifice ---
\item
  ross douthat\\
  I'll plead guilty. I'll plead guilty to that cynicism.
\item
  daniel mccarthy\\
  But look, if that's your view of Josh Hawley, that he's that cynical
  of a politician, then I don't want Josh Hawley to be the Republican
  nominee if that's who he is. I don't want a guy who is basically no
  different from a Mitt Romney, for example, who had a history of
  changing his positions and changing his ideological complexion based
  on what he thought would win. And in fact, I really dislike and will
  push back against one of the premises of this conversation, which is
  that popularity is sort of the most important thing, and the fact that
  polls show that Donald Trump is unpopular in a large part of the
  country is, therefore, the last word on Donald Trump. I don't think
  that's the case at all. It seems to me that Donald Trump, he's an
  imperfect tool, but he is actually trying to change things in a very
  dramatic way with respect to our economic policy, our foreign policy,
  and to see through the conservative promises on the Supreme Court. And
  those are going to be --- there's going to be some turbulence in that
  project. You're necessarily going to take some hits. Ronald Reagan was
  very unpopular in 1982. These are the sorts of headwinds you just have
  to confront and push through if you're ever actually going to achieve
  anything and be a success at the end of your years like Ronald Reagan
  was. And the idea that because you are unpopular at a given time means
  that you should just kind of give up and do something that's more
  popular is not only cynical, it's a recipe for losing. It's suicidal.
\item
  ross douthat\\
  I have some thoughts in response, but I'll just give Helen the last
  word for the show.
\item
  helen andrews\\
  {[}LAUGHS{]} Running to succeed a two-term president from the same
  party is always a challenge, but I don't think that's a reason for
  even the most cynical version of Josh Hawley to root for a Biden
  victory. Because after four years of Joe Biden's Supreme Court
  nominations, that's going to drastically limit what a president wholly
  can do. What the state of religious freedom precedents would be after
  those four years would probably leave his Supreme Court a lot less
  wiggle room to defend faithful Christians, or four years of Joe Biden
  foreign policy. That would drastically limit what a President Hawley
  could do in terms of furthering the populist agenda. So I stick by my
  cardinal principle of politics, which I think applies to everyone in
  both parties at all times, which is winning is always better than
  losing, and it is this time too.
\item
  ross douthat\\
  I think that's an excellent note on which to say, that's our show for
  this week. Thank you so much for listening. Dan, Helen, thank you so
  much for coming on.
\item
  daniel mccarthy\\
  Thanks, Ross.
\item
  helen andrews\\
  Thanks, Ross.
\item
  ross douthat\\
  You're very welcome. If you have a question you want to hear us debate
  in the future, share it with us in a voicemail by calling
  347-915-4324. You can also email us at
  \href{mailto:argument@nytimes.com}{\nolinkurl{argument@nytimes.com}}.
  ``The Argument'' is a production of The New York Times Opinion
  Section. Our team includes Phoebe Lett, Paula Szuchman, and Pedro
  Rafael Rosado. Special thanks to Brad Fisher and Kristin Lin. And
  don't worry, listeners, this is just a one-episode right-wing coup,
  and regular programming will resume next week. {[}MUSIC PLAYING{]}
\end{itemize}

Previous

More episodes ofThe Argument

\href{https://www.nytimes.com/2020/08/06/opinion/the-argument-trump-coronavirus-election.html?action=click\&module=audio-series-bar\&region=header\&pgtype=Article}{\includegraphics{https://static01.nyt.com/images/2020/08/06/opinion/06argSub/06argSub-thumbLarge.jpg}}

August 6, 2020Trump Supporters Make Their Case for 2020

\href{https://www.nytimes.com/2020/07/30/opinion/the-argument-authoritarianism-anne-applebaum.html?action=click\&module=audio-series-bar\&region=header\&pgtype=Article}{\includegraphics{https://static01.nyt.com/images/2020/07/31/opinion/30argumentWeb-print/30argumentWeb-thumbLarge.jpg}}

July 30, 2020When Conservatives Fall for Demagogues

\href{https://www.nytimes.com/2020/07/23/opinion/the-argument-israel-palestinian.html?action=click\&module=audio-series-bar\&region=header\&pgtype=Article}{\includegraphics{https://static01.nyt.com/images/2020/07/25/opinion/25audio/21argumentWeb-thumbLarge.jpg}}

July 23, 2020The Case for a One-State Solution

\href{https://www.nytimes.com/2020/07/16/opinion/the-argument-tammy-duckworth.html?action=click\&module=audio-series-bar\&region=header\&pgtype=Article}{\includegraphics{https://static01.nyt.com/images/2020/07/17/opinion/16argumentWeb-print/16argumentWeb-thumbLarge.jpg}}

July 16, 2020A Conversation With Tammy Duckworth

\href{https://www.nytimes.com/2020/07/09/opinion/is-trumps-fate-sealed.html?action=click\&module=audio-series-bar\&region=header\&pgtype=Article}{\includegraphics{https://static01.nyt.com/images/2020/07/10/opinion/10a2_audio/09argument1-thumbLarge.jpg}}

July 9, 2020Is Trump's Fate Sealed?

\href{https://www.nytimes.com/2020/07/02/opinion/the-argument-protest-statue-revolution.html?action=click\&module=audio-series-bar\&region=header\&pgtype=Article}{\includegraphics{https://static01.nyt.com/images/2020/07/05/opinion/02argument-eightyfive1/02argument-eightyfive1-thumbLarge.jpg}}

July 2, 2020Whose Statue Must Fall?

\href{https://www.nytimes.com/2020/06/25/opinion/the-argument-biden-vice-president-supreme-court.html?action=click\&module=audio-series-bar\&region=header\&pgtype=Article}{\includegraphics{https://static01.nyt.com/images/2020/06/28/opinion/25argument-eightyfour1/25argument-eightyfour1-thumbLarge.jpg}}

June 25, 2020Place Your Bets on Biden's V.P.

\href{https://www.nytimes.com/2020/06/18/opinion/the-argument-tom-cotton-resignation.html?action=click\&module=audio-series-bar\&region=header\&pgtype=Article}{\includegraphics{https://static01.nyt.com/images/2020/06/20/opinion/18argument-eightythree1/18argument-eightythree1-thumbLarge.jpg}}

June 18, 2020Which Opinions Are Out of Bounds?

\href{https://www.nytimes.com/2020/06/04/opinion/the-argument-protest-riot-violence.html?action=click\&module=audio-series-bar\&region=header\&pgtype=Article}{\includegraphics{https://static01.nyt.com/images/2020/06/04/opinion/04argument1/04argument1-thumbLarge.jpg}}

June 4, 2020Can Riots Force Change?

\href{https://www.nytimes.com/2020/05/28/opinion/the-argument-tara-reade-norma-mccorvey.html?action=click\&module=audio-series-bar\&region=header\&pgtype=Article}{\includegraphics{https://static01.nyt.com/images/2020/05/28/opinion/28argument1/28argument1-thumbLarge.jpg}}

May 28, 2020Credibility and Converts: Revisiting Tara Reade and Jane Roe

\href{https://www.nytimes.com/2020/05/21/opinion/the-argument-de-blasio-cuomo-coronavirus.html?action=click\&module=audio-series-bar\&region=header\&pgtype=Article}{\includegraphics{https://static01.nyt.com/images/2020/05/21/opinion/21argument-eighty1/21argument-eighty1-thumbLarge.jpg}}

May 21, 2020Did de Blasio Bungle the Crisis?

\href{https://www.nytimes.com/2020/05/14/opinion/the-argument-flynn-barr-coronavirus.html?action=click\&module=audio-series-bar\&region=header\&pgtype=Article}{\includegraphics{https://static01.nyt.com/images/2020/05/14/opinion/14arguement-seventynine1/14arguement-seventynine1-thumbLarge.jpg}}

May 14, 2020Bill Barr's Junk Justice

\href{https://www.nytimes.com/column/the-argument}{See All Episodes
ofThe Argument}

Next

Aug. 6, 2020

\begin{itemize}
\item
\item
\item
\item
\item
\end{itemize}

\emph{\textbf{Listen and subscribe to ``The Argument'' from your mobile
device:}}

\textbf{\href{https://itunes.apple.com/us/podcast/the-argument/id1438024613?mt=2}{\emph{Apple
Podcasts}}} \emph{\textbf{\textbar{}}}
\textbf{\href{https://open.spotify.com/show/6bmhSFLKtApYClEuSH8q42}{\emph{Spotify}}}
\emph{\textbf{\textbar{}}}
\textbf{\href{https://play.google.com/music/m/Idxib4hsg3yviao4gtym76knjjy?t=The_Argument}{\emph{Google
Play}}} \emph{\textbf{\textbar{}}}
\textbf{\href{https://radiopublic.com/the-argument-Wdbepr}{\emph{RadioPublic}}}
\emph{\textbf{\textbar{}}}
\textbf{\href{https://www.stitcher.com/podcast/the-new-york-times/the-argument}{\emph{Stitcher}}}
\emph{\textbf{\textbar{}}}
\textbf{\href{https://rss.art19.com/the-argument}{\emph{RSS Feed}}}

What do Trump supporters talk about when they talk about 2020? This week
on ``The Argument,'' Ross Douthat hosts a special intra-right debate
over whether conservatives should support Trump in 2020. He plays
``moderate squish'' (i.e., NeverTrumper) to Pro-Trump conservatives Dan
McCarthy, the editor of Modern Age, and Helen Andrews, a senior editor
of The American Conservative. They disagree with Ross about the
president's handling of the coronavirus and argue against his ultimate
question for Republicans in 2020: Should conservatives actually hope for
a Trump loss in November?

\includegraphics{https://static01.nyt.com/images/2020/08/06/opinion/06argSub/merlin_173801661_39539801-5ac0-4026-a458-7ae3610f6610-articleLarge.jpg?quality=75\&auto=webp\&disable=upscale}

\begin{center}\rule{0.5\linewidth}{\linethickness}\end{center}

\textbf{Background Reading:}

\begin{itemize}
\item
  Ross on how
  \href{https://www.nytimes.com/2020/08/04/opinion/trump-republicans-tea-party.html}{Republicans
  will pretend Trumpism never happened}
\item
  Daniel McCarthy's Op-Eds:
  ``\href{https://www.nytimes.com/2019/11/20/opinion/trump-impeachment.html}{Trump
  Is Doing Exactly What He Was Elected to Do}'' and
  ``\href{https://www.nytimes.com/2019/10/12/opinion/sunday/trump-impeachment-congress.html}{Trump's
  Politics Aren't Pretty. That's How His Voters Like It}''
\item
  Daniel McCarthy on ``The Argument'':
  ``\href{https://www.nytimes.com/2018/11/15/opinion/the-argument-amazon-headquarters-trump-republicans-cbd-gummies.html}{Is
  Amazon Bad for America?}''
\item
  ``The Argument'' episode,
  ``\href{https://www.nytimes.com/2019/08/15/opinion/the-argument-josh-hawley.html}{Is
  Josh Hawley the Future of the G.O.P.?}''
\end{itemize}

\begin{center}\rule{0.5\linewidth}{\linethickness}\end{center}

\textbf{How to listen to ``The Argument'':}

\emph{Press play or read the transcript (found above the center
teal-colored eye) at the top of this page, or tune in on}
\href{https://itunes.apple.com/us/podcast/the-argument/id1438024613?mt=2}{\emph{iTunes}}\emph{,}
\href{https://play.google.com/music/listen?u=0\#/ps/Idxib4hsg3yviao4gtym76knjjy}{\emph{Google
Play}}\emph{,}
\href{https://open.spotify.com/episode/5fIsHqqunLBwoxPSUUSGre?si=Rz5D9VnlRFKdGMu8ixzBOw}{\emph{Spotify}}\emph{,}
\href{https://www.stitcher.com/podcast/the-new-york-times/the-argument}{\emph{Stitcher}}
\emph{or your preferred podcast listening app. Tell us what you think
at} \href{mailto:argument@nytimes.com}{\emph{argument@nytimes.com.}}

\begin{center}\rule{0.5\linewidth}{\linethickness}\end{center}

\hypertarget{meet-the-hosts}{%
\section{Meet the Hosts}\label{meet-the-hosts}}

\hypertarget{ross-douthat}{%
\subsection{Ross Douthat}\label{ross-douthat}}

Image

I've been an Op-Ed columnist since 2009, and I write about politics,
religion, pop culture, sociology and the places where they intersect.
I'm a Catholic and a conservative, in that order, which means that I'm
against abortion and critical of the sexual revolution, but I tend to
agree with liberals that the Republican Party is too friendly to the
rich. I was against Donald Trump in 2016 for reasons specific to Donald
Trump, but in general I think the populist movements in Europe and
America have legitimate grievances, and I often prefer the populists to
the ``reasonable'' elites. I've written books about Harvard, the G.O.P.,
American Christianity and Pope Francis, and decadence. Benedict XVI was
my favorite pope. I review movies for National Review and have strong
opinions about many prestige television shows. I have four small
children, three girls and a boy, and live in New Haven with my wife.
\href{https://twitter.com/DouthatNYT}{\emph{@DouthatNYT}}

\begin{center}\rule{0.5\linewidth}{\linethickness}\end{center}

``The Argument'' is a production of The New York Times Opinion section.
The team includes Phoebe Lett, Paula Szuchman and Pedro Rafael Rosado.
Special thanks to Brad Fisher and Kristin Lin. Theme by Allison
Leyton-Brown.

Advertisement

\protect\hyperlink{after-bottom}{Continue reading the main story}

\hypertarget{site-index}{%
\subsection{Site Index}\label{site-index}}

\hypertarget{site-information-navigation}{%
\subsection{Site Information
Navigation}\label{site-information-navigation}}

\begin{itemize}
\tightlist
\item
  \href{https://help.nytimes.com/hc/en-us/articles/115014792127-Copyright-notice}{©~2020~The
  New York Times Company}
\end{itemize}

\begin{itemize}
\tightlist
\item
  \href{https://www.nytco.com/}{NYTCo}
\item
  \href{https://help.nytimes.com/hc/en-us/articles/115015385887-Contact-Us}{Contact
  Us}
\item
  \href{https://www.nytco.com/careers/}{Work with us}
\item
  \href{https://nytmediakit.com/}{Advertise}
\item
  \href{http://www.tbrandstudio.com/}{T Brand Studio}
\item
  \href{https://www.nytimes.com/privacy/cookie-policy\#how-do-i-manage-trackers}{Your
  Ad Choices}
\item
  \href{https://www.nytimes.com/privacy}{Privacy}
\item
  \href{https://help.nytimes.com/hc/en-us/articles/115014893428-Terms-of-service}{Terms
  of Service}
\item
  \href{https://help.nytimes.com/hc/en-us/articles/115014893968-Terms-of-sale}{Terms
  of Sale}
\item
  \href{https://spiderbites.nytimes.com}{Site Map}
\item
  \href{https://help.nytimes.com/hc/en-us}{Help}
\item
  \href{https://www.nytimes.com/subscription?campaignId=37WXW}{Subscriptions}
\end{itemize}
