\href{/section/arts/design}{Art \& Design}\textbar{}For Three
Suffragists, a Monument Well Past Due

\href{https://nyti.ms/3gBGdDh}{https://nyti.ms/3gBGdDh}

\begin{itemize}
\item
\item
\item
\item
\item
\item
\end{itemize}

\includegraphics{https://static01.nyt.com/images/2020/08/07/arts/06suffragist-statue1/merlin_175219929_4754839c-537e-4f93-91f4-7f93819ef33b-articleLarge.jpg?quality=75\&auto=webp\&disable=upscale}

Sections

\protect\hyperlink{site-content}{Skip to
content}\protect\hyperlink{site-index}{Skip to site index}

\hypertarget{for-three-suffragists-a-monument-well-past-due}{%
\section{For Three Suffragists, a Monument Well Past
Due}\label{for-three-suffragists-a-monument-well-past-due}}

Central Park will soon unveil its first sculpture depicting nonfictional
female figures. ``The fact that nobody even noticed that women were
missing in Central Park --- what does that say about the invisibility of
women?''

The sculptor Meredith Bergmann is putting the finishing touches on her
suffragist sculpture, slated to be presented in Central Park on Aug.
26.~It will be the park's first --- and so far only --- monument
honoring nonfictional women.Credit...Yael Malka for The New York Times

Supported by

\protect\hyperlink{after-sponsor}{Continue reading the main story}

\href{https://www.nytimes.com/by/alisha-haridasani-gupta}{\includegraphics{https://static01.nyt.com/images/2018/09/10/multimedia/author-alisha-haridasani-gupta/author-alisha-haridasani-gupta-thumbLarge-v3.png}}

By \href{https://www.nytimes.com/by/alisha-haridasani-gupta}{Alisha
Haridasani Gupta}

\begin{itemize}
\item
  Published Aug. 6, 2020Updated Aug. 7, 2020
\item
  \begin{itemize}
  \item
  \item
  \item
  \item
  \item
  \item
  \end{itemize}
\end{itemize}

Across the country, monuments honoring racist figures are being defaced
and toppled. In New York's Central Park, one statue is taking shape that
aims to amend not only racial but also gender disparities in public art:
A 14-foot-tall bronze monument of Susan B. Anthony, Sojourner Truth and
Elizabeth Cady Stanton, three of the more prominent leaders in the
nationwide fight for women's right to vote.

Called the Women's Rights Pioneers Monument, it is to be unveiled Aug.
26 to commemorate the 100th anniversary this month of the constitutional
amendment that finally guaranteed women that right. The sculpture
depicts the three figures gathered around a table for what seems to be a
discussion or a strategy meeting. Anthony stands in the middle, holding
a pamphlet that reads ``Votes for Women''; Stanton, seated to her left,
holds a pen, presumably taking notes; and Truth appears to be in
midsentence.

``I wanted to show women working together,'' said
\href{https://monumentalwomen.org/sculptors-page/}{Meredith Bergmann},
the sculptor chosen from dozens of artists to create the statue. ``I
kept thinking of women now, working together in some kitchen on a
laptop, trying to change the world.''

It will be the park's first --- and only --- monument honoring real
women, located on Literary Walk. In its 167-year history, the park has
been a leafy, lush home to about two dozen statues of men, mostly white,
and fictional or mythical female characters (Alice in Wonderland,
Shakespeare's Juliet, and
\href{https://www.nytimes.com/2019/05/29/obituaries/emma-stebbins-overlooked.html}{the
Angel of the Waters, the winged woman atop Bethesda Fountain}) but no
historical women.

\includegraphics{https://static01.nyt.com/images/2020/08/07/arts/06suffragist-statue3/06suffragist-statue3-articleLarge.jpg?quality=75\&auto=webp\&disable=upscale}

Image

Symbols of the herculean push for social change are sprinkled throughout
the work.Credit...Yael Malka for The New York Times

Image

A handbag with pamphlets and other reading material nods to the one
Susan B. Anthony carried with her around the country as she campaigned
for women's equality.Credit...Yael Malka for The New York Times

New York City as a whole hasn't been very inclusive either: Of the 150
statues honoring historical figures, only
\href{https://www.nytimes.com/2019/03/06/nyregion/women-statues-nyc.html}{five
depict women}, according to \href{https://women.nyc/she-built-nyc/}{She
Built NYC}, the city's official campaign, started last year, to increase
female representation in public art. And in 2011, just over seven
percent of the nearly 5,200 public outdoor statues across the country
represented women, according to the
\href{https://americanart.si.edu/research/inventories}{Smithsonian
American Art Museum's Art Inventories Catalog}.

``The fact that nobody, for a long time, even noticed that women were
missing in Central Park --- what does that say about the invisibility of
women?'' said Pam Elam, president of Monumental Women. ``There is a
responsibility to not only create a beautiful work of art but to have
that art reflect the reality of the lives of all the people who see
it.''

In 2014, a group of volunteers created Monumental Women (initially
called the Elizabeth Cady Stanton and Susan B. Anthony Statue Fund
Inc.), a nonprofit with a mission of campaigning and raising funds for
the suffragist statue in Central Park. Though the journey from concept
to creation ended up being a long and winding one, filled with
criticisms and setbacks.

Image

Down to the wire: The Public Design Commission approved a revamped
design in October, giving Meredith Bergmann less than a year to create
the new sculpture. We visited her recently at the foundry where she's
been working outside Newburgh, N.Y.Credit...Yael Malka for The New York
Times

Ms. Bergmann said it was ``pretty humbling'' to be making such a
monumental work, adding that every single creative decision was
carefully considered.

In the research phase, Ms. Bergmann, who in 2003 created the
\href{https://www.boston.gov/departments/womens-advancement/boston-womens-memorial}{Boston
Women's Memorial}, featuring Abigail Adams, Phillis Wheatley and Lucy
Stone, read a lot, she said, and spoke to Stanton's
great-great-granddaughter,
\href{https://www.nytimes.com/2020/07/02/style/woman-suffrage-movement-descend.html}{Coline
Jenkins-Sahlin}, for more insight.

She then spent months creating clay models of the monument, getting them
approved and then creating various different molds for the molten metal.

For their faces, she drew from multiple sources. ``I never copy a
photograph,'' she said, ``but I take all the photographs available and
study them and try to come up with a face that will express more than
one moment in the life of this person, with hints of their youthful
face, their old face, their angry face and their happy face.''

Their outfits carry Easter eggs --- symbols and clues that speak to the
social context or their personalities, Ms. Bergmann explained.
\href{https://www.nps.gov/articles/symbols-of-the-women-s-suffrage-movement.htm}{Sunflower
motifs} are carved into Stanton's dress because she had used the
pseudonym Sunflower when writing editorials for The Lily newspaper in
Seneca Falls, N.Y., Ms. Bergmann said. Anthony has a cameo around her
neck depicting Minerva --- the Roman goddess of strategy and wisdom.
Truth wears her signature shawl --- the tassels appear to be blowing in
the wind --- and a striped brocade jacket with laurel wreaths woven in
to symbolize victory and honor.

That they are all attired in long skirts and dresses is significant too.
In the late 19th and early 20th centuries, women fighting for social
reforms --- including Stanton --- adopted what came to be known as
\href{https://www.nytimes.com/1939/12/31/archives/bloomer-and-she-was-a-feminist-not-a-fashion-designee.html}{the
``Bloomer costume,}'' knee-length dresses worn over trousers, which
offered freedom and respite from the more constricting corsets and
floor-length dresses that were standard at the time.

``Stanton once said how wonderful it was to be able to climb a flight of
stairs holding a baby in one arm and a candle in the other without
having to hold up 10 pounds of wool skirt and petticoats,'' Bergmann
noted.

But the outfits were such radical departures from the norm that they
invited
\href{https://www.theatlantic.com/entertainment/archive/2019/06/american-suffragists-bloomers-pants-history/591484/}{intense
mockery} and distracted from broader conversations about women's rights,
so the suffrage fighters gave them up. Ms. Bergmann said this informed
her own choice to have the statues in voluminous skirts.

Image

A clay model of the Women's Rights Pioneers Monument.Credit...Michael
Bergmann

Though the campaign to install the statue took more than six years
(seven if you include the months of discussions that took place before
the nonprofit was formed), Monumental Women selected Ms. Bergmann's
design in 2018, giving the artist two years --- a short time in the
sculpting world, she noted --- to bring the suffragists to life.

The proposal that was approved consisted of Anthony and Stanton, and a
long scroll cascading from their work desk containing quotations from
more than 20 other suffragists.

``The initial commission was to create statues of these two women,'' Ms.
Bergmann explained, and the scroll, which included quotations from 11
women of color (including the educator
\href{https://plato.stanford.edu/entries/anna-julia-cooper/}{Anna Julia
Cooper} and the journalist
\href{https://www.nytimes.com/interactive/2018/obituaries/overlooked-ida-b-wells.html}{Ida
B. Wells}), was a way to also recognize the many other suffragists of
the movement.

The original callout for the commission noted that the sculpture should
``honor the memory of others, besides Stanton and Anthony, who helped
advance the cause of woman suffrage over the 72 -year battle.''

Image

Meredith Bergmann said that getting such a significant commission was
``pretty humbling,'' and that in approaching her work she tried to
``express more than one moment in the life of this
person.''Credit...Yael Malka for The New York Times

But when the city's Public Design Commission approved Ms. Bergmann's
design last March, it wanted her to nix the scroll and just focus on
Anthony and Stanton, Ms. Elam said. The design was also heavily
\href{https://www.nytimes.com/2019/01/17/nyregion/is-a-planned-monument-to-womens-rights-racist.html}{criticized
for placing only white women} on the pedestal --- essentially continuing
the
\href{https://www.nytimes.com/2019/05/14/opinion/central-park-suffrage-monument-racism.html}{erasure
of Black women's contributions to the suffrage movement}.

``Everything about this,'' Ms. Elam said, ``was not easy. It started
with the parks department, then it went to the Central Park Conservancy,
then the public design commission, then the Landmarks Preservation
Commission and all the community boards that surround Central Park. It
shouldn't have been so hard.''

Last August, in the wake of the controversy, Monumental Women shifted
gears and decided to include a third figure --- Truth, the
African-American abolitionist and suffragist. The commission approved
the new design in October, giving Ms. Bergmann less than a year to
create the reimagined sculpture. So far the group said they have raised
a total of \$1.5 million for the statue, including a combined \$135,000
in funding from the Manhattan borough president, Gale Brewer, and
Councilwoman Helen Rosenthal.

The city's Public Design Commission declined to comment for this
article.

When Monumental Women unveils the statue later this month (in a ceremony
at 8 a.m. on Aug. 26 that can be streamed at
\href{http://monumentalwomen.org/}{monumentalwomen.org}), the
organization said it plans to issue a challenge to municipalities all
over the country to include more women and people of color in their
public spaces. Part of the nonprofit's mission is to help other
communities navigate the kind of red tape and bureaucratic hurdles that
they encountered. The nonprofit will also kick off an online educational
campaign and has proposed providing books on women's history to all of
New York City's public school libraries.

``For the people who might think `OK, you've broken the bronze ceiling,
good for you, now your work is done' --- no, absolutely not, we are here
to stay,'' Ms. Elam said.

Advertisement

\protect\hyperlink{after-bottom}{Continue reading the main story}

\hypertarget{site-index}{%
\subsection{Site Index}\label{site-index}}

\hypertarget{site-information-navigation}{%
\subsection{Site Information
Navigation}\label{site-information-navigation}}

\begin{itemize}
\tightlist
\item
  \href{https://help.nytimes.com/hc/en-us/articles/115014792127-Copyright-notice}{©~2020~The
  New York Times Company}
\end{itemize}

\begin{itemize}
\tightlist
\item
  \href{https://www.nytco.com/}{NYTCo}
\item
  \href{https://help.nytimes.com/hc/en-us/articles/115015385887-Contact-Us}{Contact
  Us}
\item
  \href{https://www.nytco.com/careers/}{Work with us}
\item
  \href{https://nytmediakit.com/}{Advertise}
\item
  \href{http://www.tbrandstudio.com/}{T Brand Studio}
\item
  \href{https://www.nytimes.com/privacy/cookie-policy\#how-do-i-manage-trackers}{Your
  Ad Choices}
\item
  \href{https://www.nytimes.com/privacy}{Privacy}
\item
  \href{https://help.nytimes.com/hc/en-us/articles/115014893428-Terms-of-service}{Terms
  of Service}
\item
  \href{https://help.nytimes.com/hc/en-us/articles/115014893968-Terms-of-sale}{Terms
  of Sale}
\item
  \href{https://spiderbites.nytimes.com}{Site Map}
\item
  \href{https://help.nytimes.com/hc/en-us}{Help}
\item
  \href{https://www.nytimes.com/subscription?campaignId=37WXW}{Subscriptions}
\end{itemize}
