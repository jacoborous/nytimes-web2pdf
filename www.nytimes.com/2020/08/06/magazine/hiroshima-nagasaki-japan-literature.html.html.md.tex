Sections

SEARCH

\protect\hyperlink{site-content}{Skip to
content}\protect\hyperlink{site-index}{Skip to site index}

\href{https://myaccount.nytimes.com/auth/login?response_type=cookie\&client_id=vi}{}

\href{https://www.nytimes.com/section/todayspaper}{Today's Paper}

How We Retain the Memory of Japan's Atomic Bombings: Books

\href{https://nyti.ms/2PtRPMw}{https://nyti.ms/2PtRPMw}

\begin{itemize}
\item
\item
\item
\item
\item
\item
\end{itemize}

Advertisement

\protect\hyperlink{after-top}{Continue reading the main story}

Supported by

\protect\hyperlink{after-sponsor}{Continue reading the main story}

Beyond the World War II We Know

\hypertarget{how-we-retain-the-memory-of-japans-atomic-bombings-books}{%
\section{How We Retain the Memory of Japan's Atomic Bombings:
Books}\label{how-we-retain-the-memory-of-japans-atomic-bombings-books}}

Literature is a refuge we turn to when we are forced to confront
contradictions that lie beyond reason, writes the Japanese novelist Yoko
Ogawa.

\includegraphics{https://static01.nyt.com/images/2020/08/06/multimedia/06ww2-bombing-ogawa-01/merlin_175276872_dfcc7576-6b4a-46d4-9658-02bc79839fd2-articleLarge.jpg?quality=75\&auto=webp\&disable=upscale}

By Yoko Ogawa

\begin{itemize}
\item
  Aug. 6, 2020
\item
  \begin{itemize}
  \item
  \item
  \item
  \item
  \item
  \item
  \end{itemize}
\end{itemize}

\href{https://www.nytimes.com/ja/2020/08/06/magazine/atomic-bombings-japan-books-hiroshima-nagasaki.html}{日本語で読む}

\emph{\emph{\emph{In the latest article from
``}\href{https://www.nytimes.com/spotlight/beyond-wwii}{\emph{Beyond the
World War II We Know}}},'' a series by The Times that documents
lesser-known stories from the war, we asked Yoko Ogawa, an award-winning
Japanese author, to reflect on the literature unleashed by the atomic
bombings. This article was translated by Stephen Snyder.}**

The atomic bombing of Hiroshima occurred on Aug. 6. The bombing of
Nagasaki on Aug. 9. The announcement of surrender came on the 15th. In
Japan, August is the time when we remember the dead.

This year, the 75th anniversary of the atomic bombings would have been
observed during the Tokyo Olympics. But the Games
\href{https://www.nytimes.com/2020/03/24/sports/olympics/coronavirus-summer-olympics-postponed.html}{were
postponed} because of the spread of the novel coronavirus, and we will
be left instead to offer our prayers for the dead in an atmosphere of
unexpected calm.

The final torch bearer at the 1964 Tokyo Olympics was a relatively
unknown, 19-year-old, track and field competitor named
\href{https://www.japantimes.co.jp/news/2014/09/10/national/torch-runner-from-1964-tokyo-olympics-dies-at-69/}{Yoshinori
Sakai}, a young man who was born in Hiroshima on the day the bomb was
dropped. There was something extraordinary about the sight of him,
\href{https://www.worldathletics.org/news/iaaf-news/death-yoshinori-sakai-1964-olympic-games}{clad
simply in white shirt and shorts}, running up the long stairway that led
to the caldron he was meant to light. He embodied purity, a sense of
balance and an overwhelming youthfulness. Those who saw him must have
been amazed to realize that the world had gathered in Japan to celebrate
this festival of sport a mere 19 years after the end of the war. Yet
there he was, a young man born of unprecedented, total destruction, a
human being cradling a flame, advancing step by step. No doubt there
were political motivations behind the selection of the final runner, but
there was no questioning the hopeful life force personified by this
young man from Hiroshima.

Sadly, in the intervening years, we have failed to realize the dream of
a nuclear-free world. Even in Japan, the memories fade. According to
\href{https://www.nhk.or.jp/bunken/english/reports/summary/201511/01.html}{a
2015 survey} conducted by NHK, Japan's public broadcasting organization,
only 69 percent of the residents of Hiroshima and 50 percent of the
residents of Nagasaki could correctly name the month, day and year when
the Hiroshima bomb was dropped. At the national level, the rate fell to
30 percent. The cloud of oblivion rises, and the time is coming soon
when it will no longer be possible to hear directly from witnesses about
their experiences.

\includegraphics{https://static01.nyt.com/images/2020/08/06/multimedia/06ww2-bombing-ogawa-03/merlin_175232727_aa8d085c-fad1-4f77-827f-80c9b629b844-articleLarge.jpg?quality=75\&auto=webp\&disable=upscale}

So, what can those who have not seen with their own eyes do to preserve
the memories of those who have? How do we ensure that witnesses continue
to be heard? In the wake of unimaginable horrors --- endless wars, the
Holocaust, Chernobyl, Fukushima \ldots{} not to mention Hiroshima and
Nagasaki --- humankind has constantly confronted the problem of the
continuity of memory. How do we inscribe within us things that happened
long ago and far away that have no apparent connection to our lives, not
simply as learned knowledge but exactly as though we had experienced
them ourselves? How do we build a fragile bark to carry these memories
safely to the far shore, to the minds of the next generation? One thing
is certain: It is a task for which political and academic thinking and
institutions are poorly suited, quite simply because the act of sharing
the memories of another human being is fundamentally an irrational one.

So we appeal to the power of literature, a refuge we turn to when forced
to confront contradictions that lie beyond reason or theory. Through the
language of literature, we can finally come to empathize with the
suffering of nameless and unknown others. Or, at very least, we can
force ourselves to stare without flinching at the stupidity of those who
have committed unforgivable errors and ask ourselves whether the shadow
of this same folly lurks within us as well.

I myself have listened intently to the voices of those who lived during
the era of Nazi Germany, by reading and rereading Anne Frank's ``Diary
of a Young Girl,'' Victor Frankl's ``Man's Search for Meaning,'' and
Primo Levi's ``If This Is a Man.'' From Frank, for example, I learned
the invaluable truth that a human being can still grow and develop even
when living in hiding. From Frankl's observation that ``the best of us
did not return'' from the concentration camps, I learned to feel the
boundless suffering of those who survived and were forced to live on.
And when, through these books, the connection was made between my
existence here and now and that earlier time when I was not yet alive, I
could feel my horizons expanding, a new field of vision opening.

Likewise, Japanese literature continues to tell the story of the atomic
bombs. Bomb literature occupies a special place in every genre ---
fiction, poetry, drama, nonfiction. For example, anyone born in 1962, as
I was, would be familiar with Miyoko Matsutani's ``Two Little Girls
Called Iida,'' the story of a magical talking chair that unites two
girls across time in a house where the calendar is forever frozen on
Aug. 6. Or, with one of the indispensable works of modern Japanese
literature, Masuji Ibuse's ``Black Rain,'' with its excruciating account
of the aftermath of the bomb. Kenzaburo Oe, still in his 20s and barely
embarked on his literary career, visited Hiroshima and gave us
``Hiroshima Notes,'' his report on the extraordinary human dignity of
the bomb victims enduring the harsh reality of survivors. There is no
end to similar examples.

But there is one novel so admired and avidly read, even today, that it
is regularly included in school textbooks: Tamiki Hara's ``Summer
Flowers.'' A work by a bomb victim himself, it records the period and
experience in precise detail.

Born in Hiroshima in 1905, Hara had been living in Tokyo, contributing
fiction and poetry to literary magazines, when his wife died suddenly in
1944. In February 1945, he returned to his birthplace, exactly as though
he'd ``had a rendezvous with the tragedy that was coming to Hiroshima,''
as he later wrote. On the morning of Aug. 6, he was at home in his
windowless bathroom --- a fact that possibly saved his life. Fortunate
to have escaped serious injury, Hara spent the following days wandering
the burning city and recording his experiences in his notebook, a record
that later became ``Summer Flowers.''

Image

A Japanese edition of ``Summer Flowers'' by Tamiki Hara, a victim of the
bombing. It records the period and experience in precise
detail.Credit...Shinchosha

The novel begins two days before the bombing, as the protagonist pays a
visit to his wife's grave. He washes the stone and places summer flowers
on it, finding the sight cool and refreshing. But this opening passage
is haunted by sadness, a horrible premonition of the impossibility of
accounting for the loss of his beloved wife and the innumerable corpses
he will see a short time later.

The author's description of the protagonist as he flees to the river for
refuge is detailed and almost cold in tone. The language is concise, and
words that might express sentiment are nowhere to be found. Horrors of
the sort no human being had ever witnessed unfold one after the other
before the narrator's eyes, and he finds himself unable to express
anything as vague as mere emotion.

Faces so swollen that it was impossible to tell whether they were men or
women. Heads charred over with lumps like black beans. Voices crying out
again and again for water. Children clutching hands together as they
whispered faintly, ``Mother \ldots{} Father.'' People prying fingernails
from corpses or stripping off belts as keepsakes of the dead. **** The
narrator describes a city filled with the stench of death: ``In the
vast, silvery emptiness, there were roads and rivers and bridges, and
scattered here and there, raw and swollen corpses. A new hell, made real
through some elaborate technology.''

When the atomic bomb snatched away all things human, it might have
incinerated words themselves at the same time. Yet, led perhaps by the
hand of providence, he tucked a notebook and a pencil in with his food
and medicine. And what he wrote down in his notebook was not mere words.
He created a symbol for something he had heard from the dead and dying
that simply could not be expressed in words. Vestiges, scraps of
evidence that these human beings who had slipped mutely away had,
indeed, existed.

Having lost his wife to illness and then, in his solitude, encountered
the atomic bomb, Hara's creative work was constantly rooted in the
silence of the dead. He was a writer, a poet, who stood in the public
square, not to call out to his fellow man but to mutely endure the
contradiction of putting into words the voiceless voices of those whose
words had been taken from them.

Hara is the author of a short poem titled ``This Is a Human Being,'' a
work that transcends bitterness and anger, seeking to gently capture the
failing voice of someone who no longer appears human:

\begin{quote}
This is a human being.

See how the atom bomb has changed it.

The flesh is terribly bloated,

men and women all taking the same shape.

Ah! ``Help me!'' The quiet words of the voice that escapes

the swollen lips in the festering face.

This is a human being.

This is a human face.
\end{quote}

Reading it, we can't help being reminded of ``If This Is a Man,'' by
Primo Levi, chemist and concentration camp survivor. Right at the
outset, Levi poses the question:

\begin{quote}
Consider if this is a man

Who works in the mud

Who does not know peace

Who fights for a scrap of bread

Who dies because of a yes or a no
\end{quote}

I have no idea whether Levi and Hara were acquainted, but we can hear
the resonance between their words. One asks whether this is a human
being; the other answers that it is. In their work, we find the meeting
of one man who struggles to preserve the quality of humanity and another
who is determined not to lose sight of that same quality --- a meeting
of the minds that continues to reverberate into the future. In the world
of literature, the most important truth can be portrayed in a simple,
meaningless coincidence. With the help of literature, the words of the
dead may be gathered and placed carefully aboard their small boat, to
flow on to join the stream of reality.

Image

Hiroshima on Sept. 8, 1945, almost a month after the
bombing.~Credit...Wayne Miller/Magnum Photos

A further coincidence: perhaps with the sense that they had accomplished
their duty as survivors, or perhaps because the burden of living with
the horrors of their pasts was too great, the two men took their own
lives, Hara in 1951 and Levi in 1987 (some dispute that Levi's death was
a suicide).

As I write, I have in front of me
\href{https://ibashogallery.com/artists/76-hiromi-tsuchida/overview/}{Hiromi
Tsuchida's} collection of photographs of bomb artifacts offered by the
Hiroshima Peace Memorial Museum. I am struck by a picture of
\href{http://www.pcf.city.hiroshima.jp/virtual/VirtualMuseum_e/visit_e/vit_ex_e/vit_ex4_e.html}{a
lunchbox and canteen} that belonged to a middle school student named
Shigeru Orimen. His class had been mobilized for the war effort and was
working in the city on the morning of Aug. 6. Shigeru was 500 meters
from ground zero when the bomb fell. His mother discovered his body
among the corpses piled on the river bank and recovered the lunchbox and
canteen from his bag. She remembered he had left that morning saying how
much he was looking forward to lunch, since she had made roasted soybean
rice. The lunchbox was twisted out of shape, the lid cracked open, and
the contents were no more than a lump of charcoal.

But, in fact, this tiny box contained something more important: the
innocence of a young boy who had been full of anticipation for his
simple lunch, and his mother's love. Even when the last victim of the
atomic bomb has passed away and this lunchbox is no more than a
petrified relic, as long as there is still someone to hear the voice
concealed within it, this memory will survive. The voices of the dead
are eternal, because human beings possess the small boat --- the
language of literature --- to carry them to the future.

\begin{center}\rule{0.5\linewidth}{\linethickness}\end{center}

\textbf{Yoko Ogawa} is the author of numerous books, including ``The
Memory Police,'' a 2019 National Book Award finalist. Stephen Snyder is
the Dean of Language Schools and Kawashima Professor of Japanese Studies
at Middlebury College in Vermont.

Advertisement

\protect\hyperlink{after-bottom}{Continue reading the main story}

\hypertarget{site-index}{%
\subsection{Site Index}\label{site-index}}

\hypertarget{site-information-navigation}{%
\subsection{Site Information
Navigation}\label{site-information-navigation}}

\begin{itemize}
\tightlist
\item
  \href{https://help.nytimes.com/hc/en-us/articles/115014792127-Copyright-notice}{©~2020~The
  New York Times Company}
\end{itemize}

\begin{itemize}
\tightlist
\item
  \href{https://www.nytco.com/}{NYTCo}
\item
  \href{https://help.nytimes.com/hc/en-us/articles/115015385887-Contact-Us}{Contact
  Us}
\item
  \href{https://www.nytco.com/careers/}{Work with us}
\item
  \href{https://nytmediakit.com/}{Advertise}
\item
  \href{http://www.tbrandstudio.com/}{T Brand Studio}
\item
  \href{https://www.nytimes.com/privacy/cookie-policy\#how-do-i-manage-trackers}{Your
  Ad Choices}
\item
  \href{https://www.nytimes.com/privacy}{Privacy}
\item
  \href{https://help.nytimes.com/hc/en-us/articles/115014893428-Terms-of-service}{Terms
  of Service}
\item
  \href{https://help.nytimes.com/hc/en-us/articles/115014893968-Terms-of-sale}{Terms
  of Sale}
\item
  \href{https://spiderbites.nytimes.com}{Site Map}
\item
  \href{https://help.nytimes.com/hc/en-us}{Help}
\item
  \href{https://www.nytimes.com/subscription?campaignId=37WXW}{Subscriptions}
\end{itemize}
