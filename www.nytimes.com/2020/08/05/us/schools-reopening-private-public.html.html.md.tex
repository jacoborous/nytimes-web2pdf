Sections

SEARCH

\protect\hyperlink{site-content}{Skip to
content}\protect\hyperlink{site-index}{Skip to site index}

\href{https://www.nytimes.com/section/us}{U.S.}

\href{https://myaccount.nytimes.com/auth/login?response_type=cookie\&client_id=vi}{}

\href{https://www.nytimes.com/section/todayspaper}{Today's Paper}

\href{/section/us}{U.S.}\textbar{}If Public Schools Are Closed, Should
Private Schools Have to Follow?

\href{https://nyti.ms/39Y2Mj5}{https://nyti.ms/39Y2Mj5}

\begin{itemize}
\item
\item
\item
\item
\item
\item
\end{itemize}

\href{https://www.nytimes.com/news-event/coronavirus?action=click\&pgtype=Article\&state=default\&region=TOP_BANNER\&context=storylines_menu}{The
Coronavirus Outbreak}

\begin{itemize}
\tightlist
\item
  live\href{https://www.nytimes.com/2020/08/08/world/coronavirus-updates.html?action=click\&pgtype=Article\&state=default\&region=TOP_BANNER\&context=storylines_menu}{Latest
  Updates}
\item
  \href{https://www.nytimes.com/interactive/2020/us/coronavirus-us-cases.html?action=click\&pgtype=Article\&state=default\&region=TOP_BANNER\&context=storylines_menu}{Maps
  and Cases}
\item
  \href{https://www.nytimes.com/interactive/2020/science/coronavirus-vaccine-tracker.html?action=click\&pgtype=Article\&state=default\&region=TOP_BANNER\&context=storylines_menu}{Vaccine
  Tracker}
\item
  \href{https://www.nytimes.com/interactive/2020/world/coronavirus-tips-advice.html?action=click\&pgtype=Article\&state=default\&region=TOP_BANNER\&context=storylines_menu}{F.A.Q.}
\item
  \href{https://www.nytimes.com/live/2020/08/07/business/stock-market-today-coronavirus?action=click\&pgtype=Article\&state=default\&region=TOP_BANNER\&context=storylines_menu}{Markets
  \& Economy}
\end{itemize}

Advertisement

\protect\hyperlink{after-top}{Continue reading the main story}

Supported by

\protect\hyperlink{after-sponsor}{Continue reading the main story}

\hypertarget{if-public-schools-are-closed-should-private-schools-have-to-follow}{%
\section{If Public Schools Are Closed, Should Private Schools Have to
Follow?}\label{if-public-schools-are-closed-should-private-schools-have-to-follow}}

A dispute in Maryland over whether prestigious private schools can teach
in person during the coronavirus pandemic highlights a national divide.

\includegraphics{https://static01.nyt.com/images/2020/08/05/us/05VIRUS-SCHOOLS-philips/merlin_138525426_13d7ff76-2b94-4bfb-b84f-f6102769be0f-articleLarge.jpg?quality=75\&auto=webp\&disable=upscale}

By \href{https://www.nytimes.com/by/simon-romero}{Simon Romero}, Giulia
McDonnell Nieto del Rio and
\href{https://www.nytimes.com/by/patricia-mazzei}{Patricia Mazzei}

\begin{itemize}
\item
  Published Aug. 5, 2020Updated Aug. 7, 2020
\item
  \begin{itemize}
  \item
  \item
  \item
  \item
  \item
  \item
  \end{itemize}
\end{itemize}

Facing a resurgence of the coronavirus, public schools in the suburbs of
the nation's capital decided in recent weeks that more than a million
children would start the school year from home. On Friday, officials in
Maryland's most populous county said that private schools, including
some of the nation's most elite, had to join them.

Gov. Larry Hogan, a Republican, abruptly overruled that directive this
week, contending that Maryland's private schools should be allowed to
make their own reopening decisions. The governor staked out his position
on the same day that a group of parents filed a federal lawsuit seeking
to overturn the county's order, saying it discriminated against private
and religious schools.

The wrangling threw into sharp relief the challenges facing local health
officials as they piece together a response to the pandemic only to see
their efforts encounter political resistance and legal pushback.
Montgomery County officials
\href{https://www.baltimoresun.com/coronavirus/bs-md-elrich-hogan-schools-20200805-2k6mwqglknf7rhqfxecf2id64e-story.html}{tried
again on Wednesday}, issuing a new order to keep the schools closed that
cites a different source of authority under state law.

The dispute represents a contentious new front in the discussion over
inequality in American society, as some private and parochial schools
--- with their smaller class sizes, greater resources and influential
supporters --- find ways to move ahead with reopening plans that are
outside the grasp of public school systems.

``Parents in private schools are just generally more able to get their
preferences heard,'' said Christopher Lubienski, a professor of
education policy at Indiana University, adding that allowing private
schools to opt out of public health orders provided new evidence of how
schools in the United States were ``really efficient engines of
inequality.''

Mr. Hogan said on Monday that county health officers did not have the
authority to order private schools to teach online,
\href{https://twitter.com/GovLarryHogan/status/1290330304830246912/photo/1}{noting
in his statement} that school boards and superintendents have made
individual decisions on plans for reopening with the help of local
health officials. Private institutions, he said, should be allowed to do
the same.

``This had nothing to do with public health, and everything to do with
their own notions of fairness and equity,'' said Timothy Maloney, the
lawyer for parents suing the county health officer.

His clients include families whose children attend Our Lady of Mercy, a
Catholic school in Potomac, Md., which plans to offer in-person learning
options with a mask-wearing mandate and social distancing, among other
measures.

``The community was in an uproar,'' Mr. Maloney said. He noted that
private and Catholic schools had been closely following the state's
guidelines for safely reopening schools, and had invested millions of
dollars in retrofitting buildings.

Montgomery County is home to some of the most expensive and exclusive
schools in the country, including St. Andrew's Episcopal School in
Potomac, attended by President Trump's youngest son. St. Andrew's has
been preparing for scenarios that include online learning or a hybrid
model involving some instruction on campus.

Mr. Trump has inserted himself often into the debate over schools
reopening, threatening to withhold federal funds from those that do not
teach in person. ``Much of our Country is doing very well,''
\href{https://twitter.com/realDonaldTrump/status/1290257055534551043}{he
tweeted on Monday}. ``Open the Schools!''

About 90 percent of U.S. children attend public schools, which tend to
have less money and larger class sizes than private and parochial
schools, and less flexibility to make changes to their curriculum,
facilities or work force. Public schools in many places must also
negotiate with teachers' unions, many of which have
\href{https://www.nytimes.com/2020/07/29/us/teacher-union-school-reopening-coronavirus.html}{pushed
for their schools} to remain online or adopt more stringent health
measures.

``Public education is about leveling the playing field,'' said Pia
Morrison, president of the Service Employees International Union chapter
that represents some public school employees in Maryland and Washington.
But the pandemic has exacerbated the economic disparity between many
public and private school students, she said.

\includegraphics{https://static01.nyt.com/images/2020/08/05/us/05VIRUS-SCHOOLS-standrews/merlin_172035255_949df75a-239d-4150-a8dd-7b0bbe0a7f31-articleLarge.jpg?quality=75\&auto=webp\&disable=upscale}

Returning to school has already proven challenging, with some districts
that opened classrooms this week and last
\href{https://www.nytimes.com/2020/08/03/us/school-closing-coronavirus.html}{seeing
positive cases immediately} and having to quarantine students and staff
members, or even shut down temporarily. On Tuesday, the second day of
its school year, Cherokee County in Georgia
\href{https://www.ajc.com/education/cherokee-quarantines-second-grade-class-after-student-tests-positive-for-covid-19/OTD5MJKSFVFXFGPMHXUBG3INBQ/}{closed
a second-grade classroom} after a student tested positive for the virus.

\hypertarget{the-coronavirus-outbreak}{%
\subsubsection{The Coronavirus
Outbreak}\label{the-coronavirus-outbreak}}

\hypertarget{back-to-school}{%
\paragraph{Back to School}\label{back-to-school}}

Updated Aug. 8, 2020

The latest highlights as the first students return to U.S. schools.

\begin{itemize}
\item
  \begin{itemize}
  \tightlist
  \item
    Health experts say New York State schools are
    \href{https://www.nytimes.com/2020/08/07/health/coronavirus-ny-schools-reopen.html?action=click\&pgtype=Article\&state=default\&region=MAIN_CONTENT_2\&context=storylines_keepup}{in
    a good position to reopen}, and Gov. Andrew M. Cuomo has
    \href{https://www.nytimes.com/2020/08/07/nyregion/cuomo-schools-reopening.html?action=click\&pgtype=Article\&state=default\&region=MAIN_CONTENT_2\&context=storylines_keepup}{cleared
    the way}.
  \item
    Many schools spent the summer focused on reopening classrooms. What
    if they had
    \href{https://www.nytimes.com/2020/08/07/us/remote-learning-fall-2020.html?action=click\&pgtype=Article\&state=default\&region=MAIN_CONTENT_2\&context=storylines_keepup}{focused
    on improving remote learning} instead?
  \item
    A mother in Germany describes how her family
    \href{https://www.nytimes.com/2020/08/07/parenting/germany-schools-reopening-children.html?action=click\&pgtype=Article\&state=default\&region=MAIN_CONTENT_2\&context=storylines_keepup}{coped
    with the anxiety and uncertainty} of going back to school there.
  \item
    A high school freshman tested positive after two days in class. A
    yearbook editor worries about access to sporting events. We spoke to
    students about
    \href{https://www.nytimes.com/2020/08/06/us/coronavirus-students.html?action=click\&pgtype=Article\&state=default\&region=MAIN_CONTENT_2\&context=storylines_keepup}{what
    school is like in the age of Covid-19.}
  \end{itemize}
\end{itemize}

Schools in many parts of the United States face the near-certainty of
outbreaks because of the prevalence of the virus in their communities,
highlighting the tension between private school decisions and public
health directives.

In New Mexico, Albuquerque Academy, one of the most prestigious private
schools in the Southwest, developed an elaborate in-person reopening
plan that included shifting to a trimester system, installing portable
air filtration systems in every classroom and introducing touchless
water fountains.

Public schools in Albuquerque, however, opted to start the year online
as the state's coronavirus cases started climbing at a fast clip.

New Mexico's public education department does not have the authority to
tell private schools when to start classes. But Gov. Michelle Lujan
Grisham determined in July that private schools had to follow the same
public health orders that apply to other businesses in the state,
meaning they could operate only at 25 percent capacity.

Despite making several adjustments, Albuquerque Academy chose to start
the year with teachers working on campus and students taking classes
online; it will re-evaluate how things are going in several weeks.

``You need to abide by the public health order,'' said Julianne Puente,
the academy's head of school, emphasizing that she appreciated the clear
position from New Mexico's governor. ``You don't have to agree, but at a
time like this when there's clarity, at least then you know, this is the
structure.''

Several of the country's most elite boarding schools, including Phillips
Academy in Massachusetts, Choate Rosemary Hall in Connecticut, and
Phillips Exeter Academy and St. Paul's School in New Hampshire, say they
plan to reopen this fall. Those schools and others have described safety
protocols that include staggered returns to campus, reduced athletic
schedules and online classes to begin their terms.

In Florida, which is enduring some of the heaviest coronavirus caseloads
in the country, Dr. Mary Jo Trepka, chair of the epidemiology department
at Florida International University, said the decision by Miami-Dade
County Public Schools --- the nation's fourth-largest district --- to
put off opening in person until at least October was ``a really wise
move.''

Image

Jennifer Beller, the principal at Saint Mary's School, installing
plastic dividers in a kindergarten classroom. The Catholic school in
Moscow, Idaho, plans to open classrooms on Sept. 2.Credit...Geoff
Crimmins/The Moscow-Pullman Daily News, via Associated Press

The Archdiocese of Miami
\href{https://www.miamiarch.org/CatholicDiocese.php?op=Article_miami-archdiocesan-schools-to-start-aug-19-with-virtual-learning}{announced
last week} that its schools would also provide instruction online until
at least Sept. 18. And several of Miami's elite private schools said
this week that they, too, would keep teachers and students at home for
now.

But some charter schools plan to reopen. At a special meeting of the
Miami-Dade County Commission on Tuesday, Mayor Carlos Gimenez pressed
the county attorney about whether his administration would have
authority over public charter schools if they violated county rules
requiring masks and prohibiting large gatherings. The answer: probably
not.

Although the school district in Broward County, Fla., will also start
the year online, Pine Crest, a prestigious private school with campuses
in Fort Lauderdale and Boca Raton, will open on Aug. 19 with the option
for parents to send their children to classrooms. Pine Crest had the
resources to invest in equipment such as plexiglass dividers for desks,
hand-sanitizing stations for classrooms and buses, and an app for
students to screen their symptoms every morning.

In the Washington area, Georgetown Prep in North Bethesda, Md., had
planned for at least some in-person classes until Montgomery County's
order on Friday. In a letter to families, the school's president said on
Monday that it would consider the county's directive and Governor
Hogan's response and ``evaluate how best to proceed.''

Many private school decisions in the Washington area remain in flux,
just as they do across the country, said Amy McNamer, executive director
of the Association of Independent Schools of Greater Washington, which
has 76 members in the region.

``Right now, I have to tell you, it's a very stressful time to be a
school leader,'' Ms. McNamer said, adding that some private schools that
were planning two weeks ago for a hybrid opening have opted instead to
return to school virtually.

Still, Ms. McNamer acknowledged that independent schools enjoyed some
advantages, with the ability to make decisions based on the needs of a
smaller community, compared with the array of factors that public school
leaders have to consider.

``The comparison is perhaps, you know, the Titanic versus a small
sailboat,'' she said.

Colin Moynihan contributed reporting.

Advertisement

\protect\hyperlink{after-bottom}{Continue reading the main story}

\hypertarget{site-index}{%
\subsection{Site Index}\label{site-index}}

\hypertarget{site-information-navigation}{%
\subsection{Site Information
Navigation}\label{site-information-navigation}}

\begin{itemize}
\tightlist
\item
  \href{https://help.nytimes.com/hc/en-us/articles/115014792127-Copyright-notice}{©~2020~The
  New York Times Company}
\end{itemize}

\begin{itemize}
\tightlist
\item
  \href{https://www.nytco.com/}{NYTCo}
\item
  \href{https://help.nytimes.com/hc/en-us/articles/115015385887-Contact-Us}{Contact
  Us}
\item
  \href{https://www.nytco.com/careers/}{Work with us}
\item
  \href{https://nytmediakit.com/}{Advertise}
\item
  \href{http://www.tbrandstudio.com/}{T Brand Studio}
\item
  \href{https://www.nytimes.com/privacy/cookie-policy\#how-do-i-manage-trackers}{Your
  Ad Choices}
\item
  \href{https://www.nytimes.com/privacy}{Privacy}
\item
  \href{https://help.nytimes.com/hc/en-us/articles/115014893428-Terms-of-service}{Terms
  of Service}
\item
  \href{https://help.nytimes.com/hc/en-us/articles/115014893968-Terms-of-sale}{Terms
  of Sale}
\item
  \href{https://spiderbites.nytimes.com}{Site Map}
\item
  \href{https://help.nytimes.com/hc/en-us}{Help}
\item
  \href{https://www.nytimes.com/subscription?campaignId=37WXW}{Subscriptions}
\end{itemize}
