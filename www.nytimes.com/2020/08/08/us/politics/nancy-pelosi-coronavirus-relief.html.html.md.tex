Sections

SEARCH

\protect\hyperlink{site-content}{Skip to
content}\protect\hyperlink{site-index}{Skip to site index}

\href{https://www.nytimes.com/section/politics}{Politics}

\href{https://myaccount.nytimes.com/auth/login?response_type=cookie\&client_id=vi}{}

\href{https://www.nytimes.com/section/todayspaper}{Today's Paper}

\href{/section/politics}{Politics}\textbar{}Pelosi Is Playing Hardball
on Coronavirus Relief. She Thinks She'll Win.

\href{https://nyti.ms/3kseyXN}{https://nyti.ms/3kseyXN}

\begin{itemize}
\item
\item
\item
\item
\item
\end{itemize}

Advertisement

\protect\hyperlink{after-top}{Continue reading the main story}

Supported by

\protect\hyperlink{after-sponsor}{Continue reading the main story}

News analysis

\hypertarget{pelosi-is-playing-hardball-on-coronavirus-relief-she-thinks-shell-win}{%
\section{Pelosi Is Playing Hardball on Coronavirus Relief. She Thinks
She'll
Win.}\label{pelosi-is-playing-hardball-on-coronavirus-relief-she-thinks-shell-win}}

Emboldened by Republican divisions and a favorable political landscape,
the speaker is refusing to agree to a narrow relief measure, unbothered
by charges that she is an impediment to a deal.

\includegraphics{https://static01.nyt.com/images/2020/08/09/us/politics/09DC-Pelosi-print1/08DC-Pelosi1-articleLarge.jpg?quality=75\&auto=webp\&disable=upscale}

\href{https://www.nytimes.com/by/emily-cochrane}{\includegraphics{https://static01.nyt.com/images/2018/11/28/multimedia/author-emily-cochrane/author-emily-cochrane-thumbLarge-v3.png}}\href{https://www.nytimes.com/by/nicholas-fandos}{\includegraphics{https://static01.nyt.com/images/2018/11/06/multimedia/author-nicholas-fandos/author-nicholas-fandos-thumbLarge-v2.png}}

By \href{https://www.nytimes.com/by/emily-cochrane}{Emily Cochrane} and
\href{https://www.nytimes.com/by/nicholas-fandos}{Nicholas Fandos}

\begin{itemize}
\item
  Aug. 8, 2020Updated 7:54 p.m. ET
\item
  \begin{itemize}
  \item
  \item
  \item
  \item
  \item
  \end{itemize}
\end{itemize}

WASHINGTON --- As the clock ticked down Thursday on a self-imposed
deadline for a breakthrough in coronavirus relief talks with no deal in
sight, Jim Cramer, the brash CNBC host, had an on-air proposal for
Speaker Nancy Pelosi of California.

Why not try invoking the memory of
\href{https://www.nytimes.com/2020/07/17/us/john-lewis-dead.html}{the
late civil rights icon John Lewis} to try to persuade Republicans to
agree to help the most vulnerable Americans, including ``minorities''
struggling to weather a pandemic and a recession?

Ms. Pelosi flashed a forced smile. ``Perhaps,'' she deadpanned, ``you
mistook them for somebody who gives a damn for what you just
described.''

The comment --- unusually coarse for Ms. Pelosi, 80, who was educated by
nuns --- was part insult, part dare and part slogan for a woman who
believes she has the upper hand in crisis negotiations and does not
intend to lose it. And it reflected how, two weeks into stalled talks
over another round of federal assistance to prop up a battered economy,
and less than three months before Election Day, the speaker of the House
is going for the jugular.

She has publicly heaped disdain on her White House negotiating partners
as she plays hardball in daily private meetings in her Capitol office
suite, convinced that she has political leverage to force Republicans to
agree to far more generous aid than they have offered. She has been
unwilling to bow to the Trump administration's demands for a much
narrower bill or a stopgap solution.

``We're not doing short-term action, because if we do short-term action,
they're not going to do anything else,'' she said of Republicans Friday
afternoon during an interview in her office, after negotiators blew past
their own deadline without a deal. ``That's it --- like a sucker punch,
you know --- `Let us just do this little bit,' and then you know what?
We'll never see them again.''

Instead, Ms. Pelosi is pushing for a sweeping package that includes
billions of dollars for state and local governments and schools, food
and rental assistance, and additional aid for election security and the
Postal Service.

All the while, Ms. Pelosi has made it clear that she does not much trust
President Trump's advisers --- she has taken to asking negotiators to
turn over their electronic devices before entering sessions in her
office --- nor does she think highly of their ability to forge a
compromise. ``You've never done a deal,'' she has reminded Mark Meadows,
the White House chief of staff and former congressman, according to a
person familiar with the talks who described them on the condition of
anonymity.

Ms. Pelosi's strategy carries substantial political risk and real
collateral damage, at least in the short term. In holding out for a
sweeping relief package, Democrats have swatted away Republican pleas to
pass weeklong extensions of the expired
\href{https://www.nytimes.com/2020/08/08/business/economy/lost-unemployment-benefits.html}{\$600-per-week
in extra federal jobless pay} that millions of Americans have relied
upon, drawing Republican charges of obstruction.

The impasse prompted Mr. Trump
\href{https://www.nytimes.com/2020/08/07/us/politics/trump-congress-stimulus.html}{to
take unilateral action} on Saturday to provide relief on his own with a
series of executive actions --- though it remains unclear if he has the
legal authority to do so. And it has sown uneasiness even among some
rank-and-file Democrats, particularly those who represent politically
competitive districts and are eager to show voters their party is
capable of bipartisan compromise on pressing issues.

``We cannot let desperate Americans and small businesses be used as
pawns --- even in the face of a president and Senate majority leader who
appear incapable of empathy,'' said Representative Dean Phillips, a
first-term Democrat from Minnesota.

On a private conference call on Saturday, Representative Tom Malinowski
of New Jersey, another first-term Democrat, warned that a lack of an
agreement would prompt his voters to declare ``a pox on all our Houses.
Congress is broken. Washington is broken.''

``And that is great for challengers,'' he added, according to a person
familiar with the remarks, who spoke on the condition of anonymity.

Republicans have been far sharper in their criticism of her tactics,
blaming Ms. Pelosi for the lapse in jobless aid even though she included
a full extension of the payments in her May legislation, which
Republicans are trying to make deep cuts to.

``Speaker Pelosi has refused, again and again and again, to do what's
right for the country, to work together in a bipartisan way to come up
with a package to help provide relief in terms of Covid and the economic
crisis,'' Representative Liz Cheney, the No. 3 Republican, told Fox News
Radio last week.

\includegraphics{https://static01.nyt.com/images/2020/08/08/us/politics/08DC-Pelosi2/08DC-Pelosi2-articleLarge.jpg?quality=75\&auto=webp\&disable=upscale}

But Ms. Pelosi, in her second round as speaker and arguably as powerful
as she has ever been, has seen little reason to change course. Instead,
with public opinion she says is in favor of expansive government
intervention and polls showing Republicans up and down the ballot
sagging under the weight of Mr. Trump's coronavirus response, the
speaker and Democrats have been
\href{https://www.nytimes.com/2020/04/23/us/coronavirus-democrats-strategy.html}{emboldened
to press their advantage}.

``At the core of her negotiations are values, and that steers her
right,'' said Senator Chuck Schumer, Democrat of New York and the
minority leader. ``It's real. What she says out there, she says
inside.''

Ms. Pelosi's hand has been strengthened by the
\href{https://www.nytimes.com/2020/07/22/us/politics/coronavirus-stimulus.html}{divisions
among Republicans}, many of whom do not want to provide any additional
aid, meaning that the White House will need broad support from Democrats
to push through any stimulus plan.

Ms. Pelosi set the stage for the dynamic in May, when --- quick on the
heels of the enactment of nearly \$3 trillion in pandemic aid bills ---
she corralled the Democratic votes needed to approve an additional \$3.4
trillion in relief. Senate Republicans
\href{https://www.nytimes.com/2020/05/15/us/coronavirus-republicans-blowback-aid.html}{waited
until late last month} to unveil their own \$1 trillion plan, and Mr.
Trump has repeatedly undercut their position.

White House officials say it is Ms. Pelosi who has hamstrung the talks.

``It's interesting just to hear the comments from Senator Schumer and
Speaker Pelosi saying that they want a deal,'' Mr. Meadows declared on
Friday, after negotiations broke up with no resolution and Ms. Pelosi
addressed the news media. ``Their actions do not indicate the same
thing.''

Senator Marco Rubio, Republican of Florida, said Ms. Pelosi and
Democrats were motivated not by substantive policy differences, but by
politics. They ``still think it's politically beneficial for nothing to
happen,'' he said.

It is not the first time that Ms. Pelosi has found herself with
considerable leverage in a high-stakes negotiation with Republicans at a
time of crisis. During the financial meltdown of 2008, as Republicans
balked at
\href{http://archive.nytimes.com/www.nytimes.com/packages/html/national/200904_CREDITCRISIS/recipients.html}{a
\$700 billion bailout package} that George W. Bush's administration had
requested to stave off further financial ruin, Henry M. Paulson Jr.,
then the Treasury secretary, famously
\href{https://www.nytimes.com/2008/09/26/business/26bailout.html}{went
down on one knee} at the White House to beg Ms. Pelosi not to pull her
support from the plan.

``It's not me blowing this up. It's the Republicans,'' Ms. Pelosi told
him then, adding bitingly, ``I didn't know you were Catholic.''

This time, though, it has become progressively less clear whether Mr.
Trump --- who has been
\href{https://www.nytimes.com/2020/08/03/us/politics/congress-jobless-aid-talks-trump.html}{more
an irritant than an active participant} in the negotiations --- even
wants the deal that he needs Ms. Pelosi to deliver.

``Up and until now, she has rationally assumed there was some
self-interest on the part of Trump that would lead to a deal,'' said
former Representative Barney Frank, Democrat of Massachusetts, who
joined Ms. Pelosi that day at the White House in 2008. ``If, in fact,
that turns out not to be the case, you have a whole new ballgame to
think about.''

Though she acknowledges political differences with Mr. Bush, Ms. Pelosi
is far more blunt about her disdain for Mr. Trump, with whom she has
\href{https://www.nytimes.com/2019/10/18/us/politics/trump-pelosi-photo.html}{developed
a toxic relationship}.

``This president is the biggest failure in our history,'' she said on
Friday. ``I can't think of anybody worse.''

He appears to return the sentiment, referring again to Ms. Pelosi this
week as ``Crazy Nancy.''

While she said she has had productive negotiations with Steven Mnuchin,
the Treasury secretary --- so much so that Mr. Mnuchin has felt
compelled to privately answer complaints from Republicans that he has
given too much --- she is more skeptical of Mr. Meadows, who made his
name in Congress blowing up bipartisan deals from the right, not
constructing them. Talks have been ``less efficient'' than the
discussions that led to the first phases of pandemic relief, she said.

``Mark Meadows is in the room as an enforcer," she said, adding that she
was not sure whether ``he's a clone for the president, or the
president's a clone for him.''

Ms. Pelosi said she also questioned the overall approach of the
administration, comparing their negotiating tactics to ``Sophie's
Choice,'' a film in which a mother must choose which of her children to
send to their death.

At one point during one of the negotiations, Mr. Mnuchin had inquired
what WIC, a nutritional program specifically for women, infants and
children, was, according to a person familiar with the talks.

``On any given day, you might say, why am I even talking to these
people? They don't care,'' Ms. Pelosi said.

``But the fact is, we're there --- we have an opportunity to do
something.''

Luke Broadwater contributed reporting.

Advertisement

\protect\hyperlink{after-bottom}{Continue reading the main story}

\hypertarget{site-index}{%
\subsection{Site Index}\label{site-index}}

\hypertarget{site-information-navigation}{%
\subsection{Site Information
Navigation}\label{site-information-navigation}}

\begin{itemize}
\tightlist
\item
  \href{https://help.nytimes.com/hc/en-us/articles/115014792127-Copyright-notice}{©~2020~The
  New York Times Company}
\end{itemize}

\begin{itemize}
\tightlist
\item
  \href{https://www.nytco.com/}{NYTCo}
\item
  \href{https://help.nytimes.com/hc/en-us/articles/115015385887-Contact-Us}{Contact
  Us}
\item
  \href{https://www.nytco.com/careers/}{Work with us}
\item
  \href{https://nytmediakit.com/}{Advertise}
\item
  \href{http://www.tbrandstudio.com/}{T Brand Studio}
\item
  \href{https://www.nytimes.com/privacy/cookie-policy\#how-do-i-manage-trackers}{Your
  Ad Choices}
\item
  \href{https://www.nytimes.com/privacy}{Privacy}
\item
  \href{https://help.nytimes.com/hc/en-us/articles/115014893428-Terms-of-service}{Terms
  of Service}
\item
  \href{https://help.nytimes.com/hc/en-us/articles/115014893968-Terms-of-sale}{Terms
  of Sale}
\item
  \href{https://spiderbites.nytimes.com}{Site Map}
\item
  \href{https://help.nytimes.com/hc/en-us}{Help}
\item
  \href{https://www.nytimes.com/subscription?campaignId=37WXW}{Subscriptions}
\end{itemize}
