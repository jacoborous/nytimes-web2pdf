\href{/section/science}{Science}\textbar{}Your Hot-Weather Guide to
Coronavirus, Air-Conditioning and Airflow

\href{https://nyti.ms/33DTkjt}{https://nyti.ms/33DTkjt}

\begin{itemize}
\item
\item
\item
\item
\item
\end{itemize}

\href{https://www.nytimes.com/news-event/coronavirus?action=click\&pgtype=Article\&state=default\&region=TOP_BANNER\&context=storylines_menu}{The
Coronavirus Outbreak}

\begin{itemize}
\tightlist
\item
  live\href{https://www.nytimes.com/2020/08/08/world/coronavirus-updates.html?action=click\&pgtype=Article\&state=default\&region=TOP_BANNER\&context=storylines_menu}{Latest
  Updates}
\item
  \href{https://www.nytimes.com/interactive/2020/us/coronavirus-us-cases.html?action=click\&pgtype=Article\&state=default\&region=TOP_BANNER\&context=storylines_menu}{Maps
  and Cases}
\item
  \href{https://www.nytimes.com/interactive/2020/science/coronavirus-vaccine-tracker.html?action=click\&pgtype=Article\&state=default\&region=TOP_BANNER\&context=storylines_menu}{Vaccine
  Tracker}
\item
  \href{https://www.nytimes.com/interactive/2020/world/coronavirus-tips-advice.html?action=click\&pgtype=Article\&state=default\&region=TOP_BANNER\&context=storylines_menu}{F.A.Q.}
\item
  \href{https://www.nytimes.com/live/2020/08/07/business/stock-market-today-coronavirus?action=click\&pgtype=Article\&state=default\&region=TOP_BANNER\&context=storylines_menu}{Markets
  \& Economy}
\end{itemize}

\includegraphics{https://static01.nyt.com/images/2020/08/05/nyregion/05xp-virus-aircon/05xp-virus-aircon-articleLarge.jpg?quality=75\&auto=webp\&disable=upscale}

Sections

\protect\hyperlink{site-content}{Skip to
content}\protect\hyperlink{site-index}{Skip to site index}

\hypertarget{your-hot-weather-guide-to-coronavirus-air-conditioning-and-airflow}{%
\section{Your Hot-Weather Guide to Coronavirus, Air-Conditioning and
Airflow}\label{your-hot-weather-guide-to-coronavirus-air-conditioning-and-airflow}}

Indoor air is riskier than outdoor air. So what do you do if it's really
hot out?

Window air-conditioning units, like the ones in this building in
Brooklyn, are typically designed for comfort, not
health.Credit...Christina Horsten/Picture Alliance, via Getty Images

Supported by

\protect\hyperlink{after-sponsor}{Continue reading the main story}

By \href{https://www.nytimes.com/by/heather-murphy}{Heather Murphy}

\begin{itemize}
\item
  Aug. 8, 2020Updated 3:06 p.m. ET
\item
  \begin{itemize}
  \item
  \item
  \item
  \item
  \item
  \end{itemize}
\end{itemize}

Despite its critical role in our daily lives, air is not something most
of us spend a great deal of time thinking about. It's that easy to take
for granted. Unlike water, we don't need to fill up a cup to consume it.
If some escapes from the room, more will find its way back in, whether
we open the door or not.

``If you are comfortable, you ignore it,'' said
\href{https://www.ashrae.org/professional-development/learning-portal/instructor-led-training/ashrae-instructors/wade-h-conlan}{Wade
H. Conlan}, a mechanical engineer who evaluates ventilation systems on
behalf of Hanson Professional Services.

But like so many little luxuries we once took for granted, our days of
blissfully ignoring air may be numbered. Because a growing number of
scientists are convinced that a significant amount of coronavirus
transmission occurs through the air in indoor spaces, and that poor
ventilation magnifies the risk.

Not everyone has the ability or resources to make the changes to a home
or workplace to improve air circulation. But scientists and engineers
say that it's worth trying to understand the basics of how airflow works
--- in case there is a relatively easy tweak that could keep you a bit
safer.

\hypertarget{when-in-doubt-open-the-windows}{%
\subsection{When in doubt, open the
windows.}\label{when-in-doubt-open-the-windows}}

\hypertarget{and-remember-that-outdoor-air-is-good}{%
\subsubsection{And remember that outdoor air is
good.}\label{and-remember-that-outdoor-air-is-good}}

\includegraphics{https://static01.nyt.com/images/2020/07/08/multimedia/00xp-virus-aircon2/merlin_171835503_4a0c5665-902c-4aaf-8cf3-5a549b6e7f08-articleLarge.jpg?quality=75\&auto=webp\&disable=upscale}

The precise way that viral particles flow through a room when an
infected person talks, sings, exhales or eats is something that
scientists are continuing to investigate. Previous case studies have
shown that it's complicated. If there is one easy-to-understand
principle that aerosol scientists and engineers have come to agree on,
though, it's this: The more outdoor air coming into a room, the better
for dispersing that cloud of viral particles that might be lingering.
And one of the most reliable and cost-effective ways to get outdoor air
into a room is to open a window.

``If you don't know if the place is well ventilated, but you have the
ability to open a window I would do it,'' said
\href{https://www.colorado.edu/even/people/shelly-miller}{Shelly
Miller}, a professor of mechanical engineering at the University of
Colorado Boulder. That, she said, or get out quickly if you're swinging
by an indoor location with other people in it.

\hypertarget{latest-updates-the-coronavirus-outbreak}{%
\section{\texorpdfstring{\href{https://www.nytimes.com/2020/08/07/world/covid-19-news.html?action=click\&pgtype=Article\&state=default\&region=MAIN_CONTENT_1\&context=storylines_live_updates}{Latest
Updates: The Coronavirus
Outbreak}}{Latest Updates: The Coronavirus Outbreak}}\label{latest-updates-the-coronavirus-outbreak}}

Updated 2020-08-08T12:04:28.992Z

\begin{itemize}
\tightlist
\item
  \href{https://www.nytimes.com/2020/08/07/world/covid-19-news.html?action=click\&pgtype=Article\&state=default\&region=MAIN_CONTENT_1\&context=storylines_live_updates\#link-1f86d03a}{As
  the U.S. relief talks falter again, Trump says he is prepared to act
  on his own.}
\item
  \href{https://www.nytimes.com/2020/08/07/world/covid-19-news.html?action=click\&pgtype=Article\&state=default\&region=MAIN_CONTENT_1\&context=storylines_live_updates\#link-3f64a70a}{Cuomo
  says N.Y. schools can reopen in-person but leaves it up to districts
  to determine if, when and how.}
\item
  \href{https://www.nytimes.com/2020/08/07/world/covid-19-news.html?action=click\&pgtype=Article\&state=default\&region=MAIN_CONTENT_1\&context=storylines_live_updates\#link-14e70066}{Thousands
  of cases went unreported in California when a computer server failed.}
\end{itemize}

\href{https://www.nytimes.com/2020/08/07/world/covid-19-news.html?action=click\&pgtype=Article\&state=default\&region=MAIN_CONTENT_1\&context=storylines_live_updates}{See
more updates}

More live coverage:
\href{https://www.nytimes.com/live/2020/08/07/business/stock-market-today-coronavirus?action=click\&pgtype=Article\&state=default\&region=MAIN_CONTENT_1\&context=storylines_live_updates}{Markets}

The outdoor air that comes in will eventually replace the indoor air,
according to
\href{https://www.colorado.edu/chemistry/jose-luis-jimenez}{Jose-Luis
Jimenez}, an aerosol scientist at the University of Colorado Boulder.

``The more outside air you have, the more you dilute the virus,'' said
Dr. Jimenez, who was among the scientists and engineers who
\href{https://www.nytimes.com/2020/07/07/health/coronavirus-aerosols-who.html}{sent
a letter} that pushed the World Health Organization to acknowledge that
airborne transmission of the novel coronavirus is a threat in indoor
spaces.

If you want to speed up the flow of outdoor air into a room, you could
also take a box fan, place it in a window and blast it outward, Dr.
Jimenez said. When any amount of air leaves, that same amount of air
returns --- it's a fixed volume. Therefore, the fan should help pull in
the same amount of outdoor air.

\hypertarget{your-type-of-air-conditioner-matters}{%
\subsection{Your type of air-conditioner
matters.}\label{your-type-of-air-conditioner-matters}}

\hypertarget{some-pull-in-outdoor-air-others-simply-recirculate-indoor-air}{%
\subsubsection{Some pull in outdoor air. Others simply recirculate
indoor
air.}\label{some-pull-in-outdoor-air-others-simply-recirculate-indoor-air}}

If you have air-conditioning in your home, no one is saying that you
need to give up on it entirely. When it's sweltering out,
air-conditioning can be essential not only to help you function but also
to avoid heatstroke.

But if you are going to spend time in a cooled space with other people,
it may be worth understanding a bit more about the cool air you are
breathing. Basically, all air-conditioning falls into one of three
categories.

\begin{itemize}
\item
  The unit cools both indoor and outdoor air.
\item
  The unit cools and recirculates only indoor air.
\item
  The unit relies entirely on pulling in outdoor air. (These are
  uncommon outside hospitals and labs.)
\end{itemize}

Centralized-air systems, such as those common in office buildings, dorms
and some large apartment buildings, often fall in category one. Dr.
Jimenez and other building scientists involved in coronavirus prevention
are currently advising owners of businesses and buildings with category
one systems to adjust the ratio to pull in more outdoor air, an
enterprise that can be costly. Take a casino in Las Vegas, which is kept
cool enough to keep people gambling inside while it's 120 degrees
Fahrenheit outside. Cooling that hot outdoor air will be more expensive
than recirculating the already cool inside air. But given that keeping
customers healthy is also a priority, more are willing to revisit their
approach, Dr. Jimenez said.

Few of us have the ability to adjust our air-conditioning in this way.
Most window units sitting with their rears facing the outdoors, for
example, fall into category two. Instead of pulling in outdoor air, they
are dumping heat from the room outdoors, said
\href{https://iee.psu.edu/content/william-bahnfleth}{William Bahnfleth},
a professor of architectural engineering at Penn State's Institutes of
Energy and the Environment.

If you live alone, or with people you're sure aren't infectious, those
units are fine. But if you give in to throwing that birthday dinner for
your parents, or if your teenager has been less than strict about
staying home, it's worth remembering that ``any virus that's present
will be mixed in'' to the recirculating indoor air, Dr. Jimenez said.

And so, if you have to have people over, it may be preferable to revert
to rule one: When in doubt, open the windows. Or better yet, go outside.

\hypertarget{not-all-filters-are-equal}{%
\subsection{Not all filters are
equal.}\label{not-all-filters-are-equal}}

\hypertarget{but-a-good-filter-can-be-just-as-effective-as-pulling-in-outside-air}{%
\subsubsection{But a good filter can be just as effective as pulling in
outside
air.}\label{but-a-good-filter-can-be-just-as-effective-as-pulling-in-outside-air}}

So what do you do if you're stuck with a unit that primarily
recirculates indoor air and it's unrealistic to open the window? This is
where filters come in. The right filter is just as effective as pulling
in outside air, said Dr. Edward A. Nardell, a professor at Harvard
Medical School who has
\href{https://onlinelibrary.wiley.com/doi/full/10.1111/ina.12608}{written
about the role} that air-conditioning plays in spreading airborne
diseases.

Along with removing dust, pollen, cooking odors, tobacco smoke and
chemicals, filters can take viral particles from the air. Some filters
go directly in air-conditioning units and central air systems. Others
are designed to stand alone. MERV and HEPA are two widely trusted,
certified types.

\href{https://www.nytimes.com/news-event/coronavirus?action=click\&pgtype=Article\&state=default\&region=MAIN_CONTENT_3\&context=storylines_faq}{}

\hypertarget{the-coronavirus-outbreak-}{%
\subsubsection{The Coronavirus Outbreak
›}\label{the-coronavirus-outbreak-}}

\hypertarget{frequently-asked-questions}{%
\paragraph{Frequently Asked
Questions}\label{frequently-asked-questions}}

Updated August 6, 2020

\begin{itemize}
\item ~
  \hypertarget{why-are-bars-linked-to-outbreaks}{%
  \paragraph{Why are bars linked to
  outbreaks?}\label{why-are-bars-linked-to-outbreaks}}

  \begin{itemize}
  \tightlist
  \item
    Think about a bar. Alcohol is flowing. It can be loud, but it's
    definitely intimate, and you often need to lean in close to hear
    your friend. And strangers have way, way fewer reservations about
    coming up to people in a bar. That's sort of the point of a bar.
    Feeling good and close to strangers. It's no surprise, then, that
    \href{https://www.nytimes.com/2020/07/02/us/coronavirus-bars.html?action=click\&pgtype=Article\&state=default\&region=MAIN_CONTENT_3\&context=storylines_faq}{bars
    have been linked to outbreaks in several states.} Louisiana health
    officials have tied
    \href{https://www.nytimes.com/2020/06/22/us/new-coronavirus-phase.html?action=click\&pgtype=Article\&state=default\&region=MAIN_CONTENT_3\&context=storylines_faq}{at
    least 100 coronavirus cases} to bars in the Tigerland nightlife
    district in Baton Rouge. Minnesota has traced 328 recent cases to
    bars across the state.
    \href{https://www.boisestatepublicradio.org/post/bars-large-venues-close-ada-county-after-surge-coronavirus-prompts-rollback\#stream/0}{In
    Idaho}, health officials shut down bars in Ada County after
    reporting clusters of infections among young adults who had visited
    several bars in downtown Boise. Governors in
    \href{https://www.nytimes.com/2020/07/01/us/california-coronavirus-reopening.html?action=click\&pgtype=Article\&state=default\&region=MAIN_CONTENT_3\&context=storylines_faq}{California},
    \href{https://www.nytimes.com/2020/06/14/us/coronavirus-united-states.html?action=click\&pgtype=Article\&state=default\&region=MAIN_CONTENT_3\&context=storylines_faq}{Texas
    and Arizona}, where coronavirus cases are soaring, have ordered
    hundreds of newly reopened bars to shut down. Less than two weeks
    after Colorado's bars reopened at limited capacity, Gov. Jared Polis
    \href{https://www.denverpost.com/2020/06/30/colorado-bars-closed-coronavirus/}{ordered
    them to close}.
  \end{itemize}
\item ~
  \hypertarget{i-have-antibodies-am-i-now-immune}{%
  \paragraph{I have antibodies. Am I now
  immune?}\label{i-have-antibodies-am-i-now-immune}}

  \begin{itemize}
  \tightlist
  \item
    As of right now,
    \href{https://www.nytimes.com/2020/07/22/health/covid-antibodies-herd-immunity.html?action=click\&pgtype=Article\&state=default\&region=MAIN_CONTENT_3\&context=storylines_faq}{that
    seems likely, for at least several months.} There have been
    frightening accounts of people suffering what seems to be a second
    bout of Covid-19. But experts say these patients may have a
    drawn-out course of infection, with the virus taking a slow toll
    weeks to months after initial exposure. People infected with the
    coronavirus typically
    \href{https://www.nature.com/articles/s41586-020-2456-9}{produce}
    immune molecules called antibodies, which are
    \href{https://www.nytimes.com/2020/05/07/health/coronavirus-antibody-prevalence.html?action=click\&pgtype=Article\&state=default\&region=MAIN_CONTENT_3\&context=storylines_faq}{protective
    proteins made in response to an
    infection}\href{https://www.nytimes.com/2020/05/07/health/coronavirus-antibody-prevalence.html?action=click\&pgtype=Article\&state=default\&region=MAIN_CONTENT_3\&context=storylines_faq}{.
    These antibodies may} last in the body
    \href{https://www.nature.com/articles/s41591-020-0965-6}{only two to
    three months}, which may seem worrisome, but that's perfectly normal
    after an acute infection subsides, said Dr. Michael Mina, an
    immunologist at Harvard University. It may be possible to get the
    coronavirus again, but it's highly unlikely that it would be
    possible in a short window of time from initial infection or make
    people sicker the second time.
  \end{itemize}
\item ~
  \hypertarget{im-a-small-business-owner-can-i-get-relief}{%
  \paragraph{I'm a small-business owner. Can I get
  relief?}\label{im-a-small-business-owner-can-i-get-relief}}

  \begin{itemize}
  \tightlist
  \item
    The
    \href{https://www.nytimes.com/article/small-business-loans-stimulus-grants-freelancers-coronavirus.html?action=click\&pgtype=Article\&state=default\&region=MAIN_CONTENT_3\&context=storylines_faq}{stimulus
    bills enacted in March} offer help for the millions of American
    small businesses. Those eligible for aid are businesses and
    nonprofit organizations with fewer than 500 workers, including sole
    proprietorships, independent contractors and freelancers. Some
    larger companies in some industries are also eligible. The help
    being offered, which is being managed by the Small Business
    Administration, includes the Paycheck Protection Program and the
    Economic Injury Disaster Loan program. But lots of folks have
    \href{https://www.nytimes.com/interactive/2020/05/07/business/small-business-loans-coronavirus.html?action=click\&pgtype=Article\&state=default\&region=MAIN_CONTENT_3\&context=storylines_faq}{not
    yet seen payouts.} Even those who have received help are confused:
    The rules are draconian, and some are stuck sitting on
    \href{https://www.nytimes.com/2020/05/02/business/economy/loans-coronavirus-small-business.html?action=click\&pgtype=Article\&state=default\&region=MAIN_CONTENT_3\&context=storylines_faq}{money
    they don't know how to use.} Many small-business owners are getting
    less than they expected or
    \href{https://www.nytimes.com/2020/06/10/business/Small-business-loans-ppp.html?action=click\&pgtype=Article\&state=default\&region=MAIN_CONTENT_3\&context=storylines_faq}{not
    hearing anything at all.}
  \end{itemize}
\item ~
  \hypertarget{what-are-my-rights-if-i-am-worried-about-going-back-to-work}{%
  \paragraph{What are my rights if I am worried about going back to
  work?}\label{what-are-my-rights-if-i-am-worried-about-going-back-to-work}}

  \begin{itemize}
  \tightlist
  \item
    Employers have to provide
    \href{https://www.osha.gov/SLTC/covid-19/standards.html}{a safe
    workplace} with policies that protect everyone equally.
    \href{https://www.nytimes.com/article/coronavirus-money-unemployment.html?action=click\&pgtype=Article\&state=default\&region=MAIN_CONTENT_3\&context=storylines_faq}{And
    if one of your co-workers tests positive for the coronavirus, the
    C.D.C.} has said that
    \href{https://www.cdc.gov/coronavirus/2019-ncov/community/guidance-business-response.html}{employers
    should tell their employees} -\/- without giving you the sick
    employee's name -\/- that they may have been exposed to the virus.
  \end{itemize}
\item ~
  \hypertarget{what-is-school-going-to-look-like-in-september}{%
  \paragraph{What is school going to look like in
  September?}\label{what-is-school-going-to-look-like-in-september}}

  \begin{itemize}
  \tightlist
  \item
    It is unlikely that many schools will return to a normal schedule
    this fall, requiring the grind of
    \href{https://www.nytimes.com/2020/06/05/us/coronavirus-education-lost-learning.html?action=click\&pgtype=Article\&state=default\&region=MAIN_CONTENT_3\&context=storylines_faq}{online
    learning},
    \href{https://www.nytimes.com/2020/05/29/us/coronavirus-child-care-centers.html?action=click\&pgtype=Article\&state=default\&region=MAIN_CONTENT_3\&context=storylines_faq}{makeshift
    child care} and
    \href{https://www.nytimes.com/2020/06/03/business/economy/coronavirus-working-women.html?action=click\&pgtype=Article\&state=default\&region=MAIN_CONTENT_3\&context=storylines_faq}{stunted
    workdays} to continue. California's two largest public school
    districts --- Los Angeles and San Diego --- said on July 13, that
    \href{https://www.nytimes.com/2020/07/13/us/lausd-san-diego-school-reopening.html?action=click\&pgtype=Article\&state=default\&region=MAIN_CONTENT_3\&context=storylines_faq}{instruction
    will be remote-only in the fall}, citing concerns that surging
    coronavirus infections in their areas pose too dire a risk for
    students and teachers. Together, the two districts enroll some
    825,000 students. They are the largest in the country so far to
    abandon plans for even a partial physical return to classrooms when
    they reopen in August. For other districts, the solution won't be an
    all-or-nothing approach.
    \href{https://bioethics.jhu.edu/research-and-outreach/projects/eschool-initiative/school-policy-tracker/}{Many
    systems}, including the nation's largest, New York City, are
    devising
    \href{https://www.nytimes.com/2020/06/26/us/coronavirus-schools-reopen-fall.html?action=click\&pgtype=Article\&state=default\&region=MAIN_CONTENT_3\&context=storylines_faq}{hybrid
    plans} that involve spending some days in classrooms and other days
    online. There's no national policy on this yet, so check with your
    municipal school system regularly to see what is happening in your
    community.
  \end{itemize}
\end{itemize}

MERV filters are rated on how efficiently they remove particles in a
specific size range from the air.
\href{https://www.ashrae.org/about}{ASHRAE}, a professional society of
air-conditioning, heating and refrigerating engineers, recommends MERV
13 and above for filtering out the coronavirus, said Dr. Bahnfleth, who
leads the group's epidemic task force. It is what Dr. Bahnfleth has in
his own house. Any HEPA filter is even more efficient than the
highest-rated MERV filter, he added, so either should effectively
capture coronavirus particles.

Many central-air systems are designed to incorporate specialized
filters. But not all can handle the most advanced filters. Lower-rated
filters still could be helpful, Dr. Conlan said --- it's not that they
won't ever catch smaller particles; they just won't do it as frequently.
Window units are typically designed for comfort, not health, and have
even more filter limitations.

For those who can afford them --- or push their employers or landlords
to buy them --- a stand-alone HEPA filter is a good option Dr. Bahnfleth
said. Some are designed for bigger spaces than others. The key, Dr.
Jimenez added, is picking one that will filter all the air in the room
at least twice an hour.

Be aware that if an air-filtration system sounds too good to be true,
your instincts may be right. Some of them appear to rely on questionable
marketing and science, Dr. Jimenez said.

\hypertarget{theres-no-good-spot-in-a-room}{%
\subsection{There's no `good spot' in a
room.}\label{theres-no-good-spot-in-a-room}}

\hypertarget{instead-keep-your-distance-wear-a-mask-get-out-quickly-if-you-can}{%
\subsubsection{Instead, keep your distance, wear a mask, get out quickly
if you
can.}\label{instead-keep-your-distance-wear-a-mask-get-out-quickly-if-you-can}}

Now that you're an air expert, it may be tempting to think that you know
how to pick the safest position in a restaurant or other indoor space
you might find you have a reason to be in.

But even experts cannot easily eyeball the lowest risk location, said
Andrew Persily, who oversaw the development of an
\href{https://www.nist.gov/services-resources/software/fatima}{online
tool} to estimate exposure to infectious aerosols in rooms and buildings
as chief of the Energy \& Environment Division at the National Institute
of Standards and Technology.

``Depending on the airflow pattern and where the aerosols are released,
there may be regions in the room that result in higher exposure than
others,'' he said. ``It's tough to predict.''

It's also hard to gauge how many is too many people in a given space.
After all it only takes one infected person to get other people sick. If
you have a \emph{c}arbon dioxide detector, you could try a technique
previously used to manage the spread of tuberculosis, and use that to
tip you off, Dr. Miller suggests. If CO₂ levels are above 1,000 parts
per one million, you'd be wise to decrease the number of people in the
indoor space, increase the amount of outdoor air or both, she says.

An alternate approach is to look around. Do you see other people? If so,
leave.

Advertisement

\protect\hyperlink{after-bottom}{Continue reading the main story}

\hypertarget{site-index}{%
\subsection{Site Index}\label{site-index}}

\hypertarget{site-information-navigation}{%
\subsection{Site Information
Navigation}\label{site-information-navigation}}

\begin{itemize}
\tightlist
\item
  \href{https://help.nytimes.com/hc/en-us/articles/115014792127-Copyright-notice}{©~2020~The
  New York Times Company}
\end{itemize}

\begin{itemize}
\tightlist
\item
  \href{https://www.nytco.com/}{NYTCo}
\item
  \href{https://help.nytimes.com/hc/en-us/articles/115015385887-Contact-Us}{Contact
  Us}
\item
  \href{https://www.nytco.com/careers/}{Work with us}
\item
  \href{https://nytmediakit.com/}{Advertise}
\item
  \href{http://www.tbrandstudio.com/}{T Brand Studio}
\item
  \href{https://www.nytimes.com/privacy/cookie-policy\#how-do-i-manage-trackers}{Your
  Ad Choices}
\item
  \href{https://www.nytimes.com/privacy}{Privacy}
\item
  \href{https://help.nytimes.com/hc/en-us/articles/115014893428-Terms-of-service}{Terms
  of Service}
\item
  \href{https://help.nytimes.com/hc/en-us/articles/115014893968-Terms-of-sale}{Terms
  of Sale}
\item
  \href{https://spiderbites.nytimes.com}{Site Map}
\item
  \href{https://help.nytimes.com/hc/en-us}{Help}
\item
  \href{https://www.nytimes.com/subscription?campaignId=37WXW}{Subscriptions}
\end{itemize}
