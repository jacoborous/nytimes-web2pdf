Sections

SEARCH

\protect\hyperlink{site-content}{Skip to
content}\protect\hyperlink{site-index}{Skip to site index}

\href{https://www.nytimes.com/section/business}{Business}

\href{https://myaccount.nytimes.com/auth/login?response_type=cookie\&client_id=vi}{}

\href{https://www.nytimes.com/section/todayspaper}{Today's Paper}

\href{/section/business}{Business}\textbar{}Google Faces European
Inquiry Into Fitbit Acquisition

\url{https://nyti.ms/2XpxOek}

\begin{itemize}
\item
\item
\item
\item
\item
\end{itemize}

Advertisement

\protect\hyperlink{after-top}{Continue reading the main story}

Supported by

\protect\hyperlink{after-sponsor}{Continue reading the main story}

\hypertarget{google-faces-european-inquiry-into-fitbit-acquisition}{%
\section{Google Faces European Inquiry Into Fitbit
Acquisition}\label{google-faces-european-inquiry-into-fitbit-acquisition}}

Authorities are investigating how Google will use health and wellness
data collected from Fitbit's fitness tracking devices.

By \href{https://www.nytimes.com/by/adam-satariano}{Adam Satariano}

\begin{itemize}
\item
  Aug. 4, 2020
\item
  \begin{itemize}
  \item
  \item
  \item
  \item
  \item
  \end{itemize}
\end{itemize}

LONDON --- European Union authorities on Tuesday announced an
investigation into
\href{https://www.nytimes.com/2019/11/01/technology/google-fitbit.html}{Google's
\$2.1 billion purchase} of the fitness-tracking company Fitbit, raising
alarms about the health data the internet giant would be acquiring as
part of the deal.

The inquiry shows the increased scrutiny Google and other large
technology companies are facing from regulators in Europe and the United
States about their growing dominance of the digital economy. Officials
have raised concerns that the biggest tech platforms buy smaller
companies to solidify their dominance and limit competition.

Margrethe Vestager, the European Commission's top antitrust regulator,
said a preliminary investigation of the Fitbit deal had raised concerns
about how Google would use data collected from Fitbit for its online
advertising services, a market where Google is already dominant. The
health and fitness data could be used to more narrowly target ads, she
said.

``By increasing the data advantage of Google in the personalization of
the ads it serves via its search engine and displays on other internet
pages, it would be more difficult for rivals to match Google's online
advertising services,'' the commission said in a
\href{https://ec.europa.eu/commission/presscorner/detail/en/ip_20_1446}{statement
announcing the investigation}.

The commission, the executive body of the European Union, said the
investigation would be completed by Dec. 9.

Google defended the acquisition, saying it competes with companies like
Apple, Samsung and Garmin that offer fitness tracking devices.

``This deal is about devices, not data,'' Rick Osterloh, senior vice
president for devices and services,
\href{https://blog.google/around-the-globe/google-europe/update-fitbit/}{said
in a blog post}. The company said it would not use Fitbit health and
wellness data for advertising services and offered to make a legally
binding commitment to the commission to limit its use of the data.

Google said last year that it was buying Fitbit to gain a foothold in
the market for wearable devices. One of the earliest companies in the
segment, Fitbit helped popularize the goal of logging 10,000 steps a
day. More recently, the San Francisco-based company has faced stiff
competition from Apple and other makers of so-called smartwatches that
blend some of the functionality of a smartphone with tracking of fitness
activity.

Acquiring Fitbit would give Google another brand of hardware products.
In recent years, the tech giant has introduced a series of Pixel
smartphones and home appliances like its Nest thermostats and security
cameras. On Tuesday, Google also announced that it was buying a \$450
million stake in the
\href{https://www.globenewswire.com/news-release/2020/08/03/2071540/0/en/ADT-and-Google-Partner-To-Create-Leading-Smart-Home-Security-Offering.html}{home
alarm company ADT}.

The Fitbit deal had been expected to face government scrutiny as
regulators look more closely at tech-industry acquisitions. In
hindsight, many regulators view approval of past deals like Facebook's
acquisitions of Instagram and WhatsApp, or Google's purchase of the
online advertising platform DoubleClick, as having undermined
competition in the market.

The European investigation adds to the regulatory challenges facing
Google.

Last week, Sundar Pichai, the chief executive of Google's parent
company, Alphabet, was grilled by members of Congress over the company's
business practices, along with the top executives from Amazon, Apple and
Facebook. Google is facing
\href{https://www.nytimes.com/2020/06/25/technology/barr-google-investigation.html}{a
possible antitrust suit} from the Justice Department, along with an
investigation from a collection of state attorneys general.

Google has long been a target of the European Union. From 2017 to 2019,
Ms. Vestager's office issued fines totaling roughly 8.25 billion euros,
or about \$9.7 billion, in three separate cases related to its online
shopping service, Android mobile software and online advertising
business. The penalties are now under appeal.

The European Consumer Organization, a Brussels-based group pushing for
more oversight of the tech industry, cheered the Fitbit investigation.

``This takeover is likely to be a worrying game changer not only for how
consumers interact with the online world but also for how their health
data is used,'' Monique Goyens, the group's director general, said in a
statement. ``It is hugely important that the E.U. carries out this
in-depth examination because wearable devices like Fitbit's could in
future give companies details of essentially everything consumers do
24/7.''

Advertisement

\protect\hyperlink{after-bottom}{Continue reading the main story}

\hypertarget{site-index}{%
\subsection{Site Index}\label{site-index}}

\hypertarget{site-information-navigation}{%
\subsection{Site Information
Navigation}\label{site-information-navigation}}

\begin{itemize}
\tightlist
\item
  \href{https://help.nytimes.com/hc/en-us/articles/115014792127-Copyright-notice}{©~2020~The
  New York Times Company}
\end{itemize}

\begin{itemize}
\tightlist
\item
  \href{https://www.nytco.com/}{NYTCo}
\item
  \href{https://help.nytimes.com/hc/en-us/articles/115015385887-Contact-Us}{Contact
  Us}
\item
  \href{https://www.nytco.com/careers/}{Work with us}
\item
  \href{https://nytmediakit.com/}{Advertise}
\item
  \href{http://www.tbrandstudio.com/}{T Brand Studio}
\item
  \href{https://www.nytimes.com/privacy/cookie-policy\#how-do-i-manage-trackers}{Your
  Ad Choices}
\item
  \href{https://www.nytimes.com/privacy}{Privacy}
\item
  \href{https://help.nytimes.com/hc/en-us/articles/115014893428-Terms-of-service}{Terms
  of Service}
\item
  \href{https://help.nytimes.com/hc/en-us/articles/115014893968-Terms-of-sale}{Terms
  of Sale}
\item
  \href{https://spiderbites.nytimes.com}{Site Map}
\item
  \href{https://help.nytimes.com/hc/en-us}{Help}
\item
  \href{https://www.nytimes.com/subscription?campaignId=37WXW}{Subscriptions}
\end{itemize}
