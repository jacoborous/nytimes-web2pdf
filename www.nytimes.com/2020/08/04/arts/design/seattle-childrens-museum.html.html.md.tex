Sections

SEARCH

\protect\hyperlink{site-content}{Skip to
content}\protect\hyperlink{site-index}{Skip to site index}

\href{https://www.nytimes.com/section/arts/design}{Art \& Design}

\href{https://myaccount.nytimes.com/auth/login?response_type=cookie\&client_id=vi}{}

\href{https://www.nytimes.com/section/todayspaper}{Today's Paper}

\href{/section/arts/design}{Art \& Design}\textbar{}Turmoil After a
Museum Deletes `Black Lives Matter' From Postings

\url{https://nyti.ms/2XtNYU7}

\begin{itemize}
\item
\item
\item
\item
\item
\end{itemize}

\href{https://www.nytimes.com/news-event/george-floyd-protests-minneapolis-new-york-los-angeles?action=click\&pgtype=Article\&state=default\&region=TOP_BANNER\&context=storylines_menu}{Race
and America}

\begin{itemize}
\tightlist
\item
  \href{https://www.nytimes.com/interactive/2020/07/03/us/george-floyd-protests-crowd-size.html?action=click\&pgtype=Article\&state=default\&region=TOP_BANNER\&context=storylines_menu}{Black
  Lives Matter Movement}
\item
  \href{https://www.nytimes.com/interactive/2020/06/28/us/i-cant-breathe-police-arrest.html?action=click\&pgtype=Article\&state=default\&region=TOP_BANNER\&context=storylines_menu}{History
  of `I Can't Breathe'}
\item
  \href{https://www.nytimes.com/interactive/2020/06/10/upshot/black-lives-matter-attitudes.html?action=click\&pgtype=Article\&state=default\&region=TOP_BANNER\&context=storylines_menu}{How
  Public Opinion Shifted}
\item
  \href{https://www.nytimes.com/interactive/2020/07/16/us/black-lives-matter-protests-louisville-breonna-taylor.html?action=click\&pgtype=Article\&state=default\&region=TOP_BANNER\&context=storylines_menu}{45
  Days in Louisville}
\end{itemize}

Advertisement

\protect\hyperlink{after-top}{Continue reading the main story}

Supported by

\protect\hyperlink{after-sponsor}{Continue reading the main story}

\hypertarget{turmoil-after-a-museum-deletes-black-lives-matter-from-postings}{%
\section{Turmoil After a Museum Deletes `Black Lives Matter' From
Postings}\label{turmoil-after-a-museum-deletes-black-lives-matter-from-postings}}

The director of the Seattle Children's Museum faced a strike and an
internal inquiry after she edited staff postings, citing fund-raising
and other concerns.

\includegraphics{https://static01.nyt.com/images/2020/08/05/arts/04pulldown3/04pulldown3-articleLarge.jpg?quality=75\&auto=webp\&disable=upscale}

By \href{https://www.nytimes.com/by/julia-jacobs}{Julia Jacobs}

\begin{itemize}
\item
  Aug. 4, 2020
\item
  \begin{itemize}
  \item
  \item
  \item
  \item
  \item
  \end{itemize}
\end{itemize}

In the wake of the police killing of George Floyd and the protests that
followed, institutions of every kind worked to figure out what they
wanted to say.

What sort of public statement should a shoe company release about racial
injustice? How about a university? A theater?

At the Seattle Children's Museum, staff members decided to post lists of
children's books online that were anti-racist in their messaging and
featured joyful stories about Black children and their families.

Another social media post featured a museum program where children
create their own ``support signs,'' not unlike the signs that activists
bring to demonstrations, but typically softer. One declared ``I love
everything,'' with drawings of heart-shaped balloons and peace signs.

All of the posts started with a declaration: Black Lives Matter.

Until they didn't.

Hours after the postings on Instagram and Facebook on May 30, all
mentions of the phrase ``Black Lives Matter'' had been edited out of the
captions.

The museum's executive director explained her rationale for the
deletions a couple of days later on a staff video call that participants
taped. Christi Stapleton Keith, the director, said she personally
believed in the message of Black Lives Matter but the institution had a
process and needed to create a message ``that the museum could all agree
on as an organization.''``And what happens'' she went on, ``if we lose
funding? What happens if we lose donors? All of those considerations
have to be considered when we write the language around this.''

The deletions and the call that followed created a crisis at the
children's museum that is still unraveling more than two months later.
Nine employees of the museum, which had been operating online only
because of the pandemic, almost immediately went on strike.

``At that moment I was prepared to never come back,'' said Maya Burton,
who, at the time, worked in the museum's education department.

Two weeks ago, those nine employees were laid off, though the museum
said it was a preplanned layoff related to the exhaustion of its funds
from the federal Paycheck Protection Program and ``in no way tied to
recent developments.''

Now an outside investigator hired by the board of trustees is
interviewing former and current employees as part of an inquiry into the
social media incident.

Ms. Stapleton Keith, who has run the museum since 2017, has been placed
on paid leave until its conclusion. In an email, she declined an
interview request, saying that she could not go into specifics because
of the ongoing investigation.

``I do want to underscore that the Seattle Children's Museum and myself,
personally, do support Black Lives Matter and have long put forth
educational programming for children that supports a more diverse,
inclusive and equitable society,'' she said.

In a statement released after the layoffs in July, the museum sought to
explain the controversy: ``Because the content dealt with sensitive
topics and had been posted without typical discussion, review or
approval from S.C.M. leadership, it was revised and references to Black
Lives Matter were temporarily removed until a wider group of museum
stakeholders could be consulted to ensure our messaging accurately
represented our educational content.''

\includegraphics{https://static01.nyt.com/images/2020/08/05/arts/04pulldown-02/04pulldown-02-articleLarge.jpg?quality=75\&auto=webp\&disable=upscale}

In an email, the chair of the museum's board of trustees, Andrew
Mathews, said that ``prospective loss of donors would never change our
fundamental commitment to equity and social justice.''

The museum, which once featured a staff of 20, is now operating with
five people. It had laid off most of its workers earlier in the spring
as the pandemic bore down on Seattle, and it became clear that it would
be a while before children and families would converge on its vast
playground of hands-on activities, like the grocery store where visitors
can pick out imitation food items.

During the months of the pandemic, though, the museum's Instagram
account had become a virtual stand-in for its programming, filling up
over the weeks with science demonstrations about wind energy and
composting, or cooking programs that showed children how to make ``heart
healthy chocolate pudding.''

One former employee who often appeared as a host of the videos, Mimi
Santos, said that it felt natural for the museum to post lists of books
that dovetailed with the racial justice protests sweeping the country.
The lists included titles like
\href{https://anastasiahigginbotham.com/not-my-idea/}{``Not My Idea: A
Book About Whiteness''} and
\href{https://www.albertwhitman.com/book/my-hair-is-a-garden/}{``My Hair
Is a Garden,''} a picture book about a Black girl learning the beauty of
natural hair.

When Ms. Santos heard that Ms. Stapleton Keith did not approve of what
had been posted, she said that she told her supervisor that, ``If you
take it down you're telling the families that we serve and your front
facing staff, who are majority people of color, you're telling us that
you don't care.''

(Mr. Mathews said in an email that there were ``varying accounts'' of
the timeline for when staffers expressed opposition to the idea of
altering the posts.)

By the time the staff logged onto their computers to join the Zoom call
with the director in June, they had a list of demands for their bosses,
among them, that Ms. Stapleton Keith make a public apology and an
explanation as to why the posts were edited.

Ms. Stapleton Keith sought to explain what had happened on the call. She
said that the social media posts hadn't gone through the proper
processes and apologized for the ``hurt all around because of the way
these posts were handled.''

She told the staff that while ``I don't think any of us disagree with
the language around Black Lives Matter,'' releasing such a statement
required more group consultation, as well as board approval.

A former employee who wrote the captions for the posts, Meg Hesketh,
said that she did not realize that writing ``Black Lives Matter'' in a
social media post would require a review process --- or end up causing
such a stir. She said that many members of the museum staff wore pins on
their vests that say Black Lives Matter.

Particularly upsetting, several staff members said, was the suggestion
that the tone of the postings needed to be modified so as to not upset
donors. The museum says that about 40 percent of its budget, which was
roughly \$1.3 million in 2018, is contributed.

``My thought was that then we need to find better funding,'' Anthony
Noceda, a former employee, said in an interview. ``If their values don't
align with that, then we don't need their money.''

Mr. Mathews, the board chair, said that the investigator had also been
tasked with looking into what employees had recently identified as
ongoing problems with how staffers of color were treated at the museum.
He said that the board was ``taking the hurt that our community feels
very seriously.''

Ms. Burton said that, even if there is a chance some staff will be hired
back, she had decided not to return and headed home to Florida, where
she lived before moving to Seattle to attend college in 2012. On Sunday,
she was in the middle of the drive, when she reflected on the tumult of
recent months.

``It's sad because I was sure that this was going to be my forever
job,'' said Ms. Burton, who is Black. ``But this is about people's
lives. It's about my life, the lives of my family and friends.''

Since the controversy erupted in early June, the museum has used the
phrase ``Black Lives Matter'' several times on its social media accounts
and in its public statements. A
\href{https://thechildrensmuseum.org/black-lives-matter/}{new page}on
its website is called ``Because Black Lives Matter.''

\emph{Alain Delaqueriere contributed research.}

Advertisement

\protect\hyperlink{after-bottom}{Continue reading the main story}

\hypertarget{site-index}{%
\subsection{Site Index}\label{site-index}}

\hypertarget{site-information-navigation}{%
\subsection{Site Information
Navigation}\label{site-information-navigation}}

\begin{itemize}
\tightlist
\item
  \href{https://help.nytimes.com/hc/en-us/articles/115014792127-Copyright-notice}{©~2020~The
  New York Times Company}
\end{itemize}

\begin{itemize}
\tightlist
\item
  \href{https://www.nytco.com/}{NYTCo}
\item
  \href{https://help.nytimes.com/hc/en-us/articles/115015385887-Contact-Us}{Contact
  Us}
\item
  \href{https://www.nytco.com/careers/}{Work with us}
\item
  \href{https://nytmediakit.com/}{Advertise}
\item
  \href{http://www.tbrandstudio.com/}{T Brand Studio}
\item
  \href{https://www.nytimes.com/privacy/cookie-policy\#how-do-i-manage-trackers}{Your
  Ad Choices}
\item
  \href{https://www.nytimes.com/privacy}{Privacy}
\item
  \href{https://help.nytimes.com/hc/en-us/articles/115014893428-Terms-of-service}{Terms
  of Service}
\item
  \href{https://help.nytimes.com/hc/en-us/articles/115014893968-Terms-of-sale}{Terms
  of Sale}
\item
  \href{https://spiderbites.nytimes.com}{Site Map}
\item
  \href{https://help.nytimes.com/hc/en-us}{Help}
\item
  \href{https://www.nytimes.com/subscription?campaignId=37WXW}{Subscriptions}
\end{itemize}
