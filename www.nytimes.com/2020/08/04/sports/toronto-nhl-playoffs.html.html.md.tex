Sections

SEARCH

\protect\hyperlink{site-content}{Skip to
content}\protect\hyperlink{site-index}{Skip to site index}

\href{https://www.nytimes.com/section/sports}{Sports}

\href{https://myaccount.nytimes.com/auth/login?response_type=cookie\&client_id=vi}{}

\href{https://www.nytimes.com/section/todayspaper}{Today's Paper}

\href{/section/sports}{Sports}\textbar{}Toronto Tones Down Its
Postseason Party

\url{https://nyti.ms/3gvxulO}

\begin{itemize}
\item
\item
\item
\item
\item
\end{itemize}

Advertisement

\protect\hyperlink{after-top}{Continue reading the main story}

Supported by

\protect\hyperlink{after-sponsor}{Continue reading the main story}

\hypertarget{toronto-tones-down-its-postseason-party}{%
\section{Toronto Tones Down Its Postseason
Party}\label{toronto-tones-down-its-postseason-party}}

The N.H.L. playoffs are proceeding without fans in and around Scotiabank
Arena, leaving hockey to compete with other leagues for attention.

\includegraphics{https://static01.nyt.com/images/2020/08/04/sports/04nhl-toronto-3/merlin_156430548_d842714e-5d34-4577-8a1d-1ef1a7a59a45-articleLarge.jpg?quality=75\&auto=webp\&disable=upscale}

By Morgan Campbell

\begin{itemize}
\item
  Aug. 4, 2020
\item
  \begin{itemize}
  \item
  \item
  \item
  \item
  \item
  \end{itemize}
\end{itemize}

TORONTO --- On Bremner Boulevard just west of Scotiabank Arena, a few TV
news crews gathered outside the temporary fence surrounding the
building, hoping to see N.H.L. teams leaving or entering the perimeter.

In the past, barricades bounded the wide cul-de-sac, barely pinning in
the crowded street party that drew thousands of fans to the Toronto
Raptors' run to the 2019 N.B.A. championship. But in the hours before
the start of the 2020 N.H.L. playoffs on Saturday afternoon, the area
featured security, police and a solitary hot dog vendor waiting next to
his cart.

In an indoor retail space across the street, a Sport Chek apparel store
with plenty of Toronto Maple Leafs gear on display was open, but empty
of shoppers. Farther down the concourse, a liquor store had customers
lined up out the door, spaced six feet apart.

Toronto is one of two sites
\href{https://www.nytimes.com/2020/07/06/sports/hockey/nhl-playoffs.html}{hosting
the N.H.L.'s postseason} ---
\href{https://www.nytimes.com/2020/08/01/sports/hockey/edmonton-nhl-playoffs.html}{the
other is Edmonton} --- in a restart that came after a 140-day pause in
play because of the coronavirus pandemic. The culminating tournament did
not compel many locals to rearrange their priorities during the Simcoe
Day holiday weekend, but the league has put down a symbolic footprint,
if not an economically significant one. Blocks of rooms at two downtown
hotels house the teams and signage on busy sidewalks near the arena
remind pedestrians the N.H.L. has arrived.

Norm O'Reilly, director of the International Institute for Sport
Business and Leadership at the University of Guelph, said Canadians
recognized the significance of holding the games in the country's
largest city, ``even though there's very little economic impact because
nobody's traveling to watch the games.'' He added, ``For the hard-core
hockey fan, it's a no-brainer.''

Inside the empty arena, stakes remained high not only for the league,
which is looking to salvage a season upended by the pandemic, but also
for the players, who are competing for the Stanley Cup, after all. But
arena light shows and piped-in crowd noise could not make playoff hockey
without spectators feel normal.

``There's no crowd, obviously,'' Rangers goaltender Henrik Lundqvist
said Saturday after his team's 3-2 loss to the Carolina Hurricanes.
``That intensity that you feed off of playing in the playoffs, it's not
there.''

\includegraphics{https://static01.nyt.com/images/2020/08/04/sports/04nhl-toronto-2/merlin_175249056_4e84c0ea-2e8d-4110-8e17-8698c8c85d71-articleLarge.jpg?quality=75\&auto=webp\&disable=upscale}

Before restarting play, the N.H.L. published a 28-page manual outlining
how it would operate a playoff schedule while limiting player and staff
exposure to the public during the pandemic. Players and staff undergo
daily Covid-19 screenings, and stay at their hotels when not playing or
training, or at N.H.L.-designated recreational areas, including movie
theaters, patios and lounges.

The setup helps ensure that all teams feel like visitors, including the
hometown Maple Leafs, who, like every other team in the hub, are staying
in a downtown hotel. ``There will be some familiarity for us'' being in
the home arena, Maple Leafs captain John Tavares said Sunday before
Toronto's series-opening loss to the Columbus Blue Jackets. ``Things are
going to be different, even when we do get to use our own facilities.''

\hypertarget{the-coronavirus-outbreak}{%
\subsubsection{The Coronavirus
Outbreak}\label{the-coronavirus-outbreak}}

\hypertarget{sports-and-the-virus}{%
\paragraph{Sports and the Virus}\label{sports-and-the-virus}}

Updated Aug. 4, 2020

Here's what's happening as the world of sports slowly comes back to
life:

\begin{itemize}
\item
  \begin{itemize}
  \tightlist
  \item
    As the virus spreads through baseball,
    \href{https://www.nytimes.com/2020/08/03/sports/baseball/mlb-coronavirus-outbreak.html?action=click\&pgtype=Article\&state=default\&region=MAIN_CONTENT_2\&context=storylines_keepup}{so
    does frustration}. Series have been postponed, teams have been
    quarantined and road trips have been rerouted in a season that has
    been defined above all by its precariousness.
  \item
    On all but the two biggest courts, automated line calls
    \href{https://www.nytimes.com/2020/08/03/sports/tennis/us-open-hawkeye-line-judges.html?action=click\&pgtype=Article\&state=default\&region=MAIN_CONTENT_2\&context=storylines_keepup}{will
    replace human judges} at the U.S. Open to reduce the number of
    people on site during the pandemic.
  \item
    Mets star Yoenis Cespedes is healthy, but
    \href{https://www.nytimes.com/2020/08/02/sports/baseball/Yoenis-cespedes-opt-out-rule.html?action=click\&pgtype=Article\&state=default\&region=MAIN_CONTENT_2\&context=storylines_keepup}{has
    decided to opt out} of the 2020 baseball season for Covid-related
    reasons.
  \end{itemize}
\end{itemize}

But in a league where gate revenue still matters, teams also need to
adjust to a postseason without live spectators.

The Maple Leafs, for example, ranked fourth in the N.H.L. in home
attendance this season, averaging 19,301 fans per home game. Revenue
lost from those fans' ticket purchases, parking, seat licensing and
concessions adds up. The absence of fans heightens the importance of TV
ratings, even as the N.H.L. competes for North American viewers against
other leagues, like Major League Baseball and the N.B.A., that have
resumed competition this summer.

According to Sports Media Watch, the series opener between the
Pittsburgh Penguins and the Montreal Canadiens averaged 1.54 million
viewers on NBC, a markedly smaller audience than the N.B.A. and M.L.B.
restarts drew earlier in July.

Four years ago, all seven Canadian N.H.L. teams missed the playoffs and
TV ratings cratered. That year, the first week of playoff broadcasts
averaged\href{https://thehockeynews.com/news/article/playoff-tv-ratings-down-a-shocking-61-percent-in-canada}{a
reported 513,000 viewers in Canada}, down 61 percent from the previous
season.

Canadian viewership numbers won't be made available until Tuesday, but
last week, Sportsnet's Chris Johnston
\href{https://twitter.com/SportsnetPR/status/1288572556887429120}{reported
that 4.3 million people total tuned in} across the company's various
networks during an N.H.L. exhibition doubleheader July 28.

But even in hockey-mad Canada, where, in 2013, Rogers, which owns
Sportsnet, agreed to pay \$5.2 billion for 12 years of N.H.L. TV rights,
deep fan engagement and big audiences are not guaranteed for this
postseason. The only two Canadian teams at the Toronto playoff hub are
the eighth-seeded Maple Leafs and the 12th-seeded Canadiens. An early
exit for either team would eliminate two of Sportsnet's biggest TV
attractions.

O'Reilly said that outside Canada the summer restart was an opportunity
for the N.H.L. to grow its audience beyond its current fan base, but the
league won't know if it has succeeded until later in the postseason.

``Can they get that share of the market that's just a general sports
fan?'' O'Reilly said. ``If the N.H.L. can get them interested in hockey,
that's a win.''

Advertisement

\protect\hyperlink{after-bottom}{Continue reading the main story}

\hypertarget{site-index}{%
\subsection{Site Index}\label{site-index}}

\hypertarget{site-information-navigation}{%
\subsection{Site Information
Navigation}\label{site-information-navigation}}

\begin{itemize}
\tightlist
\item
  \href{https://help.nytimes.com/hc/en-us/articles/115014792127-Copyright-notice}{©~2020~The
  New York Times Company}
\end{itemize}

\begin{itemize}
\tightlist
\item
  \href{https://www.nytco.com/}{NYTCo}
\item
  \href{https://help.nytimes.com/hc/en-us/articles/115015385887-Contact-Us}{Contact
  Us}
\item
  \href{https://www.nytco.com/careers/}{Work with us}
\item
  \href{https://nytmediakit.com/}{Advertise}
\item
  \href{http://www.tbrandstudio.com/}{T Brand Studio}
\item
  \href{https://www.nytimes.com/privacy/cookie-policy\#how-do-i-manage-trackers}{Your
  Ad Choices}
\item
  \href{https://www.nytimes.com/privacy}{Privacy}
\item
  \href{https://help.nytimes.com/hc/en-us/articles/115014893428-Terms-of-service}{Terms
  of Service}
\item
  \href{https://help.nytimes.com/hc/en-us/articles/115014893968-Terms-of-sale}{Terms
  of Sale}
\item
  \href{https://spiderbites.nytimes.com}{Site Map}
\item
  \href{https://help.nytimes.com/hc/en-us}{Help}
\item
  \href{https://www.nytimes.com/subscription?campaignId=37WXW}{Subscriptions}
\end{itemize}
