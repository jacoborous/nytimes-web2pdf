Sections

SEARCH

\protect\hyperlink{site-content}{Skip to
content}\protect\hyperlink{site-index}{Skip to site index}

\href{https://www.nytimes.com/section/politics}{Politics}

\href{https://myaccount.nytimes.com/auth/login?response_type=cookie\&client_id=vi}{}

\href{https://www.nytimes.com/section/todayspaper}{Today's Paper}

\href{/section/politics}{Politics}\textbar{}Rashida Tlaib Beat Her
Primary Opponent by 900 Votes in 2018. How Will the Rematch Go?

\url{https://nyti.ms/31e4fxI}

\begin{itemize}
\item
\item
\item
\item
\item
\end{itemize}

\begin{itemize}
\item
  \href{https://www.nytimes.com/2020/08/04/us/elections/primary-election-michigan-arizona-kansas.html?action=click\&pgtype=Article\&state=default\&region=TOP_BANNER\&context=storylines_menu}{Election
  Updates}
\item
  \href{https://www.nytimes.com/article/biden-vice-president-2020.html?action=click\&pgtype=Article\&state=default\&region=TOP_BANNER\&context=storylines_menu}{Biden's
  V.P. Search}
\item
  \href{https://www.nytimes.com/interactive/2020/07/24/us/politics/trump-biden-campaign-donors.html?action=click\&pgtype=Article\&state=default\&region=TOP_BANNER\&context=storylines_menu}{Map
  of Donations}
\item
  \href{https://www.nytimes.com/interactive/2020/us/elections/delegate-count-primary-results.html?action=click\&pgtype=Article\&state=default\&region=TOP_BANNER\&context=storylines_menu}{Delegate
  Count}
\item
  \href{https://www.nytimes.com/interactive/2019/us/politics/2020-presidential-candidates.html?action=click\&pgtype=Article\&state=default\&region=TOP_BANNER\&context=storylines_menu}{The
  Candidates}
\item
  \href{https://www.nytimes.com/newsletters/politics?action=click\&pgtype=Article\&state=default\&region=TOP_BANNER\&context=storylines_menu}{Politics
  Newsletter}
\end{itemize}

Advertisement

\protect\hyperlink{after-top}{Continue reading the main story}

Supported by

\protect\hyperlink{after-sponsor}{Continue reading the main story}

\hypertarget{rashida-tlaib-beat-her-primary-opponent-by-900-votes-in-2018-how-will-the-rematch-go}{%
\section{Rashida Tlaib Beat Her Primary Opponent by 900 Votes in 2018.
How Will the Rematch
Go?}\label{rashida-tlaib-beat-her-primary-opponent-by-900-votes-in-2018-how-will-the-rematch-go}}

The Michigan Democrat is facing a rematch against a challenger who says
she hasn't done enough for her district in Detroit.

\includegraphics{https://static01.nyt.com/images/2020/08/03/us/politics/03michigan-setup1/merlin_174478425_9e8d34ea-f4ee-4e42-b3e2-742a04dcb262-articleLarge.jpg?quality=75\&auto=webp\&disable=upscale}

By Kathleen Gray

\begin{itemize}
\item
  Published Aug. 3, 2020Updated Aug. 4, 2020, 6:30 a.m. ET
\item
  \begin{itemize}
  \item
  \item
  \item
  \item
  \item
  \end{itemize}
\end{itemize}

DETROIT --- Representative Rashida Tlaib, a first-term Michigan Democrat
who rocketed to national attention as a vocal critic of President Trump,
is fighting for her political life, locked in a close
\href{https://www.nytimes.com/2020/08/04/us/elections/primary-election-michigan-arizona-kansas.html}{primary}
race that could be decided by a few hundred mail-in ballots.

One of the first Muslim women elected to the House of Representatives,
Ms. Tlaib on Tuesday faces a rematch against Brenda Jones, the Detroit
City Council president who Ms. Tlaib narrowly defeated in 2018.

It's a sequel that many more people are likely to be watching. Ms.
Tlaib's prominent role in Washington has translated to more resources
for her district, supporters say, and in this contest it's enabled her
to significantly outspend her opponent on advertising.

It's also given Ms. Jones an argument against her: that spending time
criticizing the president limits the kind of compromises necessary to
get the most done for the people who sent her to Washington. And the
contest, taking place amid national protests against racial injustice,
finds people in this majority-Black district asking themselves whether
Black voters must be represented by Black politicians to really be
heard.

Ms. Tlaib, 43, came to Washington as a member of ``the squad,'' a group
of four progressive women of color elected in 2018 who became frequent
targets of Mr. Trump, who has attacked them as foreigners who do not
love the United States. (All four are American citizens, and only one,
Ilhan Omar, was born outside the country.)

For over 50 years, the district, which includes a portion of Detroit and
a handful of surrounding suburbs, was represented by John Conyers, a
civil rights icon. Some supporters of Ms. Jones, who is Black, have said
that she would be a better fit for the district than Ms. Tlaib, who is
Palestinian by descent.

Ian Conyers, a former state senator and great-nephew of John Conyers who
ran against Ms. Jones in a special election in 2018 and endorsed her
this year, argued that her community ties would help her forge better
relationships in Washington.

``From your work in the community, you're able to directly talk to
people who already have a working relationship with you,'' he said in
his endorsement. ``You just can't make up that experience overnight.''

Ms. Tlaib's supporters, however, reject the idea that she is not an
effective advocate for her district.

For Kim McDade, 58, of Highland Park, a Detroit suburb, the information
coming from Ms. Tlaib's office on the coronavirus has been invaluable.
``She's had a couple of town halls and she always calls or texts to let
us know about those,'' said Ms. McDade. ``It's been very informative and
she also shares the sites where we can get tested.''

\hypertarget{latest-updates-2020-election}{%
\section{\texorpdfstring{\href{https://www.nytimes.com/2020/08/04/us/elections/primary-election-michigan-arizona-kansas.html?action=click\&pgtype=Article\&state=default\&region=MAIN_CONTENT_1\&context=storylines_live_updates}{Latest
Updates: 2020
Election}}{Latest Updates: 2020 Election}}\label{latest-updates-2020-election}}

Updated 2020-08-04T18:55:19.561Z

\begin{itemize}
\tightlist
\item
  \href{https://www.nytimes.com/2020/08/04/us/elections/primary-election-michigan-arizona-kansas.html?action=click\&pgtype=Article\&state=default\&region=MAIN_CONTENT_1\&context=storylines_live_updates\#link-3924dd44}{Two
  G.O.P. Senate primaries offer --- what else? --- a test of loyalty to
  Trump.}
\item
  \href{https://www.nytimes.com/2020/08/04/us/elections/primary-election-michigan-arizona-kansas.html?action=click\&pgtype=Article\&state=default\&region=MAIN_CONTENT_1\&context=storylines_live_updates\#link-32b39e33}{President
  Trump is suddenly a big supporter of mail-in voting --- in Florida.}
\item
  \href{https://www.nytimes.com/2020/08/04/us/elections/primary-election-michigan-arizona-kansas.html?action=click\&pgtype=Article\&state=default\&region=MAIN_CONTENT_1\&context=storylines_live_updates\#link-6d019753}{Election
  experts warn Congress about widespread disenfranchisement of voters of
  color in November.}
\end{itemize}

\href{https://www.nytimes.com/2020/08/04/us/elections/primary-election-michigan-arizona-kansas.html?action=click\&pgtype=Article\&state=default\&region=MAIN_CONTENT_1\&context=storylines_live_updates}{See
more updates}

Speaking on Saturday under a tarp held aloft by a group of supporters to
shield her from a sudden downpour, Ms. Tlaib tied her criticisms of the
president to local concerns about over-policing.

``I'm not going to allow this impeached president to come into my
community,'' she said to protesters attending a march organized by
Detroit Will Breathe, a group started after the killing of George Floyd
by the police. ``He thinks he can send these federal agents and troops
into our community on our watch. To think that he's going to come in and
do this to us. No.''

Ms. Tlaib caught the attention of Mr. Trump the day she was sworn into
office in January 2019, when she used an expletive to say she would work
to impeach him. She has been the target of the president's wrath ever
since, and her criticism of him won her national attention. But it also
prompted Ms. Jones to jump into the race earlier this year.

``It's so important to me that we unify the district. Sixty percent of
the district is African-American, but it's not about race, it's about
bringing home the bacon,'' Ms. Jones said during a recent virtual town
hall. ``The money we bring home is so important to me, and I'm able to
work with those who I don't always agree with.''

Ms. Jones, 60, has good name recognition in the district from her 15
years on the City Council, as well as the five weeks she served in
Congress in 2018 after winning the special election, which was held to
fill the remainder of Mr. Conyers's term after he retired amid
accusations of sexual harassment from former staffers.

In the four-way Democratic primary to complete the rest of Mr. Conyers's
term, Ms. Jones easily won the portion of the district in Detroit and
held her ground in the suburbs, edging out Ms. Tlaib by about 1,600
votes, out of about 87,000 cast. But in the six-way Democratic primary
for the next full two-year term, which was held concurrently with the
special election, Ms. Tlaib narrowly won, beating Ms. Jones by about 900
votes, out of about 89,000 cast.

\includegraphics{https://static01.nyt.com/images/2020/08/03/us/politics/03michigan-setup2/merlin_141697746_58017414-7ff9-4c52-9dcb-3b0549e6a36b-articleLarge.jpg?quality=75\&auto=webp\&disable=upscale}

Observers expect Tuesday's primary to be close as well. But Michigan
voters may not know the winners of this and many other races on election
night because of a surge in absentee voting.

In the Third Congressional District, a Republican stronghold in Western
Michigan that has been trending blue in recent years, five Republicans
are vying for the nomination to replace Representative Justin Amash, a
former Republican who left the party and briefly considered running for
president as a libertarian before deciding to retire.

And in the reliably conservative 10th District, which spans several
counties north of Detroit, three Republicans and two Democrats are
running to replace Paul Mitchell, who also decided not to run for
re-election.

The race between Ms. Tlaib and Ms. Jones, by far the most prominent, has
been characterized by the candidates' differing campaign styles and
disparities in resources.

Ms. Tlaib has emphasized in-person appearances, canvassing door to door
and attending community events. She has won the endorsement of several
labor unions, as well as Nancy Pelosi, the speaker of the House. And she
has raised about \$3 million, significantly more than Ms. Jones,
allowing her to dominate the airwaves in metro Detroit.

``I've watched a few of those town halls and some attracted a lot of
people, but others didn't,'' said Mario Morrow, a political consultant
in Detroit. ``One of the people I see most in Southeast Michigan is
Rashida Tlaib, and she usually has a bullhorn in her hand. I think she's
on her way back to Washington.''

Ms. Jones, who contracted Covid-19 in April and recovered, has limited
her campaign to virtual town halls and Zoom calls. She has also been
hampered by anemic fund-raising --- she has raised just \$165,000,
coming into a campaign that still had debt from the 2018 race.

Image

Five Republicans are fighting for the chance to represent the seat held
by Representative Justin Amash, an independent who has been critical of
President Trump and is not seeking re-election.Credit...Anna
Moneymaker/The New York Times

She has remained competitive thanks in part to the endorsements of the
other candidates in the 2018 contest, as well as a large group of
influential Black ministers in the city, including the Rev. Wendell
Anthony, the president of the Detroit chapter of the N.A.A.C.P.

``I've known Councilwoman Jones as a very solid person who does what she
says she's going to do,'' Reverend Anthony said. ``Following the tenure
of Congressman John Conyers, an icon for all of us around here, there's
a legacy that comes with that, and Congresswoman Jones would bring that
back and sustain that.''

The race may ultimately come down to whether Ms. Jones's focus on
establishment-style retail politics is enough to match Ms. Tlaib's
ability to excite voters.

Glenda McDonald, 59, who works at a charter school in Highland Park,
said she appreciated Ms. Tlaib's ``in-your-face'' style, if not always
the language that the congresswoman uses.

And while Ms. McDonald, who is Black, has taken heat from some friends
for supporting Ms. Tlaib instead of Ms. Jones, she said skin color was
not a factor in her choice.

``Most people think that the district's seat belongs to someone else,''
she said. ``But I choose who I want to choose. The person who represents
me should represent everybody.''

\hypertarget{our-2020-election-guide}{%
\section{Our 2020 Election Guide}\label{our-2020-election-guide}}

Updated Aug. 4, 2020

\begin{itemize}
\item
  \begin{center}\rule{0.5\linewidth}{\linethickness}\end{center}

  \hypertarget{the-latest}{%
  \subsection{The Latest}\label{the-latest}}

  \begin{itemize}
  \tightlist
  \item
    Five states are holding primary elections Tuesday, with voters in
    Arizona, Kansas, Michigan, Missouri and Washington State choosing
    nominees for Congress and local offices.
    \href{https://www.nytimes.com/2020/08/04/us/elections/primary-election-michigan-arizona-kansas.html?action=click\&pgtype=Article\&state=default\&region=BELOW_MAIN_CONTENT\&context=storylines_guide}{Follow
    live election updates here.}
  \end{itemize}
\item
  \begin{center}\rule{0.5\linewidth}{\linethickness}\end{center}

  \hypertarget{bidens-vp-search}{%
  \subsection{Biden's V.P. Search}\label{bidens-vp-search}}

  \begin{itemize}
  \tightlist
  \item
    \href{https://www.nytimes.com/article/biden-vice-president-2020.html?action=click\&pgtype=Article\&state=default\&region=BELOW_MAIN_CONTENT\&context=storylines_guide}{Here
    are 13 women} who have been under consideration to be Joe Biden's
    running mate, and why each might be chosen --- and might not be.
  \end{itemize}
\item
  \begin{center}\rule{0.5\linewidth}{\linethickness}\end{center}

  \hypertarget{keep-up-with-our-coverage}{%
  \subsection{Keep Up With Our
  Coverage}\label{keep-up-with-our-coverage}}

  \begin{itemize}
  \tightlist
  \item
    Get an
    \href{https://www.nytimes.com/newsletters/politics?action=click\&pgtype=Article\&state=default\&region=BELOW_MAIN_CONTENT\&context=storylines_guide}{email}
    recapping the day's news
  \end{itemize}

  \begin{itemize}
  \tightlist
  \item
    Download our mobile app on
    \href{https://apps.apple.com/us/app/nytimes/id284862083?ls=1\&mat_click_id=5c79ae7455014fd1bd66b5610c05b8f2-20191112-16948\&referrer=mat_click_id\%3D5c79ae7455014fd1bd66b5610c05b8f2-20191112-16948\%26link_click_id\%3D722930677036718082}{iOS}
    and
    \href{http://a.localytics.com/android?id=com.nytimes.android\&referrer=utm_source\%3Dother_nyt_mobile_web\%26utm_medium\%3DWeb\%2520page\%26utm_term\%3DGeneral\%2520Mobile\%2520Page\%26utm_campaign\%3DNYT\%2520Mobile\%2520General\%2520Page}{Android}
    and turn on Breaking News and Politics alerts
  \end{itemize}
\end{itemize}

Advertisement

\protect\hyperlink{after-bottom}{Continue reading the main story}

\hypertarget{site-index}{%
\subsection{Site Index}\label{site-index}}

\hypertarget{site-information-navigation}{%
\subsection{Site Information
Navigation}\label{site-information-navigation}}

\begin{itemize}
\tightlist
\item
  \href{https://help.nytimes.com/hc/en-us/articles/115014792127-Copyright-notice}{©~2020~The
  New York Times Company}
\end{itemize}

\begin{itemize}
\tightlist
\item
  \href{https://www.nytco.com/}{NYTCo}
\item
  \href{https://help.nytimes.com/hc/en-us/articles/115015385887-Contact-Us}{Contact
  Us}
\item
  \href{https://www.nytco.com/careers/}{Work with us}
\item
  \href{https://nytmediakit.com/}{Advertise}
\item
  \href{http://www.tbrandstudio.com/}{T Brand Studio}
\item
  \href{https://www.nytimes.com/privacy/cookie-policy\#how-do-i-manage-trackers}{Your
  Ad Choices}
\item
  \href{https://www.nytimes.com/privacy}{Privacy}
\item
  \href{https://help.nytimes.com/hc/en-us/articles/115014893428-Terms-of-service}{Terms
  of Service}
\item
  \href{https://help.nytimes.com/hc/en-us/articles/115014893968-Terms-of-sale}{Terms
  of Sale}
\item
  \href{https://spiderbites.nytimes.com}{Site Map}
\item
  \href{https://help.nytimes.com/hc/en-us}{Help}
\item
  \href{https://www.nytimes.com/subscription?campaignId=37WXW}{Subscriptions}
\end{itemize}
