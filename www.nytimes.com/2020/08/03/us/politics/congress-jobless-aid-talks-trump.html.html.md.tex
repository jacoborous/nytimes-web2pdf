Sections

SEARCH

\protect\hyperlink{site-content}{Skip to
content}\protect\hyperlink{site-index}{Skip to site index}

\href{https://www.nytimes.com/section/politics}{Politics}

\href{https://myaccount.nytimes.com/auth/login?response_type=cookie\&client_id=vi}{}

\href{https://www.nytimes.com/section/todayspaper}{Today's Paper}

\href{/section/politics}{Politics}\textbar{}With Jobless Aid Expired,
Trump Sidelines Himself in Stimulus Talks

\href{https://nyti.ms/31gJ5yJ}{https://nyti.ms/31gJ5yJ}

\begin{itemize}
\item
\item
\item
\item
\item
\end{itemize}

Advertisement

\protect\hyperlink{after-top}{Continue reading the main story}

Supported by

\protect\hyperlink{after-sponsor}{Continue reading the main story}

\hypertarget{with-jobless-aid-expired-trump-sidelines-himself-in-stimulus-talks}{%
\section{With Jobless Aid Expired, Trump Sidelines Himself in Stimulus
Talks}\label{with-jobless-aid-expired-trump-sidelines-himself-in-stimulus-talks}}

As his top advisers met with Democratic leaders to try to hash out a
compromise, President Trump hurled insults at Democrats and mused aloud
about short-circuiting the talks and acting on his own.

\includegraphics{https://static01.nyt.com/images/2020/08/03/us/politics/03dc-virus-stimulus01/03dc-virus-stimulus01-articleLarge.jpg?quality=75\&auto=webp\&disable=upscale}

\href{https://www.nytimes.com/by/maggie-haberman}{\includegraphics{https://static01.nyt.com/images/2018/07/12/multimedia/author-maggie-haberman/author-maggie-haberman-thumbLarge.png}}\href{https://www.nytimes.com/by/emily-cochrane}{\includegraphics{https://static01.nyt.com/images/2018/11/28/multimedia/author-emily-cochrane/author-emily-cochrane-thumbLarge-v3.png}}\href{https://www.nytimes.com/by/jim-tankersley}{\includegraphics{https://static01.nyt.com/images/2018/10/19/multimedia/author-jim-tankersley/author-jim-tankersley-thumbLarge.png}}

By \href{https://www.nytimes.com/by/maggie-haberman}{Maggie Haberman},
\href{https://www.nytimes.com/by/emily-cochrane}{Emily Cochrane} and
\href{https://www.nytimes.com/by/jim-tankersley}{Jim Tankersley}

\begin{itemize}
\item
  Aug. 3, 2020
\item
  \begin{itemize}
  \item
  \item
  \item
  \item
  \item
  \end{itemize}
\end{itemize}

On the first day of the first full week when tens of millions of
Americans went without the federal jobless aid that has cushioned them
during the pandemic, President Trump was not cajoling undecided
lawmakers to embrace a critical stimulus bill to stabilize the
foundering economy.

He was at the White House, hurling insults at the Democratic leaders
whose support he needs to strike a deal.

Mr. Trump called Speaker Nancy Pelosi ``Crazy Nancy,'' charging that she
had no interest in helping the unemployed. He said Senator Chuck Schumer
of New York, the Democratic leader, only wanted to help ``radical left''
governors in states run by Democrats. And he threatened to short-circuit
a delicate series of negotiations to produce a compromise and instead
unilaterally impose a federal moratorium on tenant evictions.

The comments came just as Mr. Trump's own advisers were on Capitol Hill
meeting with Ms. Pelosi and Mr. Schumer in search of an elusive deal,
and they underscored just how absent the president had been from the
negotiations. They also highlighted how, three months before he is to
face voters, the main role that Mr. Trump appears to have embraced in
assembling an economic recovery package is that of sniping from the
sidelines in ways that undercut a potential compromise.

On Monday, the president said he remained ``totally involved'' in the
talks, even though he was not ``over there with Crazy Nancy.'' But while
White House officials say that he is interested in the talks and is
closely monitoring them, he has not sought to use the full powers of his
office to prod a deal, and more often he has complicated the already
sensitive negotiations.

The situation reflects the dysfunctional dynamic that Mr. Trump has
developed with leaders of both parties in Congress. He has a toxic
relationship with Ms. Pelosi, with whom he has not met face-to-face
since last year. And Republicans have learned to eye their own president
warily in delicate negotiations, knowing that he is prone to changing
his position, bucking party principles and leaving them to suffer the
political consequences of high-profile collapses.

In the stimulus talks, Mr. Trump's ideas have often been out of sync
with members of his own party. On Monday, he said he was considering
acting on his own to eliminate payroll taxes, something a president does
not have the power to do himself, and an idea that his advisers had
dropped from the talks in the face of near-unanimous opposition by
Republican lawmakers. The eviction moratorium he has championed was not
a part of the Republican plan.

``I'll do it myself if I have to,'' Mr. Trump said.

While that might be possible, virtually every other measure under
discussion to stimulate the economy would require congressional
approval.

The stakes of the negotiations could not be higher. Business leaders
pleaded with lawmakers to draft a sweeping recovery package to help the
hardest-hit industries survive the crisis. And economists warned that
the expiration of the \$600-per-week enhanced unemployment payments
could already be dragging down consumer spending.

On Monday, Ms. Pelosi floated a possible compromise to extend the
benefits, saying that Democrats might be open to tying the weekly
payments, which Republicans are pressing to cut substantially, to the
unemployment rate, allowing the amount to fall in tandem with the
jobless rate.

``That's something to talk about,'' Ms. Pelosi said on CNN. ``Right now,
today, we have an emergency. A building is on fire, and they are
deciding how much water they want to have in the bucket.''

Privately, she warned House Democrats during an afternoon conference
call that while she had hoped to reach a deal with the White House this
week, she was no longer sure that was possible, according to two people
on the call who described it on the condition of anonymity.

Some lawmakers saw the glimmers of a possible bargain, although they
warned the process of striking it would not be pretty.

``This is just the painful period between people finally deciding, `OK,
we want a deal,' and what that deal ultimately looks like,'' Senator
John Cornyn, Republican of Texas, told reporters.

At the same moment that Mr. Trump was blasting her, Ms. Pelosi met on
Capitol Hill with Mr. Schumer; Mark Meadows, the White House chief of
staff; and Steven Mnuchin, the Treasury secretary, in search of a
compromise. It was the sixth such in-person meeting in eight days, and
followed a rare Saturday session with the four negotiators.

\includegraphics{https://static01.nyt.com/images/2020/08/03/us/03dc-virus-stimulus02/03dc-virus-stimulus02-articleLarge.jpg?quality=75\&auto=webp\&disable=upscale}

Mr. Trump, who spent Saturday and Sunday on his golf course in Virginia,
berated Democrats from the White House on Monday, accusing them of being
blinded by a focus on ``bailout money'' for states controlled by
Democrats, as opposed to extending unemployment benefits.

``All they're really interested in is bailout money to bail out radical
left governors and radical left mayors like in Portland and places that
are so badly run --- Chicago, New York City,'' Mr. Trump said.

In their \$3 trillion recovery package, Democrats have proposed
providing more than \$900 billion to cash-strapped states and cities
whose budgets have been devastated in the recession, while Republicans
did not include any money for them in their \$1 trillion plan. But it is
Republicans who have proposed cutting the jobless aid, while Democrats
are pushing to extend the \$600 weekly federal payments through January.

Later in the day, Mr. Trump sounded a less hostile note, even as he
repeated that he could halt evictions with an executive order.

``But we are having a very good discussion with Nancy Pelosi and Chuck
Schumer,'' he added during a late-afternoon briefing.

White House officials describe Mr. Trump as interested in the talks, but
from a distance. He calls Mr. Meadows, a former House member, for
updates nearly a dozen times on some days, and in general gets briefed
in 10-minute increments from other aides. He makes frequent calls to
allies like Representative Kevin McCarthy, the House Republican leader,
and to Senator Mitch McConnell of Kentucky, the majority leader.

But he does not reach out to members of the House he is not personally
close with to use the power of persuasion that comes with the
presidency, they concede, and he is expending little energy of his own
to move the ball forward.

Last Thursday, when Mr. Meadows was asked by reporters why the president
did not simply bring congressional leaders to the Oval Office and keep
everyone there until there was a deal, Mr. Meadows replied, ``You've
seen that movie before,'' prompting laughter.

Previous efforts by Mr. Trump to convene a bipartisan meeting of the
minds at the White House have proved disastrous.

On Capitol Hill, the group discussing a possible deal spent two hours
going over the proposal put forward by Republicans a week ago and
``going to these specific numbers and what each side thinks they can do
with their dollar allocation,'' Mr. Schumer said.

``We're really getting an understanding of each side's position and
we're making some progress on certain issues, moving closer together,''
he added afterward. ``There are a lot of issues that are still
outstanding, but I think there is a desire to get something done as soon
as we can.''

Ms. Pelosi sounded a hopeful tone, as well, saying, ``We're moving down
the track,'' even though significant differences remained between the
two proposals.

But after the meeting, Mr. Schumer said that Republicans were ``sticking
to their position'' on maintaining the \$600 weekly federal unemployment
benefits, and Ms. Pelosi added, ``We're sticking to ours.''

While White House officials and Democratic leaders reported some
progress over the weekend in their talks, they
\href{https://www.nytimes.com/2020/08/02/us/politics/coronavirus-jobless-aid.html}{still
have substantial differences}. Democrats are pressing to maintain the
enhanced federal unemployment payments, bail out strapped states and
cities, send billions of dollars to schools and extend additional health
care and food aid funds, as well as protections for workers. Republicans
want to scale back the jobless money, devote \$105 billion to schools
and include a broad liability shield to protect businesses from being
held legally liable for the spread of the virus.

Republicans initially proposed to cut the \$600-per-week unemployment
benefit and shift to a system that would pad the typical unemployed
worker's check by about \$200 per week. Last week, Senate Republicans
offered a one-week extension of the \$600 supplement, which Democrats
rejected. Administration officials later offered a longer-term extension
at a lower rate, which Democrats again rejected.

Congressional staff and lobbyists who are engaged in discussions said on
Monday that the talks between administration officials and Ms. Pelosi
and Mr. Schumer had essentially frozen negotiations between top
Democrats and Republicans on key committees who would have to hammer out
the details of any deal. That could leave the parties little time to
flesh out any compromises over additional aid to businesses or
individuals, yielding a plan that mostly consists of re-upping existing
aid programs like the Paycheck Protection Program and direct payments to
individuals.

A group of executives led by the former Starbucks chairman Howard
Schultz, which included several major business groups and top executives
from companies like Alphabet and Facebook,
\href{https://www.howardschultz.com/lettertocongress/}{sent a letter to
congressional leaders} on Monday urging more aggressive efforts like
long-term, federally guaranteed loans to help small businesses in any
new rescue package.

``This is not a call for bottomless handouts,'' they wrote. ``It is a
defining moment to show how capitalism can help all Americans,
particularly entrepreneurs who have been forced to shutter or reduce the
capacity of their businesses through no fault of their own.''

Luke Broadwater contributed reporting.

Advertisement

\protect\hyperlink{after-bottom}{Continue reading the main story}

\hypertarget{site-index}{%
\subsection{Site Index}\label{site-index}}

\hypertarget{site-information-navigation}{%
\subsection{Site Information
Navigation}\label{site-information-navigation}}

\begin{itemize}
\tightlist
\item
  \href{https://help.nytimes.com/hc/en-us/articles/115014792127-Copyright-notice}{©~2020~The
  New York Times Company}
\end{itemize}

\begin{itemize}
\tightlist
\item
  \href{https://www.nytco.com/}{NYTCo}
\item
  \href{https://help.nytimes.com/hc/en-us/articles/115015385887-Contact-Us}{Contact
  Us}
\item
  \href{https://www.nytco.com/careers/}{Work with us}
\item
  \href{https://nytmediakit.com/}{Advertise}
\item
  \href{http://www.tbrandstudio.com/}{T Brand Studio}
\item
  \href{https://www.nytimes.com/privacy/cookie-policy\#how-do-i-manage-trackers}{Your
  Ad Choices}
\item
  \href{https://www.nytimes.com/privacy}{Privacy}
\item
  \href{https://help.nytimes.com/hc/en-us/articles/115014893428-Terms-of-service}{Terms
  of Service}
\item
  \href{https://help.nytimes.com/hc/en-us/articles/115014893968-Terms-of-sale}{Terms
  of Sale}
\item
  \href{https://spiderbites.nytimes.com}{Site Map}
\item
  \href{https://help.nytimes.com/hc/en-us}{Help}
\item
  \href{https://www.nytimes.com/subscription?campaignId=37WXW}{Subscriptions}
\end{itemize}
