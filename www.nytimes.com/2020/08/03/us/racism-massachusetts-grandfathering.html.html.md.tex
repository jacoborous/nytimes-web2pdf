Sections

SEARCH

\protect\hyperlink{site-content}{Skip to
content}\protect\hyperlink{site-index}{Skip to site index}

\href{https://www.nytimes.com/section/us}{U.S.}

\href{https://myaccount.nytimes.com/auth/login?response_type=cookie\&client_id=vi}{}

\href{https://www.nytimes.com/section/todayspaper}{Today's Paper}

\href{/section/us}{U.S.}\textbar{}Massachusetts Court Won't Use Term
`Grandfathering,' Citing Its Racist Origins

\url{https://nyti.ms/2XlSML5}

\begin{itemize}
\item
\item
\item
\item
\item
\end{itemize}

Advertisement

\protect\hyperlink{after-top}{Continue reading the main story}

Supported by

\protect\hyperlink{after-sponsor}{Continue reading the main story}

\hypertarget{massachusetts-court-wont-use-term-grandfathering-citing-its-racist-origins}{%
\section{Massachusetts Court Won't Use Term `Grandfathering,' Citing Its
Racist
Origins}\label{massachusetts-court-wont-use-term-grandfathering-citing-its-racist-origins}}

The practice was ``adopted by some states after the Civil War in an
effort to disenfranchise African-American voters,'' the court noted.

\includegraphics{https://static01.nyt.com/images/2020/08/03/multimedia/03xp-grandfathering/merlin_135097122_deddcc7f-fe39-4d11-b00f-3b2be907bdef-articleLarge.jpg?quality=75\&auto=webp\&disable=upscale}

\href{https://www.nytimes.com/by/azi-paybarah}{\includegraphics{https://static01.nyt.com/images/2019/02/14/multimedia/author-azi-paybarah/author-azi-paybarah-thumbLarge.png}}

By \href{https://www.nytimes.com/by/azi-paybarah}{Azi Paybarah}

\begin{itemize}
\item
  Published Aug. 3, 2020Updated Aug. 4, 2020, 12:40 a.m. ET
\item
  \begin{itemize}
  \item
  \item
  \item
  \item
  \item
  \end{itemize}
\end{itemize}

\href{https://www.nytimes.com/2020/06/22/us/wine-master.html}{A wine
organization announced} in June that it would no longer use the term
``master'' to refer to its high-ranking experts.

This month, lawmakers in New Jersey said county elected officials should
be
\href{https://www.nytimes.com/2020/07/10/nyregion/Freeholder-new-jersey.html}{called
``commissioners'' instead of ``freeholders,''} a word that dates to a
time when only white males could own land.

And on Monday, an appeals court in Massachusetts said that it would no
longer use the term ``grandfathering'' because ``it has racist
origins.''

As Confederate monuments and other physical symbols of racism have begun
to come down across the country, some commonly used terms have begun to
drop out of circulation.

The latest example of the linguistic updating came in a footnote to
\href{https://www.mass.gov/files/documents/2020/08/03/z19P1163.pdf}{a
decision published on Monday} by the Massachusetts Appeals Court.

In the footnote, the court said the phrase ```grandfather clause'
originally referred to provisions adopted by some states after the Civil
War in an effort to disenfranchise African-American voters by requiring
voters to pass literacy tests or meet other significant qualifications,
while exempting from such requirements those who were descendants of men
who were eligible to vote before 1867.''

The court's footnote was in a case dealing with a local zoning dispute.
Instead of using the phrase, the court referred to ``a certain level of
protection'' provided to ``all structures that predate applicable zoning
restrictions.''

The court's action on Monday comes amid a national debate about racism
and whether to remove statues and public names of historical figures
with links to slavery and oppression.

Some of those figures came to prominence primarily for actions that
perpetuated slavery and inequality, like Confederate Gen.
\href{https://www.nytimes.com/2020/06/18/us/confederate-statues-monuments-removal.html}{Robert
E. Lee} and
\href{https://www.nytimes.com/2018/04/16/nyregion/nyc-sims-statue-central-park-monument.html}{Dr.
J. Marion Sims}, the 19th-century surgeon who performed experiments on
enslaved women.

Statues of Confederate figures have been toppled by demonstrators in
cities across the country. In August, a New York City commission voted
to remove a statue of Dr. Sims in Central Park.

But words with direct links to slavery and oppression may be harder to
detect, and addressing them may best be left to experts, said Nicole R.
Holliday, an assistant professor of linguistics at the University of
Pennsylvania.

``We can't know the etymology of everything,'' Dr. Holliday said.
``That's just too much to ask of speakers.''

Dr. Holliday, who is Black, said she would not correct her own mother if
she used the term ``grandfathered'' in casual conversation, because
doing so would be ``actually rude, and it doesn't accomplish the goal of
creating a more equal society.''

But Dr. Holliday agreed with the Massachusetts judges who identified the
term's roots in suppressing the rights of Black people and decided to no
longer use it. ``This is the legal system and there are wrongs to be
righted,'' she said.

``It was very explicitly a racist legal practice,'' Dr. Holliday said,
noting that the term was an example of ``professional jargon.''

Judges and legal experts can be asked to find a replacement for such
jargon, she said. ``That's not really asking too much in my opinion,''
she added.

Advertisement

\protect\hyperlink{after-bottom}{Continue reading the main story}

\hypertarget{site-index}{%
\subsection{Site Index}\label{site-index}}

\hypertarget{site-information-navigation}{%
\subsection{Site Information
Navigation}\label{site-information-navigation}}

\begin{itemize}
\tightlist
\item
  \href{https://help.nytimes.com/hc/en-us/articles/115014792127-Copyright-notice}{©~2020~The
  New York Times Company}
\end{itemize}

\begin{itemize}
\tightlist
\item
  \href{https://www.nytco.com/}{NYTCo}
\item
  \href{https://help.nytimes.com/hc/en-us/articles/115015385887-Contact-Us}{Contact
  Us}
\item
  \href{https://www.nytco.com/careers/}{Work with us}
\item
  \href{https://nytmediakit.com/}{Advertise}
\item
  \href{http://www.tbrandstudio.com/}{T Brand Studio}
\item
  \href{https://www.nytimes.com/privacy/cookie-policy\#how-do-i-manage-trackers}{Your
  Ad Choices}
\item
  \href{https://www.nytimes.com/privacy}{Privacy}
\item
  \href{https://help.nytimes.com/hc/en-us/articles/115014893428-Terms-of-service}{Terms
  of Service}
\item
  \href{https://help.nytimes.com/hc/en-us/articles/115014893968-Terms-of-sale}{Terms
  of Sale}
\item
  \href{https://spiderbites.nytimes.com}{Site Map}
\item
  \href{https://help.nytimes.com/hc/en-us}{Help}
\item
  \href{https://www.nytimes.com/subscription?campaignId=37WXW}{Subscriptions}
\end{itemize}
