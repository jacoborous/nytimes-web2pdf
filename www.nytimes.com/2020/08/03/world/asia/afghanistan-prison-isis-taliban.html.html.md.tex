Sections

SEARCH

\protect\hyperlink{site-content}{Skip to
content}\protect\hyperlink{site-index}{Skip to site index}

\href{https://www.nytimes.com/section/world/asia}{Asia Pacific}

\href{https://myaccount.nytimes.com/auth/login?response_type=cookie\&client_id=vi}{}

\href{https://www.nytimes.com/section/todayspaper}{Today's Paper}

\href{/section/world/asia}{Asia Pacific}\textbar{}29 Dead After ISIS
Attack on Afghan Prison

\url{https://nyti.ms/31fjSVs}

\begin{itemize}
\item
\item
\item
\item
\item
\end{itemize}

Advertisement

\protect\hyperlink{after-top}{Continue reading the main story}

Supported by

\protect\hyperlink{after-sponsor}{Continue reading the main story}

\hypertarget{29-dead-after-isis-attack-on-afghan-prison}{%
\section{29 Dead After ISIS Attack on Afghan
Prison}\label{29-dead-after-isis-attack-on-afghan-prison}}

The 20-hour gun battle left officials scrambling to recapture hundreds
of prisoners, including many from the Islamic State and the Taliban.

\includegraphics{https://static01.nyt.com/images/2020/08/03/world/03afghan-prison6/merlin_175263168_e480620a-c180-403d-995f-66ba81207750-articleLarge.jpg?quality=75\&auto=webp\&disable=upscale}

By Zabihullah Ghazi and
\href{https://www.nytimes.com/by/mujib-mashal}{Mujib Mashal}

\begin{itemize}
\item
  Published Aug. 3, 2020Updated Aug. 4, 2020, 1:52 a.m. ET
\item
  \begin{itemize}
  \item
  \item
  \item
  \item
  \item
  \end{itemize}
\end{itemize}

JALALABAD, Afghanistan --- A militant
\href{https://www.nytimes.com/2020/08/02/world/asia/afghan-prison-attack-prisoners.html}{assault
on a prison complex} in eastern Afghanistan ended on Monday after a
20-hour gun battle, leaving 29 people dead and officials scrambling to
recapture hundreds of prisoners, including many from the Islamic State
and the Taliban.

The attack at the prison in Jalalabad City was claimed by the Islamic
State. It began on Sunday night when a brief cease-fire between the
Afghan government and the Taliban was still in place. Its timing
underscored the complexity of a conflict that is growing deadlier by the
day, even as peace talks face obstacles.

Gen. Yasin Zia, the chief of the Afghan army who arrived in the city to
lead the last stretch of the operations, said ten assailants were
involved in the attack and all were killed. The security perimeter was
first breached with a car-bomb before attackers with assault rifles
streamed in and started a gun battle with prison guards.

At least 29 people had been killed and 48 others wounded, according to
Attaullah Khogyani, a spokesman for the governor of Nangarhar Province.
The casualties included civilians, inmates and security forces, he said.

\includegraphics{https://static01.nyt.com/images/2020/08/03/world/03afghan-prison-1/merlin_175234137_b40b0e94-cced-4c94-a215-e28b9668c33f-articleLarge.jpg?quality=75\&auto=webp\&disable=upscale}

The assault, which left much of the prison's security barriers destroyed
and brought the city to a standstill, was one of the most complicated
operations claimed by the Islamic State's chapter in Afghanistan.

As its
\href{https://www.nytimes.com/2019/12/02/world/asia/ISIS-afghanistan-baghdadi.html}{territory}
has been constricted significantly by a campaign of military operations
over the past couple of years, the group has largely turned to gruesome
attacks on
\href{https://www.nytimes.com/2017/09/03/world/asia/shoes-kabul-mosque-bombing.html}{soft-targets},
such as civilians with little protection.

But the Islamic State may not be the biggest winner in the jailbreak. A
senior Afghan official, speaking on the condition of anonymity, said
that only a third of the prison's population of about 1,800 included
ISIS loyalists. The rest were split among Taliban prisoners and
criminals. All of them got a chance to break free, at least for a while.

Mr. Khogyani said about 1,000 prisoners who had tried to escape had been
recaptured, and that another 400 --- stuck inside the jail during the
shootout --- had been rescued.

The rest of the population was still unaccounted for, but with the
prison's security still compromised, and with the area under tight
military restrictions, figures were difficult to verify.

The jailbreak came at a time when prisoners in Afghanistan have
frequently been in the news. Conflict over
\href{https://www.nytimes.com/2020/07/28/world/asia/afghanistan-cease-fire-taliban.html}{the
release of Taliban prisoners} as part of an agreement reached in
February between
\href{https://www.nytimes.com/2020/02/29/world/asia/us-taliban-deal.html}{the
United States and the Taliban} has delayed the next steps of the
agreement, mainly the start of direct talks between the Taliban and the
Afghan government.

Image

Afghan security forces walking past a building where the militants were
hiding, following the attack.Credit...Parwiz/Reuters

Afghan prisons have been crucial targets for combatants throughout
decades of war, and during the Taliban insurgency of recent years, the
insurgents have
\href{https://www.nytimes.com/2011/04/26/world/asia/26afghanistan.html}{freed
hundreds of prisoners at a time} in such attacks. On Monday, a Taliban
spokesman denied having anything to do with the attack on the Jalalabad
prison.

Nangarhar has been a stronghold of the Islamic State in Afghanistan. But
intense operations by Afghan forces, often backed by American air power,
has reduced the group's presence significantly. Afghan officials said on
Saturday that they had also killed a senior leader of the group in the
province.

While the Taliban and ISIS have fought bloody turf wars in eastern
Afghanistan, Afghan officials have long claimed that elements of the two
groups have overlapped, at times sharing networks and resources for
urban attacks.

The murky identity of the ISIS branch in the country has made it
\href{https://www.nytimes.com/2019/08/20/world/asia/isis-afghanistan-peace.html}{a
threat to the peace process}. During the first cease-fire between the
Taliban and the Afghan government in 2018, the Islamic State claimed a
deadly bombing in Nangarhar that killed
\href{https://www.nytimes.com/2018/06/16/world/asia/afghanistan-explosion-taliban-ceasefire.html}{nearly
40 people.}

The attack on the jail came during the final hours of a three-day
cease-fire between the Taliban and the Afghan government for the Muslim
festival of Eid al-Adha. Afghan officials said that violence during the
cease-fire had dropped, but the insurgents were still behind 38
incidents over the three days.

Image

Detainees who fled the prison after the militant attack but were
captured by security officials on Monday.Credit...Ghulamullah
Habibi/EPA, via Shutterstock

Afghan security officials warn that the Taliban has exploited a
complicated battlefield to strike blows with deniability so it can
maintain its deal with the U.S., often hiding behind the Islamic State
as cover.

One senior Afghan official said that detained fighters --- who have
their fingerprints checked and eyes scanned --- are often already in the
system as having been previously arrested or associated with attacks
carried out by the Haqqani network, a lethal arm of the Taliban.

Just hours before the cease-fire for Eid al-Adha went into effect on
Thursday, a deadly car-bomb in Logar province killed at least 15 people.
While the Taliban denied they were behind the attack, Massoud Andarabi,
Afghanistan's interior minister, said they had ``solid intelligence''
that the attack was designed by a local Taliban commander in the
province.

Mr. Andarabi said he had little doubt anymore that cells formerly
carrying out urban attacks for the Haqqani network were closely
cooperating with what remains of the Islamic State's chapter in the
country.

``Shahab Almahajir, the newly appointed leader of Islamic State of
Khorasan Province, or I.S.K.P., is a Haqqani member,'' Mr. Andrabi said.
``Haqqani and the Taliban carry out their terrorism on a daily basis
across Afghanistan, and when their terrorist activities do not suit them
politically, they rebrand it under I.S.K.P.''

Zabihullah Mujahid, a spokesman for the Taliban, rejected Mr. Andarabi's
claims as ``rumors to muddy the public perception.''

Image

Smoke billowing from the prison on Monday. It had held 1,500
detainees.Credit...Ghulamullah Habibi/EPA, via Shutterstock

The U.S.-Taliban deal called for the Afghan government to free 5,000
Taliban prisoners in exchange for 1,000 Taliban-held members of the
Afghan security forces. The swap was supposed to take place early this
year over 10 days, after which the Taliban and the Afghan government
were expected to have direct negotiations.

The Afghan government initially resisted the prisoner release, and then
gave in to a phased release following pressure from the Trump
administration. More recently, President Ashraf Ghani, of Afghanistan,
said that he would not release the last 400 of the 5,000 prisoners on a
list provided by the Taliban, as those prisoners were accused of serious
crimes.

It wasn't immediately clear whether any of the 400 prisoners on the list
were held in the prison in Jalalabad.

One Afghan official said that it was possible a few of those Taliban
prisoners were there, but that most high-profile prisoners are held at
the central jail in Kabul, or a highly protected facility near the U.S.
military base in Bagram.

While the Taliban has completed its release of the 1,000 prisoners, Mr.
Ghani has proposed a new deal to settle his end of the bargain. He has
committed to releasing 500 different Taliban members instead of the 400
on the list presented by the insurgents.

Mr. Ghani has called a council of elders from across Afghanistan to
consult on whether to free the 400 accused of grave crimes as well. The
grand consultation, called a Loya Jirga, is expected to happen later
this week.

Zabihullah Ghazi reported from Jalalabad, and Mujib Mashal from Kabul,
Afghanistan.

Advertisement

\protect\hyperlink{after-bottom}{Continue reading the main story}

\hypertarget{site-index}{%
\subsection{Site Index}\label{site-index}}

\hypertarget{site-information-navigation}{%
\subsection{Site Information
Navigation}\label{site-information-navigation}}

\begin{itemize}
\tightlist
\item
  \href{https://help.nytimes.com/hc/en-us/articles/115014792127-Copyright-notice}{©~2020~The
  New York Times Company}
\end{itemize}

\begin{itemize}
\tightlist
\item
  \href{https://www.nytco.com/}{NYTCo}
\item
  \href{https://help.nytimes.com/hc/en-us/articles/115015385887-Contact-Us}{Contact
  Us}
\item
  \href{https://www.nytco.com/careers/}{Work with us}
\item
  \href{https://nytmediakit.com/}{Advertise}
\item
  \href{http://www.tbrandstudio.com/}{T Brand Studio}
\item
  \href{https://www.nytimes.com/privacy/cookie-policy\#how-do-i-manage-trackers}{Your
  Ad Choices}
\item
  \href{https://www.nytimes.com/privacy}{Privacy}
\item
  \href{https://help.nytimes.com/hc/en-us/articles/115014893428-Terms-of-service}{Terms
  of Service}
\item
  \href{https://help.nytimes.com/hc/en-us/articles/115014893968-Terms-of-sale}{Terms
  of Sale}
\item
  \href{https://spiderbites.nytimes.com}{Site Map}
\item
  \href{https://help.nytimes.com/hc/en-us}{Help}
\item
  \href{https://www.nytimes.com/subscription?campaignId=37WXW}{Subscriptions}
\end{itemize}
