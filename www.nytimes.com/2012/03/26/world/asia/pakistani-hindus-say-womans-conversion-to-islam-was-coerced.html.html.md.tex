Sections

SEARCH

\protect\hyperlink{site-content}{Skip to
content}\protect\hyperlink{site-index}{Skip to site index}

\href{https://www.nytimes.com/section/world/asia}{Asia Pacific}

\href{https://myaccount.nytimes.com/auth/login?response_type=cookie\&client_id=vi}{}

\href{https://www.nytimes.com/section/todayspaper}{Today's Paper}

\href{/section/world/asia}{Asia Pacific}\textbar{}In Pakistan, Hindus
Say Woman's Conversion to Islam Was Coerced

\begin{itemize}
\item
\item
\item
\item
\item
\end{itemize}

Advertisement

\protect\hyperlink{after-top}{Continue reading the main story}

Supported by

\protect\hyperlink{after-sponsor}{Continue reading the main story}

\hypertarget{in-pakistan-hindus-say-womans-conversion-to-islam-was-coerced}{%
\section{In Pakistan, Hindus Say Woman's Conversion to Islam Was
Coerced}\label{in-pakistan-hindus-say-womans-conversion-to-islam-was-coerced}}

\includegraphics{https://static01.nyt.com/images/2012/03/26/world/PAKISTAN-1/PAKISTAN-1-articleLarge.jpg?quality=75\&auto=webp\&disable=upscale}

By \href{https://www.nytimes.com/by/declan-walsh}{Declan Walsh}

\begin{itemize}
\item
  March 25, 2012
\item
  \begin{itemize}
  \item
  \item
  \item
  \item
  \item
  \end{itemize}
\end{itemize}

GHOTKI, Pakistan --- Banditry is an old scourge in this impoverished
district of southern Pakistan, on the plains between the mighty river
Indus and a sprawling desert, where roving gangs rob and kidnap with
abandon. Lately, though, local passions have stirred with allegations of
an unusual theft: that of a young woman's heart.

In the predawn darkness on Feb. 24, Rinkel Kumari, a 19-year-old student
from a Hindu family, disappeared from her home in Mirpur Mathelo, a
small village off a busy highway in Sindh Province. Hours later, she
resurfaced 12 miles away, at the home of a prominent Muslim cleric who
phoned her parents with news that distressed them: Their daughter wished
to convert to Islam, he said.

Their protests were futile. By sunset, Ms. Kumari had become a Muslim,
married a young Muslim man, and changed her name to Faryal Bibi.

Over the past month, this conversion has generated an acrid controversy
that has reverberated far beyond its origins in small-town Pakistan,
whipping up a news media frenzy that has traced ugly sectarian divisions
and renewed a wider debate about the protection of vulnerable minorities
in a country that has so often failed them.

At its heart, though, it is a head-on clash of narratives and motives.

Hindu leaders insist that Ms. Kumari was abducted at gunpoint and forced
to abandon her religion. Local Muslim leaders say she wanted to marry
her secret sweetheart: Naveed Shah, a young neighbor who said he had
been conducting a secret courtship with her via mobile phone and the
Internet for several months. Ms. Kumari, for her part, has said in a
court filing and media interviews that she converted of her free will
--- but public figures have questioned whether she had been pressed or
intimidated into saying that.

\includegraphics{https://static01.nyt.com/images/2012/03/26/world/PAKISTAN-2/PAKISTAN-2-jumbo.jpg?quality=75\&auto=webp\&disable=upscale}

The truth may emerge Monday, when the young woman is due to testify
before the Supreme Court in Islamabad. For the past two weeks she has
been sequestered in a women's shelter in Karachi on court orders. When
she takes the stand on Monday, many Pakistanis hope she can resolve the
central mystery: where do her religious, and romantic, intentions lie?

In one sense, the drama is an old story in South Asia, where the
contours of society have been shaped by waves of conversions over the
centuries. Since the founding of Pakistan, most conversions are to
Islam, the state religion. But such conversions usually take place
quietly, even in an organized fashion, and the unusual furor surrounding
the latest case stems partly from the brash manner of her conversion at
the hands of a divisive local politician, Mian Mitho.

After Ms. Kumari declared herself a Muslim in her town court on Feb. 27,
Mr. Mitho triumphantly led the new convert from the courthouse, parading
her before thousands of cheering supporters. Then he drove her in a
caravan to an ancient Sufi religious shrine controlled by his family and
famed as a site where Hindus have been converted.

There, Ms. Kumari was welcomed by Mr. Mitho's elderly brother, Mian
Shaman --- the same cleric who had converted her three days earlier ---
who led her into the towering shrine. When she emerged, now wearing a
black veil, gunmen unleashed volleys of celebratory Kalashnikov fire
into the air and shouted ``God is calling you!''

Hindu leaders, enraged, viewed the images as a crass provocation. ``If
the couple was really in love, then why this fanfare of guns?'' said
Amarnath Motumal, a Hindu lawyer and human rights activist in Karachi.
``It clearly shows they are trying to embarrass the Hindu community and
are bent on taking our girls forcefully.''

Ms. Kumari's parents pursued the case through the courts, claiming that
their daughter had been abducted by a Muslim supremacist, and that the
police and judiciary were biased against them because they came from a
minority background.

Image

Sindh, where Ghotki is located, is known for tolerance.Credit...The New
York Times

``Mian Mitho is a terrorist and a thug. He takes the girls, and keeps
them in his home for sexual purposes,'' said Ms. Kumari's father, Nand
Lal, a government schoolteacher, noting that Mr. Mitho's armed guards
had escorted his daughter to court appearances and news conferences. His
wife, Sulachany Devi, issued an anguished appeal. ``Rinkel was my blood,
and she remains my blood. All I want is for her to return home,'' she
said.

Mr. Mitho, in an interview, denied the allegations against him. ``I am
merely protecting her human rights,'' he said. And at the Sufi shrine in
Ghotki district, his brother, the cleric who converted Ms. Kumari, was
equally unapologetic.

``We are saving them from the fires of hell,'' said Mian Shaman, a frail
man in his 70s with a mottled complexion and a wavering voice. ``We
consider they are born again, and the sins of their previous life are
washed away.''

Mr. Shaman estimated he had converted 200 people the previous year. He
insisted none had been coerced. ``Forced conversions are not permitted
in Islam,'' he said firmly.

Mr. Shaman led the way into the mosque, a spectacular building covered
in intricately patterned indigo tiles and a carved wooden roof. Then he
walked into the adjacent shrine, where murmuring pilgrims rocked back
and forth in front of four tombs containing the bones of the cleric's
ancestors.

Women are not permitted inside, he said --- they may only peek through a
small barred window in the tomb wall --- but he made an exception for
Ms. Kumari. ``She was a special lady,'' he said.

Image

Newspaper clippings show Rinkel Kumari with her new husband, Naveed
Shah.Credit...Sam Phelps for The New York Times

The case has caused division within the ruling Pakistan Peoples Party,
of which Mr. Mitho is a member. Earlier this month, President Asif Ali
Zardari privately intervened to have Ms. Kumari taken into protective
custody. Later, the president's sister, Dr. Azra Fazal Pechuho,
delivered an impassioned speech to Parliament about the plight of the
Hindu community.

``I have a lot of discomfort with this kind of behavior,'' said a senior
party member from Sindh Province, speaking on condition of anonymity
because of the political delicacy of the matter. ``The state is not
giving the Hindus an equal environment. So they are turning to a
narrative of forced conversion to fight back.''

Pir Muhammad Shah, the local police chief, agreed that Mr. Mitho's
actions had aggravated the situation. ``It teased the whole Hindu
community, and led them to believe the conversion had been done at
gunpoint.''

Although Pakistan is blighted by sectarian bloodshed, rural Sindh
Province is a relative beacon of religious tolerance. The majority of
the country's Hindus, estimated to number more than three million
people, live here, and they have a history of tranquil co-existence with
Muslims. The two communities share religious festivals, go into business
together, and attend one another's weddings and funerals.

Yet it remains a delicate social balance. In many Sindhi towns, wealthy
Hindu traders have been targeted by kidnappers. Conversions, which are
freighted with notions of collective honor, can present a jarring social
fault line. Officials with the Human Rights Commission of Pakistan have
spoken of up to 20 forced conversions a month --- and Hindu families
fleeing for India --- but they admit that the research is thin.

As Ms. Kumari's anticipated court date nears, it has revived many old
tensions. And while no one is expecting widespread violence in her case,
in some of its particulars it bears a remarkable resemblance to an
earlier conversion scandal --- one in 1936, when a British magistrate
returned a Hindu girl to her parents after she had been converted. The
result was an 11-year uprising by Muslim Pashtun tribesmen that at one
point involved 40,000 British troops.

Advertisement

\protect\hyperlink{after-bottom}{Continue reading the main story}

\hypertarget{site-index}{%
\subsection{Site Index}\label{site-index}}

\hypertarget{site-information-navigation}{%
\subsection{Site Information
Navigation}\label{site-information-navigation}}

\begin{itemize}
\tightlist
\item
  \href{https://help.nytimes.com/hc/en-us/articles/115014792127-Copyright-notice}{©~2020~The
  New York Times Company}
\end{itemize}

\begin{itemize}
\tightlist
\item
  \href{https://www.nytco.com/}{NYTCo}
\item
  \href{https://help.nytimes.com/hc/en-us/articles/115015385887-Contact-Us}{Contact
  Us}
\item
  \href{https://www.nytco.com/careers/}{Work with us}
\item
  \href{https://nytmediakit.com/}{Advertise}
\item
  \href{http://www.tbrandstudio.com/}{T Brand Studio}
\item
  \href{https://www.nytimes.com/privacy/cookie-policy\#how-do-i-manage-trackers}{Your
  Ad Choices}
\item
  \href{https://www.nytimes.com/privacy}{Privacy}
\item
  \href{https://help.nytimes.com/hc/en-us/articles/115014893428-Terms-of-service}{Terms
  of Service}
\item
  \href{https://help.nytimes.com/hc/en-us/articles/115014893968-Terms-of-sale}{Terms
  of Sale}
\item
  \href{https://spiderbites.nytimes.com}{Site Map}
\item
  \href{https://help.nytimes.com/hc/en-us}{Help}
\item
  \href{https://www.nytimes.com/subscription?campaignId=37WXW}{Subscriptions}
\end{itemize}
