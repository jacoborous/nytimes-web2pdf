Sections

SEARCH

\protect\hyperlink{site-content}{Skip to
content}\protect\hyperlink{site-index}{Skip to site index}

\href{https://myaccount.nytimes.com/auth/login?response_type=cookie\&client_id=vi}{}

\href{https://www.nytimes.com/section/todayspaper}{Today's Paper}

\href{/section/opinion}{Opinion}\textbar{}You Up? College in the Age of
Tinder

\href{https://nyti.ms/2BrUb94}{https://nyti.ms/2BrUb94}

\begin{itemize}
\item
\item
\item
\item
\item
\end{itemize}

Advertisement

\protect\hyperlink{after-top}{Continue reading the main story}

\href{/section/opinion}{Opinion}

Supported by

\protect\hyperlink{after-sponsor}{Continue reading the main story}

On campus

\hypertarget{you-up-college-in-the-age-of-tinder}{%
\section{You Up? College in the Age of
Tinder}\label{you-up-college-in-the-age-of-tinder}}

Some found love; others learned valuable lessons about time stamps.

By Phoebe Lett

\begin{itemize}
\item
  Feb. 14, 2018
\item
  \begin{itemize}
  \item
  \item
  \item
  \item
  \item
  \end{itemize}
\end{itemize}

\includegraphics{https://static01.nyt.com/images/2018/02/14/opinion/14oncampus2/14oncampus2-articleLarge.jpg?quality=75\&auto=webp\&disable=upscale}

It may not be on any syllabus, but college has always been a time for
young people to learn about relationships and sex. But as the internet
increasingly influences the ways we interact, it also transforms how
students date and find partners. We asked students at nine colleges and
universities how technology affects the campus dating scene.

\begin{center}\rule{0.5\linewidth}{\linethickness}\end{center}

\hypertarget{no-one-wants-to-be-known-as-tinder-girl}{%
\subsection{No One Wants to Be Known as Tinder
Girl}\label{no-one-wants-to-be-known-as-tinder-girl}}

\emph{Madeline Apple, University of Michigan, Class of 2018}

Dating apps may have killed the college dating scene. Because it's so
easy to swipe left or right on a seemingly endless pile of potential
partners, it's become harder to actually meet anyone. As students, we
are told over and over that college is a time for us to expand our
social groups, to meet new people and grow into adults. But the
indecisiveness that is built into dating app culture can stunt us ---
we're trapped in an endless cycle of swipes! Commitment, already a scary
concept to many, becomes even more difficult with the false illusion
that the dating possibilities are endless.

Frankly, dating apps can also just make things incredibly awkward. My
freshman year I swiped through hundreds of people. At one of the last
tailgates of the year, a random man walked by me and yelled: ``Hey! We
matched on Tinder! You are Tinder girl!''

I was mortified. Suddenly everyone around me knew that I was on Tinder.
And I had swiped through so many people, I had no idea who this guy was.
He was just another nameless ``match'' that I would never get to know.
Because, needless to say, I walked away and never spoke to that guy
again.

Tinder is supposed to bring people together, but it actually pushes them
emotionally further apart. The fact that there could be hundreds, if not
thousands, of potential dates in your pocket gives an illusion of
possibility. In reality, students just become more isolated in a world
of fake interactions and awkward run-ins with old matches. We're not
getting out of our comfort zone to meet new people. Why approach someone
in person when you can hide behind a Tinder profile?

\begin{center}\rule{0.5\linewidth}{\linethickness}\end{center}

\hypertarget{ladies-check-your-snapchat-time-stamps}{%
\subsection{Ladies, Check Your Snapchat Time
Stamps}\label{ladies-check-your-snapchat-time-stamps}}

\emph{Catherine Gumarin, Mercer University, Class of 2019}

In a romantic comedy, the female lead might scribble her phone number on
a restaurant napkin to demonstrate interest. In college, asking for
someone's Snapchat is more common than asking for his or her digits.
When Brian in the Cosine Upsilon Triathlon Whatever T-shirt starts
flirting in Environmental Communication class, he's after your Snapchat
user name, not your number. While single students at Mercer University
use dating apps like Tinder and Bumble, Snapchat reigns as the most
eye-roll-eliciting app for sparking college romance. To know if Brian is
interested in a serious relationship or a casual fling, read the time
stamp on his flirtatious Snapchat message. The same Snap asking to
``hang out'' sent at 2 p.m. can have a completely different meaning when
sent at 2 a.m.

\begin{center}\rule{0.5\linewidth}{\linethickness}\end{center}

\includegraphics{https://static01.nyt.com/images/2018/02/14/opinion/14oncampus3/14oncampus3-articleLarge.jpg?quality=75\&auto=webp\&disable=upscale}

\hypertarget{we-dont-date-we-netflix-and-chill}{%
\subsection{We Don't Date, We Netflix and
Chill}\label{we-dont-date-we-netflix-and-chill}}

\emph{Mary Walz, University of Iowa, Class of 2020}

College students don't date. Instead, we ``hang out.'' One of the most
popular ways to hang out is to ``Netflix and Chill,'' a trope so common
it became a meme. A typical hangout of the Netflix variety begins with
one student going to another's residence, which is usually small and in
a state of disarray. Next, the couple sit on the bed or futon (in the
case of nicer dorms) and decide what movie or show to watch. This
decision-making process can take up to half an hour and is often the
most stressful time. With so many different genres, there is the problem
of choice. But ultimately the most important consideration --- the
stressful element --- is this question: What will be appropriate
background noise for making out? The wrong choice could kill the mood.
You don't want to be mid-makeout while the jewel-encrusted crab from
``Moana'' is singing about how shiny he is.

\begin{center}\rule{0.5\linewidth}{\linethickness}\end{center}

\hypertarget{the-three-stages-of-hanging-out}{%
\subsection{The Three Stages of Hanging
Out}\label{the-three-stages-of-hanging-out}}

\emph{Cache' Roberts, Miami University, Class of 2021}

If I could tell my younger self one thing upon entering college, it
would be don't expect much from these campus boys. My first encounter
with college dating was with someone who was the exact Urban Dictionary
definitions of unreliable and unpredictable. Eventually his move became
frequent late-night messages. He'd text, ``You got any water?'' What
kind of question is that? It's definitely lame code for ``Can we hang
out?'' and a poor excuse for romance.

Later on I was infatuated with another guy, a charmer, to the point that
I thought it was the start of an actual relationship. From this smooth
talker, I learned the three stages of seriousness in college dating.

The first stage is ``hanging out.'' In this phase you get to know each
other as friends, and sometimes kiss. (Side note: I don't kiss my
friends.) The second stage is ``talking.'' In this phase you are not
exclusive with the person, but you're also not on the market to ``hang
out'' with anyone else. The last stage is ``snatched.'' No, ``snatched''
is not slang for any dubious behavior. It means ``in a relationship''
--- like Facebook-official status. The charmer never wanted to move past
the ``hanging out'' stage, but I hung on for a while. Hopefully, I'll
never make the mistake of investing my time in someone like that again.
The most important lesson in college dating is to make your own
experiences, and not let them make you.

\begin{center}\rule{0.5\linewidth}{\linethickness}\end{center}

Image

Credit...Danielle Chenette

\hypertarget{driving-two-hours-to-date-a-stranger}{%
\subsection{Driving Two Hours to Date a
Stranger}\label{driving-two-hours-to-date-a-stranger}}

\emph{Emma Thom, Sweet Briar College, Class of 2018}

I fell in love with the small classroom environment of Sweet Briar
College and the picturesque scenery of its surroundings in the middle of
nowhere, Virginia. But as a heterosexual female at an all-women's
college, my dating life was nonexistent until I was introduced to Tinder
and Bumble. Initially I hated the concept of dating apps. The upside to
them was blind dates (yikes) and the downside was the opportunity to get
rejected in three seconds or less by a potential match.

But as I began to create my dating profiles, choosing the most
attractive pictures of me and my golden retriever, I started to have
some fun. I hadn't yet warmed up to the idea of driving an hour or two
to grab a drink with a stranger, but the conversations were light and
the attention was wonderful. After hundreds of swipes left and right ---
and plenty of opening lines that received no response --- I finally
matched with a guy I was eager to meet.

He was a Virginia Tech student who seemed intelligent, witty and
happened to be 6-foot-4 --- tall enough for my highest heels.
Conveniently, my best friend is also a student at Tech, so when I told
her about this new guy, she immediately responded with ``Come to
Blacksburg! You can meet up with him, and if he sucks, stay with me.''
So I drove two hours to meet a guy I'd only been messaging for a week
and a half. I'd never heard the sound of his voice, or seen the way he
walked or chewed his food. What would he think about my smile or the
awkward snorting sound I make when I laugh too hard?

I pulled into the parking lot of the Thai restaurant hoping that I
didn't have pit stains and flaking mascara. When I saw him waiting for
me, I almost did a double take --- not because he didn't look like the
guy in the pictures, but because he looked better. He was tall, blond,
with green eyes and a smile wider and more welcoming than I'd imagined.
We had dinner and drinks, and several months later, we're still doing
the same. Dating apps aren't for everyone, but they gave me the
opportunity to meet someone I wasn't sure existed.

\begin{center}\rule{0.5\linewidth}{\linethickness}\end{center}

\hypertarget{i-found-my-first-date-on-an-app}{%
\subsection{I Found My First Date on an
App}\label{i-found-my-first-date-on-an-app}}

\emph{Caleb Keyes, Otterbein University, Class of 2018}

In high school I had always wanted to date but struggled to believe
anyone would want to date me. When I got to college those fears were
compounded by a feeling of trepidation that if I tried to date someone
and we broke up, it would be hard to see them around campus. A friend
encouraged me to download Coffee Meets Bagel, which was described as a
dating app for people who are easily overwhelmed.

I got a date and she suggested we get ice cream, even though it was
snowing outside. It was old-school romantic in a way I hadn't expected.
She looked beautiful with snowflakes falling on her hair and her cheeks
red from the cold.

Though college is often depicted as a place of sexual exploration, and
dating apps seem to encourage passing from one relationship to another,
my generation defies that. A study in the journal
\href{http://onlinelibrary.wiley.com/doi/10.1111/cdev.12930/abstract}{Child
Development} found that 18-year-olds today are less likely to have dated
than 15-year-olds in the 1990s. The good news is, even if we're dating
later, it's no less magical to stand in the snow with someone you like,
as the world seems to stop.

\begin{center}\rule{0.5\linewidth}{\linethickness}\end{center}

Image

Credit...Danielle Chenette

\hypertarget{losing-irl-relationships-to-someone-on-the-screen}{%
\subsection{Losing IRL Relationships to Someone on the
Screen}\label{losing-irl-relationships-to-someone-on-the-screen}}

\emph{Roxanne Powell, San Jose State University, Class of 2018}

There is something to be said for technology and the way it has made our
lives easier. But for all the time we spend on our devices, talking and
looking at people across the country or globe, we can miss the people
right in front of us. Sure, you can be attracted to someone online, but
without meeting them in person, looking them in the eyes, holding their
hand or giving them a hug, how can you know if that connection holds up
IRL?

Someone I was dating made a friend online which developed into something
more, and I was blindsided by it. It was painful to see the person I
cared about, the person I saw a future with, share more of his time with
someone he had never met than with me.

I kept wondering what I had done wrong, what I could have done
differently, what this other person might have that I lacked. But the
more I thought about it, the more I realized that the flexibility of an
online relationship simply seemed easier to him. I couldn't compete with
someone who could be accessed with the push of a button. Nor do I want
to.

\begin{center}\rule{0.5\linewidth}{\linethickness}\end{center}

\hypertarget{hope-hes-not-a-serial-killer}{%
\subsection{Hope He's Not a Serial
Killer}\label{hope-hes-not-a-serial-killer}}

\emph{Caroline Roddy, Bates College, Class of 2021}

\emph{Ping! You have a new match. Be the first one to say hello.}

During my first semester at Bates College I matched with a guy on Tinder
who plays the same sport as me, ice hockey, and also has a Labrador
retriever. Even though he lived an hour away, we agreed to meet at my
college, and later go on a surprise adventure. He drove up in a car with
a custom license plate and a CD collection stocked with Black Eyed Peas
albums and obscure metal bands. We embarked on our adventure and were
driving down a rural road in Maine when he suddenly pulled over.
``Great,'' I thought. ``I've managed to get into the hands of a serial
killer. What will my mother say now?'' He led me on a hike along a trail
to a quarry. It wasn't ideal for a first date: The exercise, coupled
with the get-to-know-you conversation, left me out of breath and
sounding like a dying cat.

As we walked along, I tried to gauge his interest in politics, mumbling
something about the upcoming local election and telling him that one of
the candidates went to my college. He didn't seem interested in this
tidbit, but otherwise, we had a good time together. We found out we both
enjoyed the artist Lorde and shared a love of Thai food. Eventually, we
turned around and he dropped me back off on campus.

After exchanging occasional texts for a month, I received a message from
him: ``Hey so can I ask you something?''

I hesitated, thinking: ``Is he defining the relationship already? That
was quick.''

I replied with a cool, ``yea what's up?'' Casual enough, I thought.
Unassuming.

He told me he's not liberal so we should avoid talking about politics.

Ah, right. Not a serial killer, but perhaps a Trump voter. That
relationship ended there.

\begin{center}\rule{0.5\linewidth}{\linethickness}\end{center}

\hypertarget{snail-mail-keeps-love-alive-from-a-distance}{%
\subsection{Snail Mail Keeps Love Alive From a
Distance}\label{snail-mail-keeps-love-alive-from-a-distance}}

\emph{Kasey Roper, University of Virginia, Class of 2021}

I'm a freshman at the University of Virginia, but my girlfriend attends
college out West. In order to sustain our relationship we rely on
technology and the Postal Service. Technology has definitely made
maintaining a relationship easier, since we can talk frequently and
immediately. But it is also prone to glitches: Messages sometimes don't
send or they get cut off because of the Apple-Android divide, which,
coupled with the fact that I refuse to update iOS, leads to accidental
miscommunication.

If we're in the middle of an important conversation, that ``unsent''
message can cause a lot of hurt feelings that don't just disappear when
one of us explains that ``I wasn't ignoring you, the message just didn't
send.'' It's a major inconvenience, but we have learned to be
understanding about it.

The saving grace of a long-distance relationship is the letters. About
every two weeks, I get an email saying I have a package, and, unless
it's the beginning of the semester and my textbooks haven't come in yet,
I know it's from her. I eagerly wait until my classes are over for the
day and rush to the mailroom to pick it up. Then I hide out in my room,
my desk full of reminders of her --- a pride flag made out of Legos, our
initials spelled out in thumbtacks, pictures of us --- and read the
letter. In these notes to each other we say everything that needs to be
expressed more intimately than can be said over a text or a video chat,
as well as random thoughts we've had that get lost in everyday
conversation. We also send care packages to cheer each other up during
difficult times. She recently sent me a mixtape of songs relevant to our
relationship, and I made one for her, too.

Advertisement

\protect\hyperlink{after-bottom}{Continue reading the main story}

\hypertarget{site-index}{%
\subsection{Site Index}\label{site-index}}

\hypertarget{site-information-navigation}{%
\subsection{Site Information
Navigation}\label{site-information-navigation}}

\begin{itemize}
\tightlist
\item
  \href{https://help.nytimes.com/hc/en-us/articles/115014792127-Copyright-notice}{©~2020~The
  New York Times Company}
\end{itemize}

\begin{itemize}
\tightlist
\item
  \href{https://www.nytco.com/}{NYTCo}
\item
  \href{https://help.nytimes.com/hc/en-us/articles/115015385887-Contact-Us}{Contact
  Us}
\item
  \href{https://www.nytco.com/careers/}{Work with us}
\item
  \href{https://nytmediakit.com/}{Advertise}
\item
  \href{http://www.tbrandstudio.com/}{T Brand Studio}
\item
  \href{https://www.nytimes.com/privacy/cookie-policy\#how-do-i-manage-trackers}{Your
  Ad Choices}
\item
  \href{https://www.nytimes.com/privacy}{Privacy}
\item
  \href{https://help.nytimes.com/hc/en-us/articles/115014893428-Terms-of-service}{Terms
  of Service}
\item
  \href{https://help.nytimes.com/hc/en-us/articles/115014893968-Terms-of-sale}{Terms
  of Sale}
\item
  \href{https://spiderbites.nytimes.com}{Site Map}
\item
  \href{https://help.nytimes.com/hc/en-us}{Help}
\item
  \href{https://www.nytimes.com/subscription?campaignId=37WXW}{Subscriptions}
\end{itemize}
