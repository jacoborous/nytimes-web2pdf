Sections

SEARCH

\protect\hyperlink{site-content}{Skip to
content}\protect\hyperlink{site-index}{Skip to site index}

\href{https://www.nytimes.com/section/politics}{Politics}

\href{https://myaccount.nytimes.com/auth/login?response_type=cookie\&client_id=vi}{}

\href{https://www.nytimes.com/section/todayspaper}{Today's Paper}

\href{/section/politics}{Politics}\textbar{}Democrats Capture Control of
House; G.O.P. Holds Senate

\url{https://nyti.ms/2yUnsHg}

\begin{itemize}
\item
\item
\item
\item
\item
\item
\end{itemize}

Advertisement

\protect\hyperlink{after-top}{Continue reading the main story}

Supported by

\protect\hyperlink{after-sponsor}{Continue reading the main story}

\hypertarget{democrats-capture-control-of-house-gop-holds-senate}{%
\section{Democrats Capture Control of House; G.O.P. Holds
Senate}\label{democrats-capture-control-of-house-gop-holds-senate}}

\includegraphics{https://static01.nyt.com/images/2018/11/07/us/politics/07midterms-dems/07midterms-dems-videoSixteenByNine3000-v3.jpg}

By \href{https://www.nytimes.com/by/jonathan-martin}{Jonathan Martin}
and \href{https://www.nytimes.com/by/alexander-burns}{Alexander Burns}

\begin{itemize}
\item
  Nov. 6, 2018
\item
  \begin{itemize}
  \item
  \item
  \item
  \item
  \item
  \item
  \end{itemize}
\end{itemize}

Democrats harnessed voter fury toward President Trump to win control of
the House and capture pivotal governorships Tuesday night as liberals
and moderates banded together to deliver a forceful rebuke of Mr. Trump,
even as Republicans held on to their Senate majority by claiming a
handful of conservative-leaning seats.

The two parties each had some big successes in the states. Republican
governors were elected in Ohio and Florida, two important battlegrounds
in Mr. Trump's 2020 campaign calculations. Democrats beat Gov. Scott
Walker, the Wisconsin Republican and a top target, and captured the
governor's office in Michigan --- two states that Mr. Trump carried in
2016 and where the left
\href{https://www.nytimes.com/2018/10/14/us/politics/wisconsin-scott-walker-tony-evers.html}{was
looking}
\href{https://www.nytimes.com/2018/09/16/us/politics/michigan-democrats-labor-unions-trump.html}{to
rebound}.

Propelled by an unusually high turnout that illustrated the intensity of
the backlash against Mr. Trump,
\href{https://www.nytimes.com/2018/11/06/us/politics/election-day.html}{Democrats
claimed at least 26 House seats} on the strength of their support in
suburban and metropolitan districts that were once bulwarks of
Republican power but where voters have recoiled from the president's
demagoguery on race.

Early Wednesday morning Democrats clinched the 218 House seats needed to
take control. There were at least 15 additional tossup seats that had
yet to be called.

From the suburbs of Richmond to the subdivisions of Chicago and even
Oklahoma City, an array of diverse candidates --- many of them women,
first-time contenders or both --- stormed to victory and ended the
Republicans' eight-year grip on the House majority.

But in an indication that the political and cultural divisions that
lifted Mr. Trump two years ago may only be deepening, the Democratic
gains did not extend to the Senate, where many of the most competitive
races were in heavily rural states. Republicans were set to build on
their one-seat majority in the chamber by winning Democratic seats in
Indiana, North Dakota and Missouri while
\href{https://www.nytimes.com/2018/11/06/us/ted-cruz-wins-texas-senate-race.html}{turning
back Representative Beto O'Rourke's spirited challenge of Senator Ted
Cruz in Texas}.

\includegraphics{https://static01.nyt.com/images/2018/11/07/us/politics/07midterms2-sub/merlin_146455359_1b4307bc-3e7a-4ea8-97c7-d78df5b24b4b-articleLarge.jpg?quality=75\&auto=webp\&disable=upscale}

\emph{{[}Make sense of the country's political landscape}
\href{https://www.nytimes.com/newsletters/politics?smid=rd\%3Faction\%3Dclick\&module=inline\&pgtype=Article}{\emph{with
our newsletter}}\emph{.{]}}

In two marquee races in the South, progressive African-American
candidates for governor captured the imagination of liberals across the
country. One fell to defeat at the hands of Trump acolytes, and the
other's future was in doubt --- a sign that steady demographic change
across the region was proceeding too gradually to lift Democrats
definitively to victory.

Secretary of State Brian Kemp of Georgia
\href{https://www.nytimes.com/elections/results/georgia-governor}{was
ahead of Stacey Abrams}, who was seeking to become the first black woman
to lead a state; early Wednesday morning, Ms. Abrams suggested the race
might go to a runoff. And former Representative Ron DeSantis
\href{https://www.nytimes.com/2018/11/06/us/florida-desantis-gillum-governor.html}{narrowly
defeated Andrew Gillum}, the mayor of Tallahassee, in the largest
presidential battleground, Florida.

At an election-night celebration in Washington, Nancy Pelosi, the
Democratic minority leader in the House who may soon return to the
office of House speaker, signaled how central the theme of checking Mr.
Trump and cleaning up government was to the party's success.

``When Democrats win --- and we will win tonight --- we will have a
Congress that is open, transparent and accountable to the American
people,'' she proclaimed. ``Are you ready for a great Democratic
victory?''

But at a meeting of Democratic donors and strategists earlier on
Tuesday, she signaled there were lines she would not cross next year.
Attempting to impeach Mr. Trump, she said, was not on the agenda.

Even so, the Democrats' House takeover represented a clarion call that a
majority of the country wants to see limits on Mr. Trump for the next
two years of his term. With the opposition now wielding subpoena power
and the investigation by the special counsel, Robert S. Mueller III,
still looming, the president is
\href{https://www.nytimes.com/2018/11/07/us/politics/trump-house-midterm-election.html}{facing
a drastically more hostile political environment} in the lead up to his
re-election.

Their loss of the House also served unmistakable notice on Republicans
that the rules of political gravity still exist in the Trump era. What
was effectively a referendum on Mr. Trump's incendiary conduct and
hard-right nationalism may make some of the party's lawmakers uneasy
about linking themselves to a president who ended the campaign showering
audiences with a blizzard of mistruths, conspiracy theories and
invective about immigrants.

Image

Representative Beto O'Rourke ran closer than expected against Senator
Ted Cruz, Republican of Texas, thanks to huge turnout throughout the
state.Credit...Todd Heisler/The New York Times

And it revealed that many of the right-of-center voters who backed Mr.
Trump in 2016, as a barely palatable alternative to Hillary Clinton,
were unwilling to give him enduring political loyalty.

The president was initially muted Tuesday night, offering only a terse
statement on Twitter, but then turned more boastful, citing others to
claim that he deserved credit for Republicans who won.

For Democrats, their House triumph was particularly redemptive --- not
only because of how crestfallen they were in the wake of Mrs. Clinton's
defeat but due to how they found success this year.

The president unwittingly galvanized a new generation of activism,
inspiring hundreds of thousands angered, and a little disoriented, by
his unexpected triumph to make their first foray into politics as
volunteers and candidates. He also helped ensure that Democratic
officeholders would more closely reflect the coalition of their party,
and that a woman may take over the House, should Ms. Pelosi secure the
voters to reclaim the speakership.

It was the party's grass roots, however, that seeded Democratic
candidates with unprecedented amounts of small-dollar contributions and
dwarfed traditional party fund-raising efforts. The so-called liberal
resistance was
\href{https://www.nytimes.com/2018/11/06/us/politics/women-midterms-historic.html}{undergirded
by women} and people of color and many of them won on Tuesday, including
Mikie Sherrill in New Jersey, Lauren Underwood in Illinois and Abigail
Spanberger in Virginia.

In next year's session of Congress, there will be 100 women in the House
for the first time in history.

The Democrats' broad gains in the House, and their capture of several
powerful governorships, in many cases represented a vindication of the
party's more moderate wing. The candidates who delivered the House
majority largely hailed from the political center, running on
clean-government themes and promises of incremental improvement to the
health care system rather than transformational social change.

Image

Supporters of the congressional candidate Harley Rouda in Newport Beach,
Calif., watch Representative Nancy Pelosi, the Democrats' leader in the
House, speak on a screen.Credit...Sam Hodgson for The New York Times

To this end, the Democratic gains Tuesday came in many of the country's
most affluent suburbs, communities Mrs. Clinton carried, but they also
surprised Republicans in some more conservative metropolitan areas.
Kendra Horn, for example, pulled off perhaps the upset of the night by
defeating Representative Steve Russell in central Oklahoma.

``Oklahoma City has grown increasingly diverse and today's Republican
Party has little to say to people of color,'' said the city's mayor,
David F. Holt, noting that Mr. Russell sought to broaden his appeal but
``was running against the national message of his party.''

And in a traditionally Republican South Carolina district where
Representative Mark Sanford had lost his primary race in June, a
Democrat, Joe Cunningham, upset a Trump enthusiast, Katie Arrington.

Indeed, the coalition of voters that mobilized against Mr. Trump was
broad, diverse and somewhat ungainly, taking in young people and
minorities who reject his culture-war politics; women appalled by what
they see as his misogyny; seniors alarmed by Republican health care
policies; and upscale suburban whites who support gun control and
environmental regulation as surely as they favor tax cuts. It will now
fall to Democrats to forge these disparate communities alienated by the
president into a durable electoral base for the 2020 presidential race
at a time when their core voters are increasingly tilting left.

Yet the theory --- embraced by hopeful liberals in states like Texas and
Florida --- that charismatic and unapologetically progressive leaders
might transmute Republican bastions into purple political battlegrounds,
proved largely fruitless. Though there were signs that demographic
change was loosening Republicans' grip on the Sun Belt, those changes
did not arrive quickly enough for candidates like Mr. Gillum and Mr.
O'Rourke. And the Democratic collapse in rural areas that began to
plague their candidates under President Obama worsened Tuesday across
much of the political map.

Polling indicated that far more voters than is typical used their
midterm vote to render a verdict on the president, and Mr. Trump
embraced the campaign as a judgment on him: the signs above the stage at
his finally rally in Missouri Monday night read, ``Promises Made,
Promises Kept,'' and made no mention of the candidate he was ostensibly
there to support.

But by maintaining the intense support of his red-state conservative
base, Mr. Trump strengthened his party's hold on the Senate and extended
Republican dominance of several swing states crucial to his re-election
campaign, including Florida, Iowa and Ohio, where the G.O.P. retained
the governorships.

Image

Abigail Spanberger, a Democratic candidate for the House, at her
election night party after winning Virginia's Seventh Congressional
District.Credit...Erin Schaff for The New York Times

Despite how inescapable the president was, Democrats
\href{https://www.nytimes.com/2018/10/28/us/politics/health-care-elections-democrats.html}{carefully
framed the election on policy issues such as health care} to win over
voters who were more uneasy with than hostile to the provocateur in the
White House. There were far more campaign advertisements on the left
about congressional Republicans endangering access to health insurance
for those with pre-existing conditions than there were about a president
who many liberals fear is a menace to American democracy.

While drawing less notice than the fight for control of Congress,
Democrats enjoyed mixed success in something of a revival in the region
that elevated Mr. Trump to the presidency by winning governor's races in
Michigan and Illinois. Beyond the symbolic importance of regaining a
foothold in the Midwest, their state house gains will also offer them a
measure of control over the next round of redistricting.

Drawing as much notice among progressives hungry for a new generation of
leaders was the Senate race in Texas, where Mr. O'Rourke, a 46-year-old
El Paso congressman, eschewed polling and political strategists to run
as an unapologetic progressive in a conservative state undergoing a
demographic shift.

Mr. O'Rourke ran closer than expected against Mr. Cruz thanks to a
historic midterm turnout, and the Democrat's unconventional success
prompted calls for him to seek the presidency long before the polls
closed Tuesday night.

In the states Mr. Trump made a priority --- Florida, Georgia, Indiana,
Missouri --- he came away with several marquee victories for Senate and
governor. But in parts of the country with many college-educated white
voters, some of whom supported Mr. Trump in 2016, his style of
leadership and his singular focus on immigration in the last weeks of
the campaign contributed to Republican House losses.

Among the major races of the night,
\href{https://www.nytimes.com/elections/results/indiana-senate}{Senator
Joe Donnelly of Indiana},
\href{https://www.nytimes.com/2018/11/06/us/politics/josh-hawley-claire-mccaskill-missouri.html}{Senator
Claire McCaskill of Missouri} and
\href{https://www.nytimes.com/2018/11/06/us/politics/heidi-heitkamp-kevin-cramer-north-dakota.html}{Senator
Heidi Heitkamp of North Dakota}, three moderate Democrats in
increasingly conservative states, were decisively defeated thanks to
Republican strength in small towns and rural areas. In Tennessee,
Representative Marsha Blackburn, a conservative Republican,
\href{https://www.nytimes.com/elections/results/tennessee-senate}{was
dominating former Gov. Phil Bredesen} in the middle and western parts of
the state that were once Democratic strongholds.

The Democrats flipped the Senate seat in Nevada, with Representative
Jacky Rosen beating Senator Dean Heller, the chamber's most endangered
Republican this year.

Image

Josh Hawley, the Republican attorney general of Missouri, after winning
his race challenging Senator Claire McCaskill.Credit...Ryan Christopher
Jones for The New York Times

In addition to beating Wisconsin's Mr. Walker, Democrats also elected
Gretchen Whitmer as governor of Michigan, a former State Senate leader
who is seen as a rising star in the party. Illinois voters
\href{https://www.nytimes.com/elections/results/illinois-governor}{elected
J.B. Pritzker}, a Democrat and Hyatt hotel heir, over the embattled
governor, Bruce Rauner.

The night began with a result in Kentucky that suggested a night of
mixed results. Republicans staved off an early setback in a
conservative-leaning House district in central Kentucky, as
\href{https://www.nytimes.com/2018/11/06/us/politics/andy-barr-amy-mcgrath-kentucky.html}{Representative
Andy Barr repelled a fierce challenge from Amy McGrath}, a former
fighter pilot running as a Democrat. Mr. Barr's survival offered some
hope to Republicans that they could hang on to a small majority in the
House.

Many voters were waiting to see if the country would place a check on
Mr. Trump and Republican power in Washington, and if antagonism toward
the president would fuel a wave of Republican losses. But just as Mr.
Trump shocked many Americans with his victory in the Electoral College
in 2016, the possibility that he might receive a political boost Tuesday
with Republican wins in the Senate --- if not a mandate for the next two
years --- was a bracing thought for Democrats, and an energizing one for
Republicans.

In Chapmanville, W.Va., a hardware store worker, Chance Bradley, said he
was voting Republican because Mr. Trump had made him ``feel like an
American again.'' But Carl Blevins, a retired coal miner, voted
Democratic and said he didn't understand how anybody could support Mr.
Trump --- or, for that matter, the Republican candidate for Senate
there, Patrick Morrisey, who went on to lose to Senator Joe Manchin.

``I think they put something in the water,'' Mr. Blevins said.

Mr. Trump had appeared sensitive in recent days to the possibility that
losing the House might be seen as a repudiation of his presidency, even
telling reporters that he has been more focused on the Senate than on
the scores of contested congressional districts where he is unpopular.
And Mr. Trump insisted that he would not take the election results as a
reflection on his performance.

``I don't view this as for myself,'' Mr. Trump said on Sunday, adding
that he believed he had made a ``big difference'' in a handful of Senate
elections.

\href{https://www.nytimes.com/interactive/2018/09/28/us/politics/the-campaign-reporter-ul.html?src=hpPromoHeadline}{}

\hypertarget{sign-up-for-the-campaign-reporter}{%
\subsection{Sign up for The Campaign
Reporter}\label{sign-up-for-the-campaign-reporter}}

\includegraphics{https://int.nyt.com/newsgraphics/push-interactive/projects/campaign-reporter/avatars/alex_burns.png}

Hey, I'm Alex Burns, a politics correspondent for The Times. Send me
your questions using the NYT app. I'll give you the latest intel from
the campaign trail.

Sign up via push alert

Early exit polls of voters, released by CNN on Tuesday night, showed a
mixed assessment of President Trump as well as of Democratic leaders,
and a generally gloomy mood in the country after months of tumultuous
campaigning marked by racial tensions and spurts of violence.

Overall, 39 percent of voters said they went to the polls to express
their opposition to the president, while 26 percent said they wanted to
show support for him. Thirty-three percent said Mr. Trump was not a
factor in their vote.

Advertisement

\protect\hyperlink{after-bottom}{Continue reading the main story}

\hypertarget{site-index}{%
\subsection{Site Index}\label{site-index}}

\hypertarget{site-information-navigation}{%
\subsection{Site Information
Navigation}\label{site-information-navigation}}

\begin{itemize}
\tightlist
\item
  \href{https://help.nytimes.com/hc/en-us/articles/115014792127-Copyright-notice}{©~2020~The
  New York Times Company}
\end{itemize}

\begin{itemize}
\tightlist
\item
  \href{https://www.nytco.com/}{NYTCo}
\item
  \href{https://help.nytimes.com/hc/en-us/articles/115015385887-Contact-Us}{Contact
  Us}
\item
  \href{https://www.nytco.com/careers/}{Work with us}
\item
  \href{https://nytmediakit.com/}{Advertise}
\item
  \href{http://www.tbrandstudio.com/}{T Brand Studio}
\item
  \href{https://www.nytimes.com/privacy/cookie-policy\#how-do-i-manage-trackers}{Your
  Ad Choices}
\item
  \href{https://www.nytimes.com/privacy}{Privacy}
\item
  \href{https://help.nytimes.com/hc/en-us/articles/115014893428-Terms-of-service}{Terms
  of Service}
\item
  \href{https://help.nytimes.com/hc/en-us/articles/115014893968-Terms-of-sale}{Terms
  of Sale}
\item
  \href{https://spiderbites.nytimes.com}{Site Map}
\item
  \href{https://help.nytimes.com/hc/en-us}{Help}
\item
  \href{https://www.nytimes.com/subscription?campaignId=37WXW}{Subscriptions}
\end{itemize}
