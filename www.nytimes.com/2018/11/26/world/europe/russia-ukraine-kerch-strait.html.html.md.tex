Sections

SEARCH

\protect\hyperlink{site-content}{Skip to
content}\protect\hyperlink{site-index}{Skip to site index}

\href{https://www.nytimes.com/section/world/europe}{Europe}

\href{https://myaccount.nytimes.com/auth/login?response_type=cookie\&client_id=vi}{}

\href{https://www.nytimes.com/section/todayspaper}{Today's Paper}

\href{/section/world/europe}{Europe}\textbar{}Russia-Ukraine Fight Over
Narrow Sea Passage Risks Wider War

\url{https://nyti.ms/2DL4FkX}

\begin{itemize}
\item
\item
\item
\item
\item
\item
\end{itemize}

Advertisement

\protect\hyperlink{after-top}{Continue reading the main story}

Supported by

\protect\hyperlink{after-sponsor}{Continue reading the main story}

\hypertarget{russia-ukraine-fight-over-narrow-sea-passage-risks-wider-war}{%
\section{Russia-Ukraine Fight Over Narrow Sea Passage Risks Wider
War}\label{russia-ukraine-fight-over-narrow-sea-passage-risks-wider-war}}

\includegraphics{https://static01.nyt.com/images/2018/11/29/world/29ukraine-2-print/merlin_147360057_8475bbf5-ada7-4696-8479-729a113b1abc-articleLarge.jpg?quality=75\&auto=webp\&disable=upscale}

By \href{https://www.nytimes.com/by/neil-macfarquhar}{Neil MacFarquhar}

\begin{itemize}
\item
  Nov. 26, 2018
\item
  \begin{itemize}
  \item
  \item
  \item
  \item
  \item
  \item
  \end{itemize}
\end{itemize}

MOSCOW --- Ukraine's president put his nation on a war footing with
Russia on Monday, as tensions over a shared waterway escalated into a
crisis that dragged in NATO and the United Nations.

Russia's seizure a day earlier of three small Ukrainian naval vessels
and 23 sailors --- including at least three wounded in a shooting by the
Russian side --- was the first overt armed conflict between the two
sides since the beginning days of the conflict in 2014, when Russian
special forces occupied Crimea.

The opening of an additional front at sea, even if Ukraine lacks a real
navy, introduced an unstable element into what had been a shadowy war.
The conflict pitting Ukrainian soldiers against Russian-backed
separatists in the breakaway Donbas region, in eastern Ukraine, has
sputtered along for almost five years with more than 10,000 people
killed.

The Kremlin, along with some Ukrainian opposition figures, called the
martial drumbeats echoing from Kiev a domestic political ploy by its
embattled president, Petro O. Poroshenko. They accused him of
fearmongering in order to delay or at least reconfigure the March 31
election that he had seemed certain to lose.

Mr. Poroshenko delivered a speech to Ukraine's Parliament asking it to
approve the declaration of martial law starting on Wednesday, with the
military already on full alert. The attack on the naval vessels near the
shared waterway, the Kerch Strait, represented a new stage of aggression
in what he called Russia's ``hybrid war'' against Ukraine.

``This is a bold and frank participation of the regular units of the
Russian Federation, their demonstrative attack on the detachment of the
Ukrainian Armed Forces,'' Mr. Poroshenko said. ``This is a qualitatively
different situation, a qualitatively different threat.''

Members of 450-member Verkhovna Rada, the Parliament, who were present
voted overwhelmingly to support the measure --- 276 to 30 --- after the
president agreed to dilute its scope.

Ukraine also received a boost from the international reaction,
underscoring both the isolation of Russia from the West over the Ukraine
conflict, and the desire to protect the international maritime
convention that allows for unimpeded shipping through any strait.

\includegraphics{https://static01.nyt.com/images/2018/11/27/world/27ukraine-1-print/merlin_147354027_699211d2-2cbc-427d-a3c3-13e48d3b66cf-articleLarge.jpg?quality=75\&auto=webp\&disable=upscale}

``What you saw yesterday was very serious, because you saw actually that
Russia used military force in an open way,'' said NATO's secretary
general, Jens Stoltenberg, during a news conference in Brussels
following a meeting requested by Ukraine. ``This is escalating the
situation in the region and confirms a pattern of behavior which we have
seen over several years.''

NATO was increasing its military presence in the area, he said, calling
on Russia to allow freedom of navigation for Ukrainian ships in the
Kerch Strait.

At the United Nations, Russia called a session of the Security Council
in an attempt to force a discussion about what it called Ukrainian
violations of Russian territorial waters. But Western nations quickly
turned the session into a long criticism of Russia for its actions
against Ukraine since 2014.

``Impeding Ukraine's lawful transit through the Kerch Strait is a
violation under international law. It is an arrogant act that the
international community must condemn and will never accept,'' Ambassador
Nikki R. Haley of the United States told the council.

``As President Trump said many times, the United States would welcome a
normal relationship with Russia, but outlaw actions like this one
continue to make that impossible,'' she said.

Mr. Trump offered his own criticism --- without specifically blaming
Russia ---
\href{https://www.reuters.com/video/2018/11/26/trump-does-not-like-russia-ukraine-situa?videoId=485722993\&videoChannel=117760\&channelName=World+News}{when
asked by reporters} in Washington about the naval confrontation. ``Not
good. We're not happy about it at all,'' he said. ``We do not like
what's happening either way.''

Various European capitals also criticized Russia, calling for it to
release the seized vessels and their crews. There were scattered calls
for new sanctions against Russia.

The Russians seemed to try to tamp down the confrontation, moving an old
cargo vessel anchored to block passage through the Kerch Strait, and
allowing commercial traffic to resume.

The Kremlin remained largely silent for much of the day. It was left to
Foreign Minister Sergey V. Lavrov to
\href{http://www.mid.ru/ru/foreign_policy/news/-/asset_publisher/cKNonkJE02Bw/content/id/3420678?p_p_id=101_INSTANCE_cKNonkJE02Bw\&_101_INSTANCE_cKNonkJE02Bw_languageId=en_GB}{address
the issue}, and his ministry accused Ukraine of creating threats to
normal shipping traffic in the strait by violating international
maritime law, and trying to foment a crisis for domestic political
purposes.

Russian political analysts suggested that the Kremlin had no need to
ratchet up its confrontation with Ukraine --- it had already achieved
what it wanted by destabilizing the country through support for the
separatists.

While President Vladimir V. Putin's flagging domestic support may
benefit from a fight with Ukraine, that would be outweighed by the risk
of greater Western sanctions. ``I do not see any benefits for the
Kremlin from this confrontation,'' said Nikolai Petrov, an analyst and
professor of political science.

Asked about events during his daily briefing, Dmitri S. Peskov, Mr.
Putin's spokesman, framed the Russian actions against the Ukrainian
boats as an interception, not an attack.

``The question here is of incursion into the territorial waters of the
Russian Federation by foreign military vessels,'' Mr. Peskov said.
``They entered the territorial waters of Russia without responding to
any queries from our border guards, in no way responded to offers to
make use of pilotage service, and so on and so forth.''

The dispute over the waterway is fundamentally unresolvable because it
hinges on different interpretations of who controls the territorial
waters around the Crimean peninsula. Both sides tried to portray
themselves as determined to protect the normal shipping that the other
side was interfering with. Ukraine had previously sought and been
granted permission for similar passages,
\href{http://www.fsb.ru/fsb/press/message/single.htm\%21id\%3D10438315\%40fsbMessage.html}{according
to official Russian accounts,} but did not this time.

But Ukraine wants to assert its continued sovereignty in areas which
Russia considers its own, analysts said. Controlling passage from the
Black Sea through the Kerch Strait into the Sea of Azov is a key element
in asserting Russia's broader claim to Crimea.

``Moscow clearly seeks to turn the Azov Sea into a Russian basin, and to
use it to bring leverage to bear on Kiev,'' wrote Mark Galeotti, an
expert on Russian intelligence services at the Institute of
International Relations in Prague, on
\href{https://twitter.com/MarkGaleotti/status/1067059451679703040}{Twitter}.
``It wants to demonstrate its capacity to act without having to worry
about external constraint.''

The two sides signed an agreement in 2003 to guarantee free passage
through the strait, but in recent months have been harassing each
other's ships. The port of Mariupol and a couple of others are important
for the Ukrainian economy for exports of steel and grain, as well as for
imports.

Steven Pifer, a former American ambassador to Ukraine, said that the
Kremlin might be testing the level of support for Ukraine using the
waterway. ``They can very easily back off,'' he said. ``But if they
sense the reaction is weak, I think that they will continue the
blockade.''

Image

During a meeting of the Security Council in New York on Monday, United
Nations representatives voted on issues surrounding the conflict between
Russia and Ukraine.Credit...Carlo Allegri/Reuters

This year Mr. Putin
\href{https://www.nytimes.com/2018/05/15/world/europe/putin-russia-crimea-bridge.html}{inaugurated
a \$7.5 billion bridge across the strait,} meant not least to assert its
claim to Crimea with a physical link.

One strategic aspect of the design shown by events on Sunday is that
Russia could block the strait merely by anchoring a cargo ship in the
one opening under the bridge --- 185 meters wide and 35 meters high ---
large enough to allow the passage of ships.

President Poroshenko sought to portray events as part of a larger
assault by Russia. During his speech to Parliament, he waved a sheaf of
papers that he said detailed Ukraine's intelligence about Russian
preparations for a ground offensive. That seemed unlikely, given the
lack of any clear military objective, analysts said.

There was, however, a widespread sense among opposition figures and
analysts that Mr. Poroshenko aimed to put off the March election, noting
that he had not called for martial law during previous points in the
conflict when the fighting was far worse.

Mr. Poroshenko tried to assuage that criticism by cutting the period of
martial law from two months to one, so it would not interfere with the
official start of the campaign season on Dec. 31.

Other compromises mean that the martial law declaration will only affect
the 10 provinces bordering Russia or Transnistria, a breakaway province
of neighboring Moldova, also controlled by Russian-backed forces.

The president also promised that martial law would not be used to curb
civil liberties or to announce a general military mobilization, and that
it would only be enforced in the case of new attacks. Still, the very
prospect of martial law could help boost support for him as a wartime
leader.

Oleg Kashin, a Russian columnist and political analyst, wrote in the
online publication Republic that the expansion of the shooting into the
Sea of Azov seemed more like an extension of the endless skirmishing in
eastern Ukraine than the start of any full-fledged war.

``The Sea of Azov is the most convenient space for the most spectacular
political wrestling,'' he wrote, calling it a ``tiny reservoir'' that
nobody had ever considered a real sea.

He wrote that the sea belongs ``only to Russia and Ukraine, and no
third-party interests will be affected, even if tomorrow the entire
surface of the Sea of Azov goes up in flames.''

Advertisement

\protect\hyperlink{after-bottom}{Continue reading the main story}

\hypertarget{site-index}{%
\subsection{Site Index}\label{site-index}}

\hypertarget{site-information-navigation}{%
\subsection{Site Information
Navigation}\label{site-information-navigation}}

\begin{itemize}
\tightlist
\item
  \href{https://help.nytimes.com/hc/en-us/articles/115014792127-Copyright-notice}{©~2020~The
  New York Times Company}
\end{itemize}

\begin{itemize}
\tightlist
\item
  \href{https://www.nytco.com/}{NYTCo}
\item
  \href{https://help.nytimes.com/hc/en-us/articles/115015385887-Contact-Us}{Contact
  Us}
\item
  \href{https://www.nytco.com/careers/}{Work with us}
\item
  \href{https://nytmediakit.com/}{Advertise}
\item
  \href{http://www.tbrandstudio.com/}{T Brand Studio}
\item
  \href{https://www.nytimes.com/privacy/cookie-policy\#how-do-i-manage-trackers}{Your
  Ad Choices}
\item
  \href{https://www.nytimes.com/privacy}{Privacy}
\item
  \href{https://help.nytimes.com/hc/en-us/articles/115014893428-Terms-of-service}{Terms
  of Service}
\item
  \href{https://help.nytimes.com/hc/en-us/articles/115014893968-Terms-of-sale}{Terms
  of Sale}
\item
  \href{https://spiderbites.nytimes.com}{Site Map}
\item
  \href{https://help.nytimes.com/hc/en-us}{Help}
\item
  \href{https://www.nytimes.com/subscription?campaignId=37WXW}{Subscriptions}
\end{itemize}
