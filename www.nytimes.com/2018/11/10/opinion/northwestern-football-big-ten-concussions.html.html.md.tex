Sections

SEARCH

\protect\hyperlink{site-content}{Skip to
content}\protect\hyperlink{site-index}{Skip to site index}

\href{https://myaccount.nytimes.com/auth/login?response_type=cookie\&client_id=vi}{}

\href{https://www.nytimes.com/section/todayspaper}{Today's Paper}

\href{/section/opinion}{Opinion}\textbar{}I Miss Northwestern Football's
Losing Tradition

\href{https://nyti.ms/2DekD6L}{https://nyti.ms/2DekD6L}

\begin{itemize}
\item
\item
\item
\item
\item
\end{itemize}

Advertisement

\protect\hyperlink{after-top}{Continue reading the main story}

\href{/section/opinion}{Opinion}

Supported by

\protect\hyperlink{after-sponsor}{Continue reading the main story}

Sporting

\hypertarget{i-miss-northwestern-footballs-losing-tradition}{%
\section{I Miss Northwestern Football's Losing
Tradition}\label{i-miss-northwestern-footballs-losing-tradition}}

My college was usually on the bottom of the Big Ten. When I think of the
players' futures, I wish it still was.

By Carmel McCoubrey

Ms. McCoubrey is a Northwestern alumna.

\begin{itemize}
\item
  Nov. 10, 2018
\item
  \begin{itemize}
  \item
  \item
  \item
  \item
  \item
  \end{itemize}
\end{itemize}

\includegraphics{https://static01.nyt.com/images/2018/11/10/opinion/10sportingWeb/10sportingWeb-articleLarge.jpg?quality=75\&auto=webp\&disable=upscale}

The football team at my alma mater, Northwestern, is having a pretty
good season. Once, that would have thrilled me. Now, it just makes me
uneasy.

The first game I attended at N.U. was a doozy: The Wildcats beat
Northern Illinois on Sept. 25, 1982, to break what remains the longest
losing streak
(\href{http://www.espn.com/page2/s/list/colfootball/teams/worst.html}{34
games}) in Division I-A history. My classmates streamed onto the field
at Dyche Stadium to dismantle the goal posts in triumph and deposit them
in Lake Michigan. The team went on to a losing season, though: It had
been a long time since the days when
\href{http://exhibits.library.northwestern.edu/archives/exhibits/football/7.html}{the
future Notre Dame legend Ara Parseghian was its relatively successful
coach}, and even longer since Northwestern had gone to the Rose Bowl.

We would have been delighted if the team had won more games (it didn't
have a winning season until 1995), but we consoled ourselves by taking a
sort of perverse pride in our losses. As the Wildcats were being pounded
by Big Ten opponents --- especially our downstate rival, the University
of Illinois --- the N.U. students in the stands would chant, ``That's
all right, that's O.K., you're going to work for us one day!'' Obnoxious
and classist, yes, but satisfying.

When I worked for the sports section of the campus newspaper, we'd
dutifully write features about the hopes and dreams of the football
players at the beginning of the season. Then, as the season rolled on,
we would just as dutifully record their losses next to accounts of the
exploits of the university's real star athletes: the field hockey team.

And that was all right, that was O.K., because nobody went to
Northwestern for its football prowess. I don't recall ever meeting a
fellow student who regretted taking a pass on Ohio State because the
football there was better. If only a few of our players got jobs in the
N.F.L., that was all right and O.K., too.

So it's been disconcerting in recent years to see Northwestern be
competitive in the Big Ten and regularly appear in bowl games. Right
now, as The Times
\href{https://www.nytimes.com/2018/10/30/sports/college-football-playoff.html}{noted
with bemusement last week}, it leads its division, with a 5-1 conference
record.

The school has invested plenty in the team; a couple of months ago an
indoor practice field on prime lakefront property opened,
\href{https://www.chicagotribune.com/sports/college/ct-spt-northwestern-athletics-facility-football-20180802-story.html}{part
of a \$270 million complex} that Northwestern hopes will lure recruits
and render practices more efficient --- and make the team more
competitive in a conference that has a lucrative television deal. It's a
commonplace for non-athletes to complain about too many resources being
devoted to athletics, but colleges \emph{should} spend money on sports
for a lively campus and to promote students' health.

And there's the problem: Football's not healthy.

So it's been much more than disconcerting that my alma mater's success,
and its big investment in the sport, comes as we are being reminded
every day of the price football players pay in traumatic brain injury.
The
\href{https://www.nytimes.com/interactive/2017/07/25/sports/football/nfl-cte.html}{rash
of chronic traumatic encephalopathy} among N.F.L. players has gotten the
most attention, but
\href{https://www.nytimes.com/2010/09/14/sports/14football.html}{college
players}
\href{https://concussionfoundation.org/CTE-resources/cte-college-football}{are}
\href{https://www.today.com/health/college-football-player-tyler-hilinski-who-died-suicide-had-cte-t131843}{hurt,
too}.
\href{https://www.nytimes.com/2018/10/26/sports/ivy-league-football-dartmouth.html?action=click\&module=RelatedCoverage\&pgtype=Article\&region=Footer}{Some
colleges, like Dartmouth}, are trying ways to reduce these injuries by
eliminating tackling in practice and
\href{http://www.espn.com/college-football/story/_/id/24859819/concussions-drop-ivy-league-football-kickoff-change}{taking
other measures}, but they remain outliers.

News of Northwestern's triumphs now just serves as a reminder that there
are real young men behind those wins whose brains are being battered. I
want the Wildcats to win less so they won't play as much.

It's also making me increasingly uncomfortable about giving to my old
school. When I donate nowadays, I make sure to earmark my gift so it
won't be applied to the football team in some way, but I'm wondering if
I'm still making myself complicit by donating at all to a university
that is willing to risk its students' health and happiness for a share
of television revenue.

The Wildcats play Iowa on Saturday. A victory would put them a step
closer to the Big Ten championship game after the regular season. I'll
root for them to play safe --- and lose.

Carmel McCoubrey is a staff editor in the Opinion section.

\emph{Follow The New York Times Opinion section on}
\href{https://www.facebook.com/nytopinion}{\emph{Facebook}}\emph{,}
\href{http://twitter.com/NYTOpinion}{\emph{Twitter (@NYTopinion)}}
\emph{and}
\href{https://www.instagram.com/nytopinion/}{\emph{Instagram}}\emph{.}

Advertisement

\protect\hyperlink{after-bottom}{Continue reading the main story}

\hypertarget{site-index}{%
\subsection{Site Index}\label{site-index}}

\hypertarget{site-information-navigation}{%
\subsection{Site Information
Navigation}\label{site-information-navigation}}

\begin{itemize}
\tightlist
\item
  \href{https://help.nytimes.com/hc/en-us/articles/115014792127-Copyright-notice}{©~2020~The
  New York Times Company}
\end{itemize}

\begin{itemize}
\tightlist
\item
  \href{https://www.nytco.com/}{NYTCo}
\item
  \href{https://help.nytimes.com/hc/en-us/articles/115015385887-Contact-Us}{Contact
  Us}
\item
  \href{https://www.nytco.com/careers/}{Work with us}
\item
  \href{https://nytmediakit.com/}{Advertise}
\item
  \href{http://www.tbrandstudio.com/}{T Brand Studio}
\item
  \href{https://www.nytimes.com/privacy/cookie-policy\#how-do-i-manage-trackers}{Your
  Ad Choices}
\item
  \href{https://www.nytimes.com/privacy}{Privacy}
\item
  \href{https://help.nytimes.com/hc/en-us/articles/115014893428-Terms-of-service}{Terms
  of Service}
\item
  \href{https://help.nytimes.com/hc/en-us/articles/115014893968-Terms-of-sale}{Terms
  of Sale}
\item
  \href{https://spiderbites.nytimes.com}{Site Map}
\item
  \href{https://help.nytimes.com/hc/en-us}{Help}
\item
  \href{https://www.nytimes.com/subscription?campaignId=37WXW}{Subscriptions}
\end{itemize}
