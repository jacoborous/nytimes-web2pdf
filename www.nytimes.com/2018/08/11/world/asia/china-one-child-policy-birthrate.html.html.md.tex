Sections

SEARCH

\protect\hyperlink{site-content}{Skip to
content}\protect\hyperlink{site-index}{Skip to site index}

\href{https://www.nytimes.com/section/world/asia}{Asia Pacific}

\href{https://myaccount.nytimes.com/auth/login?response_type=cookie\&client_id=vi}{}

\href{https://www.nytimes.com/section/todayspaper}{Today's Paper}

\href{/section/world/asia}{Asia Pacific}\textbar{}Burying `One Child'
Limits, China Pushes Women to Have More Babies

\url{https://nyti.ms/2OuXiAq}

\begin{itemize}
\item
\item
\item
\item
\item
\end{itemize}

Advertisement

\protect\hyperlink{after-top}{Continue reading the main story}

Supported by

\protect\hyperlink{after-sponsor}{Continue reading the main story}

\hypertarget{burying-one-child-limits-china-pushes-women-to-have-more-babies}{%
\section{Burying `One Child' Limits, China Pushes Women to Have More
Babies}\label{burying-one-child-limits-china-pushes-women-to-have-more-babies}}

\includegraphics{https://static01.nyt.com/images/2018/08/11/world/11chinabirths-1/11chinabirths-1-articleLarge.jpg?quality=75\&auto=webp\&disable=upscale}

By \href{https://www.nytimes.com/by/steven-lee-myers}{Steven Lee Myers}
and Olivia Mitchell Ryan

\begin{itemize}
\item
  Aug. 11, 2018
\item
  \begin{itemize}
  \item
  \item
  \item
  \item
  \item
  \end{itemize}
\end{itemize}

\href{https://cn.nytimes.com/china/20180813/china-one-child-policy-birthrate/}{阅读简体中文版}\href{https://cn.nytimes.com/china/20180813/china-one-child-policy-birthrate/zh-hant/}{閱讀繁體中文版}\href{https://www.nytimes.com/es/2018/08/16/china-poblacion-hijo-unico/}{Leer
en español}

BEIJING --- For decades, China harshly restricted the number of babies
that women could have. Now it is encouraging them to have more. It is
not going well.

Almost three years after
\href{https://www.nytimes.com/2015/10/30/world/asia/china-end-one-child-policy.html}{easing
its ``one child'' policy} and allowing couples to have two children, the
government has begun to acknowledge that its efforts to raise the
country's birthrate are faltering because parents are deciding against
having more children.

Officials are now scrambling to devise ways to stimulate a baby boom,
worried that a looming demographic crisis could imperil economic growth
--- and undercut the ruling Communist Party and its leader,
\href{https://www.nytimes.com/2018/07/31/world/asia/xi-jinping-internal-dissent.html}{Xi
Jinping}.

It is a
\href{https://www.nytimes.com/interactive/2015/10/29/world/asia/china-one-child-policy-timeline.html}{startling
reversal for the party}, which only a short time ago imposed punishing
fines on most couples who had more than one child and compelled hundreds
of millions of Chinese women to have abortions or undergo sterilization
operations.

The new campaign has raised fear that China may go from one invasive
extreme to another in getting women to have more children. Some
provinces are already tightening access to abortion or making it more
difficult to get divorced.

``To put it bluntly, the birth of a baby is not only a matter of the
family itself, but also a state affair,'' the official newspaper
People's Daily said in an editorial this week, prompting widespread
criticism and debate online.

\includegraphics{https://static01.nyt.com/images/2018/08/11/world/11chinabirths-2/11chinabirths-2-articleLarge.jpg?quality=75\&auto=webp\&disable=upscale}

In what appeared to be a trial balloon to test public sentiment, the
provincial government in Shaanxi, in central China, last month called on
Beijing \href{http://www.globaltimes.cn/content/1111866.shtml}{to
abolish all birth limits} and let people have as many children as they
want.

The proposal is politically fraught because removing the last remaining
checks on family size would be another reminder that a policy that
touched every Chinese family and
\href{https://www.nytimes.com/2015/11/06/magazine/the-long-shadow-of-chinas-one-child-policy.html}{reshaped
society} --- most Chinese millennials, for example,
\href{https://www.nytimes.com/2015/11/14/world/asia/china-one-child-policy-loneliest-generation.html}{have
no siblings} --- may have been deeply flawed.

``Among regular people, among scholars, there's enough consensus already
about the policy,'' said Wang Huiyao, president of the Center for China
and Globalization, a research organization in Beijing. ``It's just a
matter of time before they can lift this policy.''

A plan to end the two-child limit was floated during the legislative
session in Beijing last spring and now appears to be under consideration
with other measures, the National Health Commission said in a statement.

Experts say the government has little choice but to encourage more
births. China --- the world's most populous nation with more than 1.4
billion people --- is aging quickly, with a smaller work force left to
support a growing elderly population that is living longer. Some
provinces have already reported difficulties meeting pension payments.

It is unclear whether lifting the two-child limit now will make much of
a difference. As in many countries, educated women in Chinese cities are
postponing childbirth as they pursue careers. Young couples are also
struggling with economic pressures, including rising housing and
education costs.

Image

A couple posing for a wedding photographer in Shanghai in May. Officials
in parts of China are proposing new benefits for young families,
including tax breaks and subsidies for housing and
education.Credit...Johannes Eisele/Agence France-Presse --- Getty Images

The ``one child'' policy also resulted in more boys than girls being
born. Some parents obtained abortions because the fetuses were female,
reflecting traditional preferences for male children, though such
selective abortions were illegal. Because of that and other factors,
there are now simply fewer women to marry and bear children.

The number of women between the ages of 20 and 39 is expected to drop by
more than 39 million over the next decade, to 163 million from 202
million, according to He Yafu, a demographer and the author of a book on
the impact of China's population controls.

``Without the introduction of measures to encourage fertility, the
population of China will drop sharply in the future,'' he said.

In advance of any policy changes nationally, local governments are
already taking steps to promote childbirth.

In Liaoning, a province in the northeast with one of the nation's lowest
birthrates, officials last month proposed an array of new benefits for
young families, including tax breaks, housing and education subsidies
and longer maternity and paternity leaves, as well as investments in
clinics and preschools.

In Jiangxi Province, in the southeast, the government has adopted a more
intrusive approach, reissuing guidelines for when women can get
abortions. Though the rules were not new, the move raised fears that the
authorities intend to enforce them more strictly, including a
requirement that women who are more than 14 weeks pregnant obtain three
signatures from medical personnel before an abortion.

Image

The Children Baby Maternity Expo in Shanghai last month. The percentage
of families with two children has climbed from 36 percent in 2013 to 51
percent today, the National Health Commission said.Credit...Giulia
Marchi for The New York Times

Officials said the guidelines were meant to enforce the law prohibiting
couples from aborting a female fetus in hopes of having a boy --- though
they acknowledged that keeping the official birthrate up was also a
consideration.

Two other provinces have
\href{https://www.nytimes.com/2018/05/30/world/asia/china-divorce-quiz.html}{tightened
the requirements for couples to divorce}, saying the changes were made
in part to keep alive the possibility of new offspring.

Such measures have revived longstanding complaints about the
government's invasive control over women's bodies.

``Women cannot decide what happens to their own ovaries,'' one user
complained on Weibo, a popular microblogging platform, after Jiangxi
detailed the abortion guidelines in July.

The ``one child'' policy was introduced in 1979 as a way to slow
population growth and bolster the economic boom that was then just
beginning. The party built a vast bureaucracy of ``planned birth''
workers to enforce the policy, sometimes with violence. Resistance in
the countryside was especially fierce, in part because of a rural
preference for male children who could help with farm work.

In 1984, the government allowed rural couples whose first child was a
girl to have a second child, and there were other exceptions for ethnic
minorities. In 2013, recognizing the implications of an aging
population, the government allowed parents who were only children
themselves to have two children. Two years later, the limit was
\href{https://www.nytimes.com/2015/10/30/world/asia/china-end-one-child-policy.html}{raised
to two children} for everyone, effective Jan. 1, 2016.

Image

Jane Sun, chief executive of the online travel company Ctrip, which
offers financial support to employees who have children or are
considering it. ``We really need to have a sense of urgency'' around
China's birthrate, she said.Credit...Giulia Marchi for The New York
Times

The birthrate jumped that year, reflecting the exuberance of those
longing for a second child, but it dropped again in 2017, prompting the
reconsideration now underway.

One recent government study estimated that China's labor force could
\href{http://www.cppcc.gov.cn/zxww/2018/07/13/ARTI1531443023003519.shtml}{lose
100 million people} from 2020 to 2035, then another 100 million from
2035 to 2050. It warned of pressure on ``economic and social
development,'' budget resources and the environment.

The economic imperatives have prompted some private companies to act on
their own.

Ctrip, the world's second-largest online travel company after Priceline,
already offers a variety of benefits to support parents, like taxi rides
to and from the office during pregnancies and bonuses when employees'
children reach school age. Last month, it announced that it would also
begin subsidizing the cost of freezing the eggs of some managers ---
said to be a first for a Chinese company.

The company's chief executive, Jane Sun, said Ctrip was acting out of a
sense of social responsibility but also responding to economic factors:
A declining population hurts growth. James Liang, a co-founder of Ctrip,
has
\href{https://www.wiley.com/en-us/The+Demographics+of+Innovation\%3A+Why+Demographics+is+a+Key+to+the+Innovation+Race-p-9781119408925}{written
a book} warning of the impact of China's shifting demographics on
technological innovation.

``The generation before us only had one child, so in their mind having
only one child is the normal thing,'' Ms. Sun said in an interview in
the company's Shanghai headquarters.

``I think we really need to have a sense of urgency --- from the top
down and the bottom up --- to encourage families to resume a healthy
birthrate,'' she added.

Image

A new mother, left, with her baby and postpartum nanny in Beijing in
2015. Many Chinese are reluctant to have a second child, citing concerns
about cost and child care.Credit...Adam Dean for The New York Times

In a written response to questions, the National Health Commission said
the ``two child'' policy was working. While the total number of births
dipped to 17.2 million last year --- compared with nearly 17.9 million
in 2016 --- the percentage of families with two children has climbed
from 36 percent in 2013 to 51 percent today, it said.

The commission acknowledged that couples faced many obstacles to having
a second child and said the government was working on policies in areas
like taxation and education that would address them.

``To eliminate the concerns of the masses and sustain the birthrate, we
need to focus on the practical difficulties in fertility and
child-rearing,'' it said.

Demographic experts warn that it will be difficult to change people's
reproductive behavior.

Shang Xiaoyuan, a professor at the University of New South Wales in
Sydney and an expert on child welfare in China, said the government
needed to help the families most likely to have a second or third child.

``This kind of family should be given more support and should have more
invested in child welfare: early education, maternal and child health,''
she said.

Better benefits and services will not be enough to persuade everyone.

Sun Zhongyue, a 27-year-old accountant in Beijing who is pregnant with
her first child, said she had already ruled out having a second, citing
workplace discrimination, the costs of education and the social strains
on extended families.

While grandparents often help with child care in China, the majority of
Ms. Sun's generation are only children who are expected in turn to
support their aging parents.

``Although elders can help us look after the kid, they cannot once their
health worsens,'' she said during a visit to a government office to
obtain reimbursement for her maternity care.

``Raising a child is stressful,'' she added. ``It costs money and
manpower.''

Advertisement

\protect\hyperlink{after-bottom}{Continue reading the main story}

\hypertarget{site-index}{%
\subsection{Site Index}\label{site-index}}

\hypertarget{site-information-navigation}{%
\subsection{Site Information
Navigation}\label{site-information-navigation}}

\begin{itemize}
\tightlist
\item
  \href{https://help.nytimes.com/hc/en-us/articles/115014792127-Copyright-notice}{©~2020~The
  New York Times Company}
\end{itemize}

\begin{itemize}
\tightlist
\item
  \href{https://www.nytco.com/}{NYTCo}
\item
  \href{https://help.nytimes.com/hc/en-us/articles/115015385887-Contact-Us}{Contact
  Us}
\item
  \href{https://www.nytco.com/careers/}{Work with us}
\item
  \href{https://nytmediakit.com/}{Advertise}
\item
  \href{http://www.tbrandstudio.com/}{T Brand Studio}
\item
  \href{https://www.nytimes.com/privacy/cookie-policy\#how-do-i-manage-trackers}{Your
  Ad Choices}
\item
  \href{https://www.nytimes.com/privacy}{Privacy}
\item
  \href{https://help.nytimes.com/hc/en-us/articles/115014893428-Terms-of-service}{Terms
  of Service}
\item
  \href{https://help.nytimes.com/hc/en-us/articles/115014893968-Terms-of-sale}{Terms
  of Sale}
\item
  \href{https://spiderbites.nytimes.com}{Site Map}
\item
  \href{https://help.nytimes.com/hc/en-us}{Help}
\item
  \href{https://www.nytimes.com/subscription?campaignId=37WXW}{Subscriptions}
\end{itemize}
