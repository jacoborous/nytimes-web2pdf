Sections

SEARCH

\protect\hyperlink{site-content}{Skip to
content}\protect\hyperlink{site-index}{Skip to site index}

\href{https://www.nytimes.com/section/technology/personaltech}{Personal
Tech}

\href{https://myaccount.nytimes.com/auth/login?response_type=cookie\&client_id=vi}{}

\href{https://www.nytimes.com/section/todayspaper}{Today's Paper}

\href{/section/technology/personaltech}{Personal Tech}\textbar{}Are
Targeted Ads Stalking You? Here's How to Make Them Stop

\url{https://nyti.ms/2MpX0Nz}

\begin{itemize}
\item
\item
\item
\item
\item
\item
\end{itemize}

Advertisement

\protect\hyperlink{after-top}{Continue reading the main story}

Supported by

\protect\hyperlink{after-sponsor}{Continue reading the main story}

\href{/column/tech-fix}{Tech Fix}

\hypertarget{are-targeted-ads-stalking-you-heres-how-to-make-them-stop}{%
\section{Are Targeted Ads Stalking You? Here's How to Make Them
Stop}\label{are-targeted-ads-stalking-you-heres-how-to-make-them-stop}}

\includegraphics{https://static01.nyt.com/images/2018/08/16/business/16TECHFIX.illo/merlin_142389390_c45da4cf-7276-4959-a437-d2da0e773666-articleLarge.jpg?quality=75\&auto=webp\&disable=upscale}

By \href{http://www.nytimes.com/by/brian-x-chen}{Brian X. Chen}

\begin{itemize}
\item
  Aug. 15, 2018
\item
  \begin{itemize}
  \item
  \item
  \item
  \item
  \item
  \item
  \end{itemize}
\end{itemize}

\href{https://www.nytimes.com/es/2018/08/18/asi-puedes-acabar-con-el-acoso-publicitario/}{Leer
en español}

Online ads have always been annoying, but now they're worse than ever.

Consider what happens when you shop online for a wristwatch. You peruse
a few watch websites and the next thing you know, a watch advertisement
is following you everywhere. On your computer, it's loading in your
Facebook feed. On your phone, it's popping up on Instagram. In your web
browser on either, it's appearing on news sites that have nothing to do
with watches. Even if you end up ordering the watch, the ads continue
trailing you everywhere.

They're stalker ads.

And they are a symptom of how online ads are becoming increasingly
targeted and persistent. Tracking technologies like web cookies are
\href{https://www.nytimes.com/2016/02/18/technology/personaltech/free-tools-to-keep-those-creepy-online-ads-from-watching-you.html}{collecting
information about our browsing activities} from site to site. Marketers
and ad tech companies compile that data to target us across our devices.
And trackers are now so sophisticated that they can see when you are
thinking about buying something but don't follow through --- so they
tell the ads to chase you around so you make the purchase.

To the ad industry, targeted ads are better for people than the old days
of randomly blasting commercials.

``The content isn't free, so what would you rather see?'' said Sarah
Hofstetter, the chairwoman for the ad agency 360i. ``Ads that are at
least trying to be of interest to you, or ads that are spray and pray?''

That's a fair point. On the other hand, these creepy ads can be
extremely annoying, especially when they make incorrect assumptions.
They are another example, along with obnoxious
\href{https://www.nytimes.com/2018/08/01/technology/personaltech/autoplay-video-fight-them.html}{auto-play
videos} and
\href{https://www.nytimes.com/2018/08/08/technology/personaltech/internet-trolls-comments.html}{internet
trolls} taking over internet comments, of how a few bad actors are
breaking the integrity of the web.

Stalker ads also raise privacy concerns. A 2012
\href{http://www.pewinternet.org/2012/03/09/main-findings-11/}{survey}
by Pew Research Center found that 68 percent of internet users did not
like targeted advertising because they do not like having their online
behavior tracked and analyzed. Your browsing history can reveal a lot
about you, including your health issues, political affiliations and
sexual habits.

Fortunately, I have good news. After several years of interviewing
internet companies and privacy experts and testing many web tools, I
finally managed to make my stalker ads go away.

\hypertarget{why-are-ads-stalking-me}{%
\subsection{Why are ads stalking me?}\label{why-are-ads-stalking-me}}

Before you try to exorcise targeted ads, it helps to understand what's
going on behind the scenes.

Let's say you are shopping online for a blender. You load a webpage for
a blender from Brand X, then close the browser. The next time you open
the browser, ads for the blender are following you from site to site.
They're also showing up in some of your mobile apps like Facebook and
Instagram.

When you visited Brand X's website, the site stored a cookie on your
device containing a unique identifier. Brand X hired multiple ad tech
companies to do its marketing. The ad tech companies embedded trackers
that also loaded on Brand X's website, and the trackers took a look at
your cookie to pinpoint your device.

The trackers can tell if you are interested in buying something. They
look for signals --- like if you closed the browser after looking at the
blender for awhile or left the item in the site's shopping cart without
completing the purchase. From there, the ad tech companies can follow
your cookie through trackers and ad networks on various sites and apps
to serve you an ad for the blender.

Ms. Hofstetter said that among ad tech companies, there are good actors
and bad actors. The good ones will try to minimize the chances of
annoying you by showing you the blender ad only a few times and stopping
if they detect that you made the purchase. The bad ones only care to
persuade you to buy the blender, so they will relentlessly serve you the
ad and not bother to determine whether you already bought it.

Things get extra messy when brands employ multiple ad tech companies
that employ different approaches. Perhaps one ad company finished
serving you the blender ad after a few times on Facebook. But elsewhere
on the web or inside another app, another ad tech company served you
that same ad endlessly.

\hypertarget{what-are-some-easy-ways-to-sidestep-stalker-ads}{%
\subsection{What are some easy ways to sidestep stalker
ads?}\label{what-are-some-easy-ways-to-sidestep-stalker-ads}}

Here are a few simple steps you can take if you are being pestered by an
ad and want that to end:

• \textbf{Periodically, clear your cookies.} Ad trackers will have a
tougher time following you around if you delete your cookies on each of
your devices. Apple, Google and Microsoft have published instructions on
how to clear data for their browsers
\href{https://support.apple.com/kb/PH21411?locale=en_US}{Safari},
\href{https://support.google.com/accounts/answer/32050?co=GENIE.Platform\%3DDesktop\&hl=en}{Chrome}
and\href{https://privacy.microsoft.com/en-us/windows-10-microsoft-edge-and-privacy}{Edge}.
(Click the links for instructions.)

• \textbf{Reset your advertising ID.} In addition to cookies, Android
and Apple phones use a so-called advertising ID to help marketers track
you. You can reset it whenever you want. On Android devices, you can
find the reset button in the ads menu inside the Google settings app,
and on iPhones, you can find the reset button inside the settings app in
the privacy menu, under advertising.

• \textbf{Periodically purge your Google ad history.} Google offers the
My Activity tool,
\href{https://myactivity.google.com/}{myactivity.google.com}, where you
can take a deep look at the data that Google has stored about you,
including the history of ads you have loaded, and choose the data you
want to delete.

• \textbf{If possible, hide the annoying ad.} On some web ads, like
those served by Google and Facebook, there is a tiny button in the
top-right corner that you can click on to hide the ad.

\hypertarget{can-i-bring-that-up-a-notch}{%
\subsection{Can I bring that up a
notch?}\label{can-i-bring-that-up-a-notch}}

There are more extreme methods to take if you want to prevent targeted
ads from ever following you around. But this isn't for the faint of
heart: In my experience, you have to take all, not just some, of these
steps to get the pesky ads to leave you alone forever.

• \textbf{Install an ad blocker.} For your web browser, you can install
add-ons that block ads. My favorite one for computer browsers is
\href{https://chrome.google.com/webstore/detail/ublock-origin/cjpalhdlnbpafiamejdnhcphjbkeiagm?hl=en}{uBlock
Origin}, and on iPhones I recommend
\href{https://itunes.apple.com/us/app/1blocker-x/id1365531024?mt=8}{1Blocker
X}. (For Android users, Google banned many ad blockers from its official
Play app store, so the simplest way to block ads is by using a private
web browser.)

• \textbf{On mobile devices, use a private browser.}
\href{https://www.mozilla.org/en-US/firefox/mobile/\#mobile-download-buttons-focus}{Firefox
Focus},
\href{https://play.google.com/store/apps/details?id=com.duckduckgo.mobile.android\&hl=en_US}{DuckDuckGo}
and
\href{https://itunes.apple.com/us/app/ghostery-privacy-browser/id472789016?mt=8}{Ghostery
Privacy Browser} are privacy-centric mobile browsers that include
built-in ad and tracker blocking. These are handy when you want to do a
discreet web search. (They can be impractical to use as everyday
browsers because the built-in blockers can break important parts of
websites.)

• \textbf{Install a tracker blocker.} Tracker blockers detect snoopy
code on websites and prevent them from loading. My favorite tracker
blocker for desktop and mobile systems is
\href{https://disconnect.me/}{Disconnect.me}.

• \textbf{Wherever you can, opt out of interest-based advertising.} Tech
companies including
\href{https://support.google.com/ads/answer/2662922?hl=en}{Google},
\href{https://www.facebook.com/help/1075880512458213/?helpref=hc_fnav}{Facebook},
\href{https://business.twitter.com/en/help/ads-policies/other-policy-requirements/interest-based-opt-out-policy.html}{Twitter}
and \href{https://support.apple.com/en-us/HT202074}{Apple} offer
instructions on opting out of receiving ads based on your interests.

It will probably take you a couple of hours to set yourself up to
prevent ads from haunting you. I gradually made all these changes to my
devices and internet accounts over the last few years and only recently
stopped seeing targeted ads. It was a grueling process.

But I've been happy with the results. Those wristwatch ads that once
followed me are gone. And recently, I was served an ad for 7-11 on
Instagram.

Was that ad irrelevant to me? Yes. But was it a sign that I was no
longer being well tracked? Also yes. I confess I was pretty happy to see
it.

Advertisement

\protect\hyperlink{after-bottom}{Continue reading the main story}

\hypertarget{site-index}{%
\subsection{Site Index}\label{site-index}}

\hypertarget{site-information-navigation}{%
\subsection{Site Information
Navigation}\label{site-information-navigation}}

\begin{itemize}
\tightlist
\item
  \href{https://help.nytimes.com/hc/en-us/articles/115014792127-Copyright-notice}{©~2020~The
  New York Times Company}
\end{itemize}

\begin{itemize}
\tightlist
\item
  \href{https://www.nytco.com/}{NYTCo}
\item
  \href{https://help.nytimes.com/hc/en-us/articles/115015385887-Contact-Us}{Contact
  Us}
\item
  \href{https://www.nytco.com/careers/}{Work with us}
\item
  \href{https://nytmediakit.com/}{Advertise}
\item
  \href{http://www.tbrandstudio.com/}{T Brand Studio}
\item
  \href{https://www.nytimes.com/privacy/cookie-policy\#how-do-i-manage-trackers}{Your
  Ad Choices}
\item
  \href{https://www.nytimes.com/privacy}{Privacy}
\item
  \href{https://help.nytimes.com/hc/en-us/articles/115014893428-Terms-of-service}{Terms
  of Service}
\item
  \href{https://help.nytimes.com/hc/en-us/articles/115014893968-Terms-of-sale}{Terms
  of Sale}
\item
  \href{https://spiderbites.nytimes.com}{Site Map}
\item
  \href{https://help.nytimes.com/hc/en-us}{Help}
\item
  \href{https://www.nytimes.com/subscription?campaignId=37WXW}{Subscriptions}
\end{itemize}
