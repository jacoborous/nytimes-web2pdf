Sections

SEARCH

\protect\hyperlink{site-content}{Skip to
content}\protect\hyperlink{site-index}{Skip to site index}

\href{https://www.nytimes.com/section/politics}{Politics}

\href{https://myaccount.nytimes.com/auth/login?response_type=cookie\&client_id=vi}{}

\href{https://www.nytimes.com/section/todayspaper}{Today's Paper}

\href{/section/politics}{Politics}\textbar{}To Rally Voters, Democrats
Focus on Health Care as Their Closing Argument

\url{https://nyti.ms/2yFcy8e}

\begin{itemize}
\item
\item
\item
\item
\item
\end{itemize}

Advertisement

\protect\hyperlink{after-top}{Continue reading the main story}

Supported by

\protect\hyperlink{after-sponsor}{Continue reading the main story}

\hypertarget{to-rally-voters-democrats-focus-on-health-care-as-their-closing-argument}{%
\section{To Rally Voters, Democrats Focus on Health Care as Their
Closing
Argument}\label{to-rally-voters-democrats-focus-on-health-care-as-their-closing-argument}}

\includegraphics{https://static01.nyt.com/images/2018/10/25/us/politics/25healthcare-1/merlin_145548201_9884fd7e-5a6f-4208-981b-5ad23076e7fa-articleLarge.jpg?quality=75\&auto=webp\&disable=upscale}

By \href{https://www.nytimes.com/by/trip-gabriel}{Trip Gabriel}

\begin{itemize}
\item
  Oct. 28, 2018
\item
  \begin{itemize}
  \item
  \item
  \item
  \item
  \item
  \end{itemize}
\end{itemize}

COLUMBIA, Mo. --- Senator Claire McCaskill isn't subtle in reminding
voters what her campaign is all about. She's rechristened it the ``Your
Health Care, Your Vote'' tour.

The turnaround could not be more startling. After years of running as
far as they could from President Barack Obama's health care law, Ms.
McCaskill and vulnerable Senate Democrats in Florida, West Virginia and
other political battlegrounds have increasingly focused their closing
argument on a single issue: saving the Affordable Care Act.

Now, with Republicans desperate to reposition themselves and come up
with their own health care pitch, and with the elections roiled by
gale-force winds on immigration and Justice Brett M. Kavanaugh's
confirmation hearings, the question is whether health care will be
enough to save her and Democrats in other key Senate races. Most
recently, the mail bombs sent from Florida and the fatal synagogue
shooting in Pittsburgh on Saturday have added jagged new pieces with the
potential to further disrupt both parties' strategies.

Ms. McCaskill and her Republican opponent, the Missouri attorney
general, Josh Hawley, clashed sharply over health care once again at
their final debate on Thursday. She lambasted him for participating in a
lawsuit that challenges the constitutionality of the Affordable Care Act
and would end its protections for those with pre-existing conditions. He
said he supported a program that would protect patients with high
medical costs outside the current health care law.

On the same day, President Trump proposed that Medicare pay for certain
prescription drugs based on the prices paid in other industrialized
countries --- one recent initiative coming from the White House and
Republicans after years of campaigning on killing the Affordable Care
Act without offering a replacement that would include comparable
protections for people with pre-existing medical conditions.

Mr. Trump announced falsely last October he had already killed the
Affordable Care Act, saying: ``It's dead. It's gone. It's no longer.''
Senate Republicans
\href{https://www.tillis.senate.gov/public/index.cfm/2018/8/senators-introduce-legislation-to-protect-americans-with-pre-existing-conditions}{introduced
legislation} in August they said would protect people with pre-existing
conditions. But health experts say the bill would
\href{https://twitter.com/larry_levitt/status/1033020087437996033}{allow
insurers to exclude services and treatment for certain pre-existing
conditions}.

Nonetheless, Mr. Trump tweeted last week: ``Republicans will totally
protect people with Pre-Existing Conditions, Democrats will not! Vote
Republican.''

And on Friday, Mr. Obama, in speeches in Detroit and Milwaukee, mocked
Republican ads on health care, accusing them of trying to rewrite
history and their own positions after seeking for years to repeal the
Affordable Care Act.

It is unknown whether Democrats' health care message will hold up as Mr.
Trump, through almost daily rallies and frequent Twitter blasts, tries
to dominate television news and social media in the campaign's final
days. He has said the midterms would be about ``Kavanaugh, the caravan,
law and order, and common sense.''

But after years of trying and failing to rally voters behind the
complicated features of Mr. Obama's health care law, Democrats have
discovered this year the emotional power of one of its benefits,
protecting people with pre-existing illnesses.

\includegraphics{https://static01.nyt.com/images/2018/10/25/us/politics/25healthcare-2/merlin_145609686_047e3a20-7a87-43c2-86bc-59a2ed931591-articleLarge.jpg?quality=75\&auto=webp\&disable=upscale}

The subject has lit up polls, monopolized advertising budgets and driven
a national strategy for Democrats, who are defending 10 Senate seats in
states Mr. Trump won and are relying heavily on health care as a
defining issue in key states including Arizona, Florida, West Virginia
and Nevada.

``This is the message coming straight from people in the red states,''
said Senator Chris Van Hollen of Maryland, chairman of the Democrats'
Senate campaign committee.

Republicans have been put on the defensive, insisting in TV ads
featuring their family members that they, too, support affordable care
for people with pre-existing conditions.

Their claims come after years of lawsuits and congressional votes by
Republicans to gut or weaken the health care law's protections of
expensive chronic illnesses.

\emph{{[}}\href{https://www.nytimes.com/2018/10/27/reader-center/first-time-donors.html}{\emph{Read
about the political motivations of people who recently made their first
donations to an election campaign.}}\emph{{]}}

In addition to fashioning a message on health care, Mr. Hawley is
relying on a blunt appeal to partisanship. Asked about his campaign's
closing message, Mr. Hawley's campaign manager, Kyle Plotkin, pointed to
the slogan boldly emblazoned on the campaign bus: ``Stop Schumer. Fire
Claire. Don't let the liberals take over.''

Callie Glascock, an administrator at the University of Missouri who
voted for Mr. Trump, said on Sunday she would cast a ballot for Ms.
McCaskill, who is seeking a third term, because of health care.
``Everyone wants to say Obamacare was bad. Well, who's come up with a
better plan?'' she asked outside a market displaying a hay wagon of
pumpkins in Ashland, Mo.

At the same time, her father-in-law, a lifelong Democrat, is leaning
toward Mr. Hawley.

``He said, You know what, after watching all that Judge Kavanaugh stuff,
he's about changed his mind on voting Democrat,'' said his son, Mike
Glascock, who was also at the market. ``He just hated all that
rhetoric.''

The midterms ``are officially about health care,'' is the conclusion of
the \href{http://mediaproject.wesleyan.edu/releases/101818-tv/}{Wesleyan
Media Project}, whose analysis of television ads in congressional races
found that nearly half include discussion of health care.

Two years ago, during the 2016 election, health care was featured in
just 10 percent of Democrats' ads.

\href{https://www.kff.org/health-reform/press-release/poll-midterms-health-care-voters-top-issue-but-president-trump-and-other-factors-also-loom-large/}{A
survey by the Kaiser Family Foundation} last week found that in two
battleground states, Florida and Nevada, nearly seven in 10 voters
support protecting people with pre-existing conditions even if it means
healthy people would pay more.

\href{https://www.nytimes.com/interactive/2018/10/24/us/elections/2018-battle-for-congress.html}{}

\includegraphics{https://static01.nyt.com/images/2018/10/24/us/2018-battle-for-congress-promo-1540395475116/2018-battle-for-congress-promo-1540395475116-articleLarge.png}

\hypertarget{the-battle-for-congress-is-close-heres-the-state-of-the-race}{%
\subsection{The Battle for Congress Is Close. Here's the State of the
Race.}\label{the-battle-for-congress-is-close-heres-the-state-of-the-race}}

The math currently favors the Democrats in the House and the Republicans
in the Senate.

Democrats' reversal of fortune on health care astonishes former Senator
Mary Landrieu of Louisiana, who lost her seat in the last midterm cycle,
she believes, because of her vote for the Affordable Care Act in 2010.

Before the 2014 campaign season, Ms. Landrieu invited other red-state
Democrats up for re-election to her Washington home to brainstorm about
reforms they could propose without actually mentioning ``Obamacare.''
They did not succeed.

Not only Ms. Landrieu but other Democrats were swept away that year in
Arkansas, Alaska and North Carolina, handing Republicans the Senate
majority.

``The truth is finally winning out,'' Ms. Landrieu said of the
Affordable Care Act's increasing popularity in polls. ``It didn't happen
fast enough for us.''

National Democrats said the issue bubbled up this year from voters. In
fact, the party was taken by surprise last year as Congress was swamped
with phone calls and protesters when Republicans tried but failed to
repeal the Affordable Care Act in a series of dramatic votes.

Just as the issue was fading a bit from voters' minds, Republicans in 20
states filed a lawsuit this year to overturn what remains of the law. In
June, the Trump administration weighed in.
\href{https://www.justice.gov/file/1069806/download}{Citing the
president's approval, the Justice Department said it agreed} with the
need to end the health care law's guarantees that insurance companies
could not deny coverage to people with pre-existing illnesses or charge
them more.

``They gave us the political gift of the cycle,'' said Brad Woodhouse,
executive director of Protect Our Care, an advocacy group on the left.

Since then, Republicans in many close races have scrambled for cover,
insisting that they, too,
\href{https://www.nytimes.com/2018/10/16/upshot/republicans-health-care-ads-midterms.html}{want
to protect people with pre-existing conditions}.

Two Republican House members running for Senate in battleground states,
Representatives Martha McSally of Arizona and Kevin Cramer of North
Dakota, argued they voted for Obamacare repeal-and-replace bills in 2017
that would have protected people with pre-existing conditions.

Similarly, Senator Dean Heller of Nevada, the most vulnerable Republican
seeking re-election, maintained in a debate last week that he helped
write a 2017 bill protecting pre-existing conditions.

Yet the Republican bills watered down or eliminated the protections in
existing law. They would have allowed states to receive waivers to let
insurers charge higher premiums for unhealthy people.

Image

Democrats in several battleground Senate races have made health care a
key component of their campaigns as Election Day
approaches.Credit...Whitney Curtis for The New York Times

Nowhere is the issue more important than in the close race in Missouri.

Ms. McCaskill, 65, recounts on the campaign trail how, when she was a
young lawyer, her parents were forced to move in with her after her
father lost his job and his insurance because of a brain tumor. ``I
remember hearing my mother in the next room being very upset because she
was so frightened,'' she said at a rally last Friday.

Mr. Hawley, as Missouri's attorney general, is a plaintiff in the
multistate lawsuit that would terminate the Affordable Care Act
altogether.

Facing criticism on the issue, he recorded
\href{https://www.youtube.com/watch?v=XKpY8PCut2A}{an ad}in which he
tells of learning that his 5-year-old son has a chronic disease and says
he will force insurance companies to cover pre-existing conditions.

\href{https://www.news-leader.com/story/opinion/2018/10/03/josh-hawley-obamacare-isnt-needed-protect-preexisting-conditions/1508838002/}{In
a newspaper column}, Mr. Hawley proposed that the federal government pay
expenses above \$10,000 for people with pre-existing conditions. His
plan included no details on how the government would raise the money to
cover these costs.

Ms. McCaskill blasted it as ``a press release plan'' in a debate last
week. ``You can't go to court and get rid of important protections when
there is no backup, when people will be in a free fall,'' she said.

In an interview on Saturday, after he had walked in the University of
Missouri homecoming parade, Mr. Hawley, 38, said that his plan would
cost only ``a fraction'' of the Affordable Care Act. He attacked Ms.
McCaskill as unwilling to consider solutions that are not part of
Obamacare. The 2010 law offset expenses for people with existing
illnesses by requiring healthy people to buy insurance --- a mandate
since eliminated by Republicans.

For all the Democrats' focus, it is uncertain how the issue will
influence voters.

John Kosach, a 32-year-old public relations researcher in a St. Louis
suburb, said he supported Ms. McCaskill. ``Claire's message of saying
Hawley is going to repeal the law but there isn't a plan in place,
that's resonated with me,'' he said.

His wife, Sam Kosach, 31, who also works in public relations, said Mr.
Hawley's use of his family in the ad about his support for pre-existing
conditions ``just felt icky.''

\href{https://www.nytimes.com/interactive/2018/09/28/us/politics/the-campaign-reporter-ul.html?src=hpPromoHeadline}{}

\hypertarget{sign-up-for-the-campaign-reporter}{%
\subsection{Sign up for The Campaign
Reporter}\label{sign-up-for-the-campaign-reporter}}

\includegraphics{https://int.nyt.com/newsgraphics/push-interactive/projects/campaign-reporter/avatars/alex_burns.png}

Hey, I'm Alex Burns, a politics correspondent for The Times. Send me
your questions using the NYT app. I'll give you the latest intel from
the campaign trail.

Sign up via push alert

Another voter, Karen French, a 59-year-old retired nurse, is part of the
large bloc of conservatives who crossed party lines in the past to vote
for Ms. McCaskill, one of only two Missouri Democrats in statewide
office.

This year, however, she will not do so.

``I'm kind of ashamed of my country and the mess we made of Judge
Kavanaugh's life,'' she said. ``I probably would have voted for Claire
until that happened.''

Ms. French, who lives in rural Fulton, Mo., has a 28-year-old daughter
who has severe problems with her heart and lungs. Asked if she was
worried that Mr. Hawley's lawsuit would threaten protections for people
like her daughter, Ms. French broke in before a reporter could finish
the question.

``It doesn't matter to me,'' she said. ``More important is the situation
that happened with Judge Kavanaugh.''

Advertisement

\protect\hyperlink{after-bottom}{Continue reading the main story}

\hypertarget{site-index}{%
\subsection{Site Index}\label{site-index}}

\hypertarget{site-information-navigation}{%
\subsection{Site Information
Navigation}\label{site-information-navigation}}

\begin{itemize}
\tightlist
\item
  \href{https://help.nytimes.com/hc/en-us/articles/115014792127-Copyright-notice}{©~2020~The
  New York Times Company}
\end{itemize}

\begin{itemize}
\tightlist
\item
  \href{https://www.nytco.com/}{NYTCo}
\item
  \href{https://help.nytimes.com/hc/en-us/articles/115015385887-Contact-Us}{Contact
  Us}
\item
  \href{https://www.nytco.com/careers/}{Work with us}
\item
  \href{https://nytmediakit.com/}{Advertise}
\item
  \href{http://www.tbrandstudio.com/}{T Brand Studio}
\item
  \href{https://www.nytimes.com/privacy/cookie-policy\#how-do-i-manage-trackers}{Your
  Ad Choices}
\item
  \href{https://www.nytimes.com/privacy}{Privacy}
\item
  \href{https://help.nytimes.com/hc/en-us/articles/115014893428-Terms-of-service}{Terms
  of Service}
\item
  \href{https://help.nytimes.com/hc/en-us/articles/115014893968-Terms-of-sale}{Terms
  of Sale}
\item
  \href{https://spiderbites.nytimes.com}{Site Map}
\item
  \href{https://help.nytimes.com/hc/en-us}{Help}
\item
  \href{https://www.nytimes.com/subscription?campaignId=37WXW}{Subscriptions}
\end{itemize}
