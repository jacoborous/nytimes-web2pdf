Sections

SEARCH

\protect\hyperlink{site-content}{Skip to
content}\protect\hyperlink{site-index}{Skip to site index}

\href{https://www.nytimes.com/section/politics}{Politics}

\href{https://myaccount.nytimes.com/auth/login?response_type=cookie\&client_id=vi}{}

\href{https://www.nytimes.com/section/todayspaper}{Today's Paper}

\href{/section/politics}{Politics}\textbar{}U.S. Accuses Chinese
Nationals of Infiltrating Corporate and Government Technology

\url{https://nyti.ms/2GzmxCg}

\begin{itemize}
\item
\item
\item
\item
\item
\item
\end{itemize}

Advertisement

\protect\hyperlink{after-top}{Continue reading the main story}

Supported by

\protect\hyperlink{after-sponsor}{Continue reading the main story}

\hypertarget{us-accuses-chinese-nationals-of-infiltrating-corporate-and-government-technology}{%
\section{U.S. Accuses Chinese Nationals of Infiltrating Corporate and
Government
Technology}\label{us-accuses-chinese-nationals-of-infiltrating-corporate-and-government-technology}}

\includegraphics{https://static01.nyt.com/images/2018/12/21/us/politics/21dc-justice-video/21dc-justice-video-videoSixteenByNine3000.jpg}

By \href{https://www.nytimes.com/by/david-e-sanger}{David E. Sanger} and
\href{https://www.nytimes.com/by/katie-benner}{Katie Benner}

\begin{itemize}
\item
  Dec. 20, 2018
\item
  \begin{itemize}
  \item
  \item
  \item
  \item
  \item
  \item
  \end{itemize}
\end{itemize}

\href{https://cn.nytimes.com/usa/20181221/us-and-other-nations-to-announce-china-crackdown/}{阅读简体中文版}\href{https://cn.nytimes.com/usa/20181221/us-and-other-nations-to-announce-china-crackdown/zh-hant/}{閱讀繁體中文版}

WASHINGTON --- The Trump administration ramped up its pressure campaign
on Beijing on Thursday, as the Justice Department accused two Chinese
nationals with ties to the country's Ministry of State Security of
infiltrating the biggest providers of internet services and boring into
government computer systems, including a major Department of Energy
laboratory.

The indictment of the two men came just months after the Justice
Department
\href{https://www.nytimes.com/2018/10/10/us/politics/china-spy-espionage-arrest.html}{lured
one of the Chinese intelligence agency's officers to Belgium}, where he
was arrested and extradited to the United States. Both cases focus on an
intelligence effort based in Tianjin to advance Beijing's economic and
geopolitical interests with an extraordinarily broad attack on Western
companies and governments.

Just as the indictment was unsealed, Britain identified the same
intelligence operation, often named APT 10 by cybersecurity firms, as
responsible for separate attacks in that country and beyond. The
statement from Britain's Foreign Office was part of a new, collective
effort by Western allies to call out China's attempts to obtain trade
secrets and intellectual property through a state-coordinated
cyberespionage campaign, according to people involved in the planning.
\href{https://foreignminister.gov.au/releases/Pages/2018/mp_mr_181221.aspx}{Australia}
and
\href{https://www.gcsb.govt.nz/news/cyber-campaign-attributed-to-china/}{New
Zealand} on Friday issued similar statements.

The allegations highlight the tension between the United States and
China over what the White House says is a brazen effort by the Chinese
to obtain Western technology and other proprietary information. The
United States formally accused the Chinese of violating a 2015 agreement
--- brokered by President Barack Obama and President Xi Jinping --- to
cease economic espionage, saying
\href{https://www.nytimes.com/2018/11/29/us/politics/china-trump-cyberespionage.html}{Chinese
hackers have come roaring back}after two years of comity.

The Justice Department action also comes amid other Trump administration
pressure on the Chinese. The United States helped coordinate the arrest
of a top executive of Huawei, the Chinese telecom giant, on suspicion of
committing fraud related to sanctions against Iran. Her arrest, while
she was changing planes in Canada, has set off a geopolitical standoff,
with China arresting three Canadians on its own soil.

Last week, American investigators said that a long-running hack of
Starwood Hotels, now part of Marriott International, was a Chinese
intelligence-gathering operation. And the United States has taken steps
to block Chinese investment and student enrollments in the United
States.

It is not clear how, if at all, the latest indictment will affect Mr.
Trump's effort to reach a trade agreement with Mr. Xi that would end an
escalating tariff war between the world's two largest economies. The
United States has set a March 2 deadline to reach agreement with Beijing
on a range of issues, including what the White House says is a pattern
in which China has pressured American companies to hand over valuable
technology and trade secrets as a condition of doing business there.

China's Foreign Ministry said in a statement on its website that it does
not support stealing trade secrets and that the move ``seriously
violates the basic norms of international relations and has severely
damaged Sino-U.S. cooperation.''

It said that American monitoring of foreign governments, businesses and
individuals has long been an ``open secret.''

The statement was a reminder of how cyberspace has become a primary
battleground between the two nations.

The indictment unsealed on Thursday describes the broad outlines of what
it calls a yearslong campaign by China to steal American technological
secrets in a range of industries to allow Beijing's companies to
undercut international competitors and help its military erode the
United States' defensive edge. That echoes a previous indictment against
officers of the People's Liberation Army in
2014\href{https://www.nytimes.com/2014/05/20/us/us-to-charge-chinese-workers-with-cyberspying.html}{,
who were charged with stealing a variety of industrial secrets}. Safely
ensconced in China, none of them have ever been arrested or brought to
trial.

In the new case, the government said it had charged two Chinese
nationals, Zhu Hua and Zhang Shilong, with conspiracy to hack into
computer systems and commit wire fraud and identity theft. The
government accused them of targeting unnamed aviation,
telecommunications, pharmaceutical and satellite companies, and said
several government entities were attacked, including the Navy and NASA's
Goddard Space Flight Center and Jet Propulsion Laboratory.

The indictment does not describe any specific technology stolen by the
group but said they ``successfully obtained unauthorized access'' to a
range of entities, including the Lawrence Berkeley National Laboratory,
an Energy Department lab in California.

Security firms have been tracking the Chinese hackers, Mr. Zhu, also
called ``Godkiller,'' and Mr. Zhang, also called ``Baobeilong,'' for 13
years, under various names including APT10 and Stone Panda. At the
National Security Agency, intelligence analysts tracked the two, along
with a third Chinese hacker, whom they called ``Legion Opal.'' All
targeted an exhaustive list of individuals and companies in aerospace
and defense, naval, energy, natural resources, automotive, electronic
and government sectors, as well as the occasional Chinese dissident.

But as recently as 2013, intelligence officials were not sure what to
make of the hackers' relationship to Chinese state officials. All three
were based in Tianjin, and that year, a classified intelligence
assessment was vague in describing any tie to Beijing. The ``exact
affiliation with Chinese government entities is not known, but their
activities indicate a probable intelligence requirement feed,'' one
N.S.A. document obtained by The New York Times said.

The hackers worked for private tech companies, but their targets ---
particularly in aerospace and defense --- were of value to China's
civilian spy agency. Other targets in the energy, automotive, electronic
and national resources industries closely aligned with China's economic
priorities.

``This is outright cheating and theft, and it gives China an unfair
advantage at the expense of law-abiding businesses and countries that
follow the international rules in return for the privilege of
participating in the global economic system,'' Rod J. Rosenstein, the
deputy attorney general, said at a news conference.

APT10 has rapidly changed its approaches and technology after private
cybersecurity firms discovered its attacks. It constantly shifted its
internet protocol, or IP, addresses to avoid detection and bypass
security filters, the indictment alleged, allowing it to remain on its
victims' systems far longer.

Between 2006 and 2018, according to the indictment, the APT10 group
hacked computers in at least a dozen countries and broke into companies
and the American government to steal information and data on various
technologies.

Beginning in 2014, APT10 began to target companies that provide computer
services like cloud computing and networking support. It penetrated
those networks and stole confidential business data from companies
around the world, the indictment says.

In doing so, the administration said, Beijing violated its 2015
agreement not to steal American technological secrets. Not only did it
use the stolen information to give its companies a competitive
advantage, but it also used the intelligence to rapidly advance the
capabilities of the People's Liberation Army as it worked to increase
its influence in the Pacific region.

In addition to its corporate espionage, the group compromised the Navy's
computer systems, downloading the private information of more than
100,000 Navy personnel, the Justice Department said.

``China stands accused of engaging in criminal activity that victimizes
individuals and companies in the United States, violates our laws and
departs from international norms of state behavior,'' Mr. Rosenstein
said.

He accused China of trying to ``dominate'' other countries through
economic espionage, and said the response ``requires a strategic,
whole-of-government approach to the threats that China poses.''

The legal assault on China comes as Mr. Trump tries to end a trade war
with Beijing that has begun inflicting economic harm on both sides of
the Pacific. But while the administration has tried to divorce trade
talks from law enforcement actions, Mr. Trump has eagerly conflated the
two, potentially complicating an already complex negotiation.

Mr. Trump has suggested that he could intervene in the Huawei case if it
would help secure a trade agreement with China. And he previously
intervened in another sanctions case involving a Chinese telecom firm,
ZTE, which had been barred from buying American components after China's
president, Mr. Xi, personally appealed to the president.

Steven Mnuchin, the Treasury secretary, said he was ``cautiously
optimistic'' that the charges against Chinese hackers would not derail
the trade negotiations. He said that while cybersecurity had been
discussed, this specific situation had not been raised.

Advertisement

\protect\hyperlink{after-bottom}{Continue reading the main story}

\hypertarget{site-index}{%
\subsection{Site Index}\label{site-index}}

\hypertarget{site-information-navigation}{%
\subsection{Site Information
Navigation}\label{site-information-navigation}}

\begin{itemize}
\tightlist
\item
  \href{https://help.nytimes.com/hc/en-us/articles/115014792127-Copyright-notice}{©~2020~The
  New York Times Company}
\end{itemize}

\begin{itemize}
\tightlist
\item
  \href{https://www.nytco.com/}{NYTCo}
\item
  \href{https://help.nytimes.com/hc/en-us/articles/115015385887-Contact-Us}{Contact
  Us}
\item
  \href{https://www.nytco.com/careers/}{Work with us}
\item
  \href{https://nytmediakit.com/}{Advertise}
\item
  \href{http://www.tbrandstudio.com/}{T Brand Studio}
\item
  \href{https://www.nytimes.com/privacy/cookie-policy\#how-do-i-manage-trackers}{Your
  Ad Choices}
\item
  \href{https://www.nytimes.com/privacy}{Privacy}
\item
  \href{https://help.nytimes.com/hc/en-us/articles/115014893428-Terms-of-service}{Terms
  of Service}
\item
  \href{https://help.nytimes.com/hc/en-us/articles/115014893968-Terms-of-sale}{Terms
  of Sale}
\item
  \href{https://spiderbites.nytimes.com}{Site Map}
\item
  \href{https://help.nytimes.com/hc/en-us}{Help}
\item
  \href{https://www.nytimes.com/subscription?campaignId=37WXW}{Subscriptions}
\end{itemize}
