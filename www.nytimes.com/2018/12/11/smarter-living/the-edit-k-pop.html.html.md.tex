Sections

SEARCH

\protect\hyperlink{site-content}{Skip to
content}\protect\hyperlink{site-index}{Skip to site index}

\href{https://www.nytimes.com/section/smarter-living}{Smarter Living}

\href{https://myaccount.nytimes.com/auth/login?response_type=cookie\&client_id=vi}{}

\href{https://www.nytimes.com/section/todayspaper}{Today's Paper}

\href{/section/smarter-living}{Smarter Living}\textbar{}At First I Was a
K-Pop Skeptic. Not Anymore.

\url{https://nyti.ms/2zQLVha}

\begin{itemize}
\item
\item
\item
\item
\item
\end{itemize}

Advertisement

\protect\hyperlink{after-top}{Continue reading the main story}

Supported by

\protect\hyperlink{after-sponsor}{Continue reading the main story}

THe eDIT

\hypertarget{at-first-i-was-a-k-pop-skeptic-not-anymore}{%
\section{At First I Was a K-Pop Skeptic. Not
Anymore.}\label{at-first-i-was-a-k-pop-skeptic-not-anymore}}

How K-pop helped me to change how I see myself.

\includegraphics{https://static01.nyt.com/images/2018/12/11/smarter-living/11edit-1/11edit-1-articleLarge.jpg?quality=75\&auto=webp\&disable=upscale}

\includegraphics{https://static01.nyt.com/images/2018/08/30/multimedia/author-kevin-lao/author-kevin-lao-thumbLarge.png}

By Kevin Liao

\begin{itemize}
\item
  Dec. 11, 2018
\item
  \begin{itemize}
  \item
  \item
  \item
  \item
  \item
  \end{itemize}
\end{itemize}

\emph{Welcome to The Edit. Each week in our newsletter, you'll hear
from}
\href{https://www.nytimes.com/2018/08/28/smarter-living/the-edit-contributors.html?module=inline}{\emph{college
students and recent graduates}} \emph{about issues going on in their
lives.}
\href{https://www.nytimes.com/newsletters/the-edit?module=inline}{\emph{Sign
up here}} \emph{to get it in your inbox.}

I used to laugh whenever my friends would mention their obsession with
Korean bands like
\href{https://www.nytimes.com/2018/02/07/arts/music/k-pop-olympics-korea.html}{BTS,
Girls' Generation and EXO}. The colorful hair, the dated beats and the
quirkiness of East Asian pop culture all stood in stark contrast to what
I had come to expect from pop artists. In retrospect, the most foreign
aspect of it all was the fact that the Katy Perrys, Justin Biebers and
Taylor Swifts of western music were replaced by faces like my own.

Somehow, even I got sucked into this strange K-pop universe. Sure, the
quality of the music is sometimes debatable. I'm not going to try to
tell you that nine women dressed up in tight yellow skirts singing about
their love for ``Mr. Taxi'' is some sort of brilliant artistic feat.

So why am I still rambling on about K-pop?

We hear a lot of debate these days about media representation of
minorities. I cannot say that Asian-Americans have had \emph{no}
representation. But American pop culture has typically painted a
disfigured portrait of us and rarely thought to rework it.

We've seen this image in classic films like ``Breakfast at Tiffany's,''
in which white comedian Mickey Rooney wears prosthetics and squints his
eyes to transform into Holly Golightly's Japanese landlord. He also
adopts on an over-the-top accent for laughs.

Fifty years later, we still see this image in shows like ``2 Broke
Girls,'' where Han, the short Chinese-American restaurant manager, is
constantly emasculated by white women. It's hard not to take away from
this anything other than that Asians are punch lines, not people.

More recently, we've seen the romantic comedy ``Crazy Rich Asians'' **
attempt to combat these kinds of harmful stereotypes. As someone with
Chinese heritage, it was mind-blowing to see my culture on the big
screen for the first time. But ``Crazy Rich Asians'' is just one movie.
It can't alone fix decades of misrepresentation.

For years, I took these lessons to heart. In school, I'd laugh it off
whenever kids would pull their eyes back at me, make fun of Asian names,
or otherwise insinuate that we were ugly, undesirable, and foreign.
Incidents like these would shape the way I saw the world and myself.
Whatever cool looked like to me back then, it certainly didn't look
anything like me.

K-pop is not the answer to American racism, nor is it even a proper
representation of the wide diversity of people that the pan-ethnic term
``Asian-American'' represents. The South Asian and Southeast Asian
representation that Western media fails to demonstrate does not exist in
K-pop either. But for me, K-pop has helped me unlearn the lesson that my
Korean-Chinese blood somehow made me less than.

K-pop never showed me such a tightly restrained picture of what it means
to be of East Asian descent, instead its stars show a range of
diversity. Some K-pop stars can move audiences to tears with their
voices. Others can light up stages with their dance moves.

So maybe I, too, a Korean-Chinese American, could be ugly or sexy, nerdy
or cool, quiet or loud like the K-pop stars I saw in music videos. Maybe
I, too, could be seen as a person, not a punch line.

\emph{If you're interested in K-pop but unsure where to start, Kevin
made a Spotify playlist to give you a taste:}

\hypertarget{what-were-reading}{%
\subsection{What We're Reading}\label{what-were-reading}}

\href{https://www.nytimes.com/interactive/2018/12/11/style/2018-year-in-review.html}{2018:
The Year in Dissonance} We dare you to live through 2018 --- again.

\href{https://www.nytimes.com/2018/11/27/magazine/48-of-the-coolest-kids-in-new-york.html}{48
of the Coolest Kids in New York} **** We sent a photographer to look for
the most fashionable kids in the city. She began on the first day of
school and shot for two months. Here are the standouts.

\href{https://www.nytimes.com/interactive/2018/12/10/business/location-data-privacy-apps.html}{Your
Apps Know Where You Were Last Night, and They're Not Keeping It Secret}
Dozens of companies use smartphone locations to help advertisers and
even hedge funds. They say it's anonymous, but the data shows how
personal it is.

\href{https://www.nytimes.com/2018/12/09/smarter-living/why-you-start-things-youll-never-finish.html}{Why
You Start Things You'll Never Finish} One solution: Start fewer things.

\href{https://www.nytimes.com/2018/12/07/business/work-advice-liars.html}{You
Work With Liars!} A quick guide to the office fabulist.

\href{https://www.nytimes.com/2018/12/08/opinion/college-gpa-career-success.html}{What
Straight-A Students Get Wrong} **** If you always succeed in school,
you're not setting yourself up for success in life.

Kevin Liao is a student at Stanford University and a contributor to The
Edit.

Advertisement

\protect\hyperlink{after-bottom}{Continue reading the main story}

\hypertarget{site-index}{%
\subsection{Site Index}\label{site-index}}

\hypertarget{site-information-navigation}{%
\subsection{Site Information
Navigation}\label{site-information-navigation}}

\begin{itemize}
\tightlist
\item
  \href{https://help.nytimes.com/hc/en-us/articles/115014792127-Copyright-notice}{©~2020~The
  New York Times Company}
\end{itemize}

\begin{itemize}
\tightlist
\item
  \href{https://www.nytco.com/}{NYTCo}
\item
  \href{https://help.nytimes.com/hc/en-us/articles/115015385887-Contact-Us}{Contact
  Us}
\item
  \href{https://www.nytco.com/careers/}{Work with us}
\item
  \href{https://nytmediakit.com/}{Advertise}
\item
  \href{http://www.tbrandstudio.com/}{T Brand Studio}
\item
  \href{https://www.nytimes.com/privacy/cookie-policy\#how-do-i-manage-trackers}{Your
  Ad Choices}
\item
  \href{https://www.nytimes.com/privacy}{Privacy}
\item
  \href{https://help.nytimes.com/hc/en-us/articles/115014893428-Terms-of-service}{Terms
  of Service}
\item
  \href{https://help.nytimes.com/hc/en-us/articles/115014893968-Terms-of-sale}{Terms
  of Sale}
\item
  \href{https://spiderbites.nytimes.com}{Site Map}
\item
  \href{https://help.nytimes.com/hc/en-us}{Help}
\item
  \href{https://www.nytimes.com/subscription?campaignId=37WXW}{Subscriptions}
\end{itemize}
