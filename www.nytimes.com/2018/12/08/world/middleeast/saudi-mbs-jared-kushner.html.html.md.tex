Sections

SEARCH

\protect\hyperlink{site-content}{Skip to
content}\protect\hyperlink{site-index}{Skip to site index}

\href{https://www.nytimes.com/section/world/middleeast}{Middle East}

\href{https://myaccount.nytimes.com/auth/login?response_type=cookie\&client_id=vi}{}

\href{https://www.nytimes.com/section/todayspaper}{Today's Paper}

\href{/section/world/middleeast}{Middle East}\textbar{}The Wooing of
Jared Kushner: How the Saudis Got a Friend in the White House

\url{https://nyti.ms/2zLH8xu}

\begin{itemize}
\item
\item
\item
\item
\item
\item
\end{itemize}

Advertisement

\protect\hyperlink{after-top}{Continue reading the main story}

Supported by

\protect\hyperlink{after-sponsor}{Continue reading the main story}

\hypertarget{the-wooing-of-jared-kushner-how-the-saudis-got-a-friend-in-the-white-house}{%
\section{The Wooing of Jared Kushner: How the Saudis Got a Friend in the
White
House}\label{the-wooing-of-jared-kushner-how-the-saudis-got-a-friend-in-the-white-house}}

\includegraphics{https://static01.nyt.com/images/2018/12/06/world/06MBS-Kushner/merlin_145529385_b6651b22-df58-4efc-841b-08292ffd9816-articleLarge.jpg?quality=75\&auto=webp\&disable=upscale}

By \href{https://www.nytimes.com/by/david-d-kirkpatrick}{David D.
Kirkpatrick}, \href{https://www.nytimes.com/by/ben-hubbard}{Ben
Hubbard}, \href{https://www.nytimes.com/by/mark-landler}{Mark Landler}
and \href{https://www.nytimes.com/by/mark-mazzetti}{Mark Mazzetti}

\begin{itemize}
\item
  Dec. 8, 2018
\item
  \begin{itemize}
  \item
  \item
  \item
  \item
  \item
  \item
  \end{itemize}
\end{itemize}

Senior American officials were worried. Since the early months of the
Trump administration, Jared Kushner, the president's son-in-law and
Middle East adviser, had been having private, informal conversations
with Prince Mohammed bin Salman, the favorite son of Saudi Arabia's
king.

Given Mr. Kushner's political inexperience, the private exchanges could
make him susceptible to Saudi manipulation, said three former senior
American officials. In an effort to
\href{https://www.nytimes.com/2017/08/03/us/politics/john-kelly-chief-of-staff-trump.html}{tighten
practices at the White House}, a new chief of staff tried to reimpose
longstanding procedures stipulating that National Security Council staff
members should participate in all calls with foreign leaders.

But even with the restrictions in place, Mr. Kushner, 37, and Prince
Mohammed, 33, kept chatting, according to three former White House
officials and two others briefed by the Saudi royal court. In fact, they
said, the two men were on a first-name basis, calling each other Jared
and Mohammed in text messages and phone calls.

The exchanges continued even after the Oct. 2 killing of Jamal
Khashoggi, the Saudi journalist who was ambushed and dismembered by
Saudi agents, according to two former senior American officials and the
two people briefed by the Saudis.

As the killing set off a firestorm around the world and American
intelligence agencies concluded that
\href{https://www.nytimes.com/2018/11/16/us/politics/cia-saudi-crown-prince-khashoggi.html}{it
was ordered by Prince Mohammed}, Mr. Kushner became the prince's most
important defender inside the White House, people familiar with its
internal deliberations say.

Mr. Kushner's support for Prince Mohammed in the moment of crisis is a
striking demonstration of a singular bond that has helped draw President
Trump into
\href{https://www.nytimes.com/2018/11/20/world/middleeast/trump-saudi-khashoggi.html}{an
embrace of Saudi Arabia} as one of his most important international
allies.

But the ties between Mr. Kushner and Prince Mohammed did not happen on
their own. The prince and his advisers, eager to enlist American support
for his hawkish policies in the region and for his own consolidation of
power, cultivated the relationship with Mr. Kushner for more than two
years, according to documents, emails and text messages reviewed by The
New York Times.

A delegation of Saudis close to the prince visited the United States as
early as the month Mr. Trump was elected, the documents show, and
brought back a report identifying Mr. Kushner as a crucial focal point
in the courtship of the new administration. He brought to the job scant
knowledge about the region, a transactional mind-set and an intense
focus on reaching a deal with the Palestinians that met Israel's
demands, the delegation noted.

Even then, before the inauguration, the Saudis were trying to position
themselves as essential allies who could help the Trump administration
fulfill its campaign pledges. In addition to offering to help resolve
the dispute between Israel and the Palestinians, the Saudis offered
hundreds of billions of dollars in deals to buy American weapons and
invest in American infrastructure. Mr. Trump later announced versions of
some of these items with great fanfare when he made his first foreign
trip: to an Arab-Islamic summit in Riyadh, the Saudi capital. The Saudis
had extended that invitation during the delegation's November 2016
visit.

``The inner circle is predominantly deal makers who lack familiarity
with political customs and deep institutions, and they support Jared
Kushner,'' the Saudi delegation wrote of the incoming administration in
a slide presentation obtained by
\href{https://www.al-akhbar.com/Arab_Island/262756/\%D8\%A7\%D9\%84\%D8\%B3\%D8\%B9\%D9\%88\%D8\%AF\%D9\%8A\%D8\%A9-\%D9\%80\%D9\%80-\%D9\%84\%D9\%8A\%D9\%83\%D8\%B3-\%D8\%AA\%D8\%A3\%D8\%B1\%D9\%8A\%D8\%AE-\%D8\%B9\%D9\%84\%D8\%A7\%D9\%82\%D8\%A9-\%D8\%A7\%D8\%A8\%D9\%86-\%D8\%B3\%D9\%84\%D9\%85\%D8\%A7\%D9\%86-\%D9\%88\%D8\%AA\%D8\%B1\%D8\%A7\%D9\%85}{the
Lebanese newspaper Al Akhbar}, which provided it to The Times. Several
Americans who spoke with the delegation confirmed the slide
presentation's accounts of the discussions.

The courtship of Mr. Kushner appears to have worked.

Only a few months after Mr. Trump moved into the White House, Mr.
Kushner was inquiring about the Saudi royal succession process and
whether the United States could influence it, raising fears among senior
officials that he sought to help Prince Mohammed, who was not yet the
crown prince, vault ahead in the line for the throne, two former senior
White House officials said. American diplomats and intelligence
officials feared that the Trump administration might be seen as playing
favorites in the delicate internal politics of the Saudi royal family,
the officials said.

(After publication, a senior White House official said in a statement:
``Implications that Jared inquired about the possibility of influencing
the Saudi royal succession process are false.'')

By March, Mr. Kushner helped usher Prince Mohammed into a formal lunch
with Mr. Trump in a state dining room at the White House, capitalizing
on a last minute cancellation by Chancellor Angela Merkel of Germany
because of a snowstorm.

\includegraphics{https://static01.nyt.com/images/2018/12/05/world/XXMBS-Kushner4/merlin_119495039_f909b9c6-565f-4fcb-b5ed-2392d3bf1161-articleLarge.jpg?quality=75\&auto=webp\&disable=upscale}

Bending protocol, Mr. Kushner arranged for Prince Mohammed, often
referred to by his initials as M.B.S., to receive the kind of treatment
usually reserved for heads of state, with photographs and news media
coverage, according to a person involved in the arrangements. It appears
to have been the first face-to-face meeting between Mr. Kushner and the
prince, but Mr. Kushner raised eyebrows by telling others in the White
House that he and Prince Mohammed had already spoken several times
before, two people at the event recalled.

In a statement, a White House spokesman said that ``Jared has always
meticulously followed protocols and guidelines regarding the
relationship with MBS and all of the other foreign officials with whom
he interacts.''

White House officials declined to explain those protocols and
guidelines, and declined to comment on Mr. Kushner's one-on-one
communications with Prince Mohammed since the killing of Mr. Khashoggi.

Their connection, though, has been pivotal since the start of the Trump
administration.

``The relationship between Jared Kushner and Mohammed bin Salman
constitutes the foundation of the Trump policy not just toward Saudi
Arabia but toward the region,'' said Martin Indyk, a fellow at the
Council on Foreign Relations and a former Middle East envoy. The
administration's reliance on the Saudis in the peace process, its
\href{https://www.nytimes.com/2017/06/06/world/middleeast/trump-qatar-saudi-arabia.html}{support
for the kingdom's feud with Qatar}, an American ally, and its backing of
the Saudi-led
\href{https://www.nytimes.com/interactive/2018/10/26/world/middleeast/saudi-arabia-war-yemen.html}{intervention
in Yemen}, he said, all grew out of ``that bromance.''

\hypertarget{you-will-love-him}{%
\subsection{`You Will Love Him'}\label{you-will-love-him}}

Before the 2016 presidential race, Mr. Kushner's most extensive exposure
to the Middle East was through Israel. Prime Minister Benjamin Netanyahu
was a Kushner family friend, and the Kushners had contributed heavily to
Israeli nonprofits supporting Jewish settlements in the Palestinian
territories of the West Bank.

But the Arab rulers of the oil-rich Persian Gulf mainly figured in Mr.
Kushner's life as investors in American real estate, the Kushner family
business.

So
\href{https://www.nytimes.com/2018/06/13/world/middleeast/trump-tom-barrack-saudi.html}{Tom
Barrack}, a Lebanese-American real estate investor with close ties to
both Mr. Trump and the Gulf rulers, set out during the campaign to
introduce Mr. Kushner to his associates as a useful ally.

``You will love him and he agrees with our agenda!'' Mr. Barrack wrote
in May 2016 in an email to the Emirati ambassador in Washington, Youssef
Otaiba.

Image

Tom Barrack, a Lebanese-American real estate investor with close ties to
both Mr. Trump and the Gulf rulers.Credit...Chip Somodevilla/Getty
Images

Mr. Otaiba soon positioned himself as an informal adviser on the region
to Mr. Kushner.

``Thanks to you, I am in constant contact with Jared and that has been
extremely helpful,'' Mr. Otaiba wrote to Mr. Barrack in the first months
after Mr. Trump took office.

The Emirati ambassador was also eagerly promoting Prince Mohammed. Since
the prince's aging father had taken the throne in 2015, the Emiratis
were betting heavily on the prince as their preferred contender in the
succession struggles within the Saudi royal family.

``MBS is incredibly impressive,'' Mr. Otaiba wrote Mr. Barrack in June
2016, as they tried to arrange meetings between the prince and the Trump
campaign.

The month after the American election, the de facto ruler of the United
Arab Emirates --- Crown Prince Mohammed bin Zayed of Abu Dhabi ---
delivered a similar message when he made an unannounced trip to New York
for a meeting with Mr. Kushner and others about the Israeli-Palestinian
peace process.

While speaking with Mr. Kushner, the Emirati prince also recommended
Prince Mohammed of Saudi Arabia as a promising young leader, according
to a person familiar with their conversations.

Mr. Kushner seemed impressed. The meeting had been arranged in part by
Rick Gerson, a hedge fund manager who was close to Mr. Kushner and to
the Emirati crown prince. After the encounter, Mr. Gerson sent a message
to the Emirati crown prince about his success in winning over Mr.
Kushner.

``I promise you this will be the start of a special and historic
relationship,'' Mr. Gerson wrote, in a text message.

On the eve of the inauguration, Mr. Gerson wrote to the Emirati crown
prince again.

``You have a true friend in the White House,'' Mr. Gerson wrote,
recounting a visit with Mr. Kushner before Mr. Kushner departed for
Washington.

The emails with Ambassador Otaiba and the text messages with Mr. Gerson
were provided to The Times by people critical of Emirati foreign
policies and authenticated by others with direct knowledge of their
contents. Mr. Gerson declined to comment and the Emirati embassy did not
respond to requests for comment.

\hypertarget{lack-of-familiarity-with-history}{%
\subsection{`Lack of Familiarity' With
History}\label{lack-of-familiarity-with-history}}

Top aides to Saudi Arabia's Prince Mohammed also met with Mr. Kushner on
a trip to New York in November 2016, after the election.

The Saudi team included Musaad al-Aiban, a cabinet minister involved in
economic planning and national security, and Khaled al-Falih, installed
by the prince as minister of energy and chairman of the state oil
company, according to executives who met with them and a person who was
briefed on the meetings. Mr. Aiban did not respond to a request for
comment, and Mr. Falih could not be reached for comment.

The delegation made special note of what it characterized as Mr.
Kushner's ignorance of Saudi Arabia.

Image

Khaled al-Falih was part of a team of top aides to the prince who met
with Mr. Kushner in New York in November 2016.Credit...Tasneem Alsultan
for The New York Times

``Kushner made clear his lack of familiarity with the history of
Saudi-American relations and he asked about its support for terrorism,''
the team noted in the slide presentation prepared for Riyadh. ``After
the discussion, he expressed his satisfaction with what was explained
about the Saudi role in fighting terrorism'' and what the Saudis said
was their international leadership in fighting Islamist extremism.

Mr. Kushner, the Saudi report said, also questioned the delegation's
motives, asking whether the group had always been interested in working
with Mr. Trump. As a candidate, Mr. Trump had promised to ban Muslim
immigrants from entering the United States and had singled out Saudi
Arabia as a dangerous influence.

``Kushner wondered about Saudi Arabia's desire for partnership and
whether it came from opportunity or worry, and he wondered as well if it
was specific to this American administration or whether it was presented
to Hillary Clinton (for example: women driving),'' read another slide,
next to a photograph of Mr. Kushner.

But Mr. Kushner was clear about his own priorities, the report said.
``The Israeli-Palestinian conflict was among the most important issues
to draw Kushner's attention,'' the delegation reported, and therefore
the best way to win him over.

``The Palestinian issue first: there is still no clear plan for the
American administration toward the Middle East,'' the delegation wrote,
``except that the central interest is finding a historic solution to
support the stability of Israel and solve the Israeli-Palestinian
conflict.''

To cultivate ties with the Trump team, the Saudis had prepared a long
list of initiatives that they said would help Mr. Trump deliver for his
supporters.

Seizing on Mr. Trump's campaign vows for the ``extreme vetting'' of
immigrants, the Saudi delegation proposed ``establishing an intelligence
and data'' exchange ``to help the American administration carry out its
strategy of investigating those requesting residency (extreme
vetting),'' according to an Arabic version of a presentation for the
Trump team.

And the delegation pledged ``high-level coordination with the new
American administration'' to help with ``defeating extremist thought.''

Several of the Saudi proposals were evidently welcomed.

One was a ``joint center to fight the ideology of extremism and
terrorism.'' President Trump helped inaugurate a Saudi version of the
center on his trip to Riyadh the following May.

Another Saudi proposal outlined what the Trump administration later
called ``an Arab NATO.'' In their presentation, the Saudis described it
as an Islamic military coalition of tens of thousands of troops ``ready
when the president-elect wishes to deploy them.''

Other initiatives appeared timed to Mr. Trump's first term in office,
like proposals to spend \$50 billion over four years on American defense
contracts, to increase Saudi investment in the United States to \$200
billion over four years, and to invest, with other Gulf states, up to
\$100 billion in American infrastructure.

And the delegation urged Mr. Trump to come to Saudi Arabia himself to
``launch the initiatives as part of a historic welcome celebration.''

It is unclear how the delegation's slide-show presentation to and about
the Trump team were obtained by the Al Akhbar newspaper, which is
sympathetic to the Iranian-backed Lebanese movement Hezbollah and Iran,
an enemy of Saudi Arabia.

Several Americans identified in the presentation acknowledged meeting
with the delegation and confirmed broad outlines of the discussions. The
Times provided the documents and the names of delegation members to an
official of the Saudi Embassy in Washington, who declined to comment.

\hypertarget{a-saudi-role-in-mideast-peace}{%
\subsection{A Saudi Role in Mideast
Peace}\label{a-saudi-role-in-mideast-peace}}

Israel had long argued to American diplomats that Saudi Arabia's
influence in the region made it essential to any peace deal, and the
Israelis were developing high hopes for Prince Mohammed because of his
hawkish views toward Iran and his general iconoclasm (he would later
make several
\href{https://www.theatlantic.com/international/archive/2018/04/mohammed-bin-salman-iran-israel/557036/}{statements},
like affirming the Israeli ``right'' to land, that were notably
\href{https://www.nytimes.com/2017/12/03/world/middleeast/palestinian-saudi-peace-plan.html}{more
sympathetic} to the Israeli position than those of other Saudi leaders.)

Within weeks of Mr. Trump's move into the White House, Mr. Kushner had
embraced the delegation's proposal for the president to visit Riyadh,
convinced by then that the alliance with Saudi Arabia would be crucial
in his plans for the region, according to a person who discussed it with
Mr. Kushner and a second person familiar with his plans.

The secretary of state at the time, Rex W. Tillerson, opposed the idea.
It would link the administration too closely to Riyadh, these people
said, giving up flexibility and leverage. Mr. Trump initially saw little
benefit either, according to a person involved in his deliberations.

Image

Images of President Trump and King Salman were projected onto the Ritz
Hotel in Riyadh during Mr. Trump's visit last year.Credit...Stephen
Crowley/The New York Times

But by the time of the inauguration Mr. Kushner was already arguing that
under the influence of Prince Mohammed, Saudi Arabia could play a
pivotal role in advancing a Middle East peace deal, according to three
people familiar with his thinking. That would be the president's legacy,
Mr. Kushner argued, according to a person involved in the discussions.

It was around the time of the White House visit in March 2017 that
senior officials in the State Department and the Pentagon began to worry
about the one-on-one communications between Prince Mohammed --- who is
known to favor the online messaging service WhatsApp --- and Mr.
Kushner. ``There was a risk the Saudis were playing him,'' one former
White House official said, speaking on condition of anonymity to discuss
internal deliberations.

Two later face-to-face encounters with Mr. Kushner preceded key turning
points in Prince Mohammed's consolidation of power.

Shortly after Mr. Kushner visited Riyadh with the president in May 2017,
Prince Mohammed orchestrated the
\href{https://www.nytimes.com/2017/06/21/world/middleeast/saudi-arabia-crown-prince-mohammed-bin-salman.html?module=inline}{ouster
of his older cousin}, Prince Mohammed bin Nayef, removing him from
control of the Saudi Interior Ministry and replacing him as crown
prince. Prince Mohammed also announced a Saudi-led
\href{https://www.nytimes.com/2017/07/02/world/middleeast/qatar-saudi-arabia-blockade.html?module=inline}{blockade
of its neighbor and rival Qatar}, the host of a major American air base.

And days after Mr. Kushner made an unannounced visit to Riyadh in the
fall of 2017, the crown prince
\href{https://www.nytimes.com/2018/03/11/world/middleeast/saudi-arabia-corruption-mohammed-bin-salman.html?module=inline}{summarily
detained about 200 wealthy Saudis}, including several of his royal
cousins, in a Ritz-Carlton hotel in Riyadh.

After each play for power, President Trump publicly praised Prince
Mohammed.

Image

Protesting the killing of Jamal Khashoggi, the Saudi writer, in Istanbul
last month.Credit...Emrah Gurel/Associated Press

One former White House official argued that Mr. Kushner's personal ties
to Prince Mohammed had sometimes been an asset. At one point, for
example, the Saudi-led coalition fighting in Yemen had blocked a
critical port, cutting off humanitarian and medical supplies. The
national security adviser at the time, Lt. Gen. H.R. McMaster, suggested
that Mr. Kushner call Prince Mohammed to address the issue, the official
said, and General McMaster believed Mr. Kushner's intercession had
helped persuade the Saudis to loosen the restrictions.

White House officials also say that Mr. Kushner has formal conversations
with many other leaders in the region. And previous administrations have
also had close ties to the Saudi government.

Since the uproar over Mr. Khashoggi's killing, the Trump administration
has acknowledged only one conversation between Mr. Kushner and Prince
Mohammed: an Oct. 10 telephone call joined by John R. Bolton, the
national security adviser. The Americans ``asked for more details and
for the Saudi government to be transparent in the investigation
process,'' the White House said in a statement.

But American officials and a Saudi briefed on their conversations said
that Mr. Kushner and Prince Mohammed have continued to chat informally.
According to the Saudi, Mr. Kushner has offered the crown prince advice
about how to weather the storm, urging him to resolve his conflicts
around the region and avoid further embarrassments.

Few of the Saudi promises have amounted to much. The effectiveness of
the counterterrorism center in Riyadh remains doubtful. After offering
\$50 billion in new weapons contracts, the Saudis have signed only
letters of interest or intent without any firm deals. After proposing to
marshal up to \$100 billion in investments in American infrastructure,
the Saudis have announced an investment of only \$20 billion.

Inside the White House, Mr. Kushner has continued to argue that the
president needs to stand by Prince Mohammed because
\href{https://www.nytimes.com/2018/11/01/world/middleeast/with-saudi-prince-holding-on-to-power-us-seen-standing-by-him.html}{he
remains essential to the administration's broader Middle East strategy},
according to people familiar with the deliberations.

Whether Prince Mohammed can fulfill that role, however, remains to be
seen. His initial approaches to the Palestinians
\href{https://www.nytimes.com/2017/12/03/world/middleeast/palestinian-saudi-peace-plan.html}{were
rejected by their leaders}, and their resistance stiffened after the
Trump administration recognized Jerusalem as Israel's capital without
waiting for a negotiated agreement on the city's status.

Now the prince's father, King Salman, 82, who is still the official head
of state, has appeared to resist Mr. Kushner's Middle East peace plans
as well.

``The Palestinian issue will remain our primary issue,'' the king
declared in a speech last month, ``until the Palestinian people receive
all of their legal rights.''

Advertisement

\protect\hyperlink{after-bottom}{Continue reading the main story}

\hypertarget{site-index}{%
\subsection{Site Index}\label{site-index}}

\hypertarget{site-information-navigation}{%
\subsection{Site Information
Navigation}\label{site-information-navigation}}

\begin{itemize}
\tightlist
\item
  \href{https://help.nytimes.com/hc/en-us/articles/115014792127-Copyright-notice}{©~2020~The
  New York Times Company}
\end{itemize}

\begin{itemize}
\tightlist
\item
  \href{https://www.nytco.com/}{NYTCo}
\item
  \href{https://help.nytimes.com/hc/en-us/articles/115015385887-Contact-Us}{Contact
  Us}
\item
  \href{https://www.nytco.com/careers/}{Work with us}
\item
  \href{https://nytmediakit.com/}{Advertise}
\item
  \href{http://www.tbrandstudio.com/}{T Brand Studio}
\item
  \href{https://www.nytimes.com/privacy/cookie-policy\#how-do-i-manage-trackers}{Your
  Ad Choices}
\item
  \href{https://www.nytimes.com/privacy}{Privacy}
\item
  \href{https://help.nytimes.com/hc/en-us/articles/115014893428-Terms-of-service}{Terms
  of Service}
\item
  \href{https://help.nytimes.com/hc/en-us/articles/115014893968-Terms-of-sale}{Terms
  of Sale}
\item
  \href{https://spiderbites.nytimes.com}{Site Map}
\item
  \href{https://help.nytimes.com/hc/en-us}{Help}
\item
  \href{https://www.nytimes.com/subscription?campaignId=37WXW}{Subscriptions}
\end{itemize}
