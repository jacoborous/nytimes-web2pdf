Sections

SEARCH

\protect\hyperlink{site-content}{Skip to
content}\protect\hyperlink{site-index}{Skip to site index}

\href{https://www.nytimes.com/section/world/asia}{Asia Pacific}

\href{https://myaccount.nytimes.com/auth/login?response_type=cookie\&client_id=vi}{}

\href{https://www.nytimes.com/section/todayspaper}{Today's Paper}

\href{/section/world/asia}{Asia Pacific}\textbar{}India Gay Sex Ban Is
Struck Down. `Indefensible,' Court Says.

\url{https://nyti.ms/2CnBJQR}

\begin{itemize}
\item
\item
\item
\item
\item
\item
\end{itemize}

Advertisement

\protect\hyperlink{after-top}{Continue reading the main story}

Supported by

\protect\hyperlink{after-sponsor}{Continue reading the main story}

\hypertarget{india-gay-sex-ban-is-struck-down-indefensible-court-says}{%
\section{India Gay Sex Ban Is Struck Down. `Indefensible,' Court
Says.}\label{india-gay-sex-ban-is-struck-down-indefensible-court-says}}

\includegraphics{https://static01.nyt.com/images/2018/09/07/world/07india-gay-1sub/07india-gay-1sub-videoSixteenByNine3000.jpg}

By \href{https://www.nytimes.com/by/jeffrey-gettleman}{Jeffrey
Gettleman}, \href{https://www.nytimes.com/by/kai-schultz}{Kai Schultz}
and \href{https://www.nytimes.com/by/suhasini-raj}{Suhasini Raj}

\begin{itemize}
\item
  Sept. 6, 2018
\item
  \begin{itemize}
  \item
  \item
  \item
  \item
  \item
  \item
  \end{itemize}
\end{itemize}

\href{https://www.nytimes.com/es/2018/09/06/india-prohibicion-sexo-gay/}{Leer
en español}

NEW DELHI --- India's Supreme Court on Thursday unanimously struck down
one of the world's oldest bans on consensual gay sex, a groundbreaking
victory for gay rights that buried one of the most glaring vestiges of
India's colonial past.

After weeks of deliberation by the court and decades of struggle by gay
Indians, Chief Justice Dipak Misra said the law was ``irrational,
indefensible and manifestly arbitrary.''

News of the decision instantly shot around India. On the steps of an
iconic courthouse in Bangalore, people danced, kissed and hugged
tightly, eyes closed. In Mumbai, India's pulsating commercial capital,
human rights activists showered themselves in a blizzard of confetti.

The justices eagerly went further than simply decriminalizing gay sex.
From now on, they ruled, gay Indians are to be accorded all the
protections of the Constitution.

``This ruling is hugely significant,'' said Meenakshi Ganguly, the South
Asia director for Human Rights Watch. With restrictions on gay rights
toppling in country after country, the ruling in India, the world's
second-most-populous nation, may encourage still more nations to act,
she said.

Still, however historic the ruling of the court, considered a liberal
counterweight to the conservative politics sweeping India, gay people
here know that their landscape remains treacherous. Changing a law is
one thing --- changing deeply held mind-sets another. And few suggested
that other major victories, like same-sex marriage, were on the near
horizon.

Many Indians are extremely socially conservative, going to great lengths
to arrange marriages with the right families, of the right castes. Loved
ones who try to rebel are often ostracized. Countless gays have been
shunned by their parents and persecuted by society.

Much of this may also be true in other parts of the world. But what made
India stand out from most --- at least until Thursday --- was its
application of an anachronistic law drawn up by British colonizers
during the Victorian era and kept on the books for 150 years.

The law banned sex considered ``against the order of nature,'' and
thousands of people were prosecuted under it. But for gays in India,
prison was only one of the risks. The law was often used as a cudgel to
intimidate, blackmail and abuse.

Just to file the legal challenge that led to Thursday's ruling was an
act of bravery. The more than two dozen petitioners, who included gay,
lesbian, bisexual and transgender people, could have been rounded up and
arrested simply for identifying themselves as gay and coming forward.

\includegraphics{https://static01.nyt.com/images/2018/09/07/world/07india-gay-1/merlin_130126955_7be4d86c-57c8-4b68-9f1e-43a862454165-articleLarge.jpg?quality=75\&auto=webp\&disable=upscale}

On Thursday, conservative Christians, Muslims and Hindus, who often find
themselves at odds with one another, blasted the ruling as shameful and
vowed to fight it.

``We are giving credibility and legitimacy to mentally sick people,''
said Swami Chakrapani, president of All India Hindu Mahasabha, a
conservative group.

Gay activists said they needed to move carefully. The next step, they
said, will be to push for more equality in the workplace. Gay marriage,
they said, is still a long way off.

In their ruling, the justices said homosexuality was ``natural.'' They
also said that the Indian Constitution was not a ``collection of mere
dead letters,'' and that it should evolve with time.

The court did not rule that the law being challenged, known as Section
377, should be excised altogether. It can still be used, it said, in
cases of bestiality, for instance. But it can no longer be applied to
consensual gay sex.

The justices seemed moved by the stories they heard from the petitioners
about harassment, blackmail, abuse and persecution.

``History owes an apology to members of the community for the delay in
ensuring their rights,'' Justice Indu Malhotra said.

Menaka Guruswamy, one of the lead lawyers representing gay petitioners,
said that the court's extension of nondiscrimination principles to gay
people had laid a ``very powerful foundation.''

``This decision,'' she said, is basically saying: `You are not alone.
The court stands with you. The Constitution stands with you. And
therefore your country stands with you.'''

As the justices spoke, the crowd in the courtroom tried to remain
composed. Outside, a cheer went up and people hugged.

Image

Menaka Guruswamy, a lawyer for the petitioners, urged the Supreme Court
to consider the effect that ending the law would have on gay people in
their 20s. ``Tell my young clients that their lives will be different,''
she said.Credit...Vivek Singh for The New York Times

India has a complicated record on gay issues. Its dominant religion,
Hinduism, is actually quite permissive of same-sex love. Centuries-old
Hindu temples depict erotic encounters between members of the same sex,
and in some Hindu myths, men become pregnant. In others, transgender
people are given special status and praised for being loyal.

But that culture of tolerance changed drastically under British rule.
India was intensely colonized during the height of the Victorian era,
when the British Empire was at its peak and the social mores in England
were austere.

In the 1860s, the British introduced Section 377 of the Indian Penal
Code, imposing up to a life sentence on ``whoever voluntarily has carnal
intercourse against the order of nature.'' The law was usually enforced
in cases of sex between men, but it officially extended to anybody
caught having anal or oral sex.

Though in recent years more gay Indians have come out, and acceptance of
gay, lesbian and transgender people has grown to some degree, the fact
that intimate behavior was still criminalized created much shame.

\href{https://www.nytimes.com/2018/07/10/world/asia/india-gay-decriminalization.html?rref=collection\%2Fbyline\%2Fkai-schultz\&action=click\&contentCollection=undefined\&region=stream\&module=stream_unit\&version=latest\&contentPlacement=6\&pgtype=collection}{In
hearings in July}, lawyers argued that the law was out of sync with the
times and legally inconsistent with other recent court rulings,
including one made last year that guaranteed the constitutional right to
privacy.

They pointed to similar hoary laws that had been toppled in the United
States, Canada, England and Nepal, India's smaller and poorer neighbor.
And they went beyond classic legal arguments.

Ms. Guruswamy spoke of the decades-long relationship between two older
petitioners, Navtej Singh Johar and Sunil Mehra, and the sacrifices they
had made in their personal and professional lives to keep their
partnership secret.

Ms. Guruswamy encouraged the judges to think of all the young gay people
who did not want to follow that same road and spend their lives hiding
who they really were.

``Tell my young clients that their lives will be different,'' she
pleaded. ``The recognition of equal citizenship, that is the business of
life, so that they know they are loved, protected.''

Interestingly, India's leading politicians, who usually never resist an
opportunity to weigh in on a hot issue, have mostly stayed out of the
debate.

Image

Ritu Dalmia, a celebrity chef, was one of the initial group of
petitioners who challenged the law. Dozens more joined them as the
hearings approached.Credit...Vivek Singh for The New York Times

Prime Minister Narendra Modi has said very little about gay rights,
despite the conservative stance of his governing Bharatiya Janata Party
on many social issues. The central government announced in July that it
was
\href{https://timesofindia.indiatimes.com/india/decriminalising-gay-sex-centre-says-it-will-leave-decision-to-wisdom-of-court/articleshow/64942900.cms}{not
going to take a position} on Section 377.

It was some of India's Christian groups that put up the most aggressive
defense of the law. Lawyers for these groups argued that sexual
orientation was not innate and that decriminalizing gay sex would lead
to the transmission of H.I.V.

Prosecution under the law was relatively rare. In 2014, 1,148 complaints
were filed. In 2016, the number nearly doubled, to 2,187. That year,
over 1,600 cases were sent for trial.

Many gays feared that if they reported crimes like rape, they would be
the ones arrested. Some gay people shared stories about being raped by
police officers and then threatened with jail time if they ever came
forward.

For several years, Section 377 was in the cross hairs, with the courts
going back and forth.

In 2009, a court in New Delhi ruled that the law could not be applied to
consensual sex. But Hindu, Muslim and Christian groups filed appeals in
the Supreme Court, and in 2013, the court
\href{https://www.nytimes.com/2013/12/12/world/asia/court-restores-indias-ban-on-gay-sex.html}{restored}
the law, saying that Parliament, and not the courts, should take up the
issue. In its
\href{https://www.scribd.com/doc/190888550/Naz-Section377-Supreme-Court}{decision}
that year, the Supreme Court said only a ``minuscule fraction of the
country's population constitute lesbians, gays, bisexuals or
transgenders.''

But gay activists did not give up. They regrouped and began looking for
people willing to serve as petitioners and endure the personal scrutiny
the case would bring. In 2016, five gay and lesbian Indians submitted a
writ petition challenging Section 377 on the basis that it violated
their rights to equality and liberty.

The initial group of petitioners in the latest challenge included Mr.
Johar, a dancer, and his partner, Mr. Mehra, a journalist; Ritu Dalmia,
a celebrity chef; Ayesha Kapur, a businesswoman; and Aman Nath, a
hotelier.

As the Supreme Court prepared to hear the case, more than 25 other
Indians with varied social and economic backgrounds joined them.

Among those to file a petition was Anurag Kalia, 25, an engineer living
in Bangalore, who was once so afraid to say the word ``gay'' that he
practiced doing so in front of the mirror.

``I used to whisper it,'' he said.

As the court prepared to announce its verdict on Thursday, Mr. Kalia
checked his phone ``like crazy,'' he said. After the ruling, he stepped
out of his office for a few quiet moments, feeling ready to celebrate, a
bit numb, unsure of what the future held, but also feeling ``relief,
relief.''

``It feels like there's much more to come,'' he said. ``This is just the
first strike.''

Advertisement

\protect\hyperlink{after-bottom}{Continue reading the main story}

\hypertarget{site-index}{%
\subsection{Site Index}\label{site-index}}

\hypertarget{site-information-navigation}{%
\subsection{Site Information
Navigation}\label{site-information-navigation}}

\begin{itemize}
\tightlist
\item
  \href{https://help.nytimes.com/hc/en-us/articles/115014792127-Copyright-notice}{©~2020~The
  New York Times Company}
\end{itemize}

\begin{itemize}
\tightlist
\item
  \href{https://www.nytco.com/}{NYTCo}
\item
  \href{https://help.nytimes.com/hc/en-us/articles/115015385887-Contact-Us}{Contact
  Us}
\item
  \href{https://www.nytco.com/careers/}{Work with us}
\item
  \href{https://nytmediakit.com/}{Advertise}
\item
  \href{http://www.tbrandstudio.com/}{T Brand Studio}
\item
  \href{https://www.nytimes.com/privacy/cookie-policy\#how-do-i-manage-trackers}{Your
  Ad Choices}
\item
  \href{https://www.nytimes.com/privacy}{Privacy}
\item
  \href{https://help.nytimes.com/hc/en-us/articles/115014893428-Terms-of-service}{Terms
  of Service}
\item
  \href{https://help.nytimes.com/hc/en-us/articles/115014893968-Terms-of-sale}{Terms
  of Sale}
\item
  \href{https://spiderbites.nytimes.com}{Site Map}
\item
  \href{https://help.nytimes.com/hc/en-us}{Help}
\item
  \href{https://www.nytimes.com/subscription?campaignId=37WXW}{Subscriptions}
\end{itemize}
