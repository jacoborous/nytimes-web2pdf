Sections

SEARCH

\protect\hyperlink{site-content}{Skip to
content}\protect\hyperlink{site-index}{Skip to site index}

\href{https://www.nytimes.com/section/us}{U.S.}

\href{https://myaccount.nytimes.com/auth/login?response_type=cookie\&client_id=vi}{}

\href{https://www.nytimes.com/section/todayspaper}{Today's Paper}

\href{/section/us}{U.S.}\textbar{}`It's Back': Underwater Yet Again, the
Carolinas Face a New Reality

\url{https://nyti.ms/2NqDmCJ}

\begin{itemize}
\item
\item
\item
\item
\item
\item
\end{itemize}

Advertisement

\protect\hyperlink{after-top}{Continue reading the main story}

Supported by

\protect\hyperlink{after-sponsor}{Continue reading the main story}

\hypertarget{its-back-underwater-yet-again-the-carolinas-face-a-new-reality}{%
\section{`It's Back': Underwater Yet Again, the Carolinas Face a New
Reality}\label{its-back-underwater-yet-again-the-carolinas-face-a-new-reality}}

\includegraphics{https://static01.nyt.com/images/2018/09/19/us/19stormredux1/merlin_143830923_ffbf3c8a-6a7b-4b34-a851-175b75889b97-articleLarge.jpg?quality=75\&auto=webp\&disable=upscale}

By \href{http://www.nytimes.com/by/jack-healy}{Jack Healy},
\href{https://www.nytimes.com/by/richard-fausset}{Richard Fausset} and
\href{http://www.nytimes.com/by/campbell-robertson}{Campbell Robertson}

\begin{itemize}
\item
  Sept. 18, 2018
\item
  \begin{itemize}
  \item
  \item
  \item
  \item
  \item
  \item
  \end{itemize}
\end{itemize}

GARLAND, N.C. --- After Hurricane Matthew stomped into his trailer home
and pulped his floors, walls and cabinets two years ago, Bobby Barnes
Jr. spent \$90,000 to rebuild and protect himself from another flood. He
raised the house two feet onto brick pilings, bought \$1,300 worth of
flood fencing and said he complied with every federal recommendation.

But on Tuesday morning, his family was underwater again. The Black
River, 10 feet above flood stage and still rising, was now a lake that
had swallowed farm fields around the Barnes's house. The water lapped at
their front door and sloshed around the newly laid floors.

``It's back,'' Mr. Barnes said. ``Same nightmare.''

It was the kind of tragic, expensive, depressing rerun that played out
across much of the Carolinas this week, not only on the coast, but in
inland communities like this one in Sampson County, blessed with tobacco
and turkeys, not sea and sand.

Throughout the region, residents like Mr. Barnes were struggling to put
soggy homes and soggy lives back together yet again, amid palpable
anxiety that the Carolinas seem to be confronting a new normal of too
many storms, with too much water coming much too often.

In both states, the will, and the assumptions, of residents and
officials were being sorely tested.

In inland Kinston, N.C., which saw streets go underwater in 2016 during
Matthew and again last week, Tony Sears, the city manager, had accepted
the fact that this was hurricane country. ``Typically you think
Louisiana, Miami, somewhere a little south,'' he said.

Mr. Sears and many others here are well aware that hurricanes have long
been a fact of life in North and South Carolina, with nearly 500 miles
of coastline between them. But the last four years have been
particularly punishing. An unnamed weather system that drew moisture
from a hurricane in the Atlantic paralyzed much of South Carolina in
2015. Hurricane Matthew arrived the next year, drenching the Carolinas
and leaving dozens dead. And now, Florence, which has dropped more than
8 trillion gallons of rain on North Carolina alone.

\includegraphics{https://static01.nyt.com/images/2018/09/19/us/19stormredux2/merlin_143930850_c7d17c4c-e885-4940-a8b8-44f79e335fc8-articleLarge.jpg?quality=75\&auto=webp\&disable=upscale}

``The tropical systems that have affected us recently have not been wind
events. They've been rainfall events producing inland flooding,'' said
Dr. Susan Cutter, a geography professor at the University of South
Carolina and director of its Hazards and Vulnerability Research
Institute. ``This is something that these areas may have experienced,
but not with the constancy that we see now.''

Acknowledging the constancy, and planning for it, has been a theme this
week for Gov. Roy Cooper of North Carolina, who has been touring some of
his state's hardest-hit areas.

``When you have two 500-year floods within two years of each other, it's
pretty clear it's not a 500-year flood,'' the governor said at a news
conference this week. ``So as we approach recovery, both short-term and
long-term, we will have to look at flooded property, work on mitigation
and buyouts, and being smart about how we recover and make sure that
we're involving local, state and federal officials.''

But buyouts could prove tricky in rural communities where families have
deep ties to the land. On Tuesday, Mr. Barnes, 42, and his wife, Brandy,
41, sat parked on what was now the water's edge of Lisbon Bridge Road in
Garland, some 60 miles inland from where Hurricane Florence had slugged
ashore four days earlier.

They had lived in the house, just across the field from Mr. Barnes's
father, since 1996 and had never flooded out until Matthew. Mr. Barnes
didn't think twice about rebuilding after the 2016 storm, even though he
said his flood insurance only covered about \$25,000 of the damage. He
paid out of pocket to get county approvals to raise the house and even
put on an addition.

Mr. Barnes said he was denied a low-interest construction loan through
the Federal Emergency Management Agency, so he borrowed money and
drained the savings his family had built up after several good years
running a local repair business.

Image

The Baron and The Beef restaurant in Kinston flooded with more than two
feet of water.Credit...Hilary Swift for The New York Times

``I replaced everything,'' Mr. Barnes said.

Now, as he looked out at the waters and talked about boating home to
check on the house, he said they did not know what to do.

Raising the house yet again would cost thousands. They had raised a son
and a daughter there and did not want to leave. Their 16-year-old son,
Mason, had been killed in a car crash just before Matthew. When the
family had to flee again this month, the first thing their 11-year-old
daughter asked her mother was: When can we go back?

``She needs a place,'' Ms. Barnes said.

``We'll probably try to go out and try one more time,'' Mr. Barnes said.
``You can't relocate. You've got to do what you've got to do.''

While there is some research suggesting that small storms may become
less frequent because of climate change, there is also widespread
consensus among scientists that the most powerful storms --- those with
the kind of extreme rainfall brought to the Carolinas by Florence ---
are already becoming more common, said Kerry Emanuel, a professor of
environmental science at the Massachusetts Institute of Technology.

What's more, Dr. Emanuel said, there is evidence that the occurrence of
powerful storms is particularly likely to increase in places on the
margins of the tropics --- ``like the Carolinas.''

The states are among the fastest-growing in the country, and in each of
them, Republican-dominated legislatures have been accused of
prioritizing business and growth over efforts to limit the consequences
of climate change. Beginning in 2012, North Carolina lawmakers
\href{https://www.nytimes.com/2018/09/12/us/north-carolina-coast-hurricane.html}{took
actions} that forced state and local agencies that make policy on the
coast to ignore models that predict rising sea levels.

Image

(Left to right) Jack Helms, 11, Abby Barth, 12, and Eva Rambach, 13,
survey the damage done to Helms's flooded home in Conway.Credit...Tamir
Kalifa for The New York Times

Earlier this year, the South Carolina legislature changed the preamble
to a 30-year-old law governing beachfront development, striking out a
state policy of ``retreat'' from the shoreline in the face of erosion
and replacing it with a policy of ``preservation'' of the beaches. It
was a small change, but a sign of the state's approach, said Josh Eagle,
who teaches environmental law at the University of South Carolina law
school.

``The philosophy is one of, `We can beat nature,''' he said. ``The
driving forces are property rights and climate denial.''

People are moving to the Carolinas by the tens of thousands, and to
coastal areas in particular, many of them starting businesses in places
that would be right in a hurricane's path. Extreme storms like Florence
might jeopardize that growth, but then again, so would aggressive
measures to protect against those storms, said Robert Hartwig, the
director of the Risk and Uncertainty Management Center at the University
of South Carolina.

``Are city planners --- and the states and counties --- are they zoning
in a way to reflect the new reality?'' Professor Hartwig asked. ``The
answer to that is, generally speaking, no. Most local officials are
going to be loath to kill the goose that lays the golden economic egg.''

Still, a number of local governments have been forced to acknowledge
that the flood risks are real. After experiencing some of the worst
flooding in its history in 1995 and 1997, Mecklenburg County, N.C., home
to Charlotte, began regulating all development with an eye to what the
``anticipated ultimate development would look like'' in the region, said
Dave Canaan, the county's storm water services director.

Growth and development, Mr. Canaan said, remains a priority in a place
like Charlotte, the largest city in the Carolinas. But at the same time,
the county has run a rolling and aggressive buyback program that has
moved roughly 700 families and 450 structures out of the county's
floodplains.

Image

Flooding on the outskirts of Kinston.Credit...Hilary Swift for The New
York Times

``This is not supposed to be where big box stores are, or homes upon
homes upon homes are supposed to be,'' he said, ``but just to be open
space, where our streams are meant to meander back and forth.''

In Kinston, a working-class city of 21,000, the buyout program that
started after the last hurricane was not even finished by the time
Hurricane Florence arrived last week.

But the mind-set seemed to have fully changed. Local officials used to
regard hurricane response as something akin to major snow removal:
sporadic events that therefore warranted relatively little investment.

``Two years of these types of storms have really made us look internally
at our own preparation,'' Mr. Sears said. ``Now that we have a better
understanding of where people may be at risk, we can better stage assets
in that area.''

``Matthew got me once, Florence got me twice,'' Mr. Sears said Tuesday,
recounting what he told the City Council this week. ``The next hurricane
won't get me.''

Closer to the coast, in Conway, S.C., a few miles northwest of Myrtle
Beach, the weariness with the cycle of worry, flood, repeat was
palpable.

``I grew up here,'' said Matt Bruton, 34, a moving and storage business
operator, who was hanging out Tuesday in his single-story brick home in
Old Sherwood Country Club, an eclectic neighborhood of old and new
houses north of the Waccamaw River.

The waters had already risen through neighboring Crabtree Swamp to a
level higher than residents had ever seen. ``We had a record flood
during Floyd in '99, and it wasn't this high,'' Mr. Bruton said. ``The
hundred-year flood they called it the last time, and here we are again.
I'd like to put my house in a different, higher place, but I call this
the best neighborhood in Conway. We're all friends here.''

Advertisement

\protect\hyperlink{after-bottom}{Continue reading the main story}

\hypertarget{site-index}{%
\subsection{Site Index}\label{site-index}}

\hypertarget{site-information-navigation}{%
\subsection{Site Information
Navigation}\label{site-information-navigation}}

\begin{itemize}
\tightlist
\item
  \href{https://help.nytimes.com/hc/en-us/articles/115014792127-Copyright-notice}{©~2020~The
  New York Times Company}
\end{itemize}

\begin{itemize}
\tightlist
\item
  \href{https://www.nytco.com/}{NYTCo}
\item
  \href{https://help.nytimes.com/hc/en-us/articles/115015385887-Contact-Us}{Contact
  Us}
\item
  \href{https://www.nytco.com/careers/}{Work with us}
\item
  \href{https://nytmediakit.com/}{Advertise}
\item
  \href{http://www.tbrandstudio.com/}{T Brand Studio}
\item
  \href{https://www.nytimes.com/privacy/cookie-policy\#how-do-i-manage-trackers}{Your
  Ad Choices}
\item
  \href{https://www.nytimes.com/privacy}{Privacy}
\item
  \href{https://help.nytimes.com/hc/en-us/articles/115014893428-Terms-of-service}{Terms
  of Service}
\item
  \href{https://help.nytimes.com/hc/en-us/articles/115014893968-Terms-of-sale}{Terms
  of Sale}
\item
  \href{https://spiderbites.nytimes.com}{Site Map}
\item
  \href{https://help.nytimes.com/hc/en-us}{Help}
\item
  \href{https://www.nytimes.com/subscription?campaignId=37WXW}{Subscriptions}
\end{itemize}
