Sections

SEARCH

\protect\hyperlink{site-content}{Skip to
content}\protect\hyperlink{site-index}{Skip to site index}

\href{https://www.nytimes.com/section/politics}{Politics}

\href{https://myaccount.nytimes.com/auth/login?response_type=cookie\&client_id=vi}{}

\href{https://www.nytimes.com/section/todayspaper}{Today's Paper}

\href{/section/politics}{Politics}\textbar{}In Michigan, Female
Candidates Target a Key Trump Bloc: Union Voters

\url{https://nyti.ms/2Nleg8g}

\begin{itemize}
\item
\item
\item
\item
\item
\end{itemize}

Advertisement

\protect\hyperlink{after-top}{Continue reading the main story}

Supported by

\protect\hyperlink{after-sponsor}{Continue reading the main story}

\hypertarget{in-michigan-female-candidates-target-a-key-trump-bloc-union-voters}{%
\section{In Michigan, Female Candidates Target a Key Trump Bloc: Union
Voters}\label{in-michigan-female-candidates-target-a-key-trump-bloc-union-voters}}

\includegraphics{https://static01.nyt.com/images/2018/09/17/business/17michigan1/17michigan-articleLarge.jpg?quality=75\&auto=webp\&disable=upscale}

By \href{https://www.nytimes.com/by/noam-scheiber}{Noam Scheiber} and
\href{https://www.nytimes.com/by/astead-w-herndon}{Astead W. Herndon}

\begin{itemize}
\item
  Sept. 16, 2018
\item
  \begin{itemize}
  \item
  \item
  \item
  \item
  \item
  \end{itemize}
\end{itemize}

MONROE, Mich. --- It's not the smokestacks that mark this part of
southeast Michigan as a labor stronghold, or even the boxy union halls.

To Michelle LaVoy, a city treasurer running for the State Legislature,
it's the way people say ``union'' as shorthand for ``decent job.''

``My husband works at \href{https://www.yfai.com/}{Yanfeng},'' Carolina
Ricci, perched outside her front door, told Ms. LaVoy, referring to a
nearby auto parts plant. ``He's got the union, he's a steward. But we
still struggle.''

Ms. LaVoy, strait-laced in affect and business casual in dress, doesn't
present as a working-class hero. But she is trying, hard, and her pitch
has a distinctly Norma Rae vibe.

``We should be getting our fair share,'' Ms. LaVoy told Ms. Ricci,
echoing a refrain that many Americans are using this election year. She
wants more money for roads. More money for unions. More money in
workers' paychecks.

Monroe, about 45 minutes south of Detroit, was ground zero for the
political meteorite that struck Michigan in 2016: After twice backing
Barack Obama, the county went for President Trump by more than 20 points
--- netting him a lead of 16,000 votes that was equivalent to one and a
half times his
\href{https://www.nytimes.com/elections/results/michigan}{margin of
victory} statewide.

The area proved to be the leading edge of a historic labor defection
from Democrats that played out across Michigan and several other
Midwestern states where unions have long enjoyed outsize influence. Just
over one out of every two voters from union households supported Hillary
Clinton in Michigan, down from nearly three out of four who backed the
Democratic candidate for governor (who lost as well) in 2014.

Ms. LaVoy and other Michigan Democrats, like the party's nominee for
governor, Gretchen Whitmer, are determined to recapture union voters in
2018, and in so doing show national Democrats how to retake the state's
critical electoral votes in 2020. For unions, the fall election provides
a test of political strength after years of decline, and of the power of
economic issues to drive their members' votes.

Union leaders say Mrs. Clinton was a flawed champion in part because
voters viewed her as a symbol of the status quo, while Mr. Trump
exploited her vulnerabilities by swiping their economic message. ``He
came in and seduced a lot of people,'' said Ron Bieber, president of the
state's labor federation.

Michigan Democrats are now on the offensive on the economy, proposing
hundreds of millions in spending on infrastructure: ``Fix the damn
roads!'' thunders Ms. Whitmer. They rail against new taxes on pensions
and vow to stand up to scofflaw corporations. They insist, à la Mr.
Trump, that the state can once again produce good blue-collar jobs.

Mr. Trump also exploited cultural divisions against Mrs. Clinton, and
Michigan Democrats are mindful of that. Ms. LaVoy introduces herself as
a ``Monroe Democrat,'' by which she means a god-fearing, gun-wielding
moderate who distrusts trade deals and companies that ship jobs
overseas.

Even before many Monroe Democrats abandoned Mrs. Clinton --- be it over
policy, trust, gender or other reasons --- they had a conflicted
relationship with President Obama, who carried the county by a single
point in 2012. While Mr. Obama is largely credited in the Midwest for
saving the auto industry, Monroe's Ford plant shut down just before his
presidency and the county didn't fully share in the industry's recovery.
Privately, some Democrats in the area also hinted at Mr. Obama's race as
a factor in their disillusionment.

As for Republicans, they argue that Democrats are alienating culturally
conservative voters as the party drifts leftward in the Trump era ---
and that Ms. Whitmer and others will pay a price.

\includegraphics{https://static01.nyt.com/images/2018/09/17/us/politics/17michigan2/08michigan-gretchen-articleLarge-v2.jpg?quality=75\&auto=webp\&disable=upscale}

{[}\emph{Make sense of the people, issues and ideas shaping the 2018
elections}
\href{https://www.nytimes.com/newsletters/politics?smid=rd?action=click\&module=Intentional\&pgtype=Article}{\emph{with
our new politics newsletter}}\emph{.}{]}

``I don't think the average person knows how liberal Gretchen is,'' said
Randy Richardville, a former Republican senate majority leader from
Monroe, who did battle with Ms. Whitmer during her days in the
legislature. `` I don't think it will be a cakewalk for her at all.''

Still, these Monroe voters say they are amenable to politicians who
appreciate the union way of life and genuinely seem to want to protect
it.

``I'm union all the way,'' said Darryl Sims, a United Automobile Workers
member from Monroe County who retired last year as a forklift driver at
Detroit Diesel. ``I'm very appreciative when I walk out every month and
my pension is in the mail.

Mr. Sims has a favorable view of Mr. Trump, citing the president's
approach to trade and ``his philosophy that we should take care of our
own people.'' His wife, Michele, a teacher, believes the president has
done a good job as well.

But they are quick to distinguish between Mr. Trump and Michigan
Republicans like the outgoing governor, Rick Snyder, whom they criticize
for deciding to tax their pensions and enacting right-to-work
legislation, allowing workers to benefit from unions without paying dues
or fees.

``They swore up and down they wouldn't do it,'' Mr. Sims said.

A voter outreach project last fall by the Service Employees
International Union, which canvassed thousands of pro-union white
working class voters in Michigan and Wisconsin, showed that voters like
Mr. Sims are very much in play. The top issues, even among many who vote
Republican, were good-paying jobs and expanded access to health care.

Unions in the state are trying to seize this opportunity by increasing
field workers, volunteers and campaign spending --- in some cases, even
at greater levels than during the last presidential race.

``In 2016 I think I had two people working with me on politics,'' said
Lisa Canada, the political and legislative director for the state
carpenters union, referring to paid staffers. ``We have 20 this year.''

\hypertarget{initially-you-were-a-little-scary}{%
\subsection{`Initially you were a little
scary'}\label{initially-you-were-a-little-scary}}

There is something different about the Democratic candidates and message
aimed at union voters this time around. Call it populism with a female
face.

All four Democratic nominees for statewide office are women, as are
three of the party's five nominees in competitive congressional races,
and they are showing a knack for trying to increase the return on the
labor investment in their races. Many of the candidates lighten their
populist overtures with an empathy that often evades Mr. Trump --- and,
some Democrats say, evaded Mrs. Clinton, too.

Image

Leah Schneck, left, who is working on the congressional campaign of
Gretchen Driskell, a Democrat, with Bonnie Finzel-Doster, 88, while
phone banking for Ms. Driskell at the Monroe Democratic Party
offices.Credit...Laura McDermott for The New York Times

When Ms. LaVoy first introduced herself to Daniel Moran, a
Trump-supporting union member, he told her to ``Get out.'' But 30
minutes later they were still talking about his son and the job he said
he recently lost. The conversation ended with a hug.

``Did I come across harder than anyone you talked to today?'' Mr. Moran
asked.

``Initially you were a little scary,'' Ms. LaVoy confessed.

For her part, Ms. Whitmer, highlights her role as a leader on the
\href{https://www.nytimes.com/2013/08/28/us/medicaid-expansion-battle-in-michigan-ends-in-passage.html}{2013
legislation} expanding Medicaid that brought health care coverage to
more than 600,000 state residents. She supports repealing the state's
right-to-work law, and
\href{https://www.detroitnews.com/story/news/local/michigan/2018/09/03/right-work-michigan-governor-campaign/1156349002/}{spearheaded
opposition} to it as Senate minority leader.

She has discussed spending billions on infrastructure and pointedly
contrasts her proposals --- which draw inspiration from the epic
Mackinac suspension bridge --- with the president's. ``At a time when
some people want to build walls,'' she says in her Grand Rapids lilt,
``we in Michigan are going to get back to building bridges.''

On Labor Day, Ms. Whitmer circulated energetically through Detroit's
holiday cookouts, telling voters that November represented a ``once in a
generation opportunity'' because the state had not ``had a pro-labor
governor and a good economy in 25 years.''

``In the last election some of these people were just frustrated with
the whole world, and voted for the person that looked least familiar,''
Ms. Whitmer said in an interview.

Mrs. Clinton's connection to her husband's New Democrat administration
may have fed a certain mistrust, but many Clinton skeptics appear
willing to back other female candidates.

``We got a lot of pushback on Hillary,'' said Bill Black, the political
and legislative director for the Teamsters union in Michigan, which saw
more than 40 percent of its members vote for President Trump, according
to internal polling. ``We're not seeing that with Gretchen.''

\href{https://projects.fivethirtyeight.com/polls/michigan/?states=UT}{Recent
public polls} have shown Ms. Whitmer with double-digit leads over her
Republican opponent, Attorney General Bill Schuette. She also led Mr.
Schuette by 22 points among union households in
\href{https://www.detroitnews.com/story/news/local/michigan/2018/09/11/michigan-whitmer-schuette-governor-poll-september/1256313002/}{an
early September poll} commissioned by the Detroit News.

Image

Candidate signs and a decorative flag are displayed in the window of the
Monroe Democratic Party offices.Credit...Laura McDermott for The New
York Times

``I think she's resonated because she's invited labor to the table,''
said Jon Brown, a construction worker and member of a local laborer's
union, citing Ms. Whitmer's infrastructure plan.

Kevin Hertel, a Democratic state representative leading the party's
campaign to retake the chamber, said that having credible female
candidates dwell on practical economic concerns has the advantage of
appealing to two types of swing voters: those in affluent areas like
Oakland County and western Wayne County, where women are in open revolt
against the president. And those in blue-collar areas like Monroe, one
of the party's top takeover targets.

\href{https://www.nytimes.com/interactive/2018/09/14/us/women-primaries-house-senate-governor.html}{{[}}\emph{\href{https://www.nytimes.com/interactive/2018/09/14/us/women-primaries-house-senate-governor.html}{Women
have won more primaries in the 2018 midterms than ever
before.}}\href{https://www.nytimes.com/interactive/2018/09/14/us/women-primaries-house-senate-governor.html}{{]}}

Labor leaders, like Mary Kay Henry, the S.E.I.U. president, say that a
message focused on jobs, wages and health care has a shot at motivating
voters, including many union members, who didn't feel inspired to turn
out for the last election.

``The most urgent problems in Michigan in working-class communities ---
whether white, black or brown --- felt completely ignored in 2016,'' Ms.
Henry said. ``It resulted in 10,000 votes left on the table just in
Detroit.''

Michigan Republicans appear to sense that they're losing some economic
arguments, even though the state's unemployment rate is low by
historical standards. Earlier this month the Republican-controlled
legislature passed bills phasing in a \$12-per-hour minimum wage and
requiring employers to provide paid sick leave, two labor priorities
that would otherwise have appeared on the ballot this fall.

Republican leaders conceded that they did so because laws are easier to
change if legislators enact them. ``The Senate adopted the policy to
preserve the ability for this legislature and future legislatures to
amend the statute,'' the State Senate majority leader, Arlan Meekhof,
said in a statement to
\href{https://www.washingtonpost.com/business/2018/09/10/they-have-taken-away-our-vote-michigan-approves-minimum-wage-hike-paid-sick-leave-setting-up-clash/?utm_term=.745b2edafa2d}{The
Washington Post}.

``They are going to come back and gut it,'' predicted Mr. Hertel, adding
that Democrats plan to make an issue out of such maneuvering. ``I think
voters are extremely intelligent. They can see a political game for what
it is.''

\hypertarget{authentic-vs-authentic}{%
\subsection{Authentic vs. authentic}\label{authentic-vs-authentic}}

Ms. LaVoy's district is in many respects a case study of the Trump
phenomenon. In 2012, her husband, Bill, also a Democrat, won the seat by
more than 20 points with backing from labor. He was re-elected by a
similar margin.

By 2016, he was so confident of retaining the seat that he spent weeks
campaigning for Democrats in other districts. It was only when President
Obama held a
\href{https://www.detroitnews.com/story/news/politics/2016/11/07/harbaugh-obama/93426394/}{rally
in Ann Arbor} the day before the election that he had a sinking feeling.

``It hit me,'' Mr. LaVoy said. ``Why is the president here? Shouldn't he
be in Pennsylvania, Florida, Ohio? Anywhere but Michigan.'' He briefly
thought about turning around and knocking on 100 doors, but discarded
the idea. ``If I'm in trouble, I'm going to be in big trouble,'' he
concluded.

He turned out to be in big trouble. His opponent, Joseph Bellino, rode
the Trump wave to a comfortable eight-point win. After the election, Mr.
LaVoy thought back to earlier in the year, when union voters would
periodically ask what he thought of Mr. Trump. ``I said, `You know, I
don't even know. I want you to vote for me,''' he recalled. ``But they
loved him.'' (Mr. LaVoy is attempting his comeback in a State Senate
race.)

Image

State Representative Joe Bellino, a Republican, defeated a Democratic
incumbent in 2016, benefiting from the strong vote in the Monroe area
for Donald Trump.Credit...Laura McDermott for The New York Times

In defending the seat this year, Mr. Bellino enjoys some of the same
advantages that helped President Trump. He is a well-known local
businessman who many voters see as independent from the G.O.P. Jacob
Goins, a manager at a pizzeria, told Ms. LaVoy he voted for Mr. Bellino
after chatting him up at the wine and liquor store that Mr. Bellino and
his wife have owned for years. (``My blood kind of runs cold when I hear
that,'' admitted Ms. LaVoy, bemoaning her opponent's prom king-like
appeal.)

Mr. Bellino also has a certain blunt-spoken authenticity.

At a recent town-hall meeting on the opioid epidemic, which has hit
Monroe County like a modern-day plague, Mr. Bellino talked openly about
his own struggle with addiction. ``I was lucky to be a cocaine addict,''
he said. ``The healing in my brain happened a little easier.''

Image

Family members of Dylan Meiring, who died from an overdose in January at
the age of 20, listen to speakers during a town hall meeting about the
opioid crisis, held at St. Mary Catholic Central High School in
Monroe.Credit...Laura McDermott for The New York Times

In a possible foreshadowing of a strategy that Democrats may deploy in
2020, the unions targeting Mr. Bellino hope to dampen his appeal by
making him answer for the Republicans' agenda.

Exhibit A is the controversial repeal of the state's prevailing wage
law, which mandated that contractors pay union-level wages and benefits
to construction workers on state projects. Republicans undid the law
earlier this year despite intense lobbying from labor.

Mr. Bellino voted against the repeal, but a text message from one of his
Republican colleagues indicated that he had offered to support it until
the last minute. The colleague said Mr. Bellino only changed his
position after party leaders secured enough votes to pass the repeal and
wanted him to cover his political flank at home. The carpenters union
promptly withdrew an earlier endorsement. It has taken to circulating
fliers of Mr. Bellino with a Pinocchio schnoz.

Mr. Bellino waves off the controversy. ``I voted the way I told them I
was going to,'' he said.

But Ms. LaVoy is quick to invoke the un-endorsement, and the issue
appears to have emotional currency in Monroe, where many voters make
their living in the building and construction trades.

William Bentley, a millwright and union member who assembles and
maintains mechanical equipment at a steel mill, said he considers
himself a conservative --- ``my daughter's name is Reagan, for Christ's
sake.'' He voted for Mr. Trump, and says he is pleased that the
president is ``putting us back to work.''

Image

Syreeta Wheeler, 36, works on welding a chair at the MTS Seating factory
in Monroe County, Mich.Credit...Laura McDermott for The New York Times

But asked how he felt about Michigan Republicans, and Mr. Bentley became
noticeably cool. He said that he liked the effort Mr. Snyder made to
balance the budget, but took a dim view of right-to-work and the repeal
of the prevailing wage law, both of which Republicans passed on the
governor's watch.

``Rick Snyder has done stuff against the unions,'' Mr. Bentley said.
``I'm no longer with you. Now you're affecting my paycheck.''

Advertisement

\protect\hyperlink{after-bottom}{Continue reading the main story}

\hypertarget{site-index}{%
\subsection{Site Index}\label{site-index}}

\hypertarget{site-information-navigation}{%
\subsection{Site Information
Navigation}\label{site-information-navigation}}

\begin{itemize}
\tightlist
\item
  \href{https://help.nytimes.com/hc/en-us/articles/115014792127-Copyright-notice}{©~2020~The
  New York Times Company}
\end{itemize}

\begin{itemize}
\tightlist
\item
  \href{https://www.nytco.com/}{NYTCo}
\item
  \href{https://help.nytimes.com/hc/en-us/articles/115015385887-Contact-Us}{Contact
  Us}
\item
  \href{https://www.nytco.com/careers/}{Work with us}
\item
  \href{https://nytmediakit.com/}{Advertise}
\item
  \href{http://www.tbrandstudio.com/}{T Brand Studio}
\item
  \href{https://www.nytimes.com/privacy/cookie-policy\#how-do-i-manage-trackers}{Your
  Ad Choices}
\item
  \href{https://www.nytimes.com/privacy}{Privacy}
\item
  \href{https://help.nytimes.com/hc/en-us/articles/115014893428-Terms-of-service}{Terms
  of Service}
\item
  \href{https://help.nytimes.com/hc/en-us/articles/115014893968-Terms-of-sale}{Terms
  of Sale}
\item
  \href{https://spiderbites.nytimes.com}{Site Map}
\item
  \href{https://help.nytimes.com/hc/en-us}{Help}
\item
  \href{https://www.nytimes.com/subscription?campaignId=37WXW}{Subscriptions}
\end{itemize}
