Sections

SEARCH

\protect\hyperlink{site-content}{Skip to
content}\protect\hyperlink{site-index}{Skip to site index}

\href{https://www.nytimes.com/section/politics}{Politics}

\href{https://myaccount.nytimes.com/auth/login?response_type=cookie\&client_id=vi}{}

\href{https://www.nytimes.com/section/todayspaper}{Today's Paper}

\href{/section/politics}{Politics}\textbar{}Report Criticizes Comey but
Finds No Bias in F.B.I. Decision on Clinton

\url{https://nyti.ms/2HLq4Jb}

\begin{itemize}
\item
\item
\item
\item
\item
\item
\end{itemize}

Advertisement

\protect\hyperlink{after-top}{Continue reading the main story}

Supported by

\protect\hyperlink{after-sponsor}{Continue reading the main story}

\hypertarget{report-criticizes-comey-but-finds-no-bias-in-fbi-decision-on-clinton}{%
\section{Report Criticizes Comey but Finds No Bias in F.B.I. Decision on
Clinton}\label{report-criticizes-comey-but-finds-no-bias-in-fbi-decision-on-clinton}}

\includegraphics{https://static01.nyt.com/images/2018/06/16/us/16clinton-alpha/merlin_139107045_e9737120-2523-434e-a17c-22ad0464d695-articleLarge.jpg?quality=75\&auto=webp\&disable=upscale}

By \href{http://www.nytimes.com/by/matt-apuzzo}{Matt Apuzzo}

\begin{itemize}
\item
  June 14, 2018
\item
  \begin{itemize}
  \item
  \item
  \item
  \item
  \item
  \item
  \end{itemize}
\end{itemize}

WASHINGTON --- The Justice Department's inspector general on Thursday
painted a harsh portrait of the F.B.I. during the 2016 presidential
election, describing a destructive culture in which James B. Comey, the
former director, was ``insubordinate,'' senior officials privately
bashed Donald J. Trump and agents came to distrust prosecutors.

The 500-page report criticized Mr. Comey for breaking with longstanding
policy and publicly discussing --- in a news conference and a pair of
letters in the middle of the campaign --- an investigation into Hillary
Clinton's use of a private email server in handling classified
information. The report was a firm rebuke of those actions, which Mr.
Comey has tried for months to defend.

Nevertheless, the inspector general,
\href{https://www.nytimes.com/2018/06/13/us/politics/michael-horowitz-justice-department-inspector-general.html}{Michael
E. Horowitz}, did not challenge the conclusion that Mrs. Clinton should
not be prosecuted. That investigation loomed over most of the
presidential campaign, and Mr. Horowitz and his investigators uncovered
no proof that political opinions at the F.B.I. influenced its outcome.

``We found no evidence that the conclusions by department prosecutors
were affected by bias or other improper considerations,'' he wrote.
``Rather, we concluded that they were based on the prosecutor's
assessment of facts, the law and past department practice.''

But the report --- initiated in response to a chorus of requests from
Congress and the public --- was far from an exoneration. Mr. Horowitz
was unsparing in his criticism of Mr. Comey and referred five F.B.I.
employees for possible discipline over pro-Clinton or anti-Trump
commentary in electronic messages. He said agents were far too cozy with
journalists. And he described a breakdown in the chain of command,
calling it ``extraordinary'' that the attorney general acceded to Mr.
Comey during the most controversial moments of the Clinton
investigation.

The result, Mr. Horowitz said, undermined public confidence in the
F.B.I. and sowed doubt about the bureau's handling of the Clinton
investigation, which even two years later remains politically divisive.
Mrs. Clinton's supporters blame Mr. Comey for her election loss. Mr.
Trump believes that Mr. Comey and his agents conspired to clear Mrs.
Clinton of wrongdoing because they were openly hostile to his candidacy.

Mr. Horowitz repeatedly said he found no evidence that the F.B.I. rigged
the outcome. ``Our review did not find documentary or testimonial
evidence directly connecting the political views these employees
expressed in their text messages and instant messages to the specific
investigative decisions we reviewed,'' the report said.

The report is especially critical of two F.B.I. officials, Peter Strzok
and Lisa Page,
\href{https://www.nytimes.com/2017/12/12/us/fbi-trump-russia.html}{who
exchanged texts} disparaging Mr. Trump. Many of those text messages had
already been released, but the report cites a previously undisclosed
exchange:

Mr. Trump is ``not ever going to become president, right? Right?!'' Ms.
Page wrote.

``No,'' Mr. Strzok wrote. ``No he won't. We'll stop it.''

Ms. Page has left the F.B.I. and Mr. Strzok has been reassigned to human
resources. Like other top F.B.I. officials, they were involved in both
the Clinton case and the investigation into the Trump campaign's ties to
Russia. So while the inspector general's report focuses entirely on the
Clinton case, it has ramifications for the investigation being carried
out by the special counsel, Robert S. Mueller III. Any evidence of bias
or rule-breaking in one case could be used to undermine confidence in
the other.

Mr. Trump has repeatedly declared the Russia investigation a ``witch
hunt'' and was eagerly anticipating the release of Thursday's report. He
was briefed on it but was notably silent about the conclusions.

The Republican National Committee, though, distributed talking points to
supporters criticizing a ``fervent anti-Trump bias'' and calling for Mr.
Strzok's termination. The White House press secretary, Sarah Huckabee
Sanders, offered few remarks.

``It reaffirmed the president's suspicions about Comey's conduct and the
political bias among some of the members of the F.B.I.,'' she said. But
she referred questions to the current F.B.I. director, Christopher A.
Wray.

Mr. Wray, in a rare news conference, said he took the report seriously
but said that nothing in the report ``impugns the integrity'' of the
F.B.I. ``Our brand is doing just fine,'' he said.

\hypertarget{read-justice-dept-report-on-the-fbis-handling-of-clinton-inquiry}{%
\subsection{Read: Justice Dept. Report on the F.B.I.'s Handling of
Clinton
Inquiry}\label{read-justice-dept-report-on-the-fbis-handling-of-clinton-inquiry}}

The Justice Department's inspector general released a report on Thursday
detailing the F.B.I.'s handling of the Clinton email investigation
during the 2016 presidential election.

\includegraphics{https://int.nyt.com/data/documenthelper/39-justice-department-report-fbi-clinton-comey/5e54a6bfd23e7b94fbad/optimized/thumbnail.png}

Mr. Wray was confirmed last year after the abrupt firing of Mr. Comey,
and the report serves as an unflattering book end to Mr. Comey's
three-and-a-half-year tenure. The findings sharply criticize his
judgment as he injected the F.B.I. into presidential politics in ways
not seen since at least the Watergate era.

Mr. Comey
\href{https://www.nytimes.com/2016/07/06/us/politics/hillary-clinton-fbi-email-comey.html}{held
a news conference in July 2016} to announce that he was recommending no
charges against Mrs. Clinton and to publicly chastise her email
practices. It was highly unorthodox; the Justice Department, not the
F.B.I., makes charging decisions. And officials have been reprimanded
for injecting their opinions into legal conclusions. Mr. Comey withheld
his plans for a public statement from his bosses at the Justice
Department.

``It was extraordinary and insubordinate for Comey to do so,'' the
inspector general wrote, ``and we found none of his reasons to be a
persuasive basis for deviating from well-established department policies
in a way intentionally designed to avoid supervision by department
leadership.''

\includegraphics{https://static01.nyt.com/images/2018/06/15/us/politics/15dc-fbi2/merlin_113719807_bf6fae39-bd51-4715-8cb2-72dd3e72ce2a-articleLarge.jpg?quality=75\&auto=webp\&disable=upscale}

Then in late October, over the objection of top Justice Department
officials,
\href{https://www.nytimes.com/2016/10/29/us/politics/fbi-hillary-clinton-email.html}{Mr.
Comey sent a letter to Congress} disclosing that agents were
scrutinizing new evidence in the Clinton case.

That evidence did not change the outcome of the inquiry, but Mrs.
Clinton and many of her supporters
\href{https://www.nytimes.com/2016/11/13/us/politics/hillary-clinton-james-comey.html}{blame
Mr. Comey's late disclosure} for her defeat. Former campaign aides
expressed disbelief Thursday at another revelation in the report ---
that Mr. Comey had used a private email account to conduct official
F.B.I. business while he supervised the investigation into Mrs.
Clinton's email practices. ``I don't know whether to laugh or cry,''
said Brian Fallon, the former campaign spokesman.

And Mrs. Clinton herself responded on Twitter, noting only, ``But my
emails.''

Mr. Comey has defended his actions, saying he would have faced criticism
for any decision, so he opted to be transparent. F.B.I. officials have
acknowledged that they made those decisions in part because they assumed
Mrs. Clinton would win, and they worried about appearing to conceal
information to help her.

Mr. Comey and his agents also grew suspicious of Justice Department
prosecutors. Working-level agents wanted prosectors to be more
aggressive --- a tension that the inspector general found ``caused
significant strife and mistrust'' between the two groups.

Mr. Comey, too, said his decisions were influenced in part by concerns
that political appointees at the Justice Department did not have the
credibility to close the investigation.
\href{https://www.nytimes.com/2018/06/14/opinion/comey-clinton-inspector-general.html}{In
an Op-Ed published in The New York Times} responding to the report, Mr.
Comey said he believed he was making the right decisions at the time.

``As painful as the whole experience has been, I still believe that,''
he wrote. ``And nothing in the inspector general's report makes me think
we did the wrong thing.''

Mr. Comey has cultivated a reputation for fierce independence and
supreme self-confidence. Those traits were both assets and
vulnerabilities. Agents
\href{https://www.nytimes.com/2017/08/16/us/politics/comey-fbi-agents-confidence-survey.html}{widely
saw him} as a strong leader.

But Mr. Comey believed that he was the only one who could steer the
F.B.I. through the political winds of the Clinton case, and that left
him alone to answer for the bureau's actions.

Officially at least, Mr. Comey's handling of the Clinton case cost him
his job. After the firing, the White House held up as justification a
Justice Department memo that criticized many of the actions now
highlighted by the inspector general. In that regard, the inspector
general would seem to underscore the stated reason for Mr. Comey's
dismissal.

But Mr. Trump has muddied this issue. Hours after the firing, he
undercut his own staff and said that he had planned to fire Mr. Comey
even before receiving the recommendation. He said he had been thinking
about the Russia investigation when he fired Mr. Comey. His lawyer,
Rudolph W. Giuliani, added more recently that Mr. Comey was fired for
refusing to publicly exonerate Mr. Trump in the Russia case.

Those comments, along with
\href{https://www.nytimes.com/2017/05/18/us/politics/james-comey-memo-fbi-trump.html}{Mr.
Comey's account of private conversations with the president}, prompted
the appointment of a special counsel to begin investigating Mr. Trump
for possible obstruction of justice. That inquiry continues. The
inspector general's report does not directly affect that case, though
anything that undermines Mr. Comey's credibility is politically and
legally beneficial to Mr. Trump.

The inspector general is separately reviewing some aspects of the Russia
investigation, including Mr. Trump's theory --- backed up by no evidence
--- that the F.B.I. spied on his campaign for political purposes. Those
matters were not covered in Thursday's report.

Mr. Horowitz's investigation has already
\href{https://www.nytimes.com/2018/03/16/us/politics/andrew-mccabe-fbi-fired.html}{led
to the firing of one top F.B.I. official}, the former deputy director
Andrew G. McCabe. Mr. Horowitz issued a report in March that said Mr.
McCabe had been dishonest about his contacts with the news media about
Mrs. Clinton.

Mr. McCabe has been a frequent target of Mr. Trump's ire and is central
to his theory that the F.B.I. secretly worked to exonerate Mrs. Clinton.
Mr. McCabe's wife ran unsuccessfully as a Democrat for the Virginia
State Senate and received significant campaign donations from an ally of
Mrs. Clinton. Despite the president's criticism, the inspector general
said on Thursday that Mr. McCabe had not been required to recuse himself
from the Clinton case.

Among Mr. Horowitz's original tasks was to identify whether F.B.I.
agents improperly disclosed information about the Clinton case to
reporters. But his inquiry was stymied, he said, because improper
contacts with journalists were so common. ``The large number of F.B.I.
employees who were in contact with journalists during this time period
impacted our ability to identify the sources of leaks,'' he wrote.

The report omitted any discussion of a potential leak of information in
fall 2016 to Mr. Giuliani, who was then one of Mr. Trump's key campaign
surrogates but not yet his lawyer. Shortly before Mr. Comey announced
the discovery of new emails in the Clinton case, Mr. Giuliani appeared
on Fox News and hinted that major news was about to break: ``I mean, I'm
talking about some pretty big surprises,'' he said.

Mr. Horowitz has indicated that another report addressing leaks is
forthcoming. It is not clear whether Mr. Giuliani's remarks will be
addressed.

Advertisement

\protect\hyperlink{after-bottom}{Continue reading the main story}

\hypertarget{site-index}{%
\subsection{Site Index}\label{site-index}}

\hypertarget{site-information-navigation}{%
\subsection{Site Information
Navigation}\label{site-information-navigation}}

\begin{itemize}
\tightlist
\item
  \href{https://help.nytimes.com/hc/en-us/articles/115014792127-Copyright-notice}{©~2020~The
  New York Times Company}
\end{itemize}

\begin{itemize}
\tightlist
\item
  \href{https://www.nytco.com/}{NYTCo}
\item
  \href{https://help.nytimes.com/hc/en-us/articles/115015385887-Contact-Us}{Contact
  Us}
\item
  \href{https://www.nytco.com/careers/}{Work with us}
\item
  \href{https://nytmediakit.com/}{Advertise}
\item
  \href{http://www.tbrandstudio.com/}{T Brand Studio}
\item
  \href{https://www.nytimes.com/privacy/cookie-policy\#how-do-i-manage-trackers}{Your
  Ad Choices}
\item
  \href{https://www.nytimes.com/privacy}{Privacy}
\item
  \href{https://help.nytimes.com/hc/en-us/articles/115014893428-Terms-of-service}{Terms
  of Service}
\item
  \href{https://help.nytimes.com/hc/en-us/articles/115014893968-Terms-of-sale}{Terms
  of Sale}
\item
  \href{https://spiderbites.nytimes.com}{Site Map}
\item
  \href{https://help.nytimes.com/hc/en-us}{Help}
\item
  \href{https://www.nytimes.com/subscription?campaignId=37WXW}{Subscriptions}
\end{itemize}
