Sections

SEARCH

\protect\hyperlink{site-content}{Skip to
content}\protect\hyperlink{site-index}{Skip to site index}

\href{https://www.nytimes.com/section/world/europe}{Europe}

\href{https://myaccount.nytimes.com/auth/login?response_type=cookie\&client_id=vi}{}

\href{https://www.nytimes.com/section/todayspaper}{Today's Paper}

\href{/section/world/europe}{Europe}\textbar{}Russia Tried to Shut Down
Telegram. Websites Were Collateral Damage.

\href{https://nyti.ms/2JWQSIA}{https://nyti.ms/2JWQSIA}

\begin{itemize}
\item
\item
\item
\item
\item
\end{itemize}

Advertisement

\protect\hyperlink{after-top}{Continue reading the main story}

Supported by

\protect\hyperlink{after-sponsor}{Continue reading the main story}

\hypertarget{russia-tried-to-shut-down-telegram-websites-were-collateral-damage}{%
\section{Russia Tried to Shut Down Telegram. Websites Were Collateral
Damage.}\label{russia-tried-to-shut-down-telegram-websites-were-collateral-damage}}

\includegraphics{https://static01.nyt.com/images/2018/04/19/world/europe/19russia-telegram-print/merlin_136938870_df4222f6-f02f-46ae-897d-784bcd83a1a0-articleLarge.jpg?quality=75\&auto=webp\&disable=upscale}

By \href{https://www.nytimes.com/by/neil-macfarquhar}{Neil MacFarquhar}

\begin{itemize}
\item
  April 18, 2018
\item
  \begin{itemize}
  \item
  \item
  \item
  \item
  \item
  \end{itemize}
\end{itemize}

MOSCOW --- Russia's communications watchdog was locked in an intensive
game of whack-a-mole on Wednesday with Telegram, the popular and highly
secure messaging app, as its stuttering attempts to block the service
inadvertently knocked out the websites of scores of small businesses.

Alexander Zharov, the head of the watchdog agency, Roskomnadzor,
acknowledged that it had obstructed millions of IP addresses in an
attempt to shutter Telegram. Roskomnadzor took that step after the
company declined to provide encryption information, which would enable
the agency to identify users and see the content of messages.

The watchdog agency was granted authority to block the app on Friday by
a Moscow court. But the clumsy, unprecedented effort to follow through,
which started Monday, caused a widespread outcry after the unintended
consequences became apparent.

A member of the band Pussy Riot organized a small public protest. Edward
J. Snowden, the former N.S.A. contractor who leaked American
surveillance documents and now lives in Russia, expressed support for
the company. Even some usually staunch Kremlin supporters called the ban
misguided.

Internet experts in Russia saw a greater menace in the effort beyond
Telegram, speculating that if the government succeeded in silencing the
app with some 13 million Russian users, it might pursue bigger fish
next.

Indeed, in his
\href{https://iz.ru/733380/siuzanna-farizova/so-svobodoi-vse-khorosho-s-otvetstvennostiu-plokho}{interview
with} the newspaper Izvestia published on Wednesday, Mr. Zharov said
that Roskomnadzor planned to investigate Facebook before the end of the
year.

``If you have a government agency being so brutal attacking Telegram,
taking down millions of IP addresses, after that you can do what you
want,'' said Andrei Soldatov, one of the authors of ``The Red Web,'' a
history of the Russian internet. ``After they have faced this much heat,
they can go after Facebook and Google.''

Telegram tried to thwart the blockage by shifting its service to two
giant American web hosts, Google Cloud and Amazon Web Services, while at
the same time repeatedly changing its IP address to skip ahead of
Roskomnadzor.

In response, rather than chasing individual IP addresses --- a unique
set of numbers that identifies a computer, smartphone or other device
connected to the internet --- the watchdog agency elected to shut down
enormous blocks of addresses, called subnets.

The collateral damage hit a variety of other sites, like Viber, another
messaging app, as well as small businesses including a language school
and a courier service, all of which suffered financial losses.

Volvo dealerships could not access their service records, according to
press reports, and Kremlin museums had to suspend ticket sales.
Roskomnadzor said it unblocked individual sites as soon as the agency
became aware of a problem.

The Agora group of human rights lawyers, which represents Telegram in
Russian courts, said in a statement that it had received 73 complaints
about blocked websites. The organization planned to file a formal
complaint with the prosecutor general's office.

In addition to the virtual warfare, the two sides sparred publicly. Mr.
Zharov told the independent
\href{https://thebell.io/glava-roskomnadzora-telegram-zablokirovan-na-30/}{Russian
news outlet The Bell} that his agency had been able to cut off one-third
of the traffic to Telegram, while the company said the figure was 5
percent. The Bell suggested that traffic even rose on the day the
initial blocks had been imposed.

\includegraphics{https://static01.nyt.com/images/2018/04/19/world/europe/19russia-telegram2/merlin_89983057_33090939-856d-438b-a2cb-8a0e03691ea6-articleLarge.jpg?quality=75\&auto=webp\&disable=upscale}

Telegram has been sending messages to users encouraging them to use
alternative means, including Virtual Private Networks, which effectively
connect to the internet outside Russia, to evade the ban.

The app's founder, Pavel Durov, who fled Russia after losing control in
2014 of VKontakte, the Russian version of Facebook that he created, sent
a message to Telegram users saying that he would continue to engage in
``digital resistance.''

His supporters included Mr. Snowden, who wrote on Twitter that
resistance to the ``totalitarian demand'' for backdoor access to private
communications ``is the only moral response, and shows real
leadership.''

Maria V. Alyokhina, a political activist and member of the Pussy Riot
collective who has been jailed for previous demonstrations, helped to
organize a protest that consisted of tossing paper airplanes outside the
Moscow headquarters of the F.S.B., the secret police. She was sentenced
to 100 hours of compulsory community service, according to Mediazona, a
courtroom news agency.

The paper airplane is the corporate symbol of the Telegram app. The
F.S.B. has been leading the effort in Russia to gain backdoor access to
all private communications in Russia, saying it was necessary to combat
terrorism.

The ban came into effect
\href{https://www.nytimes.com/2018/04/13/world/europe/russia-telegram-encryption.html}{after
Telegram lost a court case} over demands from the Russian government
that the company provide the means for intelligence agencies to read
encrypted messages.

Telegram was widely used within the Russian government, including by
President Vladimir V. Putin's press office, to communicate with the
public and the media. Some government agencies and press offices
continued to use Telegram despite the ban.

Igor Lebedev, the deputy chairman of the Russian Parliament and the son
of nationalist leader Vladimir Zhirinovsky, accused the government
watchdog agency of ``incompetence.''

``It targeted Telegram but hit Russian business,'' he wrote on Twitter,
saying it should be illegal to limit access to such popular sites. ``We
must defend Russian citizens from pointless bans!''

The Kremlin has also come in for criticism on several other fronts: It
has suggested using older, slower apps as an alternative, especially one
owned by a Putin crony; it shut down LinkedIn and managed to limit some
access to Zello, a voice messaging app used to organize truck driver
protests; and it finds itself accused of paying lip service to the idea
that Russia should develop cutting edge technology while busily
undermining Telegram, one of the few Russian I.T. brands to achieve
global success.

Iran seemed to be following Moscow's lead, with
\href{http://www.presstv.com/Detail/2018/04/18/558899/Iran-Leader-Telegram}{news
agencies reporting on Wednesday} that it would ban government
organizations from using foreign-based messaging services including
Telegram, which is extremely popular in the Islamic Republic. More than
40 million Iranians use Telegram, which played an important role in
spreading information about antigovernment protests in December and
January.

The office of the supreme leader, Ayatollah Ali Khamenei, an active
Telegram user, said he had shut down his account to set the example. The
decision was ``in line with safeguarding national interests and removing
the monopoly of the Telegram messaging app,'' the ayatollah's last
Telegram stated, according to Iran's government-run Press TV news
website.

The Russian government has been trying for several years
\href{https://www.nytimes.com/2016/12/06/world/europe/russia-putin-cyberattacks.html}{to
force all communications giants} to store their information on servers
in Russia and to provide the security services back door access to
analyze the data. It has also suggested that Russia might unplug from
the global internet and create its own web.

Internet experts suggest that what the Kremlin really wants is leverage
to order the internet platforms to remove things that it does not like.

``They want a hotline that they can pick up to call someone to do
something,'' said Mr. Soldatov. ``It is a very Soviet approach.''

Advertisement

\protect\hyperlink{after-bottom}{Continue reading the main story}

\hypertarget{site-index}{%
\subsection{Site Index}\label{site-index}}

\hypertarget{site-information-navigation}{%
\subsection{Site Information
Navigation}\label{site-information-navigation}}

\begin{itemize}
\tightlist
\item
  \href{https://help.nytimes.com/hc/en-us/articles/115014792127-Copyright-notice}{©~2020~The
  New York Times Company}
\end{itemize}

\begin{itemize}
\tightlist
\item
  \href{https://www.nytco.com/}{NYTCo}
\item
  \href{https://help.nytimes.com/hc/en-us/articles/115015385887-Contact-Us}{Contact
  Us}
\item
  \href{https://www.nytco.com/careers/}{Work with us}
\item
  \href{https://nytmediakit.com/}{Advertise}
\item
  \href{http://www.tbrandstudio.com/}{T Brand Studio}
\item
  \href{https://www.nytimes.com/privacy/cookie-policy\#how-do-i-manage-trackers}{Your
  Ad Choices}
\item
  \href{https://www.nytimes.com/privacy}{Privacy}
\item
  \href{https://help.nytimes.com/hc/en-us/articles/115014893428-Terms-of-service}{Terms
  of Service}
\item
  \href{https://help.nytimes.com/hc/en-us/articles/115014893968-Terms-of-sale}{Terms
  of Sale}
\item
  \href{https://spiderbites.nytimes.com}{Site Map}
\item
  \href{https://help.nytimes.com/hc/en-us}{Help}
\item
  \href{https://www.nytimes.com/subscription?campaignId=37WXW}{Subscriptions}
\end{itemize}
