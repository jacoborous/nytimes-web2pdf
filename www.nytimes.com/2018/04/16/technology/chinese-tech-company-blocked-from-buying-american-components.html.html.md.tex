Sections

SEARCH

\protect\hyperlink{site-content}{Skip to
content}\protect\hyperlink{site-index}{Skip to site index}

\href{https://www.nytimes.com/section/technology}{Technology}

\href{https://myaccount.nytimes.com/auth/login?response_type=cookie\&client_id=vi}{}

\href{https://www.nytimes.com/section/todayspaper}{Today's Paper}

\href{/section/technology}{Technology}\textbar{}Chinese Tech Company
Blocked From Buying American Components

\url{https://nyti.ms/2qEDUX4}

\begin{itemize}
\item
\item
\item
\item
\item
\end{itemize}

Advertisement

\protect\hyperlink{after-top}{Continue reading the main story}

Supported by

\protect\hyperlink{after-sponsor}{Continue reading the main story}

\hypertarget{chinese-tech-company-blocked-from-buying-american-components}{%
\section{Chinese Tech Company Blocked From Buying American
Components}\label{chinese-tech-company-blocked-from-buying-american-components}}

\includegraphics{https://static01.nyt.com/images/2018/04/17/business/17dc-zte-1/merlin_134629200_93e43de3-365a-4c81-b6f1-8ba58f019e8e-articleLarge.jpg?quality=75\&auto=webp\&disable=upscale}

By \href{https://www.nytimes.com/by/paul-mozur}{Paul Mozur} and
\href{https://www.nytimes.com/by/ana-swanson}{Ana Swanson}

\begin{itemize}
\item
  April 16, 2018
\item
  \begin{itemize}
  \item
  \item
  \item
  \item
  \item
  \end{itemize}
\end{itemize}

\href{http://cn.nytimes.com/technology/20180417/chinese-tech-company-blocked-from-buying-american-components/}{阅读简体中文版}\href{http://cn.nytimes.com/technology/20180417/chinese-tech-company-blocked-from-buying-american-components/zh-hant/}{閱讀繁體中文版}

SHANGHAI --- China's second-largest maker of telecommunications
equipment will not be able to use components made in the United States
after the Commerce Department said it failed to punish employees who
violated American sanctions against Iran and North Korea.

The
\href{https://www.commerce.gov/sites/commerce.gov/files/zte_denial_order.pdf}{ban
announced Monday}, which effectively locks the company, ZTE Corporation,
out of American technology for seven years, is a blow to one of China's
few truly international technology suppliers.

ZTE's products for the infrastructure of telecommunications networks, as
well as its smartphones, use an array of American parts, like
microprocessors from the chip maker Qualcomm, glass made by Corning and
sound technology from San Francisco-based Dolby.

In a call with reporters on Monday, a senior Commerce Department
official said the action was not connected to a broader intellectual
property investigation into China.

But the tough restrictions on ZTE may be perceived as a new salvo in a
deepening economic conflict between the United States and China as the
Trump administration and the Chinese government trade threats of
increased tariffs on everything from washing machines to soy beans. The
Trump administration's move to counter what it terms unfair trade
practices by China has prompted threats of hundreds of billions of
dollars in tariffs on products that travel between the two countries.

That trade clash now centers heavily on cutting-edge technology. The
Trump administration accuses China of using coercion and illicit means
to obtain American technology. In particular, it has criticized an
industrial plan known as Made in China 2025 that seeks to make China a
world leader in industries like robotics, electric cars and medical
devices.

In a bid to stop China from dominating these industries, the White House
has proposed limiting American exports of semiconductors and advanced
machinery to the country. That could happen through new investment
restrictions, which are set to be announced in the coming months.

While China has long been viewed as the lower-cost producer for
technology companies in the United States, it has in recent years gained
considerable ground in areas like artificial intelligence. Last year,
China unveiled a plan to
\href{https://www.nytimes.com/2018/02/12/technology/china-trump-artificial-intelligence.html}{become
the world leader in artificial intelligence}and create an industry worth
\$150 billion to its economy by 2030.

In a Twitter post Monday morning, President Trump also took aim at
China, accusing the country of devaluing its currency.

The president's criticism contradicted
\href{https://www.nytimes.com/2018/04/13/us/politics/trump-china-currency-manipulator.html}{a
report released just three days earlier} by the Treasury Department that
scolded China for its lack of progress in reducing the bilateral trade
deficit with the United States, but did not find that it was improperly
devaluing its currency.

China, the Treasury report said, has allowed its currency to appreciate
only gradually, and ``the distortion in the global trading system
resulting from China's currency policy over this period imposed
significant and long-lasting hardship on American workers and
companies.''

ZTE representatives could not be reached for comment.

ZTE's troubles with the American government predate recent tensions.
Last year, the
\href{https://www.nytimes.com/2017/03/07/technology/zte-china-fine.html}{Chinese
company agreed to pay \$1.19 billion in fines}in a settlement that
followed a multiyear investigation into claims that the company sold
electronics to Iran and North Korea, a violation of economic sanctions
imposed by the United States against the countries.

Before the settlement, the Commerce Department had provided two internal
ZTE documents to bolster its case.

One, from 2011 and signed by several senior ZTE executives, detailed how
the company had ``ongoing projects in all five major embargoed countries
--- Iran, Sudan, North Korea, Syria and Cuba.'' Another document, in a
complex flow chart, laid out a method for circumventing United States
export controls.

The 2017 settlement, which did not block ZTE from buying technology made
in the United States, appeared to be a reprieve for the Chinese tech
company.

But the Commerce Department said Monday that ZTE had violated the terms
of the earlier agreement by making false statements to the United States
government. ZTE did not take a number of actions that it said it had:
the company did not reprimand employees involved in the banned sales and
did not cancel their bonuses.

``ZTE misled the Department of Commerce,'' Wilbur Ross, the commerce
secretary, said in a statement. ``Insead of reprimanding ZTE staff and
senior management, ZTE rewarded them. This egregious behavior cannot be
ignored.''

The department said it would reinstate the limits on the company that
the 2017 settlement had set aside.

Without access to key components from companies in the United States,
ZTE's smartphones will likely have to be redesigned. Even if new
suppliers can be found, the transition will represent a significant
disruption to production of the company's phones.

The ban also comes at a bad time for ZTE. Revenue from the expansion of
Chinese 4G cellular networks has slowed, and its smartphone business
faces major competition from new Chinese handset makers, as well as its
much larger Chinese rival, Huawei.

Huawei also faces an investigation into whether it broke sanctions on
Cuba, Iran, Sudan and Syria,
\href{https://www.nytimes.com/2017/04/26/business/huawei-investigation-sanctions-subpoena.html}{The
New York Times has reported}. Huawei has long been suspected of
committing espionage for China. In January,
\href{https://www.nytimes.com/2018/01/24/technology/personaltech/huawei-mate-10-pro-smartphone-review.html}{AT\&T
abruptly pulled out of a deal} to carry Huawei's top-of-the-line
smartphone, appearing to bend to pressure from Washington over security
concerns.

With the United States increasingly focused on barring China from
accessing its high-tech products through tariffs and investment
restrictions, the new punishment for the Chinese company could be viewed
as an escalation.

In response to the initial settlement, ZTE said it had strengthened its
compliance policies and changed its top management.

``ZTE acknowledges the mistakes it made, takes responsibility for them
and remains committed to positive change in the company,'' said Zhao
Xianming, chairman and chief executive of ZTE, in a statement last year.

Advertisement

\protect\hyperlink{after-bottom}{Continue reading the main story}

\hypertarget{site-index}{%
\subsection{Site Index}\label{site-index}}

\hypertarget{site-information-navigation}{%
\subsection{Site Information
Navigation}\label{site-information-navigation}}

\begin{itemize}
\tightlist
\item
  \href{https://help.nytimes.com/hc/en-us/articles/115014792127-Copyright-notice}{©~2020~The
  New York Times Company}
\end{itemize}

\begin{itemize}
\tightlist
\item
  \href{https://www.nytco.com/}{NYTCo}
\item
  \href{https://help.nytimes.com/hc/en-us/articles/115015385887-Contact-Us}{Contact
  Us}
\item
  \href{https://www.nytco.com/careers/}{Work with us}
\item
  \href{https://nytmediakit.com/}{Advertise}
\item
  \href{http://www.tbrandstudio.com/}{T Brand Studio}
\item
  \href{https://www.nytimes.com/privacy/cookie-policy\#how-do-i-manage-trackers}{Your
  Ad Choices}
\item
  \href{https://www.nytimes.com/privacy}{Privacy}
\item
  \href{https://help.nytimes.com/hc/en-us/articles/115014893428-Terms-of-service}{Terms
  of Service}
\item
  \href{https://help.nytimes.com/hc/en-us/articles/115014893968-Terms-of-sale}{Terms
  of Sale}
\item
  \href{https://spiderbites.nytimes.com}{Site Map}
\item
  \href{https://help.nytimes.com/hc/en-us}{Help}
\item
  \href{https://www.nytimes.com/subscription?campaignId=37WXW}{Subscriptions}
\end{itemize}
