Sections

SEARCH

\protect\hyperlink{site-content}{Skip to
content}\protect\hyperlink{site-index}{Skip to site index}

\href{https://www.nytimes.com/section/smarter-living}{Smarter Living}

\href{https://myaccount.nytimes.com/auth/login?response_type=cookie\&client_id=vi}{}

\href{https://www.nytimes.com/section/todayspaper}{Today's Paper}

\href{/section/smarter-living}{Smarter Living}\textbar{}The One Thing
That Protects a Laptop After It's Been Stolen

\url{https://nyti.ms/2GmTiit}

\begin{itemize}
\item
\item
\item
\item
\item
\item
\end{itemize}

Advertisement

\protect\hyperlink{after-top}{Continue reading the main story}

Supported by

\protect\hyperlink{after-sponsor}{Continue reading the main story}

\hypertarget{the-one-thing-that-protects-a-laptop-after-its-been-stolen}{%
\section{The One Thing That Protects a Laptop After It's Been
Stolen}\label{the-one-thing-that-protects-a-laptop-after-its-been-stolen}}

\includegraphics{https://static01.nyt.com/images/2018/03/15/us/15sl-encryption/merlin_134779146_cd3126d2-ead8-4f2f-9e9b-d0529d54e009-articleLarge.jpg?quality=75\&auto=webp\&disable=upscale}

By Whitson Gordon

\begin{itemize}
\item
  March 13, 2018
\item
  \begin{itemize}
  \item
  \item
  \item
  \item
  \item
  \item
  \end{itemize}
\end{itemize}

\href{https://www.nytimes.com/es/2018/03/19/computadora-encriptacion-robo/}{Leer
en español}

When your laptop is lost or stolen, you aren't just out \$800 (or more).
Your personal information is also accessible to whoever takes it, even
if you have a password.

``Unfortunately, a typical password-protected user account does nothing
to protect your data,'' says Dennis Stewart, a security engineer at
\href{https://www.ciphertechs.com/}{CipherTechs}. ``While the password
will prevent someone else from logging into your computer, an attacker
can still use other methods to copy files off.'' If thieves remove the
hard drive and put it into another computer, they have access to any
files you have stored on it. In some cases, they can even reset the
password on your PC and gain access to your email, passwords and other
personal information.

Thankfully, you can protect your data against both of these types of
attacks with encryption. ``Encryption is a mathematical process used to
jumble up data. If important files or whole devices are encrypted, there
is no way to make sense of them without the key,'' Mr. Stewart said.
That means if thieves try to access your information, they'll find only
a jumbled mess unless they have your password, and they won't be able to
simply reset that password if the device is encrypted.

Encrypting your hard drive isn't some super-technical process that only
security experts can perform, either --- anyone can do it on his or her
computer at home, and it should take only a few minutes to get up and
running.

\hypertarget{how-to-encrypt-your-hard-drive}{%
\subsection{How to Encrypt Your Hard
Drive}\label{how-to-encrypt-your-hard-drive}}

If you have a Windows laptop, you have a few options. Some Windows 10
devices come with encryption turned on by default, and you can check
this by going to Settings \textgreater{} System \textgreater{} About and
scrolling down to ``Device Encryption.'' You'll need to log into Windows
with a Microsoft account in order for this feature to work, but if your
laptop offers it, it's an easy and free way to protect your data.

If your laptop doesn't support Device Encryption, you can use Windows'
other built-in encryption tool: BitLocker. BitLocker is available only
on Professional versions of Windows and above (a \$99 upgrade for Home
edition users), but it's incredibly easy to set up. Just head to
Windows' Control Panel \textgreater{} System and Security \textgreater{}
Manage BitLocker. Select your operating system drive and click the
``Turn On BitLocker'' button, following the prompts to create a password
that will function as your encryption key. Be sure to store your
BitLocker key in a safe place --- somewhere not on that computer --- in
case something goes wrong!

If neither of those is an option, a free program called
\href{https://www.veracrypt.fr/}{VeraCrypt} can encrypt your entire hard
drive, requiring your password when you boot your computer. It's not
quite as simple, straightforward and built-in as Windows' Device
Encryption and BitLocker, but if it's your only option, it's worth
looking into.

Mac laptops are much more straightforward: All modern Macs (since about
2003) have a feature called FileVault that encrypts your entire system
drive. Just open your Mac's System Preferences, head to Security \&
Privacy and select the FileVault tab. Click the ``Turn On FileVault''
button to create a password and begin the encryption process. Store your
key in a safe place (not on that computer) in case you ever get locked
out.

Thankfully, modern iPhones and Android phones will automatically encrypt
your data as long as you use a PIN or password, so you need to worry
about enabling the feature only on your desktop and laptop computers. If
you have an Android phone with an SD card, however, you can enable
encryption for the SD card manually from the Lock Screen and Security
settings.

\hypertarget{dont-forget-your-password}{%
\subsection{Don't Forget Your
Password}\label{dont-forget-your-password}}

There is one catch to encryption: Your password is much, much more
important to remember. Data may be protected from intruders, but it's
also impossible for you to access should you forget your password or
recovery key. ``If a user forgets or loses their key, they're out of
luck,'' says Mr. Stewart. ``If a bad actor can't get at your data
without the key, you can't, either.'' Some businesses may grant IT
departments a master key, but not all do, and this can't be done
retroactively.

So if you're the type of person who forgets passwords often, it's
incredibly important you write it down --- along with the recovery key
you were given when you performed the above steps --- and keep it in a
safe place. Don't keep it on the computer you encrypted and don't keep
it out in plain sight --- put it in a physical safe that only you can
access. And no matter what, always keep a good backup of your data,
either on another hard drive (which you should also encrypt) or with a
cloud service like \href{https://www.backblaze.com/}{Backblaze} that
keeps your data secure.

Advertisement

\protect\hyperlink{after-bottom}{Continue reading the main story}

\hypertarget{site-index}{%
\subsection{Site Index}\label{site-index}}

\hypertarget{site-information-navigation}{%
\subsection{Site Information
Navigation}\label{site-information-navigation}}

\begin{itemize}
\tightlist
\item
  \href{https://help.nytimes.com/hc/en-us/articles/115014792127-Copyright-notice}{©~2020~The
  New York Times Company}
\end{itemize}

\begin{itemize}
\tightlist
\item
  \href{https://www.nytco.com/}{NYTCo}
\item
  \href{https://help.nytimes.com/hc/en-us/articles/115015385887-Contact-Us}{Contact
  Us}
\item
  \href{https://www.nytco.com/careers/}{Work with us}
\item
  \href{https://nytmediakit.com/}{Advertise}
\item
  \href{http://www.tbrandstudio.com/}{T Brand Studio}
\item
  \href{https://www.nytimes.com/privacy/cookie-policy\#how-do-i-manage-trackers}{Your
  Ad Choices}
\item
  \href{https://www.nytimes.com/privacy}{Privacy}
\item
  \href{https://help.nytimes.com/hc/en-us/articles/115014893428-Terms-of-service}{Terms
  of Service}
\item
  \href{https://help.nytimes.com/hc/en-us/articles/115014893968-Terms-of-sale}{Terms
  of Sale}
\item
  \href{https://spiderbites.nytimes.com}{Site Map}
\item
  \href{https://help.nytimes.com/hc/en-us}{Help}
\item
  \href{https://www.nytimes.com/subscription?campaignId=37WXW}{Subscriptions}
\end{itemize}
