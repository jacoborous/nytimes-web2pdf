Sections

SEARCH

\protect\hyperlink{site-content}{Skip to
content}\protect\hyperlink{site-index}{Skip to site index}

\href{https://myaccount.nytimes.com/auth/login?response_type=cookie\&client_id=vi}{}

\href{https://www.nytimes.com/section/todayspaper}{Today's Paper}

\href{/section/opinion}{Opinion}\textbar{}My Vietnam War

\href{https://nyti.ms/2DYtrur}{https://nyti.ms/2DYtrur}

\begin{itemize}
\item
\item
\item
\item
\item
\end{itemize}

Advertisement

\protect\hyperlink{after-top}{Continue reading the main story}

Supported by

\protect\hyperlink{after-sponsor}{Continue reading the main story}

\href{/section/opinion}{Opinion}

\href{/column/vietnam-67}{Vietnam '67}

\hypertarget{my-vietnam-war}{%
\section{My Vietnam War}\label{my-vietnam-war}}

By Nancy Biberman

\begin{itemize}
\item
  March 27, 2018
\item
  \begin{itemize}
  \item
  \item
  \item
  \item
  \item
  \end{itemize}
\end{itemize}

\includegraphics{https://static01.nyt.com/images/2018/03/27/opinion/27Vietnam-newsletter1/27Vietnam-newsletter1-articleLarge.jpg?quality=75\&auto=webp\&disable=upscale}

It had been a winter of pent-up tension with too much time spent
indoors. I was anxious to ditch my pea coat for a tank top and soak up
some rays and music on the broad steps outside of Low Library on
Columbia University's campus. But spring wasn't the only thing on my
mind.

In high school, our freedom anthem had been Bob Dylan's ``Like a Rolling
Stone.'' But by the time I was in college, in 1968, things no longer
felt so free, so open. Country Joe's cynical
``I-Feel-Like-I'm-Fixin'-to-die Rag'' better captured the mood.

I tried to explain my sullenness to my parents, hoping to damp down
their worries about their 19-year-old daughter, who had become an Ivy
League antiwar protester. Yes, I went to class and got good grades. But
my head and heart were in leaflet writing, dorm organizing, teach-ins
and sit-ins. My activism didn't really surprise them: I was very much my
parents' daughter. Our family was seeing this war in Vietnam through the
same eyes. What bound us was what we saw on the evening news every day:
women and children running from American aerial chemical bombardment,
their flesh burning, mouth's in silent screams. Still, they were
understandably concerned.

They say Vietnam was the first TV war, and we knew it at the time.
Television was in its infancy during my own young years. My dad built
one for our living room, his electrical engineering skills honed by
developing sophisticated weaponry for aerial bombardment fighting the
Japanese in the South Pacific during World War II.

I was born when the fight against the Nazis was barely in the rear-view
mirror, and never out of memory. Kids my age were schooled in violence:
a war that included mass genocide and the atomic bomb. Every day we were
reminded that another could start any day; in kindergarten we ``ducked''
under our desks and ``covered'' the backs of our necks to shield
ourselves against a nuclear bomb attack. The mushroom cloud had engulfed
Japan only five years earlier; it wasn't a stretch for our schools to
worry, and be prepared.

One Sunday morning in September 1963, four African American girls were
murdered by a bomb planted at a church in Birmingham, Ala. I was 15. Two
months later, on a November afternoon, my high school's public address
system summoned us to the gym for an assembly. The principal announced
that the president of the United States, John F. Kennedy, had been
assassinated in Dallas. We should get our things together and go home.
My family sat glued to the TV for days, until the funeral was over; we
saw images of a blood-soaked first lady watching the vice president,
Lyndon B. Johnson, take the oath of office on an airplane en route to
Washington.

I demanded that my dad explain why all this was happening, figuring he
knew about killing. Maybe because I was old enough, at 15, to
understand, maybe because I was insistent and inconsolable about
Birmingham and Dallas, but my father told me about his wartime
experiences, and showed me his medals and dog tags, along with my
parents' wedding photo taken June 4, 1944, two days before D-Day, my dad
looking young and cool in his Army uniform, about to be shipped out.

With one exception, my father answered my questions about the war.
First, why? Because we were attacked; the Japanese bombed our ships in
Pearl Harbor. They attacked us; we didn't start the fight. They were
fascists, he said. Dictators, tyrants. They ruled by fear, enslaving and
incinerating millions of Jews and many others thought to be different by
a racist ideology. Racism. I understood that word. ``Never again,'' he
said, would that be allowed to happen. Never would it be forgotten. My
father refused to tell me how many people he killed during the war, no
matter how many times I asked. No, he said. You know enough.

My parents, as first-generation Americans, were very much the children
of their own parents. All four of my grandparents arrived in our country
as refugees fleeing religious and political persecution, running for
their lives. They were penniless and spoke no English, but were
determined to make a better life for their children and children's
children (me). This was the family story, and even though I didn't speak
their language, I got the message.

On April 4, 1968, the Rev. Dr. Martin Luther King Jr. was assassinated
in Memphis. On hearing this news young people ran into the streets in
spontaneous grief and rage against the killing of a beloved leader. A
month later thousands of my classmates at Columbia barricaded ourselves
inside campus buildings, angry at the university's decision to build a
private gym for students in a small public park separating the campus
from Harlem, and protesting the university's involvement in Vietnam War
research for the military. The unending violence of the war, the
point-blank murders of civilians that we saw on TV, the murder of an
African American hero: With our bodies we said ``enough.'' I was
arrested with about 800 other students in April 1968 for criminal
trespass in a campus building.

What we did at Columbia wasn't the first major student strike of that
era; Howard University students deserve credit for being the first. But
we were both early to student antiwar activism, which grew into an
enormous movement of young people that eventually helped end the Vietnam
War.

It took way too long. Nearly 60,000 American soldiers died. Three
million Vietnamese were killed. More bombs were dropped on Vietnam than
in the entirety of World War II. Why? For what? Who attacked whom? What
ever happened to ``never again?''

Maybe we students should have done more to end the war sooner. More
demonstrations, more sit-ins, more tying up traffic and bringing normal
life to a standstill until the war stopped. We were young and
idealistic; we said ``enough'' to the slaughtering of innocents in our
name, and raged against the politicians who were too timid or too
corrupt, to stand up for what was right. Who refused to learn. Our
country was awash in uncontrollable violence, and unable to change.

But it did, eventually. It's a lesson that student activists today
should take to heart. Change will come.

Advertisement

\protect\hyperlink{after-bottom}{Continue reading the main story}

\hypertarget{site-index}{%
\subsection{Site Index}\label{site-index}}

\hypertarget{site-information-navigation}{%
\subsection{Site Information
Navigation}\label{site-information-navigation}}

\begin{itemize}
\tightlist
\item
  \href{https://help.nytimes.com/hc/en-us/articles/115014792127-Copyright-notice}{©~2020~The
  New York Times Company}
\end{itemize}

\begin{itemize}
\tightlist
\item
  \href{https://www.nytco.com/}{NYTCo}
\item
  \href{https://help.nytimes.com/hc/en-us/articles/115015385887-Contact-Us}{Contact
  Us}
\item
  \href{https://www.nytco.com/careers/}{Work with us}
\item
  \href{https://nytmediakit.com/}{Advertise}
\item
  \href{http://www.tbrandstudio.com/}{T Brand Studio}
\item
  \href{https://www.nytimes.com/privacy/cookie-policy\#how-do-i-manage-trackers}{Your
  Ad Choices}
\item
  \href{https://www.nytimes.com/privacy}{Privacy}
\item
  \href{https://help.nytimes.com/hc/en-us/articles/115014893428-Terms-of-service}{Terms
  of Service}
\item
  \href{https://help.nytimes.com/hc/en-us/articles/115014893968-Terms-of-sale}{Terms
  of Sale}
\item
  \href{https://spiderbites.nytimes.com}{Site Map}
\item
  \href{https://help.nytimes.com/hc/en-us}{Help}
\item
  \href{https://www.nytimes.com/subscription?campaignId=37WXW}{Subscriptions}
\end{itemize}
