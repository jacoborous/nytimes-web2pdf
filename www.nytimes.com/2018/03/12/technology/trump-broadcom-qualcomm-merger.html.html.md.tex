Sections

SEARCH

\protect\hyperlink{site-content}{Skip to
content}\protect\hyperlink{site-index}{Skip to site index}

\href{https://www.nytimes.com/section/technology}{Technology}

\href{https://myaccount.nytimes.com/auth/login?response_type=cookie\&client_id=vi}{}

\href{https://www.nytimes.com/section/todayspaper}{Today's Paper}

\href{/section/technology}{Technology}\textbar{}Trump Blocks Broadcom's
Bid for Qualcomm

\url{https://nyti.ms/2GmoNZY}

\begin{itemize}
\item
\item
\item
\item
\item
\item
\end{itemize}

Advertisement

\protect\hyperlink{after-top}{Continue reading the main story}

Supported by

\protect\hyperlink{after-sponsor}{Continue reading the main story}

\hypertarget{trump-blocks-broadcoms-bid-for-qualcomm}{%
\section{Trump Blocks Broadcom's Bid for
Qualcomm}\label{trump-blocks-broadcoms-bid-for-qualcomm}}

\includegraphics{https://static01.nyt.com/images/2018/03/13/business/13BROADCOM-1/merlin_133197015_fe02d95c-8d9a-472d-af16-7ac9bb4fe4bf-articleLarge.jpg?quality=75\&auto=webp\&disable=upscale}

By \href{https://www.nytimes.com/by/cecilia-kang}{Cecilia Kang} and
\href{https://www.nytimes.com/by/alan-rappeport}{Alan Rappeport}

\begin{itemize}
\item
  March 12, 2018
\item
  \begin{itemize}
  \item
  \item
  \item
  \item
  \item
  \item
  \end{itemize}
\end{itemize}

WASHINGTON --- President Trump on Monday blocked Broadcom's \$117
billion bid for the chip maker Qualcomm, citing national security
concerns and sending a clear signal that he was willing to take
extraordinary measures to promote his administration's increasingly
protectionist stance.

In
\href{https://www.whitehouse.gov/presidential-actions/presidential-order-regarding-proposed-takeover-qualcomm-incorporated-broadcom-limited/}{a
presidential order}, Mr. Trump said ``credible evidence'' had led him to
believe that if Singapore-based Broadcom were to acquire control of
Qualcomm, it ``might take action that threatens to impair the national
security of the United States.'' The acquisition, if it had gone
through, would have been the largest technology deal in history.

Mr. Trump's decision to prohibit the blockbuster deal underscored the
lengths that he is willing to go to shelter American companies from
foreign competition. In recent weeks, the president has turned to an
arsenal of tools --- including tariffs and an obscure government review
panel --- to ward off foreign control in American industries and, in
particular, thwart the rise of China.

The president has
\href{https://www.nytimes.com/2018/03/06/business/us-china-trade-technology-deals.html}{focused
many of these actions on the technology industry}. While the United
States has long claimed an advantage in tech, it is now facing
emboldened rivals in China, where the government has heavily invested in
everything from semiconductors to wireless networks to artificial
intelligence. Through its recent actions, the White House has revealed
its view that the country's national security is tied to its advancement
of those technologies.

National security was also cited by Mr. Trump last week when he
\href{https://www.nytimes.com/2018/03/08/us/politics/trump-tariff-announcement.html}{approved
stiff and sweeping tariffs} on imported steel and aluminum, saying those
imports were a threat to American manufacturing. Mr. Trump singled out
Chinese steel as a key factor in his decision; he has said that China
has routed steel through other countries and flooded the United States
with cheap metal.

``There is a perception within the administration that China does not
economically engage fairly with the United States and this action shows
it will exercise various remedies to adjust the playing field to even
the Sino-U.S. economic relationship,'' said Tony Balloon, the head of
the China corporate consulting practice at the law firm Alston \& Bird.

Mr. Trump was given an opening to block Broadcom's bid for San
Diego-based Qualcomm earlier this month. That was when the
\href{https://www.treasury.gov/resource-center/international/Pages/Committee-on-Foreign-Investment-in-US.aspx}{Committee
on Foreign Investment in the United States}, or Cfius, a government
panel that typically works behind closed doors and reviews deals only
after they are announced, said it would stall Broadcom's bid because of
national security concerns while it examined the deal.

Image

Hock Tan, Broadcom's chief executive, speaking alongside President Trump
at the White House in November. Days later the company announced that it
wanted to buy Qualcomm.Credit...Evan Vucci/Associated Press

Broadcom said it was reviewing Mr. Trump's order, and disputed the
notion that the bid posed a security threat.

``Broadcom strongly disagrees that its proposed acquisition of Qualcomm
raises any national security concerns,'' a company spokesman said in a
statement.

While Broadcom is based in Singapore, China was the main concern that
drove Mr. Trump's decision over the Qualcomm deal, because allowing an
American technology company to be acquired would cede its primacy in the
semiconductor and wireless industry.

Steven Mnuchin, the Treasury Secretary, said in a statement that the
decision was part of the administration's ``commitment to take all
actions necessary to protect the national security of the United
States.''

He said the order was based ``on the facts and national security
sensitivities related to this particular transaction only and is not
intended to make any other statement about Broadcom or its employees,
including its thousands of hard working and highly skilled U.S.
employees.''

Yet the order will undoubtedly raise questions about the extent to which
the Trump administration is willing to intervene in private-sector
decisions. While Qualcomm opposed Broadcom's bid and had reached out to
the foreign investment committee for a review, the proposal was
nonetheless headed to the company's shareholders for a vote. The foreign
investment committee intervened before that could happen, refusing to
let the shareholder meeting take place until after it had a chance to
investigate.

John P. Kabealo, an attorney who specializes in foreign investment
matters, said it was ``extraordinary'' that Mr. Trump would intervene in
the transaction before a full investigation by the government panel was
complete.

``It certainly aligns with the administration's willingness to be more
active in trade and implementing protectionist policies,'' he said. ``It
is definitely a much more activist policy than the previous
administration.''

Mr. Kabealo said the president's order could dramatically change the
world of mergers and acquisitions and open the door to the possibility
that more bankers and lawyers would use reviews by the foreign
investment committee to block hostile takeovers on national security
grounds.

The president said his decision to block Broadcom's bid had been based
on the review by the foreign investment committee. The panel had said
that the leadership of Qualcomm, which makes wireless chips and also
licenses key wireless patents, was too important to let go of. The
committee argued that economic leadership in
\href{https://www.nytimes.com/2018/03/06/technology/companies-countries-battling-5g.html}{next-generation
high-speed mobile networks known as 5G}, in which Qualcomm is a key
player, was also a national security interest.

``China would likely compete robustly to fill any void left by Qualcomm
as a result of this hostile takeover,'' a United States Treasury
official
\href{https://www.qcomvalue.com/wp-content/uploads/2018/03/Letter-from-Treasury-Department-to-Broadcom-and-Qualcomm-regarding-CFIUS.pdf}{wrote
in a letter} to the companies last week.

As part of the presidential order, the United States also barred the 15
individuals who Broadcom had proposed for Qualcomm's board from running,
saying they were ``disqualified from standing for election as directors
of Qualcomm.''

Qualcomm acknowledged receiving the presidential order and said it had
been told to reconvene its shareholder meeting at the earliest possible
date, March 23.

\begin{quote}
READ MORE:

\href{https://www.nytimes.com/2018/03/06/technology/companies-countries-battling-5g.html}{Why
Countries and Companies Are Battling for Ascendancy in 5G} (March 6,
2018)

\href{https://www.nytimes.com/2018/03/07/technology/broadcom-qualcomm-customers.html}{Broadcom's
Other Regulatory Hurdle: How It Treats Customers} (March 7, 2018)
\end{quote}

A presidential action against foreign investment in an American company
is rare and has only taken place four times in the past 30 years,
according to the law firm Ropes \& Gray. Scrutiny of foreign companies
buying United States assets ramped up under President Barack Obama,
including a presidential order barring a Chinese company from purchasing
Aixtron, a German company with American assets, on national security
grounds in 2016.

Under Mr. Trump, several deals involving foreign buyers have been
squelched after a review by the foreign investment committee, including
Moneygram's sale to an affiliate of the Alibaba Group and Lattice
Semiconductor's sale to an investment firm with reported ties to the
Chinese government. But the action against Broadcom was unusual because
mergers are rarely killed before a publicly traded company's
shareholders are given the chance to decide on the offer for themselves.

The decision was a blow to Broadcom, which under its chief executive,
Hock Tan, has built itself up through several acquisitions. Mr. Tan had
gone to great lengths to deflect concerns by American regulators and the
Trump administration, including appearing in a televised speech at the
White House with Mr. Trump last November, during which Mr. Tan promised
to redomicile Broadcom in the United States.

Broadcom had at one point in its negotiations with Qualcomm also offered
to raise its offer to acquire the company. After the foreign investment
committee announced its investigation into the Qualcomm bid this month,
Broadcom hastened plans to move its headquarters to the United States
and sent a letter to lawmakers
\href{https://www.nytimes.com/2018/03/07/business/dealbook/broadcom-qualcomm-5g-cfius.html}{promising
it would not slow research and development in 5G} networking technology
if the merger were approved.

Advertisement

\protect\hyperlink{after-bottom}{Continue reading the main story}

\hypertarget{site-index}{%
\subsection{Site Index}\label{site-index}}

\hypertarget{site-information-navigation}{%
\subsection{Site Information
Navigation}\label{site-information-navigation}}

\begin{itemize}
\tightlist
\item
  \href{https://help.nytimes.com/hc/en-us/articles/115014792127-Copyright-notice}{©~2020~The
  New York Times Company}
\end{itemize}

\begin{itemize}
\tightlist
\item
  \href{https://www.nytco.com/}{NYTCo}
\item
  \href{https://help.nytimes.com/hc/en-us/articles/115015385887-Contact-Us}{Contact
  Us}
\item
  \href{https://www.nytco.com/careers/}{Work with us}
\item
  \href{https://nytmediakit.com/}{Advertise}
\item
  \href{http://www.tbrandstudio.com/}{T Brand Studio}
\item
  \href{https://www.nytimes.com/privacy/cookie-policy\#how-do-i-manage-trackers}{Your
  Ad Choices}
\item
  \href{https://www.nytimes.com/privacy}{Privacy}
\item
  \href{https://help.nytimes.com/hc/en-us/articles/115014893428-Terms-of-service}{Terms
  of Service}
\item
  \href{https://help.nytimes.com/hc/en-us/articles/115014893968-Terms-of-sale}{Terms
  of Sale}
\item
  \href{https://spiderbites.nytimes.com}{Site Map}
\item
  \href{https://help.nytimes.com/hc/en-us}{Help}
\item
  \href{https://www.nytimes.com/subscription?campaignId=37WXW}{Subscriptions}
\end{itemize}
