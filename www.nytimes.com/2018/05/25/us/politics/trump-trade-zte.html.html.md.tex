Sections

SEARCH

\protect\hyperlink{site-content}{Skip to
content}\protect\hyperlink{site-index}{Skip to site index}

\href{https://www.nytimes.com/section/politics}{Politics}

\href{https://myaccount.nytimes.com/auth/login?response_type=cookie\&client_id=vi}{}

\href{https://www.nytimes.com/section/todayspaper}{Today's Paper}

\href{/section/politics}{Politics}\textbar{}Trump Administration Plans
to Revive ZTE, Prompting Backlash

\url{https://nyti.ms/2J5NrBT}

\begin{itemize}
\item
\item
\item
\item
\item
\end{itemize}

Advertisement

\protect\hyperlink{after-top}{Continue reading the main story}

Supported by

\protect\hyperlink{after-sponsor}{Continue reading the main story}

\hypertarget{trump-administration-plans-to-revive-zte-prompting-backlash}{%
\section{Trump Administration Plans to Revive ZTE, Prompting
Backlash}\label{trump-administration-plans-to-revive-zte-prompting-backlash}}

\includegraphics{https://static01.nyt.com/images/2018/05/26/business/26dc-zte-1-print/26dc-zte-articleLarge.jpg?quality=75\&auto=webp\&disable=upscale}

By \href{https://www.nytimes.com/by/ana-swanson}{Ana Swanson}

\begin{itemize}
\item
  May 25, 2018
\item
  \begin{itemize}
  \item
  \item
  \item
  \item
  \item
  \end{itemize}
\end{itemize}

\href{http://cn.nytimes.com/usa/20180528/trump-trade-zte/}{阅读简体中文版}\href{http://cn.nytimes.com/usa/20180528/trump-trade-zte/zh-hant/}{閱讀繁體中文版}

WASHINGTON --- The Trump administration told lawmakers it had reached a
deal that would keep the Chinese telecom firm ZTE alive, a person
familiar with the matter said, a move that could clear the way for
further trade talks with China but provoke anger in Congress.

Under the agreement brokered by the Commerce Department, ZTE would pay a
substantial fine, hire American compliance officers to be placed at the
firm and make changes to its current management team.

In return, the Commerce Department would lift a so-called denial order
that is preventing the company from buying American products, the person
said.

President Trump confirmed the news late Friday evening on Twitter, as he
criticized Senator Chuck Schumer of New York, the Democratic leader, and
the previous administration for their dealings with the company.

``Senator Schumer and Obama Administration let phone company ZTE
flourish with no security checks,'' he wrote. ``I closed it down then
let it reopen with high level security guarantees, change of management
and board, must purchase U.S. parts and pay a \$1.3 Billion fine.''

``Dems do nothing but complain and obstruct,'' he added. ``They made
only bad deals (Iran) and their so-called Trade Deals are the laughing
stock of the world!''

The deal would allow ZTE to once again begin doing business with
American companies, including Qualcomm, the chip maker based in San
Diego that is a primary ZTE supplier. The Chinese company was recently
banned from buying American technology components for seven years as
punishment for violating United States sanctions against Iran and North
Korea, a penalty that industry analysts say threatened to put the
company out of business within weeks.

\emph{{[}}\href{https://www.nytimes.com/2018/06/07/business/what-is-zte.html}{\emph{Catch
up}} \emph{on what ZTE is, and why President Trump wants to help it.{]}}

The collapse of ZTE would be an embarrassing outcome for China, and the
company's fate has become a hurdle in trade negotiations between the two
countries. President Trump directed the Commerce Department to
re-examine ZTE's penalty based on a personal request from President Xi
Jinping of China, setting off a fierce pushback from some of Mr. Trump's
national security advisers, as well as lawmakers from both parties.

Mr. Trump, however, has appeared unmoved by those concerns and has been
pushing to reach some type of trade resolution with China,
\href{https://www.nytimes.com/2018/05/21/us/politics/trump-trade-china.html}{which
has so far proved elusive}. The administration has been seeking to cut a
deal on ZTE in exchange for trade concessions from China, including
purchases of American agriculture and energy products, people familiar
with the discussions said. Wilbur Ross, the commerce secretary, is
scheduled to travel to China on June 2 to begin another round of the
talks with top Chinese officials.

Such an agreement is likely to face fierce resistance on Capitol Hill.
Top lawmakers, including Mr. Schumer and Senator Marco Rubio, Republican
of Florida, have urged the administration not to bend on ZTE, which they
consider a law enforcement and national security issue.

``ZTE presents a national security threat to the United States --- and
nothing in this reported deal addresses that fundamental fact,'' Senator
Chris Van Hollen, a Maryland Democrat, said in a statement. ``If
President Trump won't put our security before Chinese jobs, Congress
will act on a bipartisan basis to stop him.''

Lawmakers, including Mr. Van Hollen, have rolled out a variety of
measures aimed at clipping the administration's authority to ease
penalties on ZTE and have publicly criticized the administration's
consideration of a deal.

On Thursday, the House passed a bill that would prevent the
administration from easing restrictions on ZTE, and on Tuesday, the
Senate Banking Committee approved a similar amendment that would prevent
the president from modifying penalties on Chinese telecom companies that
had violated American law in the past year. A group of 27 bipartisan
senators also sent administration officials a letter last week warning
them not to ``compromise lawful U.S. enforcement actions against serial
and premeditated violators of U.S. law, such as ZTE.''

``Yes they have a deal in mind,'' Mr. Rubio said in a tweet on Friday.
``It is a great deal... for \#ZTE \& China.''

``Now congress will need to act,'' he added.

The telecom company's fate has consumed top administration officials,
who have tried to defuse lawmakers' concerns about a deal while
responding to Mr. Trump's entreaties to ``get it done.'' On Wednesday
afternoon, Mr. Ross and Steven Mnuchin, the secretary of treasury,
traveled to Capitol Hill to brief a group of Senate Republicans,
including Mr. Rubio, John Cornyn of Texas and Bob Corker of Tennessee,
on their plans for ZTE. Mr. Ross and Mr. Mnuchin sought to assure the
lawmakers that they were planning on harsh penalties for ZTE, and
appealed to Republicans to dampen their public criticism so a deal could
be reached, a person briefed on the discussions said.

``If the administration goes through with this reported deal, President
Trump would be helping make China great again,'' Mr. Schumer said in a
statement Friday. ``Simply a fine and changing board members would not
protect America's economic or national security, and would be a huge
victory for President Xi, and a dramatic retreat by President Trump.''

Defense officials have also been concerned about the Chinese telecom
firm and its products, which they believe may be vulnerable to Chinese
espionage or disruption. In early May, a spokesman for the Department of
Defense said the Pentagon was stopping the sale of phones made by ZTE
and a Chinese competitor, Huawei, in stores on American military bases
around the world because of security concerns.

The Chinese telecommunications firm has been on the brink of shutting
down, following penalties
\href{https://www.commerce.gov/news/press-releases/2018/04/secretary-ross-announces-activation-zte-denial-order-response-repeated}{imposed
by the Commerce Department in April} that severed important links in its
supply chain.

ZTE agreed to a \$1.19 billion fine and other penalties in March 2017,
after it was found to have violated American sanctions by selling
products with American-made parts to Iran and North Korea. In April, the
Commerce Department said it had found that ZTE had also made false
statements relating to disciplining senior officials, and announced a
seven-year ban on the company's purchases of American products.

That ban has crippled the Chinese firm and threatened to put tens of
thousands of Chinese employees of the company out of work. The Chinese
government had made clear that lifting ZTE's penalty would be a
condition for continuing with trade talks, and that if the penalty was
not lifted, American companies operating in China might face further
retaliation, people briefed on the discussions said.

Trump administration officials have said repeatedly in the last week
that ZTE is a law enforcement issue, and that it is being considered
independently from trade negotiations with China. But trade experts say
that the administration's actions and the president's own statements
indicate that ZTE's fate has become inextricably linked to Mr. Trump's
goal of reaching a trade deal with China.

On Thursday, Mr. Ross said that the administration was considering
installing a compliance team inside ZTE. ``We're developing a matrix of
things and while we haven't come quite to a final decision yet, we think
there may very well be an alternative that will be quite punitive to
them, but really modify behavior,'' Mr. Ross said on CNBC.

Advertisement

\protect\hyperlink{after-bottom}{Continue reading the main story}

\hypertarget{site-index}{%
\subsection{Site Index}\label{site-index}}

\hypertarget{site-information-navigation}{%
\subsection{Site Information
Navigation}\label{site-information-navigation}}

\begin{itemize}
\tightlist
\item
  \href{https://help.nytimes.com/hc/en-us/articles/115014792127-Copyright-notice}{©~2020~The
  New York Times Company}
\end{itemize}

\begin{itemize}
\tightlist
\item
  \href{https://www.nytco.com/}{NYTCo}
\item
  \href{https://help.nytimes.com/hc/en-us/articles/115015385887-Contact-Us}{Contact
  Us}
\item
  \href{https://www.nytco.com/careers/}{Work with us}
\item
  \href{https://nytmediakit.com/}{Advertise}
\item
  \href{http://www.tbrandstudio.com/}{T Brand Studio}
\item
  \href{https://www.nytimes.com/privacy/cookie-policy\#how-do-i-manage-trackers}{Your
  Ad Choices}
\item
  \href{https://www.nytimes.com/privacy}{Privacy}
\item
  \href{https://help.nytimes.com/hc/en-us/articles/115014893428-Terms-of-service}{Terms
  of Service}
\item
  \href{https://help.nytimes.com/hc/en-us/articles/115014893968-Terms-of-sale}{Terms
  of Sale}
\item
  \href{https://spiderbites.nytimes.com}{Site Map}
\item
  \href{https://help.nytimes.com/hc/en-us}{Help}
\item
  \href{https://www.nytimes.com/subscription?campaignId=37WXW}{Subscriptions}
\end{itemize}
