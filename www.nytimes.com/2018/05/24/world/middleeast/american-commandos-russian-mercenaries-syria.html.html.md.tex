Sections

SEARCH

\protect\hyperlink{site-content}{Skip to
content}\protect\hyperlink{site-index}{Skip to site index}

\href{https://www.nytimes.com/section/world/middleeast}{Middle East}

\href{https://myaccount.nytimes.com/auth/login?response_type=cookie\&client_id=vi}{}

\href{https://www.nytimes.com/section/todayspaper}{Today's Paper}

\href{/section/world/middleeast}{Middle East}\textbar{}How a 4-Hour
Battle Between Russian Mercenaries and U.S. Commandos Unfolded in Syria

\url{https://nyti.ms/2GMKOj0}

\begin{itemize}
\item
\item
\item
\item
\item
\item
\end{itemize}

Advertisement

\protect\hyperlink{after-top}{Continue reading the main story}

Supported by

\protect\hyperlink{after-sponsor}{Continue reading the main story}

\hypertarget{how-a-4-hour-battle-between-russian-mercenaries-and-us-commandos-unfolded-in-syria}{%
\section{How a 4-Hour Battle Between Russian Mercenaries and U.S.
Commandos Unfolded in
Syria}\label{how-a-4-hour-battle-between-russian-mercenaries-and-us-commandos-unfolded-in-syria}}

\includegraphics{https://static01.nyt.com/images/2018/05/25/world/middleeast/25dc-military1-print/merlin_133450685_869b2ae9-9cfa-4652-8421-c157d579e986-articleLarge.jpg?quality=75\&auto=webp\&disable=upscale}

By \href{https://www.nytimes.com/by/thomas-gibbons-neff}{Thomas
Gibbons-Neff}

\begin{itemize}
\item
  May 24, 2018
\item
  \begin{itemize}
  \item
  \item
  \item
  \item
  \item
  \item
  \end{itemize}
\end{itemize}

WASHINGTON --- The artillery barrage was so intense that the American
commandos dived into foxholes for protection, emerging covered in flying
dirt and debris to fire back at a column of tanks advancing under the
heavy shelling. It was the opening salvo in a nearly four-hour assault
in February by around 500 pro-Syrian government forces --- including
Russian mercenaries --- that threatened to inflame already-simmering
tensions between Washington and Moscow.

In the end, 200 to 300 of the attacking fighters were killed. The others
retreated under merciless airstrikes from the United States, returning
later to retrieve their battlefield dead. None of the Americans at the
small outpost in eastern Syria --- about 40 by the end of the firefight
--- were harmed.

The details of the Feb. 7 firefight were gleaned from interviews and
documents newly obtained by The New York Times. They provide the
Pentagon's first public on-the-ground accounting of one of the single
bloodiest battles the American military has faced in Syria since
deploying to fight the Islamic State.

The firefight was described by the Pentagon as an act of self-defense
against a unit of pro-Syrian government forces. In interviews, United
States military officials said they had watched --- with dread ---
hundreds of approaching rival troops, vehicles and artillery pieces in
the week leading up to the attack.

The prospect of Russian military forces and American troops colliding
has long been feared as the Cold War adversaries take opposing sides in
Syria's seven-year civil war.

At worst, officials and experts have said, it could plunge both
countries into bloody conflict. And at a minimum, squaring off in
crowded battlefields has added to heightened tensions between Russia and
the United States as they each seek to exert influence in the Middle
East.

Commanders of the rival militaries had long steered clear of the other
by speaking through often-used deconfliction telephone lines. In the
days leading up to the attack, and on opposite sides of the Euphrates
River, Russia and the United States were backing separate offensives
against the Islamic State in Syria's oil-rich Deir al-Zour Province,
which borders Iraq.

American military officials repeatedly warned about the growing mass of
troops. But Russian military officials said they had no control over the
fighters assembling near the river --- even though American surveillance
equipment monitoring radio transmissions had revealed the ground force
was speaking in Russian.

\includegraphics{https://static01.nyt.com/images/2018/05/25/world/middleeast/25dc-military2-print/merlin_133469702_585b5c27-d434-4ff4-95ed-7a563eee24de-articleLarge.jpg?quality=75\&auto=webp\&disable=upscale}

The documents described the fighters as a ``pro-regime force,'' loyal to
President Bashar al-Assad of Syria. It included some Syrian government
soldiers and militias, but American military and intelligence officials
have said a majority were private Russian paramilitary mercenaries ---
and most likely a part of the Wagner Group, a company often used by the
Kremlin to carry out objectives that officials do not want to be
connected to the Russian government.

``The Russian high command in Syria assured us it was not their
people,'' Defense Secretary Jim Mattis told senators in testimony last
month. He said he directed Gen. Joseph F. Dunford Jr., the chairman of
the Joint Chiefs of Staff, ``for the force, then, to be annihilated.''

``And it was.''

\hypertarget{amassing-forces}{%
\subsection{Amassing forces}\label{amassing-forces}}

The day began with little hint of the battle that was about to unfold.

A team of about 30 Delta Force soldiers and Rangers from the Joint
Special Operations Command were working alongside Kurdish and Arab
forces at a small dusty outpost next to a Conoco gas plant, near the
city of Deir al-Zour.

Roughly 20 miles away, at a base known as a mission support site, a team
of Green Berets and a platoon of infantry Marines stared at their
computer screens, watching drone feeds and passing information to the
Americans at the gas plant about the gathering fighters.

At 3 p.m. the Syrian force began edging toward the Conoco plant. By
early evening, more than 500 troops and 27 vehicles --- including tanks
and armored personnel carriers --- had amassed.

In the American air operations center at Al Udeid Air Base in Qatar, and
at the Pentagon, confounded military officers and intelligence analysts
watched the scene unfold. Commanders briefed pilots and ground crews.
Aircraft across the region were placed on alert, military officials
said.

Back at the mission support site, the Green Berets and Marines were
preparing a small reaction force --- roughly 16 troops in four
mine-resistant vehicles --- in case they were needed at the Conoco
plant. They inspected their weapons and ensured the trucks were loaded
with anti-tank missiles, thermal optics and food and water.

At 8:30 p.m., three Russian-made T-72 tanks --- vehicles weighing nearly
50 tons and armed with 125-millimeter guns --- moved within a mile of
the Conoco plant. Bracing for an attack, the Green Berets prepared to
launch the reaction force.

Image

Russian trucks heading to Deir al-Zour, an oil-rich Syrian province that
borders Iraq, last year.Credit...Omar Sanadiki/Reuters

At the outpost, American soldiers watched a column of tanks and other
armored vehicles turn and drive toward them around 10 p.m., emerging
from a neighborhood of houses where they had tried to gather undetected.

A half-hour later, the Russian mercenaries and Syrian forces struck.

The Conoco outpost was hit with a mixture of tank fire, large artillery
and mortar rounds, the documents show. The air was filled with dust and
shrapnel. The American commandos took cover, then ran behind dirt berms
to fire anti-tank missiles and machine guns at the advancing column of
armored vehicles.

For the first 15 minutes, American military officials called their
Russian counterparts and urged them to stop the attack. When that
failed, American troops fired warning shots at a group of vehicles and a
howitzer.

Still the troops advanced.

\hypertarget{from-the-horizon-a-barrage-of-artillery}{%
\subsection{From the horizon, a barrage of
artillery}\label{from-the-horizon-a-barrage-of-artillery}}

American warplanes arrived in waves, including Reaper drones, F-22
stealth fighter jets, F-15E Strike Fighters, B-52 bombers, AC-130
gunships and AH-64 Apache helicopters. For the next three hours,
American officials said, scores of strikes pummeled enemy troops, tanks
and other vehicles. Marine rocket artillery was fired from the ground.

The reaction team sped toward the fight. It was dark, according to the
documents, and the roads were littered with felled power lines and shell
craters. The 20-mile drive was made all the more difficult since the
trucks did not turn on their headlights, relying solely on
thermal-imaging cameras to navigate.

As the Green Berets and Marines neared the Conoco plant around 11:30
p.m., they were forced to stop. The barrage of artillery was too
dangerous to drive through until airstrikes silenced the enemy's
howitzers and tanks.

At the plant, the commandos were pinned down by enemy artillery and
burning through ammunition. Flashes from tank muzzles, antiaircraft
weapons and machine guns lit up the air.

At 1 a.m., with the artillery fire dwindling, the team of Marines and
Green Berets pulled up to the Conoco outpost and began firing. By then,
some of the American warplanes had returned to base, low on either fuel
or ammunition.

The United States troops on the ground, now roughly 40 in all, braced
their defenses as the mercenaries left their vehicles and headed toward
the outpost on foot.

A handful of Marines ran ammunition to machine guns and Javelin missile
launchers scattered along the berms and wedged among the trucks. Some of
the Green Berets and Marines took aim from exposed hatches. Others
remained in their trucks, using a combination of thermal screens and
joysticks to control and fire the heavy machine guns affixed on their
roofs.

A few of the commandos, including Air Force combat controllers, worked
the radios to direct the next fleet of bombers flying toward the
battlefield. At least one Marine exposed himself to incoming fire as he
used a missile guidance computer to find targets' locations and pass
them on to the commandos calling in the airstrikes.

An hour later, the enemy fighters had started to retreat and the
American troops stopped firing. From their outpost, the commandos
watched the mercenaries and Syrian fighters return to collect their
dead. The small team of American troops was not harmed. One allied
Syrian fighter was wounded.

\hypertarget{who-led-the-ill-fated-attack}{%
\subsection{Who led the ill-fated
attack?}\label{who-led-the-ill-fated-attack}}

The number of casualties from the Feb. 7 fight is in dispute.

Initially, Russian officials said only
\href{https://www.nytimes.com/2018/02/13/world/europe/russia-syria-dead.html}{four
Russian citizens --- but perhaps dozens more} --- were killed; a Syrian
officer said around 100 Syrian soldiers had died. The documents obtained
by The Times estimated 200 to 300 of the ``pro-regime force'' were
killed.

The outcome of the battle, and much of its mechanics, suggest that the
Russian mercenaries and their Syrian allies were in the wrong part of
the world to try a simple, massed assault on an American military
position. Since the 2003 invasion of Iraq, the United States Central
Command has refined the amount of equipment, logistics, coordination and
tactics required to mix weapons fired from both the air and ground.

Questions remain about exactly who the Russian mercenaries were, and why
they attacked.

American intelligence officials say that the Wagner Group, known by the
nickname of the retired Russian officer who leads it, is in Syria to
seize oil and gas fields and protect them on behalf of the Assad
government. The mercenaries earn of a share of the production proceeds
from the oil fields they reclaim, officials said.

The mercenaries loosely coordinate with the Russian military in Syria,
although Wagner's leaders have reportedly received awards in the
Kremlin, and its mercenaries are trained at the Russian Defense
Ministry's bases.

Russian government forces in Syria maintain they were not involved in
the battle. But in recent weeks, according to United States military
officials, they have jammed the communications of smaller American
drones and gunships such as the type used in the attack.

``Right now in Syria, we're in the most aggressive E.W. environment on
the planet from our adversaries,'' Gen. Tony Thomas, the head of United
States Special Operations Command, said recently, referring to
electronic warfare. ``They're testing us every day.''

Advertisement

\protect\hyperlink{after-bottom}{Continue reading the main story}

\hypertarget{site-index}{%
\subsection{Site Index}\label{site-index}}

\hypertarget{site-information-navigation}{%
\subsection{Site Information
Navigation}\label{site-information-navigation}}

\begin{itemize}
\tightlist
\item
  \href{https://help.nytimes.com/hc/en-us/articles/115014792127-Copyright-notice}{©~2020~The
  New York Times Company}
\end{itemize}

\begin{itemize}
\tightlist
\item
  \href{https://www.nytco.com/}{NYTCo}
\item
  \href{https://help.nytimes.com/hc/en-us/articles/115015385887-Contact-Us}{Contact
  Us}
\item
  \href{https://www.nytco.com/careers/}{Work with us}
\item
  \href{https://nytmediakit.com/}{Advertise}
\item
  \href{http://www.tbrandstudio.com/}{T Brand Studio}
\item
  \href{https://www.nytimes.com/privacy/cookie-policy\#how-do-i-manage-trackers}{Your
  Ad Choices}
\item
  \href{https://www.nytimes.com/privacy}{Privacy}
\item
  \href{https://help.nytimes.com/hc/en-us/articles/115014893428-Terms-of-service}{Terms
  of Service}
\item
  \href{https://help.nytimes.com/hc/en-us/articles/115014893968-Terms-of-sale}{Terms
  of Sale}
\item
  \href{https://spiderbites.nytimes.com}{Site Map}
\item
  \href{https://help.nytimes.com/hc/en-us}{Help}
\item
  \href{https://www.nytimes.com/subscription?campaignId=37WXW}{Subscriptions}
\end{itemize}
