Sections

SEARCH

\protect\hyperlink{site-content}{Skip to
content}\protect\hyperlink{site-index}{Skip to site index}

\href{https://www.nytimes.com/section/politics}{Politics}

\href{https://myaccount.nytimes.com/auth/login?response_type=cookie\&client_id=vi}{}

\href{https://www.nytimes.com/section/todayspaper}{Today's Paper}

\href{/section/politics}{Politics}\textbar{}Trump Mulls Options on ZTE
as Congress Tries to Force Tough Penalty

\url{https://nyti.ms/2J8h3Pe}

\begin{itemize}
\item
\item
\item
\item
\item
\end{itemize}

Advertisement

\protect\hyperlink{after-top}{Continue reading the main story}

Supported by

\protect\hyperlink{after-sponsor}{Continue reading the main story}

\hypertarget{trump-mulls-options-on-zte-as-congress-tries-to-force-tough-penalty}{%
\section{Trump Mulls Options on ZTE as Congress Tries to Force Tough
Penalty}\label{trump-mulls-options-on-zte-as-congress-tries-to-force-tough-penalty}}

\includegraphics{https://static01.nyt.com/images/2018/05/25/business/25dc-zte-1/25dc-zte-1-articleLarge.jpg?quality=75\&auto=webp\&disable=upscale}

By \href{https://www.nytimes.com/by/alan-rappeport}{Alan Rappeport}

\begin{itemize}
\item
  May 24, 2018
\item
  \begin{itemize}
  \item
  \item
  \item
  \item
  \item
  \end{itemize}
\end{itemize}

The Trump administration, over the objections of lawmakers, is
continuing to try to soften its punishment of ZTE, the Chinese
telecommunications firm that has emerged as a key sticking point in
trade negotiations between China and the United States.

The company is on the verge of shutting down after the United States
banned it from buying American components and has become a bargaining
chip between President Trump and his Chinese counterpart, Xi Jinping, as
the two leaders work to resolve a brewing trade war. Mr. Xi asked Mr.
Trump to revisit the ZTE decision because thousands of Chinese jobs
could be lost if the company is shuttered.

On Thursday, Wilbur Ross, the commerce secretary, said the
administration was considering placing an American compliance team
inside ZTE to ensure that the company was meeting its requirements and
not violating sanctions. The company was punished for violating United
States sanctions against Iran and North Korea and then lying about it.

``We're developing a matrix of things and while we haven't come quite to
a final decision yet, we think there may very well be an alternative
that will be quite punitive to them, but really modify behavior,'' Mr.
Ross said in an interview with CNBC on Thursday.

It is not clear if China would allow the United States to exert such a
high level of control over a company that is controlled by the Chinese
state, but Mr. Ross insisted that the Trump administration had
substantial leverage in the matter.

ZTE buys many of its components from American companies, including the
San Diego-based chip maker Qualcomm. In return for lifting the ban on
purchasing American components, the administration could be considering
a substantial fine, changes to ZTE's management and other limits on its
ability to operate in the United States.

\emph{{[}}\href{https://www.nytimes.com/2018/06/07/business/what-is-zte.html}{\emph{Catch
up}} \emph{on what ZTE is and why President Trump wants to help it.{]}}

``I envision a very large fine,'' the president said on Tuesday, as he
remarked that ZTE's purchases supported American jobs. ``I envision,
perhaps, new management, new board of directors, very tight security
rules.''

\includegraphics{https://static01.nyt.com/images/2018/05/25/business/25dc-zte-2/25dc-zte-2-articleLarge.jpg?quality=75\&auto=webp\&disable=upscale}

Mr. Trump's willingness to reconsider the company's penalty has drawn
bipartisan backlash in Congress, and lawmakers fear that the
administration is unwisely linking national security and trade.

``How is China supposed to think about what it is the United States is
trying to accomplish with respect to China right now?'' said Senator
Michael Bennet, Democrat of Colorado. ``First he says he's going to
sanction them, then he doesn't sanction them.''

On Thursday, the House took steps to hamstring the Trump
administration's flexibility to maneuver on ZTE. As part of a defense
measure that passed the House with overwhelming bipartisan support, the
Defense Department would be prohibited from renewing contracts that do
business with ZTE.

\emph{{[}To understand why President Trump might cave to China in trade
talks,}
\href{https://www.nytimes.com/2018/06/07/business/trump-trade-china-iowa-soybeans.html}{\emph{look
to Iowa soybean farmers}}\emph{.{]}}

On Tuesday, the Senate Banking Committee voted to approve an amendment
to a bill on foreign investment controls that would require the
president to certify that the company was no longer violating United
States law, had not done so for a year and was fully cooperating with
investigators before changing its penalties.

In an attempt to assuage concerns on Capitol Hill, Mr. Ross and Steven
Mnuchin, the Treasury secretary, met with a group of Senate Republicans
on Wednesday afternoon. According to an administration official, the
intent of the meeting was to explain the Commerce Department's thinking
on ZTE as an enforcement matter and to answer questions.

However, a person briefed on the discussions said that the
administration officials encouraged Republicans to stand down on their
criticism of ZTE. The administration is seeking to make a deal soon with
China because ZTE most likely has just weeks left in operation as a
result of the penalty, and the Chinese have made it clear they will not
deliver trade-related concessions like purchasing more agricultural
goods and energy from the United States without an agreement that lifts
the denial order, the person said. American technology firms in China
may also face retaliation from the Chinese government if ZTE does shut
down.

After the meeting, Senator Marco Rubio, Republican of Florida, said the
Trump administration was listening to the concerns of lawmakers, but he
did not appear to be completely reassured.

``I think they would prefer us not to act on it, but I think Congress is
going to do what it needs to do,'' Mr. Rubio said. ``This is a national
security issue.''

Advertisement

\protect\hyperlink{after-bottom}{Continue reading the main story}

\hypertarget{site-index}{%
\subsection{Site Index}\label{site-index}}

\hypertarget{site-information-navigation}{%
\subsection{Site Information
Navigation}\label{site-information-navigation}}

\begin{itemize}
\tightlist
\item
  \href{https://help.nytimes.com/hc/en-us/articles/115014792127-Copyright-notice}{©~2020~The
  New York Times Company}
\end{itemize}

\begin{itemize}
\tightlist
\item
  \href{https://www.nytco.com/}{NYTCo}
\item
  \href{https://help.nytimes.com/hc/en-us/articles/115015385887-Contact-Us}{Contact
  Us}
\item
  \href{https://www.nytco.com/careers/}{Work with us}
\item
  \href{https://nytmediakit.com/}{Advertise}
\item
  \href{http://www.tbrandstudio.com/}{T Brand Studio}
\item
  \href{https://www.nytimes.com/privacy/cookie-policy\#how-do-i-manage-trackers}{Your
  Ad Choices}
\item
  \href{https://www.nytimes.com/privacy}{Privacy}
\item
  \href{https://help.nytimes.com/hc/en-us/articles/115014893428-Terms-of-service}{Terms
  of Service}
\item
  \href{https://help.nytimes.com/hc/en-us/articles/115014893968-Terms-of-sale}{Terms
  of Sale}
\item
  \href{https://spiderbites.nytimes.com}{Site Map}
\item
  \href{https://help.nytimes.com/hc/en-us}{Help}
\item
  \href{https://www.nytimes.com/subscription?campaignId=37WXW}{Subscriptions}
\end{itemize}
